\chapter{الجداول}
هذا الدرس هو ملحق مباشر للدرس المتعلق بالمؤشرات، و سيعلّمك أهميتها أكثر. إن كنت تعتقد بأنك قادر على تفادي المؤشرات فأنت مخطئ ! هي في كلّ مكان في لغة الـ\textenglish{C}. لقد حذّرتك !

سنتعلم في هذا الدرس كيف ننشئ متغيرات من نوع "جداول". الجدوال مهمّة للغاية في لغة الـ\textenglish{C} لأنها تساعد في تنظيم سلسلة من القيم.

نبدأ هذا الدرس ببعض الشروحات و التفسيرات حول كيفية عمل الجداول في الذاكرة (سأقدم لك الكثير من المخططات التفسيرية). هذه المقدمات حول الذاكرة مهمة جداً : ستساعدك في في معرفة عمل الجداول. فمن المستحسن أن يعرف المبرمج ما يقوم به كي يتحكم في برامجه أكثر، أليس كذلك ؟

\section{الجداول في الذاكرة}
\textit{"الجداول هي تتابع متغيرات من نفس النوع، موجودة في مكان متواصل من الذاكرة."}

أعرف أن هذا التعريف يشبه قليلا تعريف القاموس. لهذا فسأوضح بطريقة أخرى، فعلياّّ، الجدول عبارة عن "متغيّرات ضخمة" يمكن لها أن تحتوي على أعداد كبيرة من نفس النوع
(\InlineCode{char}،
\InlineCode{long}،
\InlineCode{int}،
\InlineCode{double}...).

للجدول طول محدد. يمكنه أن يكون 2، 3، 10 خانات، 150، 2500 خانة، أنت من يحدد العدد. المخطط التالي مثال عن جدول يحجز 4 خانات بدءاً بالعنوان 1600 :
\Picture{Chapter_II-3_Array-Adresses}
عندما تطلب إنشاء جدول يحجز 4 خانات في الذاكرة، سيطلب برنامجك من نظام التشغيل أن يسمح له باستغلال 4 خانات في الذاكرة، و يجب ان تكون هذه الخانات متتالية يعني الواحدة بجانب الأخرى. و كما ترى أعلاه فالخانات متتابعة 1600, 1601, 1602, 1603 فلا يوجد "فراغ" بينها.

أخيراً، كل خانة تحتوي عددا من نفس النوع. فإن كان الجدول من نوع
\InlineCode{int}
فإن كلّ خانة يجب أن تحتوي عددا من نوع
\InlineCode{int}.
و بهذا نفهم أنه لا يمكننا وضع نوع
\InlineCode{int}
مع
\InlineCode{double}
في الجدول نفسه.

و كتلخيص، هذا أهم ما يجب أن تعرفه بخصوص الجداول :
\begin{itemize}
  \item عندما يتم إنشاء جدول، يأخذ مكانا متواصلاً في الذاكرة. بحيث تكون الخانات متجاورة الواحدة تلو الأخرى.
  \item كل خانات الجدول تكون من نفس النوع، فجدول
الـ\InlineCode{int}
يمكن أن يحمل فقط
\InlineCode{int}،
و لا أي نوع آخر.
\end{itemize}
