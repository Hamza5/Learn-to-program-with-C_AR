\chapter{السلاسل المتّصلة}

لكي نخزّن المعلومات في الذاكرة، استعملنا متغيّرات بسيطة (من نوع 
\InlineCode{int}، \InlineCode{double} \dots)،
كما استعملنا جداول و هياكل مخصّصة. إذا أردت تخزين سلسلة من البيانات، فالأبسط غالباً هو استعمال جداول. 

لكن تصبح الجداول أحياناً محدودة جداً. مثلاً، إذا أنشأت جدولاً ذو 10 خانات ثم تبيّن لك لاحقاً في البرنامج أنك تحتاج إلى حجم أكبر، سيكون من المستحيل تكبير حجم الجدول. و أيضاً لا يمكنك إدخال خانة إلى وسط الجدول.

تمثّل السلاسل المتصّلة طريقة لتنظيم البيانات في الذاكرة بطريقة أكثر مرونة. و بما أن لغة 
\textenglish{C}
لا تقترح قاعدياً هذا النظام من التخزين، سيكون علينا أن ننشئه بأنفسنا. سيكون تمريناً ممتازاً يساعدك على أن ترتاح أكثر مع هذه اللغة.
