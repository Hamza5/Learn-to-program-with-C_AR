\chapter{القوائم المتسلسلة (\textenglish{Linked lists})}

لكي نخزّن المعلومات في الذاكرة، استعملنا متغيّرات بسيطة (من نوع
\InlineCode{int}، \InlineCode{double} \dots)،
كما استعملنا جداول و هياكل مخصّصة. إذا أردت تخزين سلسلة من البيانات، فالأبسط غالباً هو استعمال جداول. 

لكن تصبح الجداول أحياناً محدودة جداً. مثلاً، إذا أنشأت جدولاً ذو 10 خانات ثم تبيّن لك لاحقاً في البرنامج أنك تحتاج إلى حجم أكبر، سيكون من المستحيل تكبير حجم الجدول. و أيضاً لا يمكنك إدخال خانة إلى وسط الجدول.

تمثّل القوائم المتسلسلة طريقة لتنظيم البيانات في الذاكرة بطريقة أكثر مرونة. و بما أن لغة الـ\textenglish{C}
لا تقترح قاعدياً هذا النظام من التخزين، سيكون علينا أن ننشئه بأنفسنا. سيكون تمريناً ممتازاً يساعدك على أن ترتاح أكثر مع هذه اللغة.

\section{تمثيل قائمة متسلسلة}

ماهي القائمة المتسلسلة ؟ أقترح عليك أن تنطلق من نموذج الجدول. يمكن تمثيل الجدول في الذاكرة بالطريقة التي توضّحها الصورة التالية. نتكلّم هنا عن جدول يحتوي على خانات من نوع
\InlineCode{int}.

\Picture{Chapter_IV-1_Array}

\begin{information}
اخترت هنا تمثيل الجدول أفقياً، لكن يمكن تمثيله عمودياً كذلك، هذا لا يهم.
\end{information}


كما قلت لك في المقدّمة، مشكل الجداول يمكن في كونها ثابتة. لا يمكن تكبير حجمها، إلا إذا فكّرنا في إعادة إنشائها من جديد و تكون أكبر (لاحظ الشكل التالي). أيضاً، لا يمكن أن نضيف عنصُراً في وسط الجدول إلا إذا قمنا بإزاحة كلّ العناصر الأخرى.

\Picture{Chapter_IV-1_Array-add}

لا تقترح علينا لغة الـ\textenglish{C}
نظاماً آخراً لتخزين البيانات، لكن من الممكن أن ننشئ بأنفسنا هذا النظام بعناصره الكاملة : ستكون الغاية من هذا الفصل و الفصول الموالية اكتشاف حلول لهذا المشكل.

القائمة المتسلسلة هي طريقة لتنظيم سلسلة من البيانات في الذاكرة. هذا يسمح بجمع هياكل
(\textenglish{structures})
مرتبطة ببعضها البعض بواسطة مؤشّرات. يمكننا تمثيلها كالتالي :

\Picture{Chapter_IV-1_Linked-list}

يمكن لكلّ عنصر أن يحتوي على ما نريد : قيمة من نوع 
\InlineCode{int}
أو أكثر،
\InlineCode{double} \dots
بالإضافة إلى ذلك، كلّ عنصر يحتوي على مؤشّر نحو العنصر الموالي :

\Picture{Chapter_IV-1_Linked-list-data}

أعرف بأن كلّ هذه المعلومات نظرية و ربّما تبدو لك غير واضحة الآن. احفظ فقط طريقة اتّصال العناصر ببعضها : هي تشكل 
\textbf{سلسلة من المؤشّرات}،
و من هنا نجد الإسم "قائمة متسلسلة". 

\begin{information}
على عكس الجداول، لا تتموضع عناصر السلسلة المتّصلة جنباً إلى جنب في الذاكرة. كل خانة تؤشّر نحو خانة أخرى لا تتواجد ضرورياً بجنب الأخرى.
\end{information}
