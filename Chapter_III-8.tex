\chapter{تشغيل الصوت بـ\textenglish{FMOD}}

منذ أن اكتشفنا الـ\textenglish{SDL}،
تعلّمنا موضعة صور على النافذة، التفاعل مع المُستعمل بالفأرة و لوحة المفاتيح، كتابة نصوص، لكن ينقص أمر بالتأكيد : الصوت !

سيسدّ هذا الفصل ذلك النقص. بما أن الإمكانيّات التي توفّرها لنا الـ\textenglish{SDL}
من ناحية الصوت محدودة جداً، سنكتشف هنا مكتبة متخصصة في الصوت :
\textenglish{FMOD}.

\section{تثبيت \textenglish{FMOD}}

\subsection{لماذا \textenglish{FMOD} ؟}

أنت تعرف ذلك الآن: الـ\textenglish{SDL}
ليست فقط مكتبة رسومية. هي تسمح أيضاً بمعالجة الصوت عن طريق وحدة تسمّى
\textenglish{SDL\_audio}.
فلماذا إذاً سنحضّر مكتبة خارجية لا علاقة لها بالـ\textenglish{SDL}
كـ\textenglish{FMOD} ؟

في الواقع هو اختيار قمتُ به بعد عدّة اختبارات. كان بإمكاني أن أشرح لك طريقة معالجة الصوت بالـ\textenglish{SDL}
لكنّي فضّلت عدم فعل ذلك.\\
سأشرح موقفي أكثر.

\subsection{لماذا قمتُ بتجنّب \textenglish{SDL\_audio} ؟}

يعتبر التحكم في الصوت بالـ\textenglish{SDL}
"منخفض المستوى". هذا يعني أنه يجب القيام بالعديد من التعامُلات الدقيقة كي نستطيع تشغيل الصوت. بمعنى آخر، سيكون الأمر صعباً و لا أجد ذلك ممتعاً. توجد مكتبات أخرى تسمح بتشغيل الصوت بشكل بسيط.

\begin{information}
تذكير بسيط : مكتبة "منخفضة المستوى" هي مكتبة قريبة من الحاسوب. يجب أن نتعرّف إذا على قليل من العمل الداخلي للحاسوب كي نستفيد منها و يتطلب الأمر في الواقع وقتاً أكثر من الوقت اللازم للقيام بنفس الشيء مع مكتبة "عالية المستوى".\\
لا تنس أنّ كلّ شيء نسبيّ : لا توجد مكتبات منخفضة المستوى من جهة و أخرى عالية المستوى من جهة أخرى. هي فقط أكثر أو أقل من بعضها البعض في المستوى. مثلا، المكتبة
\textenglish{FMOD}
عالية المستوى مقارنة بالوحدة
\textenglish{SDL\_audio}
من الـ\textenglish{SDL}.
\end{information}

تفصيل آخر مهم، تسمح الـ\textenglish{SDL}
بتشغيل صوت بصيغة
\textenglish{WAV}
فقط. صيغة الصوت هذه ليست مضغوطة. أي أن موسيقى من 3 دقائق تأخذ عشرات الميغا أوكتي،
على عكس الصيغ المضغوطة مثل
\textenglish{MP3}
أو
\textenglish{Ogg}
التي تحجز حجم ذاكرة أقلّ بكثير (من 2 إلى 3 ميغا أوكتي).

في الواقع، لو نفكّر في الأمر جيًداً، كان الأمر مشابهاً بالنسبة للصور، فالـ\textenglish{SDL}
لا تتعامل إلا مع الصيغة
\textenglish{BMP}
(صُور غير مضغوطة) بشكل مبدئي. مما استوجب علينا تسطيب مكتبة إضافية و هي
\textenglish{SDL\_image}
لنتمكّن من قراءة صيغ الصور الأخرى كـ\textenglish{JPEG}، \textenglish{PNG}، \textenglish{GIF}،
إلخ.

اعلم أنه هناك مكتبة مكافئة بالنسبة للصوت و هي :
\textenglish{SDL\_mixer}.
هي قادرة على قراءة عدد كبير من صيغ الصوت، من بينها
\textenglish{MP3}، \textenglish{Ogg}، \textenglish{Midi} \dots
و رغم ذلك، لم أكلّمك عن هذه المكتبة. لماذا ؟

\subsection{لماذا قمتُ بتجنّب \textenglish{SDL\_mixer} ؟}

\textenglish{SDL\_mixer}
هي مكتبة نضيفها للـ\textenglish{SDL}،
بطريقة 
\textenglish{SDL\_image}.
هي سهلة للاستعمال و تقرأ العديد من صيغ الصوت المختلفة. لكن، و بعد الاختبارات التي قمتُ بها، تبيّن لي أن هذه المكتبة تحتوي عِلَلا مزعجة بالإضافة إلى كونها محدودة من ناحية المزايا التي تمنحها.

من أجل هذه الأسباب توجّهت مباشرة إلى
\textenglish{FMOD}،
مكتبة لا علاقة لها بالـ\textenglish{SDL}
بالتأكيد لكن لها الأفضلية كونها قوية و متداولا عليها.

\subsection{تنزيل \textenglish{FMOD}}

إن كنت قد حكيت لك كل هذا، فهذا فقط لأخبرك بأن اختيار
\textenglish{FMOD}
لم يكن عشوائياً. ببساطة هي أفضل مكتبة مجانية استطعت إيجادها.\\
كما أنها سهلة الإستخدام كـ\textenglish{SDL\_mixer}
بأفضلية لا يمكن تجاهلها : لا توجد بها مشاكل برمجية.

تسمح
\textenglish{FMOD}
بالقيام بالعديد من الوظائف التي لا تسمح بها
\textenglish{SDL\_mixer}،
كالتأثيرات الصوتيّة ثلاثية الأبعاد.

\begin{warning}
\textenglish{FMOD}
هي مكتبة مجانية لكن ليست تحت رخصة
\textenglish{LGPL}
على عكس الـ\textenglish{SDL}.
هذا يعني أنه بإمكانك أن تستخدمها مادامت لم تحقق بها برامج مدفوعة. إذا أردت أن يكون البرنامج غير مجاني، يجب أن تدفع رسوماً لمؤلّف المكتبة (سأتركك تطّلع على الأسعار من خلال الموقع الرسمي لـ\textenglish{FMOD}).\\
كثير من الألعاب التجارية تستعمل
\textenglish{FMOD}
و من أشهر هذه الألعاب :
\textenglish{Starcraft II}، \textenglish{World of Warcraft : Cataclysm}، \textenglish{Crysis 2}،
إلخ.
\end{warning}

تتوفر العديد من نسخ
\textenglish{FMOD}،
و النسخة الموجهّة إلى الاستعمال في أنظمة التشغيل المألوفة
(\mbox{\textenglish{GNU/Linux}}، \textenglish{Windows}، \mbox{\textenglish{Mac OS X}}، \dots)
تُدعى
\textenglish{FMOD Ex Programmers API}.

نزّل إذا نسخة
\textenglish{FMOD Ex}
المناسبة لنظام التشغيل الخاص بك. خذ النسخة المسمّاة "مستقرة"
(\textenglish{stable}).

و تأكد بشكل خاص ما إن كان لديك نظام تشغيل
\textenglish{32 bits}
أو
\textenglish{64 bits}
(في
\textenglish{Windows}،
 قم بنقر يميني على جهاز الكمبيوتر
(\textenglish{Computer})
ثم في قسم الخصائص
(\textenglish{Properties})
تجد المعلومة اللازمة).

\url{http://www.fmod.org/fmod-downloads.html#FMODExProgrammersAPI}

\subsection{تثبيت \textenglish{FMOD}}

يعمل التثبيت بنفس مبدأ عمل المكتبات السابقة، أي مثل الـ\textenglish{SDL}.

يجدر بالملف الذي حمّلته أن يكون ملفاً تنفيذياً (في
\textenglish{Windows})،
 أو أن يكون أرشيفا 
(\InlineCode{.dmg}
في
\mbox{\textenglish{Mac OS X}}
و
\InlineCode{.tar.gz}
في
\mbox{\textenglish{GNU/Linux}}).

\begin{enumerate}
	\item ثبّت
	\mbox{\textenglish{FMOD Ex}}
على قرصك الصلب. الملفات التي نحتاجها يجب أن تتواجد في مجلّد يشبه هذا :\\
	\InlineCode{C:\textbackslash Program Files\textbackslash FMOD SoundSystem\textbackslash FMOD Programmers API Win32\textbackslash api}.
	\item في هذا المجلّد تجد الـ\textenglish{DLL}
	الخاصّ بـ\mbox{\textenglish{FMOD Ex}}
	(\InlineCode{fmodex.dll})
	و يجب أن يوضع في مجلّد المشروع. الـ\textenglish{DLL}
	الأخرى، أي
	\InlineCode{fmodexL.dll}
	تعمل على تنقيح العلل البرمجية. لن نقوم بذلك هنا. تذكّر فقط بأن الملف
	\InlineCode{fmodex.dll}
	هو الذي يجب أن تُعطيه مع الملف التنفيذي للبرنامج.
	\item في المجلّد
	\InlineCode{api/inc}،
	تجد الملفات
	\InlineCode{.h}.
	ضعها كلّها إلى جانب الملفات الرأسية التي هي في مجلّد البيئة التطويرية. مثلا :
	\InlineCode{Code Blocks/mingw32/include/fmodex}
	(لقد أنشأت مجلّدا خصيصاً لأجل
	\textenglish{FMOD}
	كما مثل الـ\textenglish{SDL}).
	\item في المجلّد
	\InlineCode{api/lib}،
	استرجع الملف الموافق للمترجم. يجدر بملف نصّي أن يشير إلى أي ملف يجب أن نأخذ.
	\begin{itemize}
		\item إذا كنت تستعمل
		\textenglish{Code::Blocks}،
		فالمترجم هو
		\textenglish{mingw}.
		أنسخ الملف
		\InlineCode{libfmodex.a}
		في المجلّد
		\InlineCode{lib}
		للبيئة التطويرية.
		
		في
		\textenglish{Code::Blocks}،
		إنه المجلّد
		\InlineCode{CodeBlocks/mingw32/lib}.
		\item إذا كنت تستعمل
		\textenglish{Visual C++}،
		استرجع الملف
		\InlineCode{fmodex\_vc.lib}.
	\end{itemize}
	\item أخيراً، الشيء الأكثر أهمية ربّما، يوجد مجلّد 
	\InlineCode{documentation}
	في المجلّد
	\textenglish{FMOD Ex}.
	من المفروض أن تتم إضافة اختصارات إلى قائمة "إبدأ" نحو هذه الملفات التوجيهية. أبق نظرك عليها لأننّا لا يمكن ألا نكتشف كلّ ميزات
	\textenglish{FMOD Ex}
	في هذا الفصل. ستحتاج إلى هذه الملفات في أقرب الآجال بالتأكيد.
	
	يبقى أن نخصص المشروع. هنا أيضاً و مثل كلّ مرة : افتح المشروع بواسطة البيئة التطويرية المفضّلة و أضف الملف
	\InlineCode{.a}
	(أو
	\InlineCode{.lib})
	إلى قائمة الملفات التي يجب أن يسترجعها محرر الروابط.\\
	في
	\textenglish{Code::Blocks}
	(يخالجني شعور بأنني أقوم بالتكرار)، إذهب إلى قائمة
	\InlineCode{Project} / \InlineCode{Build Options}
	ثم قسم
	\InlineCode{Linker}،
	أنقر على
	\InlineCode{Add}
	و أشر إلى المسار الذي يوجد به الملف
	\InlineCode{.a}
	إذا ظهرت لك الرسالة~:
	"\textenglish{Keep as a relative path ?}"،
	أنصحك بأن تجيب بالسلب لكن يجدر بالأمور أن تشتغل في كلتا الحالتين.
	
	تم تثبيت
	\textenglish{FMOD Ex}،
	فلنَرَ بسرعة مما هي مُشَكَّلَة.
\end{enumerate}

\section{تهيئة و تحرير غرض نظامي}

المكتبة
\textenglish{FMOD Ex}
متوفّرة من أجل اللغتين
\textenglish{C}
و
\textenglish{C++}.\\
الشيء الخاص فيها هو أن مطوّري هذه المكتبة احتفظوا ببعض التناسق في "تركيب الكلمات"
(\textenglish{syntax})
بين اللغتين. الميزة الأولى هي أنه إذا تعلّمت التعامل مع
\textenglish{FMOD Ex}
في لغة الـ\textenglish{C}
ستتمكن من فعل ذلك في الـ\textenglish{C++}
بنسبة 95\%.

\subsection{تضمين الملف الرأسي}

قبل كلّ شيء، يلزمك أن تقوم بتضمين الملف الرأسي الخاص بـ\textenglish{FMOD}.
 لابأس في التذكير بكتابته :

\begin{Csource}
#include <fmodex/fmod.h>
\end{Csource}

لقد وضعت هذا الملف في المجلّد الداخلي
\InlineCode{fmodex}.
عدّل على هذا السطر من الشفرة على حسب المسار الذي يتواجد به الملف عندك.\\
إذا كنت تعمل على
\mbox{\textenglish{GNU/Linux}}،
 يجدر بالتسطيب أن يتم تلقائيّا في المجلّد
\InlineCode{fmodex}.

\subsection{إنشاء و تهيئة غرض نظامي}

الغرض النظامي هو عبارة عن متغير نستفيد منه على طول البرنامج لكي نعرّف معاملات المكتبة.\\
تذكّر أنه بالـ\textenglish{SDL}
مثلاً، كان يجب أن نهيّئ المكتبة بشكل مباشر بواسطة دالة. هنا، دليل الاستعمال مختلف قليلاً : في عوض تهيئة كلّ المكتبة، لن نعمل إلا بغرض
(\textenglish{Object})
دوره تعريف سلوك هذه الأخيرة.

لكي ننشئ غرضا نظاميا، يكفي أن نعرّف مؤشّرا من نوع
\InlineCode{FMOD\_SYSTEM}.
مثلا :

\begin{Csource}
FMOD_SYSTEM *system;
\end{Csource}

لكي نحجز مكاناً في الذاكرة من أجل هذا الغرض النظامي، نستعمل الدالة
\InlineCode{FMOD\_System\_Create}
و التي نموذجها هو الآتي :

\begin{Csource}
FMOD_RESULT FMOD_System_Create(FMOD_SYSTEM ** system);
\end{Csource}

لاحظ أن هذه الدالة تأخذ مؤشّرا نحو مؤشّر يؤشّر نحو
\InlineCode{FMOD\_SYSTEM}.
القرّاء الأكثر حرصاً كانوا قد لاحظوا أنه لدى تعريف المؤشّر
\InlineCode{FMOD\_SYSTEM}،
لم يتم حجزه بواسطة
\InlineCode{malloc}
أو أي دالة أخرى. لهذا السبب تماماً تأخذ الدالة
\InlineCode{FMOD\_SYSTEM}
معاملا من ذلك النوع لكي تحجز مكاناً للمؤشّر النظامي.

بعد تعريف الغرض النظامي ، تكفي كتابة :

\begin{Csource}
FMOD_SYSTEM *system;
FMOD_System_Create(&system);
\end{Csource}

هكذا إذا، بما أننا نتوفّر الآن على الغرض النظامي، لم يتبّق علينا سوى تهيئته. لفعل هذا، نستعمل الدالة
\InlineCode{FMOD\_System\_Init}
ذات النموذج :
\begin{Csource}
FMOD_RESULT FMOD_System_Init(
	FMOD_SYSTEM *  system,
	int  maxchannels,
	FMOD_INITFLAGS  flags,
	void *  extradriverdata
);
\end{Csource}

\begin{itemize}
	\item المعامل
	\InlineCode{system}
	هو المعامل الذي يهمّنا أكثر، لأنه المؤشّر الذي سنقوم بتهيئته.
	\item المعامل
	\InlineCode{maxchannels}
	يمثّل العدد الأقصى للقنوات التي يجب أن تديرها
	\InlineCode{FMOD}.
	بمعنى آخر، هو العدد الأقصى للأصوات التي يمكن أن يتم تشغيلها في نفس الوقت. هذا يعتمد على قوة بطاقة الصوت لديك . ننصح عادة بقيمة 32 (قيمة كافية من أجل معظم الألعاب البسيطة). لمعلوماتك، يمكن نظرياً لـ\InlineCode{FMOD}
	إدارة 1024 قناة مختلفة، لكن بهذا المستوى ستخاطر بجعل حاسوبك يشتغل كثيرا !
	\item المعامل
	\InlineCode{flag}
	لا يهمّنا كثيراً في هذا الدرس، سنكتفي بإعطائها القيمة
	\InlineCode{FMOD\_INIT\_NORMAL}.
	\item المعامل
	\InlineCode{extradriverdata}
	لا يهمّنا أيضاً، سنعطيه القيمة
	\InlineCode{NULL}.
\end{itemize}

مثلا، لكي نعرّف، نحجز، و نهيّئ غرضا نظاميا،  نقوم بكتابة التالي :

\begin{Csource}
FMOD_SYSTEM *system;
FMOD_System_Create(&system);
FMOD_System_Init(system, 2, FMOD_INIT_NORMAL, NULL);
\end{Csource}

نتوفّر الآن على غرض نظامي جاهز للإستعمال.

\subsection{غلق و تحرير غرض نظامي}

نغلق ثمّ نحرر الغرض النظامي بواسطة دالتين :

\begin{Csource}
FMOD_System_Close(system);
FMOD_System_Release(system);
\end{Csource}

هل يجدر بي أن أعلّق على هذه الشفرة ؟

\section{الأصوات القصيرة}

فلنبدأ بدراسة الأصوات قصيرة المدّة.\\
"الصوت القصير"  كما أسمّيه، هو صوت يستمرّ غالباً بضعة ثوانٍ (أحياناً أقل من ثانية) و غالباً ما يُوجه للاستعمال المنتظم.

أمثلة عن أصوات قصيرة :

\begin{itemize}
	\item صوت إطلاق رصاصة.
	\item صوت مشي اللاعب.
	\item صوت 
	\textenglish{tic-tac}
	(لكيّ نوتّر اللاعب قبل انتهاء العد العكسي).
	\item صوت التصفيق.
	\item إلخ.
\end{itemize}

باختصار، كل صوت لا يعتبر موسيقى.\\
بشكل عام، هذه الأصوات قصيرة المدة إلى درجة أنه لا نحتاج إلى ضغطها. نجدها إذا في غالب الأحيان بصيغة
\textenglish{WAV}
غير مضغوطة.

\subsection{إيجاد الأصوات القصيرة}

 قبل أن نبدأ، سيكون من الجيد أن نتعرّف على بعض المواقع التي تقترح بنوكاً من الأصوات. بالفعل، لا أحد يريد أن يبدأ في تسجيل الأصوات بنفسه في المنزل.
 
سيكون الأمر جيداً فالأنترنت تقترح أصواتا قصيرة، غالباً بصيغة 
\textenglish{WAV}.\\
أين نجدها ؟ قد يبدو الأمر سخيفاً، لا يجب علينا أن نفكّر في ذلك (مع أنه لازم)، لكن
\textenglish{Google}
صديقنا. بشكل عشوائي، أكتب : 
"\textenglish{Free Sounds}"
و التي تعني "أصوات مجانية" بالإنجليزيّة، ستظهر لي ملايين النتائج.

لا شيء تحتاجه أكثر من الصفحة الأولى للبحث، ستجد ضالّتك هناك.\\
شخصياً حفظت الموقع 
\href{http://www.findsounds.com/}{\textenglish{FindSounds.com}}،
محرّك بحث متخصص في الأصوات. لا أدري إن كان الأفضل، لكن على أي حال هو موقع كامل.

\begin{information}
إذا لم تعرف  ماهي الكلمات المفتاحية التي تستعملها في البحث، توجّه إلى الصفحة الخاصة بأمثلة عن الكلمات المفتاحية للبحث.
يجب عليك أن تجيد بعض الكلمات الإنجليزية بالتأكيد (لكن على أي حال إن كنت تريد أن تصبح مبرمجاً، كيف ستفعل لو أنك لا تجيد على الأقل اللغة الإنجليزية ؟).
\end{information}

بالبحث عن كلمة 
"\textenglish{gun}"،
سنجد أطنان من أصوات إطلاق النار بالبندقية، لو نكتب
"\textenglish{door}"
سنجد أصوات تحرّك الباب (الصورة التالية)، إلخ.

\Picture{Chapter_III-8_FindSounds}

\subsection{الخطوات التي يجب إتّباعها لتشغيل الصوت}

 الخطوة الأولى تنصّ على تحميل الصوت الذي نريد تشغيله في الذاكرة.\\
أنصحك بتحميل كلّ الأصوات التي ترى أنك ستستعملها كثيراً منذ بداية البرنامج. تقوم بتحريرها في النهاية. في الواقع، ما إن يتم تحميل الصوت في الذاكرة، ستكون قراءته سريعة جداً.

\subsubsection{المؤشّر}

الخطوة الأولى : إنشاء المؤشّر من نوع
\InlineCode{FMOD\_SOUND}
و الذي يمثّل الصوت.

\begin{Csource}
FMOD_SOUND *fire = NULL;
\end{Csource}

\subsubsection{تحميل الصوت}

الخطوة الثانية : تحميل الصوت بواسطة الدالة
\InlineCode{FMOD\_System\_CreateSound}.
هي تأخذ \dots خمسة معاملات :

\begin{itemize}
	\item \textbf{الغرض النظامي}
	 الذي تحدّثنا عنه سابقاً.
	
	بالطبع يجب أن يكون هذا الغرض جاهزاً للاستعمال (معرّفا، محجوزاً و مهيّئاً).
	\item \textbf{اسم الملف الصوتي}
	الذي نريد تحميله. يمكن أن يكون ذو صيغة
	\textenglish{WAV}، \textenglish{MP3}، \textenglish{OGG}،
	إلخ. من المستحسن دائماً أن يتم تحميل أصوات قصيرة (بضع ثوانٍ كحدّ أقصى) على أن يتم تحميل أصوات طويلة. في الواقع، تحمّل الدالة و تفك تشفير كلّ الصوت في الذاكرة، مما قد يأخذ مكانا كبيرا لو أن الصوت هو موسيقى !
	\item المعامل الثالث هو عَلم. 
	
	يهمّنا بشكل خاص هنا لأنه بفضله يمكننا أن نقول لـ\textenglish{FMOD}
	أن الصوت الذي ستشّغله هو عبارة عن صوت قصير. من أجل هذا نستعمل القيمة
	\InlineCode{FMOD\_CREATESAMPLE}.
	\item و الرابع لا يهمّنا، سنعطيه القيمة 
	\InlineCode{NULL}.
	\item المعامل الأخير هو من نوع
	\InlineCode{FMOD\_SOUND ** sound}،
	و هذا المؤّشر سنستعمله لاحقاً من أجل تشغيل الصوت.
	
	بشكل ما، يمكننا القول بأن هذا المؤشّر سيؤشّر على الصوت الذي نريد تشغيله.
\end{itemize}

 هذا مثال عن تحميل :

\begin{Csource}
FMOD_System_CreateSound(system, "pan.wav", FMOD_CREATESAMPLE, 0, &fire);
\end{Csource}

هنا، أقوم بتحميل الصوت
\InlineCode{pan.wav}.
المؤشّر
\InlineCode{fire}
سيعتبر كمرجع للصوت لاحقاً.

\begin{information}
 إذا كنت تريد أن تختبر الشفرات في نفس الوقت الذي أعطيها لك، أنصحك بتنزيل الصوت
\InlineCode{pan.wav}\\
 (\url{http://www.siteduzero.com/uploads/fr/ftp/mateo21/pan.wav})\\
 و الذي سأستعمله أيضاً في بقيّة هذا الفصل.
\end{information}

إذا اشتغل كلّ شيء على ما يُرام، تُرجع الدالة القيمة
\InlineCode{FMOD\_OK}
و إلا، فهذا يعني انه حدث مشكل خلال فتح الملف الصوتي (ملف تالف أو غير موجود مثلا).

\subsubsection{تشغيل الصوت}

تريد تشغيل الصوت ؟ لا يوجد مشكل مع الدالة
\InlineCode{FMOD\_System\_PlaySound} !\\
يكفي أن تعطيها غرضا نظاميا جاهزا للإستعمال، رقم القناة التي نريد أن يُلعب فيها الصوت و أيضاً المؤشّر نحو الصوت، إضافة إلى معاملات أخرى لا تهمّنا سنعطيها القيمة
\InlineCode{NULL}
أو 0. بالنسبة لرقم القناة، لا تشغل بالك بالتفكير و ابعث القيمة
\InlineCode{FMOD\_CHANNEL\_FREE}
و اترك
\textenglish{FMOD}
تتحكّم في ذلك.

\begin{Csource}
FMOD_System_PlaySound(system, FMOD_CHANNEL_FREE, fire, 0, NULL);
\end{Csource}

\subsubsection{تحرير الصوت من الذاكرة}

حينما تصبح غير محتاجاً للصوت، يجب عليك تحريره. \\
لا يوجد ما هو أسهل، يكفي أن تشير إلى المؤشّر الذي تريد تحريره بواسطة الدالة
\InlineCode{FMOD\_Sound\_Release}.


\subsection{مثال : لعبة إطلاق النار}

الأفضل الآن هو أن نلخّص كل ما تعلّمناه، عن طريق مثال واضح عن برنامج مكتوب بالـ\textenglish{SDL}.\\
لا يوجد شيء معقّد هنا و يجدر ألا تصادفك أية مشكلة في تحقيق هذا التمرين.

\subsubsection{الموضوع}

مهمتك سهلة : إنشاء لعبة إطلاق النار.\\
حسناً، لن ننشئ لعبة كاملة هنا، لكننا سنتحكّم في
\underline{المصوّب}
 لقد صممتُ لك مصوّباً بسيطاً بواسطة برنامج الرسام :

\Picture{Chapter_III-8_Aim}

باختصار، إليك المهام :

\begin{itemize}
	\item خلفية النافذة : سوداء.
	\item مؤشّر الفأرة : غير مرئي.
	\item يتم تسوية صورة المصوّب على وضعية الفأرة حينما نقوم بتحريكه. احذر : يجب أن يتم لصق مركز الصورة على مستوى مؤشّر الفأرة.
	\item حينما ننقر بالفأرة، يجب تشغيل الصوت
	\InlineCode{pan.wav}.
\end{itemize}

قد تكون هذه البداية لصنع لعبة إطلقا نار كاملة.\\
سهل جداً ؟ حسناً، حان وقت العمل إذا !

\subsubsection{التصحيح}

هذه هي الشفرة المصدرية الكاملة :

\begin{Csource}
#include <stdlib.h>
#include <stdio.h>
#include <SDL/SDL.h>
#include <SDL/SDL_image.h>
#include <fmodex/fmod.h>
int main(int argc, char *argv[])
{
	SDL_Surface *screen = NULL, *gunSight = NULL;
	SDL_Event event;
	SDL_Rect position;
	int cont = 1;
	FMOD_SYSTEM *system;
	FMOD_SOUND *fire;
	FMOD_RESULT result;
	// Initializing and creating a system object
	FMOD_System_Create(&system);
	FMOD_System_Init(system, 1, FMOD_INIT_NORMAL, NULL);
	// Loading the sound and checking the loading
	result = FMOD_System_CreateSound(system, "pan.wav", FMOD_CREATESAMPLE, 0, &fire);
	if (result != FMOD_OK)
	{
		fprintf(stderr, "Can't read pan.wav\n");
		exit(EXIT_FAILURE);
	}
	// Initializing the SDL
	SDL_Init(SDL_INIT_VIDEO);
	SDL_ShowCursor(SDL_DISABLE);
	screen = SDL_SetVideoMode(640, 480, 32, SDL_HWSURFACE | SDL_DOUBLEBUF);
	SDL_WM_SetCaption("Managing sound with FMOD", NULL);
	gunSight = IMG_Load("viseur.png");
	while (cont)
	{
		SDL_WaitEvent(&event);
		switch(event.type)
		{
			case SDL_QUIT:
			cont = 0;
			break;
			case SDL_MOUSEBUTTONDOWN:
			// When we click, we play the sound
			FMOD_System_PlaySound(system, FMOD_CHANNEL_FREE , fire, 0, NULL);
			break;
			case SDL_MOUSEMOTION:
			// When we move the mouse, we move the gun sight too. To do this we have to use gunSight->w/2, gunSight->h/2
			position.x = event.motion.x - (gunSight->w / 2);
			position.y = event.motion.y - (gunSight->h / 2);
			break;
		}
		SDL_FillRect(screen, NULL, SDL_MapRGB(screen->format, 0, 0, 0));
		SDL_BlitSurface(gunSight, NULL, screen, &position);
		SDL_Flip(screen);
	}	
	// We close the SDL
	SDL_FreeSurface(gunSight);
	SDL_Quit();
	// We free the sound, free and close the system object
	FMOD_Sound_Release(fire);
	FMOD_System_Close(system);
	FMOD_System_Release(system);
	return EXIT_SUCCESS;
}
\end{Csource}

الصورة التالية تعطيك لمحة عن اللعبة المصغّرة، لكن الأفضل أن ترى الفيديو بالصوت على الأنترنت :

\Picture{Chapter_III-8_Window-aim}

مشاهدة الفيديو هنا :\\
\url{https://openclassrooms.com/uploads/fr/ftp/mateo21/viseur.html}

 هنا، حمّلت
\textenglish{FMOD}
 قبل الـ\textenglish{SDL}
 و حررتها بعدها. لا توجد قواعد من ناحية الترتيب (كان بإمكاني القيام بالعكس). اخترتُ تحميل الـ\textenglish{SDL}
 و فتح النافذة بعد تحميل
\textenglish{FMOD}
 لكي تكون اللعبة جاهزة للاستعمال ما إن يتم فتح النافذة (و إلا كان بالإمكان أن تنتظر بعض الميلي ثواني ريثما يتم تحميل
\textenglish{FMOD}).\\
 و مع ذلك أنت حر باختيار الترتيب، هذه تفاصيل ليس إلا.

أعتقد أنني علّقت كفاية على الشفرة. لا يوجد فخ معيّن، و لا يوجد جديد يذكر. يمكننا أن نذكر الصعوبة "الصغيرة" التي تنص على لصق مركز المصوّب على مستوى مؤشّر الفأرة. يتم حساب وضعية الصورة بدلالة ذلك.

بالنسبة لمن لم يفهم الفرق بعد، سنرى ذلك الآن. بالمناسبة، لقد أعدت تفعيل الإظهار الخاص بالمؤشّر كي نرى كيف يتموضع المصوّب بالنسبة لمؤشّر الفأرة.

\begin{Table*}{3}
شفرة خاطئة (المصوب في وضعية خاطئة) &
\includegraphics[width=0.07\textwidth]{Chapter_III-8_Aim-misplaced} &
\parbox{0.45\textwidth}{\small\setLTR
\textenglish{\texttt{position.x = event.motion.x;\\
position.y = event.motion.y;}
}
\unsetLTR}
\\
شفرة صحيحة (المصوب في وضعية صحيحة) &
\includegraphics[width=0.07\textwidth]{Chapter_III-8_Aim-wellplaced} &
\parbox{0.45\textwidth}{\small\setLTR
\textenglish{\texttt{position.x = event.motion.x - (gunSight->w / 2);\\
position.y = event.motion.y - (gunSight->h / 2);}
}
\unsetLTR}
\\
\end{Table*}

\subsubsection{أفكار للتحسين}

هذه اللعبة تعتبر قاعدة للعبة إطلاق النار. لديك المصوّب، صوت إطلاق النار، لم يتبقّ لك سوى إظهار أو تمرير أعداء لكي يقوم اللاعب بإطلاق النار عليهم ثم يتم تسجيل النتيجة. كالعادة، عليك بالعمل وحدك. تريد صنع لعبة ؟ لا شيء يمنعك، بل لديك المستوى الكافي الآن، و لديك أيضاً شفرة مبدئية للعبة إطلاق النار ! ماذا تنتظر، صدقاً ؟ 

\begin{information}
طبعاً، منتديات
\href{http://www.siteduzero.com/forum-81-126-langage-c.html}{\textenglish{OpenClassrooms}}
ستساعدك في حال علقت في أيّ لحظة أثناء إنشائك للعبة. مواجهة الصعوبات أمر عادي مهما كان المستوى الذي أنت فيه.
\end{information}
