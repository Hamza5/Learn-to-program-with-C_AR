\chapter{الحصول على الأدوات اللازمة}

بعد تجاوزنا لدرس تمهيدي مليئ بالثرثرة سوف نبدأ بالدخول في صلب الموضوع. سوف نجيب عن السؤال التالي : "ما هي البرامج التي نحتاج إليها للبدء في البرمجة ؟".

لا يوجد شيء صعب في هذا الدرس، سوف نأخذ وقتنا للتأقلم على هذه البرامج الجديدة.

اغتنموا الفرصة ! في الفصل التالي سنبدأ حقّا في البرمجة و لن يكون هناك وقت للقيلولة !

\section{الأدوات اللازمة للمبرمج}

إذن ما هي الأدوات التي نحتاج إليها ؟
إذا تابعت الفصل السابق جيّدا، فستعرف واحدا على الأقل !

هل تعلم عمّا أتحدّث ؟ حقّا لا ؟

حسنا، نحن نتحدّث عن
\textbf{المترجم}
الذي يمكّن من ترجمة لغة الـ\textenglish{C}
إلى اللغة الثنائيّة !

كما قلت لكم في الفصل الأوّل، يوجد العديد من المترجمات للغة الـ\textenglish{C}.
سنرى أن اختيار المترجم ليس أمرا معقّدا في حالتنا هذه.

ما الذي نحتاج إليه أيضا ؟ لن أتركك تخمّن كثيرا و سأعطيك القائمة :

\begin{itemize}
  \item \textbf{محرّر نصوص }
(\textenglish{Text Editor})
لكتابة الشفرة المصدرية الخاصّة بالبرنامح. نظريّا برنامج تحرير نصوص بسيط مثل
\textenglish{Notepad}
على
\textenglish{Windows}
أو
\textenglish{vi}
على
\textenglish{Unix}
يكفي، لكن من الأحسن استخدام محرّر نصوص ذكيّ يقوم بتلوين الشفرة المصدرية لكي يسهّل عليك العمل.
  \item \textbf{مترجم}
  لتحويل الشفرة المصدرية إلى ملف ثنائي.
  \item \textbf{المنقّح}
(\textenglish{Debugger})
لمساعدك على كشف الأخطاء في برنامجك. لسوء الحظ، لم نتمكّن بعد من ابتكار "المصحّح" الّذي يصحّح أخطائك لوحده. لكن، إن أحسنت استخدام المنقّح، يمكنك ببساطة إيجاد الأخطاء.
\end{itemize}

وجود مكتشف الأخطاء لا يعنى أن تتصرف بتهوّر و تسرع في كتابة برنامج مليء بالأخطاء، بل تريّث و كن هادئاً.

من الآن لدينا خياران :

\begin{itemize}
  \item إمّا أن نحصل على البرامج الثلاثة متفرّقة و هذه هي الطريقة الأكثر تعقيدا، و لكنّها تعمل. على
\textenglish{GNU/Linux}
تحديدا، عدد كبير من المبرمجين يفضّلون استخدام كلّ برنامج على حدة. لن أشرح هذه الطريقة هنا، بل سأتحدّث عن الطريقة الأسهل.
  \item أو أن تحصل على برنامج "ثلاثة في واحد" يتضمّن محرّر النصوص و المترجم و المنقّح. هذا النوع من البرامج يعرف باسم "بيئات التطوير المتكاملة"
(\textenglish{Integrated Development Environments})
و تسمّى اختصارا
\textenglish{IDE}.
\end{itemize}

يوجد العديد من بيئات التطوير. بداية قد تواجه صعوبة في اختيار البيئة الملائمة لك. الشيء الأكيد هو : أي بيئة مهما كانت ستحقق لك العمل المطلوب.

\subsection{اختيار البيئة الخاصة بك}

بدا لي أنه من الأفضل أن أريك بعضا من البيئات الشهيرة و المجانيّة في نفس الوقت. شخصيّا، أنا أستخدمها جميعا و أختار في كل يوم  واحدا منها.

\begin{itemize}
  \item أحد هذه البيئات الّتي أفضّلها هو
\textbf{\textenglish{Code::Blocks}}.
هو مجّاني و يعمل على أغلب أنظمة التشغيل. أنصح كلّ مبتدئ أن يختاره للبدء (و في ما بعد أيضا إذا شعرت أنّه يلائمك جيّدا !).

يعمل على أنظمة التشغيل
\textenglish{Windows}،
\textenglish{Mac OS}
و
\textenglish{GNU/Linux}.
  \item الأكثر شهرة على
\textenglish{Windows}
هو الّذي أنشأته
\textenglish{Microsoft}،
إنّه
\textbf{\textenglish{Visual C++}}.
هو برنامج مدفوع (و باهظ الثمن) لكن لحسن الحظّ توجد نسخة مجانية منه تسمّى
\textbf{\textenglish{Visual Studio Express}}
(أنا أستخدم النسخة القديمة
\textbf{\textenglish{Visual C++ Express}}
في هذا الكتاب). و هي ممتازة جدّا (بينها و بين النسخة المدفوعة فوارق طفيفة). إنه برنامج كامل و يملك منقّحا قويّا.

يعمل على
\textenglish{Windows}
فقط.
  \item على
\textenglish{Mac OS X}
يمكنك استخدام
\textbf{\textenglish{XCode}}
الّذي يفترض أن يكون متوفّرا على قرص تثبيت النظام. يناسب كثيرا مبرمجي
\textenglish{Mac}.

يعمل على
\textenglish{Mac OS X}
فقط.
\end{itemize}

\begin{information}
  ملاحظة لمستخدمي
  \textenglish{GNU/Linux} :
  يوجد العديد من البيئات لهذا النظام، و لكن المبرمجين المحترفين قد يفضّلون تجاوز البيئات و القيام بالترجمة "يدويّا"، و هو شيء أصعب قليلا. نحن سنبدأ باستخدام بيئات التطويرية. لذلك أنصحك بثبيت
  \textenglish{Code::Blocks}
  إن كنت على
  \textenglish{GNU/Linux}
  لكي تتمكن من متابعة شروحاتي.
\end{information}

\begin{question}
من هي البيئة الأفضل من بين كلّ بيئات التطوير هذه؟
\end{question}

كل واحدة من هذه البيئات تمكنك من البرمجة و متابعة بقيّة الدرس من دون أيّة مشاكل. بعضها كامل أكثر من ناحية المميزات، و أخرى سهلة الإستخدام أكثر، ولكن في كلّ الأحوال البرامج الّتي تنشؤها تكون ذاتها أيّا كانت البيئة التي اِخْتَرْتها. فهذا الخيار ليس بالأهمّية الّتي تعتقدها.

في هذا الكتاب سوف أستخدم
\textenglish{Code::Blocks}.
فإن أردت الحصول على نفس لقطات الشاشة خاصّتي، خصوصا لكي لا تضيع في البداية، أنصحك بِدَايَةً بتثبيت
\textenglish{Code::Blocks}.

\section{\textenglish{Code::Blocks (Windows, Mac OS X, GNU/Linux)}}

\textenglish{Code::Blocks}
هي بيئة تطوير متكاملة حرّة و مجانيّة، متوفّرة للـ\textenglish{Windows}،
\textenglish{Mac}
و
\textenglish{GNU/Linux}.

حاليّا
\textenglish{Code::Blocks}
متوفّر بالإنجليزيّة فقط. لكن هذا ليس أمرا يدعوك إلى تجنّب استخدامه ! فنحن قلّما نحتاج إلى العمل بقوائم واجهته، فلغة
\textenglish{C}
هي الّتي تهمّنا.

كن على علم أنه عندما تبرمج سوف تقابل عادة توثيئقا بالإنجليزية. هذا سبب آخر يدفعك للتدرّب على استخدام هذه اللغة.

\subsection{تنزيل \textenglish{Code::Blocks}}

توجّه إلى صفحة تنزيل
\textenglish{Code::Blocks}

\url{http://www.codeblocks.org/downloads/binaries}

ثمّ نزّل الملف الذي يناسب نظامك :

\begin{itemize}
  \item إذا كنت تستخدم
\textenglish{Windows}،
إذهب إلى القسم
"\textenglish{Windows}"
في أسفل الصفحة. نزّل البرنامج الّذي يحوي
\InlineCode{mingw}
في اسمه (مثلا :
\InlineCode{codeblocks-10.05mingw-setup.exe}).
النسخة الأخرى لا تحوي مترجما، لن تتمكن في حال استخدمتها من ترجمة برامجك !
  \item إذا كنت تستخدم
\textenglish{GNU/Linux}،
اختر الحزمة الّتي تناسب توزيعتك.

  \item إذا كنت تستخدم
\textenglish{Mac}،
اختر الملف الأحدث في القائمة، مثلا :
\InlineCode{codeblocks-10.05-p2-mac.zip}.

\end{itemize}

\begin{critical}
أقول و أكرّر : إذا كنت تستخدم
\textenglish{Windows}
فيجب عليك تنزيل النسخة الّتي يتضمَن اسمها كلمة
\textenglish{mingw}
لأنّه إذا اخترت النسخة الخاطئة فلن تتمكّن من ترجمة برامجك فيما بعد !
\end{critical}

التثبيت بسيط و سريع.  أترك جميع الخيارات كما هي و شغّل البرنامج. سوف تظهر لك نافذة شبيهة بهذه :

\Picture{Chapter_I-2_CodeBlocks}

نميّز أربعة أقسام رئيسية في واجهة البرنامج، و هي مرقّمة في الصورة :

\begin{enumerate}
  \item شريط الأدوات
(\textenglish{Toolbar}) :
يحتوي على كثير من الأزرار و لكنّنا سوف نستخدم بعضها فقط باستمرار، سأعود للحديث عن هذا فيما بعد.
  \item قائمة ملفات المشروع : توجد بيسار النافذة، تحتوي على كلّ الملفات المصدريّة المتعلقة بالبرنامج الذي تعمل عليه. تكون فارغة في البداية لأننا لم ننشئ أي ملف لحد الآن. سوف نبدأ بملأها خلال خمس دقائق من الآن بتقدّمك في هذا الدرس.
  \item المنطقة الرئيسية : هنا المساحة التي تكتب فيها الشفرة المصدرية الخاصة ببرنامجك بلغة الـ\textenglish{C}.
  \item منطقة الإشعار : و يدعوها البعض "منطقة الموت"، هنا تُعْرَضُ أخطاء الترجمة إذا كانت شفرة البرنامج تحوي خطأً ما. هذا الشيء يحدث كثيرا !
\end{enumerate}

ما يهمّنا الآن هو قسم محدد من شريط الأدوات. تجد فيه الأزرار التالية (بهذا الترتيب) :
\InlineCode{Compile}،
\InlineCode{Execute}،
\InlineCode{Compile \& Execute}،
\InlineCode{Recompile everything}
تذكّرهم جيّدا لأننا سنستخدمهم بانتظام.

\Picture{Chapter_I-2_Compile-toolbar}

و هذا شرح عمل كلّ واحد من هذه الأزرار :

\begin{itemize}
  \item \textbf{\textenglish{Compile}} :
كل ملفّات الشفرة المصدرية الّتي كتبتها يتم ارسالها إلى المترجم الّذي يقوم بإنشاء الملف التنفيذي في حالة عدم وجود أخطاء. أمّا في حالة العثور على أخطاء (و هذا سيحدث عاجلا أم آجلا !) فلن يتمّ إنشاؤه بل يعرض رسائل خطأ في منطقة الإشعار.
  \item \textbf{\textenglish{Execute}} :
يقوم بتشغيل آخر ملف تنفيذي تمّت ترجمته. يعني أنك ستستخدم هذا الزر لاختبار برامجك الّتي أنشأتها. طبعا يجب عليك ترجمة البرنامج قبل تشغيله. يمكننا أيضا استخدام الزرّ الثالث.
  \item \textbf{\textenglish{Compile \& Execute}} :
لا يجب أن تكون عبقريا لكي تعرف أنّه ليس إلا مجموع الزرّين السابقين. في الواقع هذا هو الزرّ الّذي ستستخدمه أكثر. في حالة ما إذا حدثت أخطاء في الترجمة لن يتمّ تشغيل البرنامج بل ستُعرض قائمة جميلة من الأخطاء التي يجب عليك تصحيحها أوّلا !
  \item \textbf{\textenglish{Recompile everything}} :
 عندما نستخدم
\textbf{\textenglish{Compile}}
يقوم
\textenglish{Code::Blocks}
في الحقيقة بترجمة الملفات الّتي عدّلتها فقط. أحيانا -فقط أحيانا- قد تحتاج إلى إعادة ترجمة جميع الملفّات. سنتحدّث لاحقا عن فائدة هذا الزر و أيضا عن كيفية عمل الترجمة بمزيد من التفصيل. حاليا لكي لا تختلط الأمور عليك يمكنك أن تعتبر أنّه غير مهمّ.
\end{itemize}

\begin{information}
أنصحك باستخدام اختصارات لوحة المفاتيح بدلا من الضغط على الأزرار، لأنّه شيء نكرّره كثيرا. تذكّر خصوصا أنّه يمكنك الضغط على
\InlineCode{F9}
بدل الزر
\InlineCode{Compile \& Execute}.
\end{information}

\subsection{انشاء مشروع جديد}

لإنشاء مشروع جديد إذهب إلى قائمة
\InlineCode{File}
ثمّ
\InlineCode{New}
ثمّ
\InlineCode{Project}.
من النافذة الّتي تظهر اختر
\InlineCode{Console application}.

\Picture{Chapter_I-2_CodeBlocks-New-project}

\begin{information}
كما تَرَى،
\textenglish{Code::Blocks}
يقترح عليك إنشاء عدد معتبر من أنواع البرامج الّتي تستخدم مكتبات
(\textenglish{Libraries})
معروفة مثل
\textenglish{SDL}
للـ\textenglish{2D}
و
\textenglish{OpenGL}
للـ\textenglish{3D}
و
\textenglish{Qt}
و
\textenglish{wxWidgets}
لإنشاء النوافذ الرسوميّة. حاليّا، هذه الأيقونات ليست سوى للزينة لأنّ المكتبات السابقة غير مثبّتة على حاسوبك لهذا لا يمكنك أن تجعلها تعمل. سوف نعود لهذه الأنواع الأخرى لاحقا. في هذه الأثناء لا يمكننا سوى أن نستخدم الـ\textenglish{Console}
لأنّك لا تملك بعد المستوى اللازم لانشاء أنواع أخرى من البرامج.
\end{information}

أنقر على
\InlineCode{Go}
لانشاء المشروع الجديد. ثم أنقر على
\InlineCode{Next}
فالصفحة الأولى ليس مهمّة. بعدها سيأتيك اختيار بين لغتي الـ\textenglish{C}
أو الـ\textenglish{C++}،
اختر الـ\textenglish{C}.

\Picture{Chapter_I-2_CodeBlocks-C}

سيُطْلَبُ منك الآن إدخال إسم المشروع، و كذلك مسار المجلّد الذي تختاره لحفظ الملفّات فيه.

\Picture{Chapter_I-2_CodeBlocks_project-path}

آخر خطوة تُطلب منك هي ، كيف ينبغي أن يترجم البرنامج، يمكنك ترك الخيارات على حالها، لن يكون لهذا أي تأثير على ما سنقوم به الآن (تأكّد أن إحدى الخانتين
"\textenglish{Release}"
أو
"\textenglish{Debug}"
تكون محدّدة على الأقل).

\Picture{Chapter_I-2_CodeBlocks_Compiler}

إضغط على
\InlineCode{Finish}،
إنتهى !\\
لقد قام
\textenglish{Code::Blocks}
بانشاء المشروع الأوّل و ملئه ببعض الشفرة المصدرية.

في الخانة الخاصة بالمشاريع على اليسار، قم بتوسيعها بالضغط على
'\InlineCode{+}'
لكي تظهر قائمة الملفات في المشروع. سيكون لديك على الأقل ملف يسمّى
\InlineCode{main.c}.
هذا هو كلّ شيء !
