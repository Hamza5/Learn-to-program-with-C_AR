\chapter{حسابات سهلة}

كما قلت لك في الدرس السابق : جهازك ماهو إلا آلة حاسبة كبيرة. سواء كنت تسمع الموسيقى، تشاهد فلماً أو تلعب لعبة، فإن الحاسوب ينجز الحسابات طيلة الوقت.

هذا الدرس سيساعدك على التعرف على معظم الحسابات التي يقوم بها الجهاز. سنعيد استعمال ما نحن بصدد تعلّمه عن عالم المتغيرات. الفكرة هي أننا سنقوم بعمليات على المتغيرات : نجمعها، نضربها، نخزّن النتائج في متغيرات أخرى، الخ.

حتى و إن لم تكن من هواة الرياضيات، فإن هذا الدرس إلزامي و لا مفرّ منه.

\section{الحسابات القاعدية}

بالرغم من قدرة الجهاز الواسعة إلا أنه في الأساس يتعمد في حساباته على عمليات بسيطة للغاية و هي :

\begin{itemize}
  \item الجمع،
  \item الطرح،
  \item القسمة،
  \item الضرب،
  \item الترديد
(\textenglish{modulo})
(سأشرح لاحقا ما الّذي يعنيه إذا لم تكن تعرفه الآن).
\end{itemize}

إن كان بودك القيام بحسابات أكثر تعقيدا (كالأسس و اللوغاريثم و ماشابه)، يجب عليك إذا برمجتها أو بمعنى آخر :
\textbf{توضح للجهاز كيف يقوم بها}.\\
لحسن الحظ، سترى لاحقاً في هذا الفصل أنه توجد مكتبة في لغة الـ\textenglish{C}،
تحتوي على دوال رياضية جاهزة. لن يكون عليك إعادة كتابتها إلا إذا أردت فعل ذلك تطوّعيا أو كنت أستاذ رياضيات.

لنبدأ بالعملية الأسهل و هي الجمع.\\
طبعا في الجمع نحتاج الرمز +.\\
يجب وضع نتيجة الجمع في متغير و لهذا سنقوم بإنشاء متغير اسمه مثلا
\InlineCode{result}
من نوع
\InlineCode{int}
و نقوم بالحساب :

\begin{Csource}
  int result = 0;
  result = 5 + 3;
\end{Csource}

لا يجب أن تكون محترفا في الحساب الذهني لتعرف أن النتيجة ستكون 8 بعد تشغيل البرنامج.\\
بالطبع البرنامج لن يظهر أية نتيجة باستعمال هذه الشفرة المصدرية. إذا أردت معرفة محتوى المتغير
\InlineCode{result}
عليك باستعمال الدالة
\InlineCode{printf}
التي تجيد كيفية استخدامها جيداً الآن :

\begin{Csource}
  printf("5 + 3 =  %d", result);
\end{Csource}

و هذا ما سيظهر على الشاشة :

\begin{Console}
  5 + 3 = 8
\end{Console}

و هكذا ننهى عملية الجمع بسهولة.\\
الأمر مماثل بالنسبة للعمليات الأخرى، نحتاج تغيير الرمز ليس إلا :

\begin{Table}{2}
  العمليّة & الرمز\\
  الجمع & \texttt{+}\\
  الطرح & \texttt{-}\\
  الضرب & \texttt{*}\\
  القسمة & \texttt{/}\\
  الترديد & \texttt{\%}\\
\end{Table}

إذا كنت قد استعملت من قبل الآلة الحاسبة الخاصة بحاسوبك، فيفترض بك أن تكون متعوّدا على هذه الإشارات. لا يوجد أي شيء صعب بخصوصها باستثناء القسمة و الترديد اللذان سأشرحهما فيما يلي بالتفصيل.

\subsection{القسمة}

ينجز الحاسوب عملية القسمة بشكل طبيعيّ عندما لا يوجد أي باق. مثلا العملية
\InlineCode{6 / 2}
تعطينا النتيجة 3، النتيجة صحيحة. حتّى الآن، لا مشكلة.

لكن لو نأخذ الآن عملية قسمة بباقٍ مثل
\InlineCode{5 / 2}
\dots
نتوقع أن النتيجة ستكون 2.5، و لكن أنظر إلى ما تعطيه الشفرة :

\begin{Csource}
int result = 0;
result = 5 / 2;
printf("5 / 2 =  %d", result);
\end{Csource}

\begin{Console}
  5 / 2 = 2
\end{Console}

هناك مشكل كبير، فنحن نتوقّع أن نحصل على القيمة 2.5، لكن الحاسوب أعطى القيمة 2 !

هل يا ترى أجهزتنا غبية لهذه الدرجة ؟\\
في الواقع، يقوم الجهاز بعملية قسمة صحيحة (إقليدية) أي أنه يحتفظ بالجزء الصحيح فقط الذي هو 2.

\begin{question}
  هه أنا أعرف السبب ! لأن المتغير
  \InlineCode{result}
  الذي استخدمناه هو من نوع
  \InlineCode{int} !
  لو استخدمنا النوع
  \InlineCode{double}
  لاستطاع تخزين العدد العشري !
\end{question}

لا، ليس هذا هو السبب ! جرب  نفس االشفرة بتغيير نوع النتيجة إلى
\InlineCode{double}
و ستجد بأننا نتحصّل على نفس النتيجة 2 لأن طرفا العملية من نوع
\InlineCode{int}
فإن الحاسوب سيعيد نتيجة من نوع
\InlineCode{int}.

إن أردنا أن يظهر لنا الجهاز القيمة الصحيحة، يجب أن نغير العددين 2 و 5 إلى عددين عشريين كالتالي : 2.0 و 5.0 (قيمتهما هي نفسها لكن الجهاز سيعتبرهما عددين عشريين، و بالتالي هو يظن بأنه يقوم بقسمة عددين عشريين) :

\begin{Csource}
  double result = 0;
  result = 5.0 / 2.0;
  printf("5 / 2 =  %f", result);
\end{Csource}

\begin{Console}
  5 / 2 = 2.500000
\end{Console}

هنا العدد صحيح بالرغم من وجود عدة أصفار في نهاية العدد، لكنّ القيمة تبقى نفسها.

فكرة القسمة الإقليدية التي يقوم بها الحاسوب مهمة، تذكّر أنه بالنسبة للحاسوب :

\begin{itemize}
  \item 5 / 2 = 2،
  \item 10 / 3 = 3،
  \item 4 / 5 = 0.
\end{itemize}

هذا مفاجئ بعض الشيء، لكنّها طريقته في التعامل مع الأعداد الصحيحة.

إن أردت الحصول على نتيجة عشريّة، فيجب أن يكون حدّا العملية عشريّين :

\begin{itemize}
  \item 5.0 / 2.0 = 2.5،
  \item 10.0 / 3.0 = 3.33333،
  \item 4.0 / 5.0 = 0.8.
\end{itemize}

يمكن القول أن الجهاز يطرح على نفسه السؤال : "كم يوجد من 2 في العدد 5 ؟" طبعا يوجد 2 فقط.

و لكن أين الباقي من العملية ؟ لأنني لما أقول 5 هي أثنين من 2 ، يبقى 1 طبعا، كيف لنا أن نسترجعه ؟\\
هنا يتدخل الترديد الذي كلمتك عنه.

\subsection{الترديد}

هو عبارة عن عملية حسابية تسمح بالحصول على باقي عملية القسمة، و هي عملية غير معروفة مقارنة بالعمليات الأربع الأخرى، لكن الجهاز يعتبرها من العمليات القاعدية، و يمكن اعتبارها حلا لمشكل قسمة الأعداد الطبيعية.

كما قلت لكم الترديد يمثل بالرمز
\InlineCode{\%}.\\
إليكم بعض الأمثلة :

\begin{itemize}
  \item 5 \% 2 = 1،
  \item 14 \% 3 = 2،
  \item 4 \% 2 = 0.
\end{itemize}

الترديد
\InlineCode{5 \% 2}
هو باقي العملية
\InlineCode{5 / 2}
مما يعني أن الجهاز يقوم بالعملية
\InlineCode{5 = 2 * 2 + 1}
حيث أن 1 هو الباقي و الذي يقوم بإرجاعه الترديد.

نفس الشيء بالنسبة للعملية
\InlineCode{14 \% 3}،
العملية هي
\InlineCode{14 = 3 * 4 + 2}
(الترديد يعطي القيمة  2). أخيرا، من أجل
\InlineCode{4 \% 2}،
القسمة تامة، فلا يوجد باقي، لهذا يعطي الترديد القيمة 0.

حسنا، لا يوجد ما يمكنني إضافته بخصوص عملية الترديد. كان هذا فقط شرحا لمن لا يعرفها.

لدي خبر جيد آخر، و هو أننا أتممنا كلّ عمليات الحساب القاعدية و تخلصنا من درس الرياضيات !

\subsection{عمليات على المتغيرات}

الشيء الجيد هو أنه بعد أن تعلمت كيف تستخدم العمليات القاعدية، يمكنك الآن أن تتعلّم كيفية القيام بهذه العمليات على المتغيرات.\\
لا شيء يمكنه منعك من كتابة الشفرة التالية :

\begin{Csource}
  result = number1 + number2;
\end{Csource}

هذا السطر يعمل على جمع المتغيرين
\InlineCode{number1}
و
\InlineCode{number2}
ثم يخزن النتيجة في المتغير
\InlineCode{result}.

هنا بدأت الامور الممتعة تظهر، و حقيقة، مستواك الحالي يسمح لك ببرمجة آلة حاسبة بسيطة. نعم، نعم، أؤكّد لك ذلك !

تخيل وجود برنامج يطلب من المستخدم إدخال عددين، ثم يقوم بتخزينهما في متغيرين، ثم يجمع هذين المتغيرين و يخزن النتيجة في متغير اسمه
\InlineCode{result}.
لم يبق سوى إظهار النتيجة على الشاشة في وقت لا يتمكّن فيه المستخدم حتى من تخمين النتيجة.

حاول كتابة هذا البرنامج البسيط، إنّه سهل و سيكون تدريبا لك !

إليك الجواب :

\begin{Csource}
int main(int argc, char * argv[])
{
  int result = 0, number1 = 0, number2 = 0;

  // We request the two numbers from the user :

  printf("Enter the first number : ");
  scanf("%d", &number1);
  printf("Enter the second number : ");
  scanf("%d", &number2);

  // We calculate the result:

  result = number1 + number2;

  // We display the result on the screen:

  printf("%d + %d = %d\n", number1, number2, result);

  return 0;
}
\end{Csource}

\begin{Console}
  Enter the first number : 30
  Enter the second number : 25
  30 + 25 = 55
\end{Console}

بدون أن تشعر، لقد أنشأت أول برنامج لك ذو فائدة. إنّه قادر على جمع عددين و إظهارا النتيجة على الشاشة~!

يمكنك التجريب باستخدام أعداد أخرى (يجب ألا تتجاوز الحد الأقصى لتحمّل نوع الـ\InlineCode{int})
و سيقوم الحاسوب بالحساب بشكل سريع جداً لا يتجاوز بعض أجزاء من المليار من الثانية !

أنصحك أيضاً بتجريب العمليات الأخرى (الطرح، القسمة و الضرب) لكي تتدرب. لن يكون هذا متعبا إلّا بقدر تغيير إشارة أو اثنتين. يمكنك أيضاً إضافة متغير ثالث و جمع ثلاثة متغيرات دفعة واحدة. سيشتغل البرنامج دون مشاكل :

\begin{Csource}
result = number1+ number2+ number3;
\end{Csource}
