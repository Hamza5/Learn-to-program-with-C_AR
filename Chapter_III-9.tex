\chapter{عمل تطبيقي : الإظهار الطيفي للصوت}

هذا العمل التطبيقي سيقترح عليك التعامل مع الـ\textenglish{SDL}
و الـ\textenglish{FMOD}
في نفس الوقت. هذه المرّة، لن نعمل على لعبة. كما نعرف فالـ\textenglish{SDL}
مخصصة لهذا، لكن يمكن استعمالها في ميادين أخرى. سيقوم هذا الفصل بإثبات أنها صالحة لأجل أشياء أخرى.

سنحقق هنا إظهاراً للطيف الصوتي بالـ\textenglish{SDL}.
يتوقّف هذا على إظهار تركيبة الصوت الذي نشغّله، مثلاً موسيقى. نجد هذه الخاصية في كثير من برامج قراءة الأصوات. إنه أمرٌ ممتع و ليس بقدر الصعوبة التي يبدو عليها !

سيسمح لك هذا الدرس بالعمل على مفاهيم قُمنا باستكشافها مؤخّراً :

\begin{itemize}
	\item التحكّم في الوقت.
	\item المكتبة 
	\textenglish{FMOD}.
\end{itemize}

سنتعرّف علاوة على ذلك، على كيفية التعديل على مساحة بيكسلا ببيكسل.

الصورة التالية تعطيك مظهراً للبرنامج الذي سنكتبه في هذا الفصل.

\Picture{Chapter_III-9_Window-spectral}

هو نوع الإظهار الذي نجده في قارئي الأصوات كـ\textenglish{Winamp}،
\textenglish{Windows Media Player} أو \textenglish{AmaroK}.\\
كما قلتُ لك إن الأمر ليس صعبٌ التحقيق. على عكس العمل التطبيقي الخاص بـ\textenglish{Mario Sokoban}،
هذه المرّة ستقوم بنفسك بالعمل. سيمثّل هذا بالنسبة إليك تمريناً جيداً.
