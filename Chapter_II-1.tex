\chapter{البرمجة المجزّأة
(\textenglish{Modular programming})}
في هذه المرحلة الثانية، سنكتشف مبادئ متقدّمة في لغة الـ
\textenglish{C}
 لن أخفي عليك، هذه المرحلة صعبة الفهم و تحتاج منك التركيز. في نهاية المرحلة، ستكون قادراً على تدبّر أمرك في معظم البرامج المكتوبة بلغة السي. في المرحلة التي تليها نتعلّم كيف نفتح نافذة، كيف ننشئ لعبة ثنائية الأبعاد... الخ

لحدّ الآن عملنا في ملف واحد سمّيناه
\InlineCode{main.c}
. كان أمراً مقبولاً لحدّ الآن لأن برامجنا كانت صغيرة، لكنها ستصبح في القريب العاجل مركّبة من عشرات، لن أقول من مئات الدوال، و إن كنت تريد وضعها كلّها في نفس الملف، فإن هذا الأخير سيصبح ضخماً جداً. لهذا السبب تم اختراع ما نسمّيه بالبرمجة المجزّأة. المبدأ سهل: بدل أن نضع كل الشفرة المصدرية في ملف واحد
\InlineCode{main.c}
، سنقوم بتفريقها إلى عدة ملفات.

\section{النماذج (\textenglish{prototypes})}
لحدّ الآن، كنت عندما تنشئ دالة، أطلب منك وضعها قبل الدالة الرئيسية
\InlineCode{main}
. لماذا؟

لأن للترتيب أهمية حقيقية هنا: فإن قمت بوضع الدالة قبل الـ
\InlineCode{main}
في الشفرة المصدرية، سيقرؤها الجهاز و يتعرف عليها. حينما تقوم باستدعاء الدالة داخل الـ
\InlineCode{main}
، سيعرفها الجهاز و يعرف أيضاً أين يبحث عليها.\\
بالعكس، لو تضع الدالة بعد الـ
\InlineCode{main}
، لن يعمل البرنامج لأن الجهاز لم يتعرّف بعد على الدالة. جرّب ذلك و سترى.
\begin{question}
  لكنه تصميم سيّء نوعاً ما، أليس كذلك ؟
\end{question}
أنا متفق معك! لكن انتبه المبرمجون لهذه النقطة قبلك و عملوا على حلّ المشكل.

بفضل ما سأعلمك إياه الآن، ستتمكن من الدوال في أي ترتيب كان في الشفرة المصدرية، هكذا لن تقلق من هذه الناحية.

\subsection{استعمال النموذج للتصريح عن دالة}
سنقوم بتصريح دوالنا للحاسوب، و هذا بكتابة ما نسميه بـ
\textbf{النماذج}
.لا تنبهر بهذا الاسم، إنه يخبّئ معلومة بسيطة جداً.

تأمل في السطر الأول من دالتنا
\InlineCode{rectangleSurface}
\begin{Csource}
double rectangleSurface(double width, double height)
{
	return width * height;
}
\end{Csource}
قم بنسخ السطر الأول
(\InlineCode{double rectangleSurface...})
المتواجد أعلى الشفرة المصدرية (مباشرة بعد تعليمات التضمين
\InlineCode{\#include}
). أضف
\textbf{فاصلة منقوطة}
في نهاية هذا السطر.\\
و هكذا يمكنك أن تضع الدالة الخاصة بك
\InlineCode{rectangleSurface}
بعد الدالة
\InlineCode{main}
ان أردت !

هذا ما يجب أن تكون عليه الشفرة المصدرية :
\begin{Csource}
#include <stdio.h>
#include <stdlib.h>
// The next line represents the prototype of the function rectangleSurface :
double rectangleSurface(double width, double height);
int main(int argc, char *argv[])
{
	printf("width = 5 and height = 10. Surface = %f\n", rectangleSurface(5, 10));
	printf("width = 2.5 and height = 3.5. Surface = %f\n", rectangleSurface(2.5, 3.5));
	printf("width = 4.2 and height = 9.7. Surface = %f\n", rectangleSurface(4.2, 9.7));

	return 0;
}
// Now, we can put our function wherever we want in the source code:
double rectangleSurface(double width , double height )
{
	return width * height ;
}
\end{Csource}
الشيء الذي تغيّر هنا هو إضافة النموذج أعلى الشفرة المصدرية.\\
النموذج هو عبارة عن إشارة للجهاز، يوحي إليه بوجود دالة تسمى
\InlineCode{rectangleSurface}
و التي تأخذ معاملات إدخال معينة و تُرجِع مخرج من نوع أنت من تحدده.  هذا يساعد الجهاز على تنظيم نفسه.

بفضل ذلك السطر، يمكنك الآن وضع دوالك في أي ترتيب كان دون أي تفكير زائد.

أكتب دائما النموذج الخاص بدوالك. البرامج التي ستكتبها من الآن و صاعداً ستصبح أكثر تعقيداً و تستعمل الكثير من الدوال: من الأحسن أن تتعلّم منذ الآن العادة الجيدة  بوضع نموذج لكل دالة في الشفرة المصدرية.

كما ترى، الدالة
\InlineCode{main}
لا تملك أي نموذج، و كمعلومة فهي الوحيدة التي لا تملك نموذجاً ! لأن الجهاز يعرفها (فهي نفسها مكررة في جميع البرامج).

عليك أن تعرف أنه في سطر النموذج، لست مضطراً إلى تحديد المعاملات التي تتلقاها الدالة كمدخل. الجهاز يحتاج أن يتعرّف إلى نوع المداخل فقط.

يمكننا أن نكتب ببساطة :
\begin{Csource}
double rectangleSurface (double, double);
\end{Csource}
و مع ذلك، فالطريقة التي أريتك إياها أعلاه تعمل أيضاً. الشيء الجيد فيها هو أن كلّ ما عليك فعله هو نسخ و لصق السطر الأول الخاص بالدالة مع إضافة فاصلة منقوطة (طريقة سهلة و سريعة).
\begin{critical}
  لا تنس
\underline{أبدا}
وضع فاصلة منقوطة بعد النموذج، هذا يمكّن الحاسوب من التفريق بين النموذج و بداية الدالة.\\
إن لم تفعل، ستعترضك أخطاء غير مفهومة أثناء عملية الترجمة.
\end{critical}

\section{الملفات الرأسية
(\textenglish{headers})}
لحدّ الآن لا نملك غير ملف مصدري واحد في مشروعنا و هو الذي كنّا نسمّيه
\InlineCode{main.c}.

\subsection{عدة ملفات في مشروع واحد}
تطبيقياً، برامجك لن تكون مكتوبة في ملف واحد
\InlineCode{main.c}.
بالطبع يمكن فعل ذلك، لكن لن يكون من الممتع أن تتجوّل في ملف به 10000 سطر (شخصياً أعتقد هذا). و لهذا فإنه في العادة ننشئ العديد من الملفات في المشروع الواحد.
\begin{question}
  عفوا ... ماهو المشروع ؟
\end{question}
لا ! هل نسيت بسرعة ؟ سأعيد الشرح لأنه من اللازم أن نتّفق على هذا المصطلح.

المشروع هو مجموع الملفات المصدرية الخاصة ببرنامجك. لحد الآن برنامجنا لم تتكون إلا من ملف واحد. و يمكنك التحقق من هذا بالنظر في البيئة التطويرية الخاصة بك، غالبا ما يظهر المشروع في القائمة على اليسار (الصورة الموالية):
\Picture{Chapter_II-1_Project}
كما يمكنك رؤيته في يسار الصورة، هذا المشروع ليس مكوّنا إلا من الملف
\InlineCode{main.c}.

إسمح لي الآن أن أُرِيَكَ صورة لمشروع حقيقي ستقوم به في وقت لاحق من الدروس : لعبة سوكوبان (الصورة الموالية) :
\Picture{Chapter_II-1_Project-Sokoban}
كما ترى، هناك ملفات عديدة. هذا ما يكون عليه المشروع الحقيقي، أي تتواجد به ملفات عديدة في القائمة اليسارية  يمكن التعرّف على الملف
\InlineCode{main.c}
من بين القائمة و الذي يحتوي الدالة
\InlineCode{main}.
بصورة عامة في برامجي، لا أضع إلّا الدالة
\InlineCode{main}
في الـملف
\InlineCode{main.c}.
لمعلوماتك، هذا ليس أمراً إجبارياً، كل واحد ينظّم ملفاته بالشكل الذي يريد. لكن لكي تتبعني جيّداً أنصحك بفعل ذلك.
\begin{question}
  لكن لم يجب عليّ إنشاء ملفات عديدة ؟ و كم من ملف يجب علىّ أن أنشئ في مشروعي ؟
\end{question}
هذا يبقى اختيارك أنت، في الغالب نجمع في نفس الملف المصدري الدوال التي تشترك في الموضوع الذي تعالجه. و هكذا ففي الملف
\InlineCode{editeur.c}
جمعت كلّ الدوال الخاصة ببناء المستوى، و في الملف
\InlineCode{jeu.c}
قمت بتجميع الدوال الخاصة باللعبة نفسها و هكذا ...

\subsection{الملفات
\texttt{\textenglish{.c}}
و
\texttt{\textenglish{.h}}}
كما يمكنك أن تلاحظ، يوجد نوعان مختلفان من الملفات في الصورة السابقة.
\begin{itemize}
  \item \textbf{ملفات ذات الإمتداد
\texttt{\textenglish{.c}}}
: الملفات المصدرية، تحتوي الدوال نفسها.
  \item \textbf{ملفات ذات الإمتداد
\texttt{\textenglish{.h}}}
: تسمى الملفات الرأسية و هي تحتوي النماذج الخاصة بالدوال.
\end{itemize}
عموما، انه لمن النادر وضع نماذج في الملفات من صيغة
\InlineCode{.c}
مثلما فعلنا للتوّ في الـملف
\InlineCode{main.c}
(إلا إذا كان برنامجك صغيرا).

من أجل كل ملف
.\InlineCode{c}
هناك ملف مكافئ له، و الذي يحتوي نماذجا للدوال الموجودة في الملف
\InlineCode{.c}
، تمعّن في الصورة السابقة.
\begin{itemize}
  \item هناك
\InlineCode{editeur.c}
(الشفرة الخاصة بالدوال) و
\InlineCode{editeur.h}
(ملف النماذج الخاصة بالدوال).
  \item هناك
\InlineCode{jeu.c}
و
\InlineCode{jeu.h}.
  \item إلخ...
\end{itemize}
\begin{question}
  لكن كيف يعرف الحاسوب بأن نماذج الدوال موجودة في ملف آخر خارج الملف
\InlineCode{.c}
؟
\end{question}
يجب عليك تضمين الملف الرأسي
\InlineCode{.h}.
مستعيناً بتوجهات المعالج القبلي.\\
كن مستعداً لأنّي سأعطيك الكثير من المعلومات في وقت قصير.

كيف نقوم بتضمين ملف رأسي ؟ أنت تجيد فعل ذلك لأنك قمت بذلك من قبل.

أنظر مثالاً من بداية الملف
\InlineCode{jeu.c} :
\begin{Csource}
#include <stdlib.h>
#include <stdio.h>
#include "jeu.h"
void play(SDL_Surface* screen)
{
// ...
\end{Csource}
التضمين يتم عن طريق توجيهات المعالج القبلي
\InlineCode{\#include}
التي يجدر بك أن تكون قد تعلّمتها من قبل.\\
تمعن في التالي :
\begin{Csource}
#include <stdlib.h>
#include <stdio.h>
#include "jeu.h" // We include jeu.h
\end{Csource}
قمنا بتضمين ثلاثة ملفات من صيغة
\InlineCode{.h}
و هي :
\InlineCode{stdio}، \InlineCode{stdlib} و \InlineCode{jeu}.\\
لاحظ الفرق : الملفات التي قمت بإنشاءها ووضعها في الـمجلّد الخاص بمشروعك يجب أن تكون مضمّنة مع اشارات الاقتباس
(\InlineCode{"jeu.h"})
بينما ملفات المكتبات (التي توجد عادة في البيئة التطويرية الخاصة بك) تكون مضمّنة بعلامات الترتيب
(\InlineCode{<stdio.h>}).

تستعمل إذا :
\begin{itemize}
  \item علامتي الترتيب
\InlineCode{< >}
: لتضمين الملفات المتواجدة في المجلّد
\InlineCode{include}
الخاص بالبيئة التطويرية.
  \item علامتي الاقتباس
\InlineCode{" "}
 : لتضمين  الملفات المتواجدة في مجلّد المشروع (و غالبا بجانب الملفات
\InlineCode{.c}).
\end{itemize}
الأمر
\InlineCode{\#include}
يطلب إدخال محتوى ملف معيّن في الملف
\InlineCode{.c}
فهي تعليمة تقول :"أدخل الملف
\InlineCode{jeu.h}
هنا" مثلا .

\begin{question}
  و في الملف
\InlineCode{jeu.h}
ماذا نجد ؟
\end{question}
لا نجد إلا نماذج خاصة بدوال الملف
\InlineCode{jeu.c}
!
\begin{Csource}
void play(SDL_Surface* screen);
void movePlayer(int map[][NB_BLOCS_HEIGHT], SDL_Rect *pos, int direction);
void moveBox(int *firstBox, int *secondeBox);
\end{Csource}
هكذا يعمل المشروع الحقيقي !
\begin{question}
  ما الهدف من وضع نماذج في ملفات من نوع
  \InlineCode{.h}
  ؟
\end{question}
السبب بسيط للغاية، عندما تستدعي دالة في الشفرة المصدرية الخاصة بك، ينبغى لجهازك أن يكون متعرفا عليها من قبل، و يعرف كم من المعاملات تستعمل...الخ. إن هذا هو الهدف وراء وجود النماذج، انه دليل الاستخدام الخاص بالدالة بالنسبة للجهاز.

كلّ هذا هو مسألة تنظيم، عندما تضع نماذجك في ملفات
\InlineCode{.h}
(ملفات رأسية) مضمّنة في أعلى الملفات
\InlineCode{.c}
، سيعرف جهازكم طريقة استخدام الدوال الموجودة في الملف ما إن يبدأ في قراءته.

عند القيام بهذا، لن يكون عليك القلق حيال الترتيب الذي ستكون عليه دوالك في الملفات
\InlineCode{.c}.
اذا كنت قمت الآن بإنشاء برنامج صغير يحتوي على دالتين أو ثلاث يمكنك أن تفكّر أنه من الممكن للبرنامج أن يتشغل دون وجود النماذج، لكن هذا لن يستمر ذلك طويلا ! فما إن يكبر البرنامج و إن لم تنظّم النماذج في ملفات رأسيّة فستفشل الترجمة دون أدنى شك.
\begin{information}
  عندما تستدعي دالة متواجدة في الملف
  \InlineCode{functions.c}
  إنطلاقا من الملف
  \InlineCode{main.c}
  سيكون عليك تضمين النماذج الخاصة بالملف
  \InlineCode{functions.c}
  في الملف
  \InlineCode{main.c}
  يجب إذن وضع
  \InlineCode{\#include "functions.h"}
  في أعلى الـملف
  \InlineCode{main.c}.\\
  تذكر هذه القاعدة : "في كلّ مرة تستدعي الدالة
  \textenglish{X}
  في ملف، يجب عليك إدراج نموذج هذه الدالة في ملفكم" هذا ما يسمح للـمترجم بمعرفة ما إن كنت قد استدعيتها بشكل صحيح.
\end{information}
\begin{question}
  كيف أقوم بإضافة ملفات
\InlineCode{.c}
 و
\InlineCode{.h}
 إلى مشروعي ؟
\end{question}
هذا راجع للـبيئة التطويرية التي تستخدمها. لكن المبدأ هو نفسه في جميع البرامج :
\InlineCode{File} / \InlineCode{New} / \InlineCode{Source File}\\
هذا يسمح بإنشاء ملف جديد فارغ. هذا الملف ليس حاليا من النوع
\InlineCode{.c}
ولا
\InlineCode{.h}
أنت من يحدد ذلك أثناء عملية حفظ الملف. قم إذن بحفظه (حتّى و إن كان لا يزال فارغا !) و هنا يطلب منكم إدخال اسم للملف، يمكنك هنا اختيار صيغة الملف :
\begin{itemize}
  \item إذا سميته
\InlineCode{file.c}
فسيكون بامتداد
\InlineCode{.c}.
  \item إذا سميته
\InlineCode{file.h}
فسيكون بامتداد
\InlineCode{.h}.
\end{itemize}
هذا سهل. قم بحفظ الملف في المجلّد أين تتواجد باقي الملفات الخاصة بمشروعك (نفس المجلّد أين يتواجد الملف
\InlineCode{main.c}). عموما كل ملفات المشروع تقوم بحفظها في نفس المجلّد سواء كانت ذات صيغة
\InlineCode{.c}
أو
\InlineCode{.h}.

مجلّد المشروع في النهاية سيكون مثل هذا :
\Picture{Chapter_II-1_Project-Sokoban-Folder}
الملف الذي أنشأته محفوظ لكن لم تتم إضافته إلى مشروعك بعد !\\
لإضافته قم بالنقر يمينا على القائمة أيسر الشاشة (الخاصة بملفات المشروع) و اختر
\InlineCode{Add files}
كالتالي :
\Picture{Chapter_II-1_Project-Add-File}
ستظهر لك نافذة تطلب منك اختيار الملفات التي تريد أن تدخلها للمشروع، اختر الملف الذي قمت بإنشاءه، للتو، و سيتم إدخاله أخيرا في المشروع.  ستجده حاضراً في القائمة اليسارية !
