\chapter{البرمجة المجزّأة (\textenglish{Modular Programming})}

في هذه المرحلة الثانية، سنكتشف مبادئ متقدّمة في لغة \textenglish{C}
 لن أخفي عليك، هذه المرحلة صعبة الفهم وتحتاج منك التركيز. في نهاية المرحلة، ستكون قادرًا على تدبّر أمرك في معظم البرامج المكتوبة بلغة \textenglish{C}. 
في المرحلة التي تليها نتعلّم كيف نفتح نافذة، كيف ننشئ لعبة ثنائية الأبعاد\dots إلخ.

لحدّ الآن عملنا في ملف واحد سمّيناه
\InlineCode{main.c}.
كان أمرًا مقبولًا لحدّ الآن لأن برامجنا كانت صغيرة، لكنها ستصبح في القريب العاجل مركّبة من عشرات، لن أقول من مئات الدوال، وإن كنت تريد وضعها كلّها في نفس الملف، فإن هذا الأخير سيصبح ضخمًا جدًا. لهذا السبب تم اختراع ما نسمّيه بالبرمجة المجزّأة. المبدأ سهل: بدل أن نضع كل الشفرة المصدرية في ملف واحد
\InlineCode{main.c}
، سنقوم بتفريقها إلى عدة ملفات.

\section{النماذج (\textenglish{Prototypes})}

لحدّ الآن، كنت عندما تنشئ دالة، أطلب منك وضعها قبل الدالة الرئيسية
\InlineCode{main}. لماذا؟

لأن للترتيب أهمية حقيقية هنا: فإن قمت بوضع الدالة قبل \InlineCode{main}
في الشفرة المصدرية، سيقرأها الجهاز ويتعرف عليها. حينما تقوم باستدعاء الدالة داخل \InlineCode{main}
، سيعرفها الجهاز ويعرف أيضًا أين يبحث عليها.\\
بالعكس، لو تضع الدالة بعد \InlineCode{main}
، لن يعمل البرنامج لأن الجهاز لم يتعرّف بعد على الدالة. جرّب ذلك وسترى.

\begin{question}
  لكنه تصميم سيّء نوعًا ما، أليس كذلك؟
\end{question}

أنا متفق معك! لكن انتبه المبرمجون لهذه النقطة قبلك وعملوا على حلّ المشكل.

بفضل ما سأعلمك إياه الآن، ستتمكن من الدوال في أي ترتيب كان في الشفرة المصدرية، هكذا لن تقلق من هذه الناحية.

\subsection{استعمال النموذج للتصريح عن دالة}

سنقوم بتصريح دوالنا للحاسوب، وهذا بكتابة ما نسميه بـ\textbf{النماذج}
.لا تنبهر بهذا الاسم، إنه يخبّئ معلومة بسيطة جدًا.

تأمل في السطر الأول من دالتنا
\InlineCode{rectangleSurface}

\begin{Csource}
double rectangleSurface(double width, double height)
{
	return width * height;
}
\end{Csource}

قم بنسخ السطر الأول
(\InlineCode{double rectangleSurface...})
المتواجد أعلى الشفرة المصدرية (مباشرة بعد تعليمات التضمين
\InlineCode{\#include}). أضف
\textbf{فاصلة منقوطة}
في نهاية هذا السطر.\\
و هكذا يمكنك أن تضع الدالة الخاصة بك
\InlineCode{rectangleSurface}
بعد الدالة
\InlineCode{main}
ان أردت!

هذا ما يجب أن تكون عليه الشفرة المصدرية:

\begin{Csource}
#include <stdio.h>
#include <stdlib.h>
// The next line represents the prototype of the function rectangleSurface :
double rectangleSurface(double width, double height);
int main(int argc, char *argv[])
{
	printf("width = 5 and height = 10. Surface = %f\n", rectangleSurface(5, 10));
	printf("width = 2.5 and height = 3.5. Surface = %f\n", rectangleSurface(2.5, 3.5));
	printf("width = 4.2 and height = 9.7. Surface = %f\n", rectangleSurface(4.2, 9.7));

	return 0;
}
// Now, we can put our function wherever we want in the source code :
double rectangleSurface(double width , double height )
{
	return width * height ;
}
\end{Csource}

الشيء الذي تغيّر هنا هو إضافة النموذج أعلى الشفرة المصدرية.\\
النموذج هو عبارة عن إشارة للجهاز، يوحي إليه بوجود دالة تسمى
\InlineCode{rectangleSurface}
و التي تأخذ معاملات إدخال معينة وتُرجِع مخرجا من نوع أنت من تحدده. هذا يساعد الجهاز على تنظيم نفسه.

بفضل ذلك السطر، يمكنك الآن وضع دوالك في أي ترتيب كان دون أي تفكير زائد.

أكتب دائما النموذج الخاص بدوالك. البرامج التي ستكتبها من الآن وصاعدًا ستصبح أكثر تعقيدًا وتستعمل الكثير من الدوال: من الأحسن أن تتعلّم منذ الآن العادة الجيدة  بوضع نموذج لكل دالة في الشفرة المصدرية.

كما ترى، الدالة
\InlineCode{main}
لا تملك أي نموذج، وكمعلومة فهي الوحيدة التي لا تملك نموذجًا! لأن الجهاز يعرفها (فهي نفسها مكررة في جميع البرامج).

عليك أن تعرف أنه في سطر النموذج، لست مضطرًا إلى تحديد المعاملات التي تتلقاها الدالة كمدخل. الجهاز يحتاج أن يتعرّف إلى نوع المداخل فقط.

يمكننا أن نكتب ببساطة:

\begin{Csource}
double rectangleSurface (double, double);
\end{Csource}

و مع ذلك، فالطريقة التي أريتك إياها أعلاه تعمل أيضًا. الشيء الجيد فيها هو أن كلّ ما عليك فعله هو نسخ ولصق السطر الأول الخاص بالدالة مع إضافة فاصلة منقوطة (طريقة سهلة وسريعة).

\begin{critical}
  لا تنس
\underline{أبدًا}
وضع فاصلة منقوطة بعد النموذج، هذا يمكّن الحاسوب من التفريق بين النموذج وبداية الدالة.\\
إن لم تفعل، ستعترضك أخطاء غير مفهومة أثناء عملية الترجمة.
\end{critical}

\section{الملفات الرأسية (\textenglish{Headers})}

لحدّ الآن لا نملك غير ملفٍ مصدريٍ واحد في مشروعنا وهو الذي كنّا نسمّيه
\InlineCode{main.c}.

\subsection{عدة ملفات في مشروع واحد}

عَمَلِيًا، برامجك لن تكون مكتوبة في ملف واحد
\InlineCode{main.c}.
بالطبع يمكن فعل ذلك، لكن لن يكون من الممتع أن تتجوّل في ملف به 10000 سطر (شخصيًا أعتقد هذا). ولهذا فإنه في العادة ننشئ العديد من الملفات في المشروع الواحد.
\begin{question}
  عفوا\dots ماهو المشروع؟
\end{question}
لا! هل نسيت بسرعة؟ سأعيد الشرح لأنه من اللازم أن نتّفق على هذا المصطلح.

المشروع هو مجموع الملفات المصدرية الخاصة ببرنامجك. لحد الآن برنامجنا لم تتكون إلا من ملف واحد. ويمكنك التحقق من هذا بالنظر في البيئة التطويرية الخاصة بك، غالبا ما يظهر المشروع في القائمة على اليسار (الصورة الموالية):

\begin{figure}[H]
	\centering
	\includegraphics[width=0.4\textwidth]{Chapter_II-1_Project}
\end{figure}

كما يمكنك رؤيته في يسار الصورة، هذا المشروع ليس مكوّنا إلا من الملف
\InlineCode{main.c}.

اسمح لي الآن أن أُرِيَكَ صورة لمشروع حقيقي ستقوم به في وقت لاحق من الكتاب: لعبة 
\textenglish{Sokoban}:

\begin{figure}[H]
	\centering
	\includegraphics[width=0.4\textwidth]{Chapter_II-1_Project-Sokoban}
\end{figure}

كما ترى، هناك ملفات عديدة. هذا ما يكون عليه المشروع الحقيقي، أي تتواجد به ملفات عديدة في القائمة اليسارية  يمكن التعرّف على الملف
\InlineCode{main.c}
من بين القائمة والذي يحتوي الدالة
\InlineCode{main}.
بصورة عامة في برامجي، لا أضع إلّا الدالة
\InlineCode{main}
في الملف
\InlineCode{main.c}.
لمعلوماتك، هذا ليس أمرًا إجباريًا، كل واحد ينظّم ملفاته بالشكل الذي يريد. لكن لكي تتبعني جيّدًا أنصحك بفعل ذلك.

\begin{question}
  لكن لم يجب عليّ إنشاء ملفات عديدة؟ وكم من ملف يجب علىّ أن أنشئ في مشروعي؟
\end{question}

هذا يبقى اختيارك أنت، في الغالب نجمع في نفس الملف المصدري الدوال التي تشترك في الموضوع الذي تعالجه. وهكذا ففي الملف
\InlineCode{editeur.c}
جمعت كلّ الدوال الخاصة ببناء المستوى، وفي الملف
\InlineCode{jeu.c}
قمت بتجميع الدوال الخاصة باللعبة نفسها وهكذا\dots

\subsection{الملفات \texttt{.c} و\texttt{.h}}

كما يمكنك أن تلاحظ، يوجد نوعان مختلفان من الملفات في الصورة السابقة.

\begin{itemize}
  \item \textbf{ملفات ذات الامتداد
\InlineCode{.c}}:
 الملفات المصدرية، تحتوي الدوال نفسها.
  \item \textbf{ملفات ذات الامتداد
\InlineCode{.h}}:
 تسمى الملفات الرأسية وهي تحتوي النماذج الخاصة بالدوال.
\end{itemize}

عموما، إنه لمن النادر وضع نماذج في الملفات من صيغة
\InlineCode{.c}
مثلما فعلنا للتوّ في الملف
\InlineCode{main.c}
(إلا إذا كان برنامجك صغيرا).

من أجل كل ملف
\InlineCode{.c}
هناك ملف مكافئ له، والذي يحتوي نماذجا للدوال الموجودة في الملف
\InlineCode{.c}،
تمعّن في الصورة السابقة.

\begin{itemize}
  \item هناك
\InlineCode{editeur.c}
(الشفرة الخاصة بالدوال) و
\InlineCode{editeur.h}
(ملف النماذج الخاصة بالدوال).
  \item هناك
\InlineCode{jeu.c}
و
\InlineCode{jeu.h}.
  \item إلخ.
\end{itemize}

\begin{question}
  لكن كيف يعرف الحاسوب بأن نماذج الدوال موجودة في ملف آخر خارج الملف
\InlineCode{.c}؟
\end{question}

يجب عليك تضمين الملف الرأسي
\InlineCode{.h}.
مستعينًا بتوجهات المعالج القبلي.\\
كن مستعدًا لأنّي سأعطيك الكثير من المعلومات في وقت قصير.

كيف نقوم بتضمين ملف رأسي؟ أنت تجيد فعل ذلك لأنك قمت بذلك من قبل.

أنظر مثالًا من بداية الملف
\InlineCode{jeu.c}:

\begin{Csource}
#include <stdlib.h>
#include <stdio.h>
#include "jeu.h"
void play(SDL_Surface* screen)
{
// ...
\end{Csource}

التضمين يتم عن طريق توجيهات المعالج القبلي
\InlineCode{\#include}
التي يجدر بك أن تكون قد تعلّمتها من قبل.\\
تمعن في التالي:

\begin{Csource}
#include <stdlib.h>
#include <stdio.h>
#include "jeu.h" // We include jeu.h
\end{Csource}

قمنا بتضمين ثلاثة ملفات من صيغة
\InlineCode{.h}
و هي:
\InlineCode{stdio}، \InlineCode{stdlib} و\InlineCode{jeu}.\\
لاحظ الفرق: الملفات التي قمت بإنشائها ووضعها في المجلّد الخاص بمشروعك يجب أن تكون مضمّنة بإشارات الاقتباس
(\InlineCode{"jeu.h"})
بينما ملفات المكتبات (التي توجد عادة في البيئة التطويرية الخاصة بك) تكون مضمّنة بعلامات الترتيب
(\InlineCode{<stdio.h>}).

تستعمل إذا:

\begin{itemize}
  \item علامتي الترتيب
\InlineCode{< >}:
 لتضمين الملفات المتواجدة في المجلّد
\InlineCode{include}
الخاص بالبيئة التطويرية.
  \item علامتي الاقتباس
\InlineCode{" "}:
 لتضمين  الملفات المتواجدة في مجلّد المشروع (وغالبا بجانب الملفات
\InlineCode{.c}).
\end{itemize}

الأمر
\InlineCode{\#include}
يطلب إدخال محتوى ملف معيّن في الملف
\InlineCode{.c}
فهي تعليمة تقول: "أدخل الملف
\InlineCode{jeu.h}
هنا" مثلا.

\begin{question}
  وفي الملف
\InlineCode{jeu.h}
ماذا نجد؟
\end{question}

لا نجد إلا نماذج خاصة بدوال الملف
\InlineCode{jeu.c}!

\begin{Csource}
void play(SDL_Surface* screen);
void movePlayer(int map[][NB_BLOCS_HEIGHT], SDL_Rect *pos, int direction);
void moveBox(int *firstBox, int *secondeBox);
\end{Csource}

هكذا يعمل المشروع الحقيقي!

\begin{question}
  ما الهدف من وضع نماذج في ملفات من نوع
  \InlineCode{.h}؟
\end{question}

السبب بسيط للغاية، عندما تستدعي دالة في الشفرة المصدرية الخاصة بك، ينبغى لجهازك أن يكون متعرفا عليها من قبل، ويعرف كم من المعاملات تستعمل\dots إلخ. إن هذا هو الهدف وراء وجود النماذج، إنه دليل الاستخدام الخاص بالدالة بالنسبة للجهاز.

كلّ هذا هو مسألة تنظيم، عندما تضع نماذجك في ملفات
\InlineCode{.h}
(ملفات رأسية) مضمّنة في أعلى الملفات
\InlineCode{.c}،
سيعرف جهازك طريقة استخدام الدوال الموجودة في الملف ما إن يبدأ في قراءته.

عند القيام بهذا، لن يكون عليك القلق حيال الترتيب الذي ستكون عليه دوالك في الملفات
\InlineCode{.c}.
إذا كنت قمت الآن بإنشاء برنامج صغير يحتوي على دالتين أو ثلاث يمكنك أن تفكّر أنه من الممكن للبرنامج أن يشتغل دون وجود النماذج، لكن هذا لن يستمر طويلا! فما إن يكبر البرنامج وإن لم تنظّم النماذج في ملفات رأسيّة فستفشل الترجمة دون أدنى شك.

\begin{information}
  عندما تستدعي دالة متواجدة في الملف
  \InlineCode{functions.c}
  انطلاقا من الملف
  \InlineCode{main.c}
  سيكون عليك تضمين النماذج الخاصة بالملف
  \InlineCode{functions.c}
  في الملف
  \InlineCode{main.c}
  يجب إذن وضع
  \InlineCode{\#include "functions.h"}
  في أعلى الملف
  \InlineCode{main.c}.\\
  تذكر هذه القاعدة: "في كلّ مرة تستدعي الدالة
  \textenglish{X}
  في ملف، يجب عليك إدراج نموذج هذه الدالة في ملفك" هذا ما يسمح للـمترجم بمعرفة ما إن كنت قد استدعيتها بشكل صحيح.
\end{information}

\begin{question}
  كيف أقوم بإضافة ملفات
\InlineCode{.c}
 و
\InlineCode{.h}
 إلى مشروعي؟
\end{question}

هذا راجع للـبيئة التطويرية التي تستخدمها. لكن المبدأ هو نفسه في جميع البرامج:
\InlineCode{File} / \InlineCode{New} / \InlineCode{Source File}\\
هذا يسمح بإنشاء ملف جديد فارغ. هذا الملف ليس حاليا من النوع
\InlineCode{.c}
ولا
\InlineCode{.h}
أنت من يحدد ذلك أثناء عملية حفظ الملف. قم إذن بحفظه (حتّى وإن كان لا يزال فارغا!) وهنا يطلب منكم إدخال اسم للملف، يمكنك هنا اختيار صيغة الملف:

\begin{itemize}
  \item إذا سميته
\InlineCode{file.c}
فسيكون بامتداد
\InlineCode{.c}.
  \item إذا سميته
\InlineCode{file.h}
فسيكون بامتداد
\InlineCode{.h}.
\end{itemize}

هذا سهل. قم بحفظ الملف في المجلّد أين تتواجد باقي الملفات الخاصة بمشروعك (نفس المجلّد أين يتواجد الملف
\InlineCode{main.c}). عموما كل ملفات المشروع تقوم بحفظها في نفس المجلّد سواء كانت ذات صيغة
\InlineCode{.c}
أو
\InlineCode{.h}.

مجلّد المشروع في النهاية سيكون مثل هذا:

\begin{figure}[H]
	\centering
	\includegraphics[width=0.8\textwidth]{Chapter_II-1_Project-Sokoban-Folder}
\end{figure}

الملف الذي أنشأته محفوظ لكن لم تتم إضافته إلى مشروعك بعد!\\
لإضافته قم بالنقر يمينا على القائمة أيسر الشاشة (الخاصة بملفات المشروع) واختر
\InlineCode{Add files}
كالتالي:

\begin{figure}[H]
	\centering
	\includegraphics[width=0.4\textwidth]{Chapter_II-1_Project-Add-File}
\end{figure}

ستظهر لك نافذة تطلب منك اختيار الملفات التي تريد أن تدخلها للمشروع، اختر الملف الذي قمت بإنشائه، للتو، وسيتم إدخاله أخيرا في المشروع.  ستجده حاضرًا في القائمة اليسارية!

\subsection{\texttt{include} الخاصّة بالمكتبات النموذجية}

يفترض أنّ لديك سؤالا يدور في رأسك الآن\dots
إذا ضمّنا الملفات
\InlineCode{stdio.h}
و
\InlineCode{stdlib.h}
فهذا يعني أنهما موجودان في مكان ما ويمكننا البحث عنهما، أليس كذلك؟

نعم بالطبع!\\
يفترض أنهما مسطّبان في المكان الذي تتواجد به البيئة التطويرية الخاصة بك، بالنسبة للبيئة
\textenglish{Code::Blocks}
أجدهم هنا:

\InlineCode{C:\textbackslash Program Files\textbackslash CodeBlocks\textbackslash MinGW\textbackslash include}

على العموم يجب البحث عن مجلد يحمل اسم
\InlineCode{include}.\\
بداخله تجد كمّا هائلا من الملفات، وهي ملفات رأسية
(\InlineCode{.h})
خاصة بمكتبات نموذجية أي مكتبات متوفرة في كل مكان (سواء في 
\textenglish{Windows}
أو
\textenglish{\mbox{Mac OS X}}
أو 
\textenglish{\mbox{GNU/Linux}}\dots)،
و ستجد داخلها الملفات
\InlineCode{stdio.h}
و
\InlineCode{stdlib.h}
مع ملفّات أخرى.

يمكنك فتحها إذا أردت، لكن ستُفاجئ بالعديد من الأشياء التي لم أدرّسها لك من قبل خاصة بما يتعلق ببعض توجيهات المعالج القبلي. يمكنك أن تلاحظ بأن الملف مليء بنماذج لدوال نموذجية مثل
\InlineCode{printf}.

\begin{question}
  حسنًا، الآن عرفت أين أجد نماذج الدوال النموذجية لكن ألا يمكنني رؤية الشفرة المصدرية الخاصة بالدوال؟ أين هي الملفات
\InlineCode{.c}؟
\end{question}

إنها غير موجودة أساسا، لأنها مترجمة (إلى ملفات ثنائية
(\textenglish{binary files})،
يعني إلى لغة الحاسوب). ولهذا فإنه من المستحيل أن تقرأها.

يمكنك إيجاد الملفات المترجمة في المجلّد المسمى
\InlineCode{lib}
(والذي هو اختصار لكلمة
\InlineCode{library}
أي مكتبة)، بالنسبة لي هي موجودة في المسار:

\InlineCode{C:\textbackslash Program Files\textbackslash CodeBlocks\textbackslash MinGW\textbackslash lib}

ملفات المكتبات المترجمة لها الصيغة
\InlineCode{.a}
في البيئة
\textenglish{Code::Blocks}
و التي تستخدم
\InlineCode{mingw}
كمترجم. ولها صيغة
\InlineCode{.lib}.
في برنامج
\textenglish{Visual C++}
الذي يستخدم المترجم
\InlineCode{Visual}.
لا تحاولوا قراءتها لأنها غير قابلة للقراءة من طرف إنسان عادي.

باختصار، يجب عليك أن تضمّن الملفات الرأسية
\InlineCode{.h}
في الملفات
\InlineCode{.c}
تتمكن من استخدام الدوال النموذجية مثل
\InlineCode{printf}
و كما تعرف فالجهاز على اطلاع على النماذج فهو يعرف ما إن كنت قد طلبت الدوال بشكل صحيح (إن لم تنس أحد المعاملات مثلا).

\section{الترجمة المنفصلة}

الآن وبعدما عرفت أن المشروع مبنى على أساس ملفات مصدرية عديدة، يمكننا الدخول الآن في تفاصيل عملية الترجمة فلحد الآن لم نر سوى مخطط مبسط عنها.

سأعطيك الآن مخططا مفصلا عنها ومن المستحسن أن تحفظه عن ظهر قلب:

\begin{figure}[H]
	\centering
	\includegraphics[width=0.7\textwidth]{Chapter_II-1_Compilation-Schema}
\end{figure}

هذا مخطط حقيقي عمّا يجري بالضبط أثناء التجميع وسأشرحه لك:

\begin{enumerate}
  \item \textbf{المعالج القبلي}:
المعالج القبلي هو برنامج ينطلق قبل عملية الترجمة وهو مخصص للقيام بتشغيل تعليمات نطلبها منه عن طريق ما سميناه بتوجيهات المعالج القبلي، وهي الأسطر الشهيرة التي تبدأ بإشارة
\InlineCode{\#}.

لحد الآن توجيهة المعالج القبلي الوحيدة الّتي نعرفها هي
\InlineCode{\#include}
و الّتي تسمح بإدراج ملف في ملف آخر. طبعا للمعالج القبلي مهام أخرى سنتعرف إليها لاحقا لكن ما يهمنا الآن هو ما أعطيتك إياه.
المعالج القبلي يقوم إذن بـ"استبدال" أسطر
\InlineCode{\#include}
بملفات أخرى نحددها، فهو يضمّن داخل كل الملفات
\InlineCode{.c}
الملفات
\InlineCode{.h}
التي نعينها ونطلب منه تضمينها في السابقة.
  \item \textbf{الترجمة}: هذه الخطوة المهمة التي تسمح بتحويل ملفاتك إلى شفرات ثنائية مفهومة للحاسوب. فالمترجم يقوم بتجميع الملفات
\InlineCode{.c}
واحدا بواحد حتى ينهيها جميعها، ولضمان ذلك يجب أن تكون كل الملفات موجودة في المشروع (بحيث تظهر في القائمة اليسارية).

سيقوم المترجم بتوليد ملف
\InlineCode{.o}
أو
\InlineCode{.obj}
و هذا راجع لنوع المترجم وهي ملفات ثنائية مؤقتة، وعلى أي حال تحذف هذه الملفات في نهاية الترجمة ولكن بتعديل الخيارات يمكنك الإبقاء عليها لكن لن يكون هناك من داع.
  \item \textbf{إنشاء الروابط}:
محرر الروابط
(\textenglish{Linker})
هو برنامج يعمل على جمع الملفات الثنائية من نوع
\InlineCode{.o}
في ملف واحد كبير: الملف التنفيذي النهائي! هذا الملف يحمل الصيغة
\InlineCode{.exe}
في \textenglish{Windows}. إن كنت تملك نظام تشغيل آخر فسيأخذ الصيغة المناسبة له.
\end{enumerate}

و هكذا تكون قد تعرفت على الطريقة الحقيقية لعمل الترجمة. كما قلتها وأكررها المخطط أعلاه مهم للغاية، فهو يفرق بين مبرمج يقوم بجمع ونسخ الشفرة المصدرية دون فهم وبين مبرمج يعرف تماما ما عليه فعله!

معظم الأخطاء تحدث في الترجمة وقد تأتي من محرر الروابط وهذا يعني أنه لم يتمكن من تجميع كل الملفات
\InlineCode{.o}
بطريقة صحيحة (ربمّا لفقدان إحداها).

لا يزال المخطط أعلاه غير كامل، إذ أن المكتبات لم تظهر فيه! إذن كيف تحدث العملية عندما نستخدم مكتبات برمجية؟

تبقى بداية المخطط هي نفسها، لكن يقوم محرر الروابط بأعمال أخرى، سيقوم بتجميع ملفاتك
\InlineCode{.o}
(المؤقتة) مع مكتبات جاهزة تحتاجها 
(\InlineCode{.a}
أو
\InlineCode{.lib}
وفقا للمترجم):

\begin{figure}[H]
	\centering
	\includegraphics[width=0.7\textwidth]{Chapter_II-1_Compilation-Schema-Libraries}
\end{figure}

هكذا ننتهي ويكون مخططنا هذه المرة كاملا، ملفاتك من المكتبات
\InlineCode{.a}
(أو
\InlineCode{.lib})
يتم تجميعها في الملف التنفيذي مع الملفات
\InlineCode{.o}.

فبهذه الطريقة نتحصل في النهاية على برنامج كامل
100\%
و الذي يحتوي كل التعليمات اللازمة للجهاز لتشرح له كيف يعرض نصّا!\\
كمثال، الدالة
\InlineCode{printf}
توجد في ملف
\InlineCode{.a}
و طبعا سيتم تجميعها مع الشفرة المصدرية الخاصّة بنا في الملف التنفيذي.

لاحقا سنتعلم كيف نستخدم المكتبات الرسومية التي نجدها أيضا في ملفات
\InlineCode{.a}
و تعطي للجهاز تعليمات خاصة بكيفية إظهار نافذة على الشاشة كمثال. لكن طبعا، لن ندرسها الآن ، كلّ شيء في وقته.

\section{نطاق الدوال والمتغيرات}

لننهي هذا الفصل، يجب أن أطلعكم عما يسمى بـ\textbf{نطاق}
المتغيرات والدوال، سنعرف إمكانية الوصول للدوال والمتغيرات، يعني متى يمكننا استدعاؤها.

\subsection{المتغيرات الخاصة بدالة}

عندما تصرّح عن متغير في داخل دالة يتم حذف هذا المتغير من الذاكرة مع نهاية الدالة.

\begin{Csource}
int triple(int number)
{
	int result = 0; // The variable result is created in the memory
	result = 3 * number;
	return result;
} // The function finished, the variable result is destroyed
\end{Csource}

كلّ متغير تم التصريح عنه في دالة، لا يكون موجودا سوى حينما تكون الدالة في طور الاشتغال.\\
لكن ماذا يعني هذا تحديدا؟ أنه لا يمكن الوصول إليه من خلال  دالة أخرى!

\begin{Csource}
int triple(int number);
int main(int argc, char *argv[])
{
	printf("The triple of 15 = %d\n", triple(15));
	printf("The triple of 15 = %d",  result); // Error
	return 0;
}

int triple(int number)
{
	int result = 0;
	result = 3 * number;
	return result;
}
\end{Csource}

في الدالة الرئيسية أحاول الوصول إلى المتغير
\InlineCode{result}
و بما أن هذا المتغير تم التصريح عنه داخل الدالة
\InlineCode{triple}
فطبعا لا يمكنني الوصول إليه من خلال الدالة
\InlineCode{main}!

\textbf{تذكّر جيّدا}
: كل متغير تم التصريح عنه داخل دالة، لا يسرى مفعوله إلا في داخل هذه الدالة نفسها! ونقول أن المتغير محلّي
(\textenglish{Local}).

\subsection{المتغيرات الشاملة (\textenglish{Global variables}): فلتتجنّبها}

\subsubsection{متغير شامل قابل للوصول إليه من خلال كلّ الملفات}

إنه من الممكن التصريح عن متغير يمكن الوصول إليه من خلال كل الدوال من ملفات المشروع.
سأريك كيفية فعل ذلك كي تعرف بأنه أمر موجود، لكن عموما تجنب القيام بذلك.
قد يظهر أنها ستسهل لك التعامل مع الشفرة المصدرية لكن قد يؤدي بك هذا لوجود العديد من المتغيرات التي يمكننا الوصول إليها من كلّ مكان مما سيصعّب عليك عملية إدارتها.

للتصريح عن متغير
\textbf{شامل}
(\textenglish{Global})،
يجب أن تقوم بذلك خارج كلّ الدوال، يعني في أعلى الملف، وعموما بعد أسطر \InlineCode{\#include}.

\begin{Csource}
#include <stdio.h>
#include <stdlib.h>

int result = 0; // Declaration of a global variable
void triple(int number ); // Prototype of the function
int main(int argc, char *argv[])
{
	triple(15); // We call the function triple which is going to modify the variable result
	printf("The triple of 15 = %d\n", result); // We can access to the variable result
	return 0;
}

void triple(int number)
{
	result = 3 * number;
}
\end{Csource}

في هذا المثال، الدالة
\InlineCode{triple}
لا تُرجع أي شيء
(\InlineCode{void}).
إنها تقوم بتعديل قيمة المتغير الشامل
\InlineCode{result}
التي يمكن للدالة
\InlineCode{main}
أن تسترجعه.

المتغير
\InlineCode{result}
يمكن الوصول إليه من خلال كل الملفات في المشروع ومنه يمكننا استدعاؤها من خلال
\underline{كلّ}
دوال البرنامج.

\begin{warning}
  هذا شيء يجب ألا يتواجد في برامج \textenglish{C}
الخاصة بك. من المستحسن استعمال التعليمة
\InlineCode{return}
لإرجاع النتيجة بدل التعديل عليه كمتغير شامل.
\end{warning}

\subsubsection{متغير شامل قابل للوصول إليه من خلال ملف واحد}

المتغير الشامل الذي أريتك إياه قبل قليل يمكن الوصول إليه من خلال كل الملفات الخاصة بالمشروع.\\
يمكننا جعل متغيّر شامل مرئيا فقط في الملف الذي تتواجد به. ولكنه يبقى متغيرا شاملا على أية حال حتى وإن كنا نقول أنه ليس كذلك إلا على الدوال المتواجدة في ذات الملف وليس على كل دوال البرنامج.

لإنشاء متغير شامل مرئي في ملف واحد نستعمل الكلمة المفتاحية
\InlineCode{static}
قبله:

\begin{Csource}
static int result = 0;
\end{Csource}

\subsubsection{متغير ساكن (\textenglish{static}) بالنسبة لدالة}

حذار: الأمر هنا حساس قليلًا. إن استعملت الكلمة المفتاحية
\InlineCode{static}
عند التصريح عن متغير في داخل دالة، فلهذا معنى آخر غير الخاص بالمتغيرات الشاملة.\\
في هذه الحالة، لا يتم حذف المتغير الساكن مع نهاية الدالة، بل حينما نستدعي الدالة مرّة أخرى، سيحفظ المتغير قيمته مثلا:
\begin{Csource}
int triple(int number)
{
	static int result = 0; // The first time when the variable is created
	result = 3 * number;
	return result;
} // When we exit the function, the variable is not destroyed
\end{Csource}
ماذا يعني هذا بالضبط؟\\
يعني أنه يمكننا استدعاء الدالة لاحقا ويبقى المتغير
\InlineCode{result}
محتفظا بنفس القيمة الأخيرة.

و هذا مثال آخر للفهم أكثر:
\begin{Csource}
int increment();

int main(int argc, char *argv[])
{
	printf("%d\n", increment());
	printf("%d\n", increment());
	printf("%d\n", increment());
	printf("%d\n", increment());

	return 0;
}

int increment()
{
	static int number= 0;

	number++;
	return number;
}
\end{Csource}

\begin{Console}
1
2
3
4
\end{Console}

هنا، في المرة الأولى التي نطلب فيها الدالة
\InlineCode{increment}،
يتم إنشاء المتغير
\InlineCode{number}.
ثم نقوم بزيادة 1 إلى قيمته. وما إن تنتهي الدالة لا يمسح المتغير.

عندما نطلب الدالة للمرة الثانية، يتم ببساطة قفز السطر الخاص بالتصريح بالمتغير، ولا نقوم بإعادة إنشاء المتغير بل فقط نعيد استعمال المتغير الذي أنشأناه سابقا. عندما يأخذ المتغير القيمة 1، تصبح قيمته 2 ثم 3 ثم 4\dots إلخ.

هذا النوع من المتغيرات ليس مستعملا بكثرة، لكن يمكنه مساعدتك في بعض الأحيان ولهذا ذكرته في هذا الكتاب.

\subsubsection{الدوال المحلية لملف}

لإنهاء حديثنا عن المتغيرات والدوال،  نتكلم قليلا حول نطاق الدوال. من المفروض، عندما تنشئ دالة، والتي هي شاملة بالنسبة لكل البرنامج، يمكن الوصول إليها من أي ملف
\InlineCode{.c}.

قد تحتاج أحيانًا إلى إنشاء دوال يمكن الوصول إليها فقط في الملفات التي تتواجد بها، وللقيام بذلك، أضف الكلمة المفتاحيّة
\InlineCode{static}
قبل الدالة كالتالي:

\begin{Csource}
static int triple(int number)
{
	// Instructions
}
\end{Csource}

و لا تنس النموذج أيضا:

\begin{Csource}
static int triple(int number);
\end{Csource}

انتهى! دالتك الساكنة
\InlineCode{triple}
لا يمكن استدعاؤها إلا من داخل دوال تنتمى لنفس الملف (مثلا
\InlineCode{main.c}).
إذا حاولت استدعاء الدالة
\InlineCode{triple}
من خلال دالة أخرى من ملف آخر (مثلا
\InlineCode{display.c})،
لن يشتغل البرنامج لأن
\InlineCode{triple}
وقتها لن تكون مرئية.

لنلخص كلّ شيء يتعلّق بنطاق المتغيرات:

\begin{itemize}
  \item المتغير الذي يتم التصريح عنه داخل دالة يتم حذفه في نهاية الدالة مباشرة. لا يمكن أن يكون مرئيا إلا داخل هذه الدالة.
  \item المتغير الذي  يتم التصريح عنه في داخل دالة بالكلمة المفتاحية
\InlineCode{static}،
لا يحذف في نهاية الدالة لكنه يحتفظ بقيمته ما دام البرنامج يعمل.
  \item المتغير الذي يتم التصريح عنه خارج الدوال يسمى متغيرا شاملا، يكون مرئيا في كلّ الدوال وفي كلّ ملفات المشروع.
  \item  المتغير الشامل المصرّح عنه بالكلمة المفتاحية
\InlineCode{static}
مرئي فقط في الملف الّذي يتواجد به، ولا يكون مرئيا في دوال باقي الملفات.
\end{itemize}

و بالمثل، هذه هي النطاقات الممكنة للدوال:

\begin{itemize}
  \item الدالة الّتي يتم التصريح عنها عاديًا تكون مرئية في كل ملفات المشروع، يمكننا طلبها إذا من أي ملف كان.
  \item إذا أردنا أن تكون الدالة مرئية فقط في الملف الذي تتواجد به فيجب إضافة الكلمة المفتاحيّة
\InlineCode{static}
قبلها.
\end{itemize}

\section*{ملخّص}

\begin{itemize}
  \item البرنامج يحتوي العديد من الملفات
\InlineCode{.c}.
كقاعدة عامة، لكل ملف
\InlineCode{.c}
ملف مرافق له يحمل الامتداد
\InlineCode{.h}
(الذي يعني ملفّا رأسيّا
(\textbf{\textenglish{Header}})).
\InlineCode{.c}
يحوي الدوال بينما
\InlineCode{.h}
يحوي
\textbf{النماذج (\textenglish{Prototypes})}،
أي توقيعات هذه الدوال.
  \item يتم تضمين محتوى الملفات
\InlineCode{.h}
في أعلى الملفات
\InlineCode{.c}
بالاستعانة ببرنامج يسمّى
\textbf{المعالج القبلي}.
  \item يتم تحويل الملفات
\InlineCode{.c}
إلى ملفات ثنائية
\InlineCode{.o}
من طرف
\textbf{المترجم}.
  \item يتم تجميع الملفات
\InlineCode{.o}
في ملف تنفيذي
(\InlineCode{.exe})
من طرف
\textbf{محرر الروابط
(\textenglish{Linker})}.
  \item المتغير الّذي يتم التصريح عنه داخل دالة غير قابل للوصول إلا من داخل هذه الدالة. نتكلّم هنا عن
\textbf{نطاق المتغيرات}.
\end{itemize}
