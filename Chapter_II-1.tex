\chapter{البرمجة المجزّأة
(\textenglish{Modular programming})}
في هذه المرحلة الثانية، سنكتشف مبادئ متقدّمة في لغة الـ
\textenglish{C}
 لن أخفي عليك، هذه المرحلة صعبة الفهم و تحتاج منك التركيز. في نهاية المرحلة، ستكون قادراً على تدبّر أمرك في معظم البرامج المكتوبة بلغة السي. في المرحلة التي تليها نتعلّم كيف نفتح نافذة، كيف ننشئ لعبة ثنائية الأبعاد... الخ

لحدّ الآن عملنا في ملف واحد سمّيناه
\InlineCode{main.c}
. كان أمراً مقبولاً لحدّ الآن لأن برامجنا كانت صغيرة، لكنها ستصبح في القريب العاجل مركّبة من عشرات، لن أقول من مئات الدوال، و إن كنت تريد وضعها كلّها في نفس الملف، فإن هذا الأخير سيصبح ضخماً جداً. لهذا السبب تم اختراع ما نسمّيه بالبرمجة المجزّأة. المبدأ سهل: بدل أن نضع كل الشفرة المصدرية في ملف واحد
\InlineCode{main.c}
، سنقوم بتفريقها إلى عدة ملفات.

\section{النماذج (\textenglish{prototypes})}
لحدّ الآن، كنت عندما تنشئ دالة، أطلب منك وضعها قبل الدالة الرئيسية main. لماذا؟

لأن للترتيب أهمية حقيقية هنا: فإن قمت بوضع الدالة قبل الـ
\InlineCode{main}
في الشفرة المصدرية، سيقرؤها الجهاز و يتعرف عليها. حينما تقوم باستدعاء الدالة داخل الـ
\InlineCode{main}
، سيعرفها الجهاز و يعرف أيضاً أين يبحث عليها.\\
بالعكس، لو تضع الدالة بعد الـ
\InlineCode{main}
، لن يعمل البرنامج لأن الجهاز لم يتعرّف بعد على الدالة. جرّب ذلك و سترى.
\begin{question}
  لكنه تصميم سيّء نوعاً ما، أليس كذلك ؟
\end{question}
أنا متفق معك! لكن انتبه المبرمجون لهذه النقطة قبلك و عملوا على حلّ المشكل.

بفضل ما سأعلمك إياه الآن، ستتمكن من الدوال في أي ترتيب كان في الشفرة المصدرية، هكذا لن تقلق من هذه الناحية.
