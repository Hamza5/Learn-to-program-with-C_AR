\chapter{الشروط (\textenglish{Conditions})}

لقد رأينا فيما سبق بأن هناك العديد من لغات البرمجة. بعضها متشابه: الكثير منها مستلهم من \textenglish{C}.

في الواقع، لغة  \textenglish{C}
أُنشئت منذ زمن طويل، وهذا جعلها نموذجا للّغات الجديدة.

لغات البرمجة تختلف في بعض الأمور، لكن هناك مبادئ لا يمكن أن تخلو منها أية لغة برمجية. لقد رأينا كيف ننشئ المتغيّرات، كيف نقوم بالحسابات، والآن سنمرّ إلى 
\textbf{الشروط}.\\
من دون استعمال الشروط، برامجنا ستقوم دائما بنفس العمل!

\section{الشرط \texttt{if\dots else}}

الشروط تسمح لنا بأن نقوم باختبارات على المتغيرات. مثلا يمكننا القول، "إن كان المتغيّر 
\InlineCode{machine}
يساوي 50، يجب أن نقوم بكذا وكذا. ولكن من المؤسف عدم إمكانية اختبار سوى المساواة! يجب أيضا اختبار ما إن كان المتغيّر، أقل من 50، أقل أو يساوي 50، أكبر، أكبر أو يساوي\dots
لا تقلق، في \textenglish{C}
كل شيء مُعَد!

لدراسة الشرط 
\InlineCode{if\dots else}
يجب أن نتبع المخطط التالي:

\begin{itemize}
\item نتعلم بعض الرموز قبل البدأ،
\item الشرط 
\InlineCode{if}،
\item الشرط 
\InlineCode{else}،
\item الشرط
\InlineCode{else if}،
\item كثير من الشروط في مرة واحدة،
\item بعض الأخطاء لنتجنبها.
\end{itemize}

قبل أن نرى كيف نكتب شرطا من النوع
\InlineCode{if\dots else}
في \textenglish{C}، يجب أن تعرف ثلاثة رموز أساسية. هذه الرموز ضرورية لإنشاء الشروط.

\subsection{بعض الرموز للتعلّم}

الجدول التالي يحوي رموز لغة \textenglish{C} الّتي
\textbf{يجب حفظها عن ظهر قلب}:

\begin{Table}{2}
الرمز & المعنى\\
\texttt{{=}{=}} & يساوي\\
\texttt{>} & أكبر\\
\texttt{<} & أصغر\\
\texttt{<=} & أصغر أو يساوي\\
\texttt{>=} & أكبر أو يساوي\\
\texttt{!=} & لا يساوي\\
\end{Table}

\begin{critical}
انتبه جيدا، هناك رمزا مساواة
\InlineCode{{=}{=}}
لنقوم باختبار المساواة. فالمبتدؤون يقومون غالبا باقتراف خطأ وضع إشارة واحدة
\InlineCode{=}،
و هذا لديه معنى مختلف في \textenglish{C}. سأذكّرك بهذا لاحقا.
\end{critical}

\subsection{\texttt{if} بسيط}

فلنبدأ، لنقم باختبار بسيط، يقول للحاسوب: إن كان المتغير يساوي كذا فلنقم بكذا.

في الإنجليزيّة، كلمة "إذا" تُتَرجم إلى
\InlineCode{if}.
و هذا ما الّذي نستخدمه في لغة \textenglish{C} لإنشاء اختبار.\\
أكتب إذن
\InlineCode{if}
ثم الأقواس وفي داخلها الشرط.

بعد ذلك افتح حاضنة
\InlineCode{\{}
و اغلقها لاحقا
\InlineCode{\}}.
كلّ ما يوجد بين الحاضنتين سيتمّ تشغيله فقط في حالة تحقق الشرط.

هذا يعطينا إذن:
\begin{Csource}
if (/* Your condition */)
{
	// Instructions to be executed
}
\end{Csource}

في مكان التعليق
"\textenglish{Your condition}"
سنكتب شرطا للتحقّق من متغيّر.\\
كمثال سنحاول أن نقوم باختبار متغيّر
\InlineCode{age}
لاحتواء العمر. سنختبر ما إن كان المستعمل راشدا أم قاصرا استنادا إلى 
\textbf{إذا ما كان عمره يساوي أو أكبر من 18}:

\begin{Csource}
if (age >= 18)
{
	printf("You are major !");
}
\end{Csource}

\InlineCode{>=}
يعني "أكبر أو يساوي"، كما رأينا في الجدول السابق.

\begin{information}
عندما لا يكون هناك سوى تعليمة واحدة بين الحاضنتين، يمكننا أن نتخلى عنهما. لكني أفضل أن تضعهما دائما من أجل قراءة أفضل للشفرة.
\end{information}

\subsubsection{جرّب هذه الشفرة}
لكي نجرّب الشفرات السابقة ونفهم كيف يعمل \InlineCode{if}، يجب أن نضعه داخل الدالة
\InlineCode{main}
و لا ننس التصريح بالمتغير 
\InlineCode{age}
و إعطاءه قيمة ابتدائية.

هذا قد يبدو بديهيّا للبعض، ولكنّ كثيرا من القرّاء قد ضاعوا في هذه الأسطر وهذا ما دفعني لإضافة هذا الشرح. هذه شفرة كاملة يمكنك تجريبها:

\begin{Csource}
#include <stdio.h>
#include <stdlib.h>
int main(int argc, char *argv[])
{
	int age = 20;
	
	if (age >= 18)
	{
		printf("You are major !\n");
	}
	return 0;
}
\end{Csource}

هنا،
\InlineCode{age}
يساوي 20 إذن سيتم عرض
"\textenglish{You are major !}".\\
جرّب تغيير القيمة إلى 15 مثلا، سيصبح الشرط خاطئًا وبالتالي لن يُعرض شيء هذه المرّة.

استخدم هذه الشفرة لتجريب الأمثلة اللاحقة في هذا الفصل.

\subsubsection{مسألة نظافة}

الطريقة الّتي تفتح بها الحاضنات ليست مهمّة. سيعمل برنامجك إن كتبت كلّ شيء على نفس السطر. مثال:

\begin{Csource}
if (age >= 18) { printf("You are major !"); }
\end{Csource}

و على الرغم من أنّه ممكن، فهذا أمر
\textbf{غير منصوح به مطلقا}.\\
في الواقع، الكتابة على نفس السطر تجعل شفرتك صعبة القراءة. إذا لم تتعوّد من الآن على تهوية شفرتك، فلاحقا عندما تكتب شفرات كبيرة ستضيع بالتأكيد!

حاول عرض شفرتك بنفس طريقتي: حاضنة على سطر، ثمّ التعليمات (مسبوقة بجدولة 
(\textenglish{tabulation}))،
 ثمّ حاضنة الإغلاق على سطر آخر.

\begin{information}
توجد طرق عديدة لعرض الشفرة بشكل جيّد. هذا لا يغيّر في عمل البرنامج شيئا، لكنّها مسألة "نمط معلوماتيّ" إن أردت. إذا رأيت الشفرة شخص آخر معروضة بشكل مختلف قليلا، فهذا لأنّه يبرمج بنمط مختلف. الهدف من هذا كلّه هو أن تبقى الشفرة مهوّاة ومقروءة.
\end{information}

\subsection{\texttt{else}
لكي نقول: وإلا}

الآن وبما أنّك تعرف كيف تقوم باختبار بسيط، سنذهب بعيدا: إن لم يعمل الشرط (كان خاطئا)، فسنطلب من الحاسوب تشغيل تعليمات أخرى.

هذا يشبه ما يلي: إن كان المتغير يساوي كذا فلنقم بكذا، وإلا فلنقم بكذا.

يكفي أن نضيف
\InlineCode{else}
بعد الحاضنة الأخيرة لأوامر \InlineCode{if}،
مثلا:
\begin{Csource}
if (age >= 18)
{
	printf("You are major !");
}
else {
	printf("You are minor !");
}
\end{Csource}

الأمر سهل: إن كان المتغيّر
\InlineCode{age}
أكبر من أو يساوي 18 سنكتب على الشاشة 
"\textenglish{You are major !}"،
و إن لم يكن الأمر كذلك فسنكتب 
"\textenglish{You are minor !}".

\subsection{\texttt{else if}
لكي نقول: وإلا فإذا}

سنرى كيف نقوم بـ"إذا" و"إلّا". يمكن أيضًا القيام بـ وإلاّ فإذا" للقيام باختبار آخر في حالة ما إذا لم ينجح الأوّل. هذا الاختبار يوضع بين
\InlineCode{if}
و
\InlineCode{else}.

يمكن ترجمة ذلك بما يلي: إن كان المتغير يساوي كذا، فافعل كذا وإلا فإن كان يساوي كذا، فافعل كذا وإلا (جميع الحالات المتبقية)، افعل كذا.

الترجمة بلغة
\textenglish{C}:

\begin{Csource}
if (age >= 18)
{
	printf("You are major !");
}
else if (age > 4)
{
	printf("Well you're not too young anyway...");
}
else
{
	printf("Aga gaa aga gaaa");
	// Baby language, you can't understand !
}
\end{Csource}

الجهاز يقوم بالاختبارات التالية بالترتيب:
\begin{itemize}
\item سيختبر \InlineCode{if}
الأول، إن كان الشرط محققا، فسيقوم بتنفيذ التعليمات الموجودة بين الحاضنتين الأولتين.
\item إن لم يكن الشرط الأول محققا (يعني أننا في \InlineCode{else if})
سيختبر الشرط الثاني، إن كان هذا الأخير محققا فإنه سينفذ التعليمات الموجودة بين الحاضنتين الثانيتين.
\item في حالة ما لم يكن الشرط الأول محققا ولا حتى الثاني، فإنه سينفذ التعليمات الموجودة بين الحاضنتين الأخيرتين.
\end{itemize}

\begin{information}
\InlineCode{else}
و \InlineCode{else if}
ليسلا ضروريّين. لإنشاء شرط، فقط
\InlineCode{if}
هو الضروري (هذا منطقي، وإلّا فكيف سيكون هناك شرط!).
\end{information}

لاحظ أنّه يمكننا أن نستعمل الكمّ الّذي نريده من \InlineCode{else if}.

\subsection{عدّة شروط في مرّة واحدة}

قد يكون من المهم القيام بعدّة اختبارات في شرط واحد. مثلا، قد تريد اختبار ما إن كان العمر أكبر من 18 والعمر أصغر من 25.\\
لهذا علينا بتعلم رموز جديدة:

\begin{Table}{2}
الرمز & المعنى\\
\texttt{\&\&} & و \\
\texttt{||} & أو \\
\texttt{!} & لا (عكس الشرط)\\
\end{Table}

\subsubsection{الاختبار بالواو (\textenglish{and})}

لنختبر إن كان العمر في نفس الوقت أكبر من 18 وأقل من 25 يجب كتابة:

\begin{Csource}
if (age > 18 && age < 25)
\end{Csource}

\InlineCode{\&\&}
يعنيان "و". بالعربية هذا يعني: "إذا كان العمر أكبر من 18 وإذا كان العمر أصغر من 25".

\subsubsection{الاختبار بالأو (\textenglish{or})}

لاستخدام "أو"، يجب كتابة الرمزين
\InlineCode{||}.
لكتابة الرمز 
\InlineCode{|}،
نضغط في لوحة المفاتيح على
\InlineCode{Alt Gr} + \InlineCode{6}
(باعتبار أن تخطيط لوحة المفاتيح
\textenglish{AZERTY}
فرنسي). 

فلنعتبر برنامجا بسيطا، يختبر ما إن كان الشخص قادرًا على فتح حساب بنكي أو لا. فلكي يكون قادرًا على ذلك يجب أن لا يكون شابّا كثيرا (فلنقل مثلا، ليس تحت 30 سنة) أو لديه الكثير من المال:

\begin{Csource}
if (age > 30 || argent > 100000)
{
	printf("Welcome to PicsouBank !");
}
else
{
	printf("Get out of my sight, miserable !");
}
\end{Csource}

الاختبار سيكون ناجحا فقط إذا كان الشخص بعمر أكبر من 30 سنة، أو على الأقل يملك مبلغًا أكبر من 100000 دينار مثلًا.

\subsubsection{الإختبار بـلا (\textenglish{not})}

الرمز الموافق لهذا الاختبار هو علامة التعجّب (!). في علوم الحاسوب، علامة التعجّب تعني "لا". يجب  وضع هذه العلامة من أجل عكس الشرط لقول  "إن لم يكن هذا صحيحا":
\begin{Csource}
if (!(age < 18))
\end{Csource}

يمكن أن نترجم هذا إلى "إن كان الشخص غير قاصر". لو حذفنا
\InlineCode{!}
فسيعكس المعنى: "إن كان الشخص قاصرا".
\subsection{بعض الأخطاء التي يرتكبها المبتدؤون}

\subsubsection{لا تنس الإشارتين \texttt{==}}

مثلما قلت سابقا، لكي نختبر ما إن كان العمر يساوي 18 نكتب:

\begin{Csource}
if (age == 18)
{
	printf("You have just become major !");
}
\end{Csource}

\textbf{لا تنس}
وضع علامتي "يساوي" داخل
\InlineCode{if}،
هكذا:
\InlineCode{==}.

إن وضعت علامة
\InlineCode{=}
واحدة فإن المتغير
\InlineCode{age}
سيأخذ القيمة 18 (مثلما تعلمنا ذلك سابقا في فصل المتغيرات). نحن نريد أن نختبر قيمة المتغير وليس تغييرها فاحذر! الكثير يقع في هذا الخطأ وبالتأكيد فبرامجهم لن تعمل بالشكل المطلوب!

\subsubsection{الفاصلة المنقوطة الزائدة}

بعض المبتدئين يقومون بإضافة فاصلة منقوطة في نهاية سطر \InlineCode{if}، ولكن \InlineCode{if} هو شرط ولا نضع فاصلة منقوطة إلّا في نهاية تعليمة. الشفرة التالية لن تعمل كما هو متوقّع لأنّه يوجد
\InlineCode{;}
في نهاية الشرط.

\begin{Csource}
if (age == 18); // Note the semicolon that mustn't be here
{
	printf("You are just major !");
}
\end{Csource}

\section{المتغيرات المنطقية (\textenglish{Booleans})، أساس الشروط}
سندخل الآن في تفاصيل عمل شرط من نوع
\InlineCode{if... else}.\\
في الشروط نعتمد كثيرًا على نوع آخر من أنواع البيانات نسميه \textbf{\textenglish{boolean}}.

\subsection{بعض الأمثلة البسيطة للفهم}

سنبدأ ببعض التجارب قبل أن نقدّم هذا المفهوم. إليك شفرة مصدريّة بسيطة جدّا أقترح عليك تجريبها:

\begin{Csource}
if (1);
{
	printf("It is true !");
}
else
{
	printf("It is false !");
}
\end{Csource}

النتيجة: 

\begin{Console}
It is true !
\end{Console}

\begin{question}
لكن، لم نضع شرطا داخل
\InlineCode{if}،
هذا عدد فقط. مالّذي يعنيه؟ لا معنى لهذا.
\end{question}

بلى، ستفهم. استبدل الواحد بالصفر:

\begin{Csource}
if (0);
{
	printf("It is true !");
}
else
{
	printf("It is false !");
}
\end{Csource}

النتيجة:
\begin{Console}
It is false !
\end{Console}

حاول الآن استبدال الصفر بأي عدد صحيح، مثل $ 4 $، $ 15 $، $ 226 $، $ -10 $، $ -36 $، إلخ. النتيجة دائما:
"\textenglish{It is true !}".

\paragraph{ملخّص اختباراتنا:}


إذا وضعنا $ 0 $، الاختبار سيعتبر خاطئا، أمّا إن وضعنا $ 1 $ أو أيّ عدد آخر، فالاختبار سيكون صحيحا.

\subsection{الشرح واجب}

في الواقع أنه في كلّ مرة تقوم فيها باختبار شرط، فإن الشرط سيقوم بإرجاع القيمة $ 1 $ إن كان صحيحًا والقيمة $ 0 $ إن كان خاطئًا.

مثلا بالنسبة لهذا الشرط الّذي هو
\InlineCode{age >= 18}:

\begin{Csource}
if (age >= 18)
\end{Csource}

لنفرض أن
\InlineCode{age}
كان 23. إذن، الشرط صحيح، والحاسوب "سيستبدل" بطريقة ما
\InlineCode{age >= 18}
بـ1.\\
بعد ذلك، سيحصل الحاسوب (في رأسه) على
\InlineCode{if (1)}.
عندما يكون 1، كما رأينا، فسيعتبره صحيحا.

بالمثل، إذا كان الشرط خاطئا، فسيستبدل
\InlineCode{age >= 18}
بـ0، فالشرط خاطئ، والحاسوب سيقرأ تعليمات
\InlineCode{else}.

\subsection{فلنجرب مع متغير}

فلنجرّب الآن شيئا آخر: وضع نتيجة الشرط في متغير، إن هذا الأمر ممكن مادام الشرط معتبرًا من طرف الحاسوب كتعليمة.

\begin{Csource}
int age = 20;
int major = 0;

major = age >= 18;
printf("Major equals : %d\n", major);
\end{Csource}

كما تلاحظ فإن الشرط
\InlineCode{age >= 18}
أعطى 1 وبالتالي فإن المتغير 
\InlineCode{major}
أخذ القيمة 1 يعني صحيح. يمكنك التأكد بـ\InlineCode{printf}.

قم بنفس الاختبار مع وضع
\InlineCode{age == 10}.
هذه المرّة،
\InlineCode{major}
سيكون 0.

\subsection{المتغير
\texttt{major}
متغير منطقي}

تذكّر هذا جيّدا: نقول عن متغيّر نجعله يأخذ القيم $ 1 $ و$ 0 $ أنّه
\textbf{متغيّر منطقيّ}
(\textenglish{boolean}).

و هذا كما يلي:

\begin{itemize}
	\item $ 0 $ = خطأ،
	\item $ 1 $ = صحيح.
\end{itemize}

في الواقع 0 يمثّل خطأ، وكلّ الأعداد الأخرى تعني صحيح (كما رأينا سابقا). ولكن من أجل تبسيط الأمور، لن نستخدم سوى 0 و1 لقول إن كان الشرط صحيحا أو خاطئا.

في لغة
\textenglish{C}
 لا يوجد نوع
\InlineCode{boolean}.\\
على الرغم من ذلك، يمكننا تغطية ذلك باستعمال النوع
\InlineCode{int}.

\subsection{المتغيرات المنطقية في الشروط}

عادة، نقوم باختبار
\InlineCode{if}
على متغيّر منطقيّ:

\begin{Csource}
int major = 1;
if (major)
{
	printf("You are major !");
}
else {
	printf("You are minor !");
}
\end{Csource}

بما أن المتغير
\InlineCode{major}
يساوي 1 فإن الشرط محقق وبالتالي سيظهر على الشاشة:
"\textenglish{You are major !}".

و هذا عمليّ جدّا، فالشرط أصبح مفهوما بشكل أفضل، فنقرأ
\InlineCode{if (major)}،
و الّذي يعني "إن كنت بالغًا". الشروط على المتغيّرات المنطقيّة هي إذن سهلة للقراءة والفهم، ما دمت قد أعطيت أسماء واضحة لمتغيّراتك كما طلبت منك من البداية.

يمكننا أيضا أن نجد شفرة كالتالي:

\begin{Csource}
if (major && boy)
\end{Csource}

يعني "إن كان الشخص راشدا وذكرًا". في هذه الحالة، المتغير 
\InlineCode{boy}
أيضا هو متغير منطقي، يساوي 1 إذا كان الشخص ولدًا و0 إذا كان بنتًا. أعتقد أنك فهمت المقصود.

باختصار، المتغيرات المنطقية تسمح لنا بمعرفة ما إن كان الاختبار صحيحًا أم خاطئًا.\\
هذا مهمّ جدّا وما شرحته لك سيمكّنك من فهم كثير من الأمور الّتي ستأتي لاحقا.

\begin{question}
سؤال صغير: إن كتبنا:
\InlineCode{if (major == 1)}
فسيعمل، أليس كذلك؟
\end{question}

هذا صحيح، لكن مبدأ المتغيّرات هي أن نستطيع اختصار عبارة الشرط وجعلها أكثر قابليّة للقراءة. اعترف أنّ
\InlineCode{if (major)}
تُفهم أحسن، أليس كذلك؟

\paragraph{تذكّر إذن:}
إذا كان متغيّرك يحمل عددا (مثل العمر)، قم باختبار من الشكل
\InlineCode{if (variable == 1)}.\\
بالمقابل،إذا كان المتغيّر منطقياّ، (أي إمّا 0 أو 1 لقول صحيح أو خطأ)، قم باختبار من الشكل
\InlineCode{if (variable)}.

\section{الشرط \texttt{switch}}

الشرط
\InlineCode{if\dots else}
الّذي رأيناه، هو أكثر أنواع الشروط استخداما.\\
في الواقع، لا يوجد 36 طريقة لكتابة شرط في \textenglish{C}.
\InlineCode{if\dots else}
يُمكّن من التعامل مع كلّ الحالات.

مع ذلك،
\InlineCode{if\dots else}
يبدو قليلا\dots تكراريّا. فلنَرَ هذا المثال:

\begin{Csource}
if (age == 2)
{
	printf("Hello baby !");
}
else if (age == 6)
{
	printf("Hello kid !");
}
else if (age == 12)
{
	printf("Hello young !");
}
else if (age == 16)
{
	printf("Hello teenager !");
}
else if (age == 18)
{
	printf("Hello adult !");
}
else if (age == 68)
{
	printf("Hello grandpa !");
}
else
{
	printf("I have no words ready for your age");
}
\end{Csource}

\subsection{بناء \texttt{switch}}

المبرمجون يكرهون الأشياء التكراريّة، لقد رأينا ذلك من قبل.

إذن، من أجل تجنّب التكرار في اختبار قيمة متغيّر وحيد، فقد اخترعوا تعليمة مثل
\InlineCode{if\dots else}.
هذه التعليمة الخاصّة تدعى
\InlineCode{switch}.
إليكم مثالا نقوم فيه باستعمال
\InlineCode{switch}.
على المثال السابق:

\begin{Csource}
switch (age)
{
	case 2 :
	printf("Hello baby !");
	break;
	
	case 6 :
	printf("Hello kid !");
	break;
	
	case 12 :
	printf("Hello young !");
	break;
	
	case 16 :
	printf("Hello teenager !");
	break;
	
	case 18 :
	printf("Hello adult !");
	break;
	
	case 68 :
	printf("Hello grandpa !");
	break;
	
	default :
	printf("I have no words ready for your age");
}
\end{Csource}


استلهم من مثالي لإنشاء
\InlineCode{switch}
الخاصة بك. نستخدمها نادرا، لكنّها عمليّة كثيرا لأنّها تجعلنا نكتب قدرا أقل (قليلا) من الشفرة.

الفكرة هي أن نكتب 
\InlineCode{switch (myVariable)}
 لكي نقول: سنقوم باختبار حول قيمة المتغير
\InlineCode{myVariable}.
ثم نقوم بفتح حاضنتين نغلقهما في الأسفل.

بعد ذلك، في داخل الحاضنتين، تتعامل مع كل الحالات:
\InlineCode{case 2}، \InlineCode{case 4}، \InlineCode{case 5}، \InlineCode{case 45}،\dots

يجب عليك في نهاية كل حالة، أن تضع التعليمة 
\InlineCode{break;}.
إن لم تفعل فإن الحاسوب سينتقل لقراءة التعليمات الموالية التي هي  من المفروض محجوزة للحالات الأخرى! أي أن التعليمة 
\InlineCode{break;}
تجبر الحاسوب على الخروج من الحاضنتين.

في النهاية،
\InlineCode{default}
هي مثابة \InlineCode{else}
الذي تعرفه جيدًا الآن. أي أنه إن لم يساوي المتغير أي من الحالات المذكورة فإن الحاسوب يقوم بتشغيل الحالة
\InlineCode{default}.

\subsection{التحكم بـقائمة بواسطة \texttt{switch}}

\InlineCode{switch}
يُستخدم بكثرة لإنشاء قوائم في الكونسول.\\
أعتقد أنّه الوقت المناسب لبعض التطبيق!

\subsubsection{إلى العمل!}

نريد أن نظهر في الكونسول قائمة للمستعمل، نستخدم
\InlineCode{printf}
لعرض مختلف الخيارات المتوفّرة. كل اختيار يرافقه رقم، وعلى المستعمل أن يدخل رقم الاختيار الذي يريد.\\
هذا على سبيل المثال ما يجب أن يظهر:

\begin{Console}
=== Menu ===
1. Royal Cheese
2. Mc Deluxe
3. Mc Bacon
4. Big Mac
Your choice ?
\end{Console}

\paragraph{مهمّتك (إن قبلتها):}
ستقوم بإظهار القائمة بالاستعانة بـ\InlineCode{printf} 
و ستستعمل 
\InlineCode{scanf}
لاسترجاع اختيار المستعمل  في متغيّر
\InlineCode{choice}
و من ثمّ تستعمل
\InlineCode{switch}
لنتحقق من الاختيار الذي قام به المستعمل. وفي النهاية حسب الحالة ستقول للمستعمل ماذا اختار، هل اختار
"\textenglish{Big Mac}"
أو
"\textenglish{Mc Bacon}"
مثلا.

هيّا، إلى العمل!

\subsubsection{تصحيح}

هذا هو التصحيح (أتمنّى أنّك قد وجدته بنفسك!):

\begin{Csource}
#include <stdio.h>
#include <stdlib.h>
int main(int argc, char *argv[])
{
	int choiceMenu;
	
	printf("=== Menu ===\n\n");
	printf("1. Royal Cheese\n");
	printf("2. Mc Deluxe\n");
	printf("3. Mc Bacon\n");
	printf("4. Big Mac\n");
	printf("\nYour choice ? ");
	scanf("%d", &choiceMenu);
	
	printf("\n");
	
	switch (choiceMenu)
	{
		case 1:
		printf("You have chosen the Royal Cheese. Good choice !");
		break;
		case 2:
		printf("You have chosen the Mc Deluxe. Berk, too much sauce...");
		break;
		case 3:
		printf("You have chosen the Mc Bacon. Well, it goes ;o)");
		break;
		case 4:
		printf("You have chosen the Big Mac. You must be very hungry!");
		break;
		default:
		printf("You haven't specified a correct number. You shall not eat anything!");
		break;
	}
	
	printf("\n\n");
	
	return 0;
}
\end{Csource}
و هذا هو العمل!

أتمنى أنك لم تنس \InlineCode{default}
 في نهاية \InlineCode{switch}!
في الواقع، عندما تبرمج يجب عليك التفكير في كلّ الحالات،  فحتّى لو طلبت عددا بين 1 و4، فسيأتيك دائما أحمق ليكتب 10 أو حتّى "مرحبا"،  وهذا ليس ما كنت تنتظره. 

باختصار، عليك دائما أن تكون حذرا ولا تثق في المستخدم، لأنّه يمكنه إدخال أيّ شيء. يجب عليك اختبار حالة
\InlineCode{default}
أيضًا أو
\InlineCode{else}
في الشروط التي تنشئها بـ\InlineCode{if}.

\begin{information}
أنصحك أن تألف عمل القوائم في الكونسول، لأنّنا سننشئ عادة برامج فيها وستحتاجها حتما.
\end{information}

\section{الثلاثيات: الشروط المختصرة}

توجد طريقة أخرى لإنشاء الشروط، أكثر ندرة.

هذا ما يعرف بـ\textbf{العبارات الثلاثيّة}.\\
في الواقع، هي تشبه
\InlineCode{if\dots else}،
باستثناء أنها تكتب على سطر واحد!

و لأنّ إعطاء مثال واحد أحسن من كتابة خطاب طويل، فسوف أعطيك نفس الشرط مرّتين: الأولى باستخدام
\InlineCode{if\dots else}،
و الثانية مطابقة لها باستخدام عبارة ثلاثيّة.
\subsection{شرط \texttt{if\dots else} معروف}

فلنفرض أنّ لدينا متغيّرا منطقيّا
\InlineCode{major}
يأخذ القيمة 1 إن كان الشخص راشدًا و0 إن كان قاصرًا.\\
نريد تغيير قيمة المتغيّر 
\InlineCode{age}
حسب المتغيّر المنطقي، نضع فيها 18 إن كان راشدا و17 إن كان قاصرا. أوافقك الرأي على أنّه مثال غبيّ، ولكنّه يسمح لنا بفهم آلية عمل العبارات الثلاثيّة.

هكذا يكون باستعمال \InlineCode{if\dots else}:

\begin{Csource}
if(major)
	age = 18;
else
	age = 17;
\end{Csource}

\begin{information}
تلاحظ أنني قمت بنزع الحاضنتين لأنه لا توجد سوى تعليمة واحدة داخل 
\InlineCode{if}
و أخرى داخل 
\InlineCode{else}،
كما قد شرحت لك سابقا.
\end{information}

\subsection{نفس الشرط بالثلاثيّة}

هذه الشفرة تقوم تماما بنفس عمل الشفرة السابقة، لكنّها مكتوبة بالشكل الثلاثي: 

\begin{Csource}
age = (major) ? 18 : 17;
\end{Csource}

الثلاثيات تسمح، بسطر واحد، بتغيير قيمة متغيّر حسب شرط معيّن. هنا، الشرط هو
\InlineCode{major}
ببساطة، ولكن كان بالإمكان وضع أي شرط مهما كان طوله. تريد مثالا آخرا؟\\ 
\InlineCode{authorization = (age >= 18) ? 1: 0;}.

علامة الاستفهام تمكّن من قول "هل هو راشد؟". إن كان الأمر كذلك فنضع القيمة 18 في
\InlineCode{age}،
و إلّا (النقطتان تعنيان
\InlineCode{else}
هنا) نضع القيمة 17.

الثلاثيّات ليست ضروريّة جدّا، شخصيّا، لا أستخدمها إلّا قليلا لأنّها تجعل قراءة الشفرة أكثر صعوبة نوعا ما. لكن يجب عليك دراستها تحسّبا ليوم تقع فيه على شفرة مليئة بالثلاثيّات!

\section*{ملخّص}

\begin{itemize}
	\item \textbf{الشروط}
	أمور قاعدية في البرمجة، وباستخدامها 
	\textbf{نتخّذ قرارات}
	 على حسب قيمة متغيّر ما.
	\item الكلمات المفتاحية
	 \InlineCode{if}،
	 \InlineCode{else}،
	 \InlineCode{else if}
	 تعني -على التوالي- إذا، وإلا ، وإلا فإذا. يمكننا استعمال
	 \InlineCode{else if}
	 بالقدر الذي نريد.
	 \item \textbf{المتغير المنطقي}
	 هو متغير يشير إذا ما كان الشيء صحيحا أو خاطئًا، الصفر يعني خطأ والواحد يعني صحيح (أي قيمة مختلفة عن 0 تعتبر صحيح). نستخدم النوع
	 \InlineCode{int}
	 للتصريح عن هذه المتغيّرات لأنها ليست سوى أعداد في الواقع.
	 \item \InlineCode{switch}
	 هي بديل لـ\InlineCode{if}،
	  تستخدم عندما نريد دراسة حالات قيمة متغيّر ما. يسمح بجعل الشفرة أكثر وضوحًا إذا كنت تريد اختبار عدد معتبر من الحالات. إذا كنت تستخدم كثيرا من
	 \InlineCode{else if}
	 فهذه عادة ما تكون إشارة إلى أن 
	 \InlineCode{switch}
	 تكون أحسن في جعل الشفرة أسهل للقراءة.
	 \item \textbf{الثلاثيات} 
	 هي شروط مختصرة جدّا تسمح بإعطاء قيمة لمتغيّر حسب نتيجة اختبار. نستخدمها بشكل قليل لأنها قد تجعل الشفرة أقل وضوحًا.
\end{itemize}
