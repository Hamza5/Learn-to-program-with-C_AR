\chapter{برمجة لعبة
الـ\textenglish{Pendu}}
أكرر دائما : التطبيق شيء ضروريّ. هو ضروريّ لك لأنك اكتشفت كثيرا من المفاهيم النظرية و، أيّا كان ما تقول، لن تفهمها حقّا بدون تطبيق.

في هذا العمل التطبيقي، أقترح عليك إنشاء لعبة الـ\textenglish{Pendu}.
و هي لعبة حروف تقليديّة يتمّ فيها تخمين كلمة سريّة حرفا بحرف. و الـ\textenglish{Pendu}
سيكون إذن لعبة في الكونسول بلغة
\textenglish{C}.

الهدف هو جعلك تستخدم كلّ ما تعلّمته حتّى الآن : المؤشرات، السلاسل المحرفيّة، الملفات، الجداول... باختصار، الأشياء الجيّدة فقط !

\section{التعليمات}
سأقوم بشرح قواعد الـ\textenglish{Pendu}
الواجب إنشاءه. سأعطيك هنا التعليمات، أي سأشرح لك بدقّة كيف يجب أن تعمل اللعبة التي ستُنشئها.

أعتقد أن الجميع يعرف
الـ\textenglish{Pendu}،
أليس كذلك ؟ هيّا، تذكير صغير لا يمكن أن يحدث ضررا : هدف الـ\textenglish{Pendu}
هو إيجاد الكلمة المخبّأة في أقلّ من عشر محاولات (يمكنك تغيير العدد الأقصى لتغيير صعوبة اللعبة، بالطبع !).
