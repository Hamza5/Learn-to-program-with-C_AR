\chapter{أنشئ أنواع متغيرات خاصّة بك}
تسمح لغة الـ\textenglish{C}
بالقيام بشيء يعتبر قوياً جداً : و هو أن ننشئ أنواعاً خاصة بنا، "أنواع متغيّرات مخصّصة". سنرى نمطين : الـهياكل
(\textenglish{Structures})
و التعدادات
(\textenglish{Enumerations}).

 إن إنشاء أنواع خاصّة بنا يعتبر أمراً ضروريا خاصة إذا أردنا إنشاء برامج أكثر تعقيداً.

الأمر ليس  (لحسن الحظّ) بالصعب، لكن ركّز جيّدا لأننا سنستعمل الهياكل كل الوقت انطلاقا من الفصل القادم.\\
يجب أن تعلم أنّ المكتبات تنشئ غالبا أنواعها الخاصّة. لن يمرّ وقت كثير حتّى تستخدم نوعا يدعى "ملف"، و بعده بقليل، أنواع أخرى مثل "نافذة"، "صوت"، "لوحة مفاتيح"، إلخ.
