\chapter{عمل تطبيقي : \textenglish{Mario Sokoban}}

المكتبة
\textenglish{SDL}
تقدّم، مثلما رأينا، عددا كبيراً من الدوال الجاهزة للاستعمال. يمكن ألا نستطيع التعوّد عليها في البداية لقلّة التطبيق.

هذا العمل التطبيقي الأول في هذا الجزء من الكتاب سيعطيك فرصة التطبيق و اختبار أشياء لم تسنح لك فرصة تجريبها.
أعتقد أنه بإمكانك التخمين، فهذه المرة لن يكون التطبيق عبارة عن كونسول و إنما سيتحتوي على واجهة رسومية !

ماذا سيكون موضوع هذا العمل التطبيقي ؟ لعبة السوكوبان !\\
قد لا يعني لك هذا العنوان شيئاً، لكن هذه هي لعبة ذكاء تقليديّة. إنّها تنصّ على دفع صناديق لوضعها في أماكن محددة في متاهة.

\section{مواصفات \textenglish{Sokoban}}

\subsection{بخصوص \textenglish{Sokoban}}

الكلمة
"\textenglish{Sokoban}"
هي كلمة يابانية تعني "صاحب محلّ".\\
إنّها عبارة عن لعبة ذكاء تم اختراعها في الثمانينات بواسطة
\textenglish{Hiroyuki Imabayashi}.
و قد مثّلت برمجة هذه اللعبة تحدّيا كبيراً في ذلك الزمن.

\subsubsection{الهدف من اللعبة}

المبدأ بسيط : تقوم بتحريك شخصية في متاهة. يجدر بالشخصية أن تقوم بدفع صناديق إلى مواقع محددة. لا يمكن للاعب أن يدفع صندوقين في آن واحد.

حتى و إن كان المبدأ مفهوماً و بسيطاً، فهذا لا يعني أن اللعبة في حدّ ذاتها سهلة ! إذ أنه يجب عليك أحيانا تكسير رأسك بالتفكير لحلّ اللغز.

الصورة الموالية تُريك كيف تبدو اللعبة التي سنقوم ببرمجتها :

\Picture{Chapter_III-5_Mario-Sokoban}

\subsubsection{لماذا اخترتُ هذه اللعبة بالذات ؟}

لأنها لعبة شعبية، جيدة لأن تكون موضوعاً برمجياً و يمكننا إنشاؤها بواسطة ما تعلّمناه من الدروس السابقة.\\
يجب هنا أن نكون منظّمين. إذ أن الصعوبة لا تكمُن في برمجة اللعبة في حدّ ذاتها لكن في ما إن نظّمنا العمل. و لهذا فسنقوم بتقسيم البرنامج إلى عدّة ملفات
\InlineCode{.c}
بطريقة ذكيّة و نحاول إنشاء الدوال المناسبة.

من أجل هذا الأمر، قررت تغيير الطريقة بالنسبة لهذا العمل  التطبيقي : لن أقدّم لك توجيهات و أقدّم التصحيح في النهاية. بالعكس، سأريك كيف نقوم ببناء المشروع كلّه من الألف إلى الياء.

\begin{question}
ماذا لو كنتُ أريد التدرّب لوحدي ؟
\end{question}

حسناً إذا فلتنطلق لوحدك، هذا أمر جيد !\\
ستحتاج ربّما وقتاً أكثر : لقد استغرقت شخصيا يوماً كاملاً لبرمجة اللعبة، هذا ليس بالوقت الكثير ربّما لأنه جرت العادة أن أقوم بالبرمجة و و أن أتحاشى الوقوع في بعض الأفخاخ المتداولة (لكنّ هذا لم يمنعني من إعادة التفكير عدّة مرّات).

اِعلم بأنه توجد الكثير من الطرق التي يمكن بها برمجة هذه اللعبة. سأعطيك طريقتي في برمجتها : ليست أحسن طريقة و لكنها بالتأكيد ليست أسوء واحدة.\\
سننتهي من هذا التطبيق بقائمة من الإقتراحات لتحسين اللعبة، كما أنني سأعطيك الرابط لتحميل اللعبة و الشفرة المصدرية الكاملة.

أنصحك مجدداً أن تحاول برمجة اللعبة لوحدك، حتى لو استغرقت 3 أو 4 أيام. إفعل أحسن ما لديك. من المهم جدّا أن تقوم بالتطبيق.

\subsection{المواصفات}

المواصفات هي عبارة عن وثيقة نكتب فيها كل ما يجب على البرنامج أن يستطيع فعله.

هذا ما أقترحه :

\begin{itemize}
	\item يجب أن يتمكن اللاعب من التحرّك في المتاهة و دفع الصناديق.
	\item لا يمكنه أن يدفع صندوقين معاً.
	\item تُربح الجولة إذا تواجدت كلّ الصناديق في الأماكن المخصصة لها.
	\item سيتم حفظ كلّ مستويات اللعبة في ملف، (ليكن مثلا 
	\InlineCode{levels.lvl}).
	\item يجب أن يتم دمج مـُنشئ المستويات 
	(\textenglish{Levels editor})
	في البرنامج ليتمكن أي شخص كان من صنع مستويات خاصة به (هذا ليس أمراً ضرورياً لكنه يعتبر إضافة مميزة !).
\end{itemize}

هذا كافٍ لنعمل كثيراً.

يجب أن تعرف أنه هناك أشياء لا يجيد البرنامج القيام بها، و يجب ذِكرُ هذا الأمر أيضاً.

\begin{itemize}
	\item برنامجنا قادر على التحكّم في مرحلة واحدة في المرّة الواحدة. إن أردت أن تكون اللعبة عبارة عن تتالي جولات، فما عليك سوى برمجة ذلك بنفسك في نهاية هذا العمل التطبيقي.
	\item البرنامج لا يقوم بحساب الوقت المٌستغرق في كلّ جولة (نحن لا نجيد فعل ذلك بعد) و لا يمكنه حساب النقاط.
\end{itemize}

على أي حال، فكلّ الأشياء التي نريد القيام بها (خاصة مـُنشِئ المراحل) تأخذ منا وقتاً لابأس به.

\begin{information}
سأعطيك في نهاية العمل التطبيقي، جملة التحسينات التي تُمكن إضافتها إلى اللعبة. و هذه ليست كلمات في الهواء، لأنّها أفكار طبّقتها أنا شخصيّا في نسخة كاملة من اللعبة  سأقترح عليك تنزيلها.\\
بالمقابل، لن أعطيك الشفرة المصدرية الخاصة بالنسخة الكاملة لأنني أريدك أن تعمل بنفسك و تتدرّب (لن أعطيك كلّ شيء على طبق من فضّة !).
\end{information}

\subsection{الحصول على  الصور اللازمة لللعبة}

في معظم الألعاب ثنائية الأبعاد، أيّا كان نوعها، نسمّي الصور التي تشكّل اللعبة 
\textenglish{\textit{Sprites}}.\\
 في حالتنا، قرّرت إنشاء
\textenglish{Sokoban}
 و وضع الشخصية 
\textenglish{Mario}
لتكون اللاعب الرئيسي فيها (من هنا جاء اسم اللعبة 
\textenglish{Mario Sokoban}).
بما أن 
\textenglish{Mario}
شخصية لها شعبية كبيرة في عالم الألعاب 
\textenglish{2D}،
لن نتعب في الحصول على الـ\textenglish{sprites}
الخاصّة بهذه الشخصيّة. سنحتاج أيضا إلى
\textenglish{sprites}
خاصة بالجدران، الصناديق، الأماكن المستهدفة، إلخ.

إذا بحثت في
\textenglish{Google}
عن
"\textenglish{sprites}"
فستحصل على عدّة نتائج. توجد العديد من المواقع التي توفّر
\textenglish{sprites}
خاصّة بألعاب
\textenglish{2D}
قد تكون لعبتها في السابق.

و هذه هي الّتي سنحتاج إليها :

\begin{Table}{2}
\textenglish{Sprite} & الشرح\\
\includegraphics{Chapter_III-5_Wall}&
جدار\\
\includegraphics{Chapter_III-5_Box}&
صندوق\\
\includegraphics{Chapter_III-5_Box2}&
 صندوق متموضع فوق منطقة مستهدفة\\
\includegraphics{Chapter_III-5_Mario-down}&
بطل اللعبة
(\textenglish{Mario})
باتجاه الأسفل\\
\includegraphics{Chapter_III-5_Mario-right}&
بطل اللعبة باتجاه اليمين\\
\includegraphics{Chapter_III-5_Mario-left}&
بطل اللعبة باتجاه اليسار\\
\includegraphics{Chapter_III-5_Mario-up}&
بطل اللعبة باتجاه الأعلى\\
\end{Table}

الأسهل هو أن تقوم بتحميل الحزمة التي أعددتها لك.

\url{https://openclassrooms.com/uploads/fr/ftp/mateo21/sprites_mario_sokoban.zip}

\begin{information}
كان من الممكن أن أستعمل 
\textenglish{sprite}
واحداً خاصاً باللاعب. كان بإمكاني جعله موجّهاً إلى الأسفل فقط، لكن إضافة امكانية توجيهه في الإتجاهات الأربعة تضيف القليل من الواقعية. و هذا يشكّل تحدّيا آخر لنا !
\end{information}

قمت أيضاً بإنشاء صورة أخرى لتكون عبارة عن الواجهة الأساسية للعبة حين تبدأ، لقد أرفقت لك الصورة بالحزمة الّتي يفترض بك تنزيلها. لاحظ الصورة التالية :

\Picture{Chapter_III-5_Home}

ستلاحظ بأن الصور تأخذ صيغا مختلفة. يوجد منها ماهو
\textenglish{GIF}،
ماهو
\textenglish{PNG}
و حتى ماهو
\textenglish{JPEG}.
و لهذا فنحن بحاجة إلى استعمال المكتبة
\textenglish{SDL\_Image}.\\
فكّر في جعل مشروعك يعمل مع الـ\textenglish{SDL}
و الـ\textenglish{SDL\_Image}.
إذا نسيت كيف تفعل ذلك، فراجع الفصول السابقة. إذا لم تقم بتخصيص المشروع بشكل صحيح، سيشير المُترجم بأن الدوال التي تستعملها (مثل
\InlineCode{IMG\_Load})
غير موجودة !
