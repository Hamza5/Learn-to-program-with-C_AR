\documentclass[11pt, a4paper]{book}
\usepackage[svgnames,table,dvipsnames]{xcolor} % For color names
\usepackage{fontspec} % font selecting commands
\usepackage[breakable, skins]{tcolorbox} % For colored boxes
\usepackage{graphicx} % For pictures
\usepackage{polyglossia} % Babel alternative in XeLaTex
\usepackage{fancyhdr} % For page headers
\usepackage[top=2.5cm, bottom=2.5cm, left=3cm, right=2cm]{geometry} % To control page margins
\usepackage{listings} % For source code
\usepackage[breaklinks,
			colorlinks,
			linkcolor=Black,
			urlcolor=RedOrange,
			unicode=true,
			pdflang=ar,
            pdftitle={تعلّم البرمجة بلغة الـC},
            pdfauthor={عدن بلواضح,حمزة عباد,أحمد زبوشي},
            pdfdisplaydoctitle=true,
            pdfduplex=DuplexFlipLongEdge]{hyperref} % For hyperlinks and PDF metadata
\usepackage{float}
\usepackage{tabu,booktabs}
\usepackage{bidi}

% Font settings
\defaultfontfeatures{Ligatures=TeX, Path=Fonts/, Extension=.ttf}
\newfontfamily\arabicfont{amiri-}[
	Script=Arabic,
	Scale=1.2,
	UprightFont=*regular,
	BoldFont=*bold,
	ItalicFont=*slanted,
	BoldItalicFont=*boldslanted
] % An arabic font
\newfontfamily\englishfont{LiberationSans-}[
	Script=Latin,
	UprightFont=*Regular,
	BoldFont=*Bold,
	ItalicFont=*Italic,
	BoldItalicFont=*BoldItalic
] % Font used for latin text in the document
\newfontfamily\arabicfonttt{LiberationMono-}[
	Script=Latin,
	UprightFont=*Regular,
	BoldFont=*Bold,
	ItalicFont=*Italic,
	BoldItalicFont=*BoldItalic
] % Monospace font, for displaying codes

\renewcommand{\dots}{...\enskip}

% Colors
\colorlet{questionback}{white!90!black}
\colorlet{questionframe}{white!95!black}
\colorlet{questiontitle}{white!60!black}

\colorlet{criticalback}{red!20}
\colorlet{criticalframe}{red!10}
\colorlet{criticaltitle}{red!80}

\colorlet{warningback}{orange!20}
\colorlet{warningframe}{orange!10}
\colorlet{warningtitle}{orange!80}

\colorlet{infoback}{cyan!20}
\colorlet{infoframe}{cyan!10}
\colorlet{infotitle}{cyan!80}

\colorlet{chapter}{green!30!blue!80}
\colorlet{section}{orange}

% Language settings
\setmainlanguage[locale=algeria]{arabic}
\setotherlanguage{english}
\addto\captionsarabic{ % Without this, changes won't take effect, because of Polyglossia
	\renewcommand{\partname}{الجزء}
	\renewcommand{\chaptername}{الفصل}
	\renewcommand{\contentsname}{جدول المحتويات}
}

% Boxes definitions
\tcbset{arc=2pt, enhanced,
		attach boxed title to top right={yshift=-\tcboxedtitleheight/3*2, xshift=\tcboxedtitlewidth/5*2},
		boxrule=0.5mm, fonttitle=\Large, boxed title style={circular arc, square, halign=center, valign=center}}
	
\newtcolorbox{question}{colback=questionback, colframe=questionframe, title=؟, coltitle=questiontitle,
						boxed title style={colback=questionback}} % Used for question boxes
					
\newtcolorbox{critical}{colback=criticalback, colframe=criticalframe, title=\textenglish{x}, coltitle=criticaltitle,
						boxed title style={colback=criticalback}} % Used for critical warning boxes

\newtcolorbox{warning}{colback=warningback, colframe=warningframe, title=!, coltitle=warningtitle, fonttitle=\LARGE,
						boxed title style={colback=warningback}} % Used for warning boxes

\newtcolorbox{information}{colback=infoback, colframe=infoframe, title=م, coltitle=infotitle,
							boxed title style={colback=infoback}} % Used for information boxes

\newcommand\InlineCode[1]{\fcolorbox{LightGray}{Snow}{\ttfamily \LR{#1}}}

% Titles settings (Make them orange or blue)
\makeatletter
\let\oldpart\part
\newcommand{\@partstar}[1]{\cleardoublepage\phantomsection\addcontentsline{toc}{part}{#1}{\color{section}\oldpart*{#1}}}
\newcommand{\@partnostar}[1]{{\color{section}\oldpart{#1}}}
\renewcommand{\part}{\@ifstar{\@partstar}{\@partnostar}}
\let\oldchapter\chapter
\newcommand{\@chapterstar}[1]{\cleardoublepage\phantomsection\addcontentsline{toc}{chapter}{#1}{\color{chapter}\oldchapter*{#1}}\markboth{#1}{}}
\newcommand{\@chapternostar}[1]{{\color{chapter}\oldchapter{#1}}}
\renewcommand{\chapter}{\@ifstar{\@chapterstar}{\@chapternostar}}
\let\oldsection\section
\newcommand{\@sectionstar}[1]{\phantomsection\addcontentsline{toc}{section}{#1}{\color{section}\oldsection*{#1}}\markright{#1}}
\newcommand{\@sectionnostar}[1]{{\color{section}\oldsection{#1}}}
\renewcommand\section{\@ifstar{\@sectionstar}{\@sectionnostar}}
\makeatother

% Pictures settings
\graphicspath{{Pictures/}} % Folder of pictures

\newcommand\Picture[2][]{ % This command automatically centers the picture and fits its size to the page. It supports captions too.
  \begin{center}
    \includegraphics[max size={0.8\textwidth}{0.5\textheight}]{#2}\\
    #1
  \end{center}
}

% Paragraphs settings
\setlength{\parskip}{4mm plus 2mm minus 2mm} % Spacing between paragraphs (+/-)

% Page header and footer settings
\setlength{\headheight}{15pt}
\pagestyle{fancy}
\renewcommand{\chaptermark}[1]{ \markboth{{\chaptername~\thechapter.~#1}}{} }
\renewcommand{\sectionmark}[1]{ \markright{\thesection.~#1} }
\fancyhead{}
\fancyhead[OR]{\rightmark}
\fancyhead[EL]{\leftmark}
\setlength{\footskip}{1.5cm}

% Fixing the issues of the numbering
\renewcommand{\thepart}{\Alph{part}}
%\renewcommand{\thechapter}{\Alph{part}.\arabic{chapter}}
\renewcommand{\thesection}{\arabic{section}.\arabic{chapter}}
%\renewcommand{\thesection}{\Alph{part}.\arabic{section}.\arabic{chapter}}
%\renewcommand{\thesubsection}{\Alph{part}.\arabic{subsection}.\arabic{section}.\arabic{chapter}}
%\renewcommand{\thesubsubsection}{\Alph{part}.\arabic{subsubsection}.\arabic{subsection}.\arabic{section}.\arabic{chapter}}
\setcounter{secnumdepth}{1}
\setcounter{tocdepth}{1}

% Global settings for code and console
\lstset{frame=single, basicstyle=\ttfamily, breaklines=true, showlines, aboveskip=\parskip, extendedchars}

% C source code
\lstdefinestyle{C}{language=C, showstringspaces=false, numbers=left, escapechar=§,
        keywordstyle=\bfseries\color{RoyalBlue}, commentstyle=\itshape\color{Gray},
        numberstyle=\color{Gray}, stringstyle=\color{Crimson},
        directivestyle=\color{DarkOrange},
        deletekeywords={return,if,else,switch,for,while,do,const,static,sizeof},
    	morekeywords={SDL_Surface,Uint32,SDL_Rect,SDL_Event,SDL_TimerID,SDL_NewTimerCallback,TTF_Font,SDL_Color,
    	FMOD_SYSTEM,FMOD_RESULT,FMOD_SOUND,FMOD_CHANNEL,FMOD_CHANNELGROUP,FMOD_BOOL,FMOD_DSP_FFT_WINDOW,size_t},
        morekeywords=[2]{return,if,else,switch,for,while,do,const,static,sizeof}, keywordstyle=[2]\bfseries\color{Magenta},
        morekeywords=[3]{printf,scanf,fprintf,fscanf,fputc,fgetc,fputs,fgets,fopen,fclose,fseek,ftell,rewind,srand,rand,
        time,SDL_Delay,SDL_GetTicks,SDL_AddTimer,SDL_RemoveTimer,TTF_OpenFont,TTF_CloseFont,TTF_Init,TTF_GetError,TTF_Quit,
        malloc,free,SDL_CreateRGBSurface,SDL_FreeSurface,SDL_BlitSurface,SDL_LoadBMP,SDL_WM_SetIcon,IMG_Load,SDL_GetError,
        SDL_Init,SDL_Quit,SDL_SetVideoMode,SDL_WM_SetCaption,SDL_FillRect,SDL_Flip,SDL_MapRGB,SDL_SetAlpha,SDL_SetColorKey,
    	SDL_PollEvent,SDL_WaitEvent,SDL_EnableKeyRepeat,SDL_ShowCursor,SDL_WarpMouse,TTF_RenderText_Blended,
    	TTF_SetFontStyle,sprintf,TTF_RenderText_Shaded,FMOD_System_Create,FMOD_System_Init,FMOD_System_Close,
    	FMOD_System_Release,FMOD_System_CreateSound,FMOD_System_PlaySound,FMOD_Sound_Release,FMOD_System_GetChannel,
    	FMOD_System_GetMasterChannelGroup,FMOD_Sound_SetLoopCount,FMOD_Channel_GetPaused,FMOD_Channel_SetPaused,
    	FMOD_Sound_Release,FMOD_ChannelGroup_GetPaused,FMOD_ChannelGroup_SetPaused,FMOD_Channel_GetSpectrum,
    	SDL_LockSurface,SDL_UnlockSurface},
        keywordstyle=[3]\color{RoyalBlue},
}
\lstnewenvironment{Csource}{\lstset{style=C}\setLTR}{\unsetLTR}

% Console
\lstnewenvironment{Console}{\setLTR}{\unsetLTR}

% Table settings
\setlength{\tabulinesep}{2pt}
\setlength{\arrayrulewidth}{2pt}
\taburulecolor{White}
\newenvironment{Table}[1]{ % Accepts 1 parameter which is the number of columns
\taburowcolors[2] 2{LightGray!40 .. LightGray!80}
\begin{center}
  \begin{tabu}{*{#1}{|r}|}
    \toprule
    \rowfont{\bfseries\color{White}}
    \rowcolor{OrangeRed}
    \everyrow{\hline}
}{
  \end{tabu}
\end{center}
}
\newenvironment{Table*}[1]{
\taburowcolors[1] 2{LightGray!40 .. LightGray!80}
\begin{center}
  \begin{tabu}{*{#1}{|r}|}
    \toprule
    \everyrow{\hline}
}{
  \end{tabu}
\end{center}
}

% Footnote rule
\renewcommand{\footnoterule}{\rule[5pt]{0.6\textwidth}{0.5pt}}

\begin{document}
  \setcounter{page}{-2}
  \thispagestyle{empty}
\cleardoublepage
\thispagestyle{empty}
\begin{center}
{\fontsize{0.7cm}{1.4cm}\selectfont\bfseries
\textcolor{section}{\textenglish{Mathieu Nebra}}
}

\vspace{1cm}
\textcolor{chapter}{\mdseries
{\fontsize{2cm}{4cm}\selectfont
تَعَلَّم البَرْمَجَة بلُغَة}\\
{\fontsize{3cm}{6cm}\selectfont
\textenglish{C}
}
}

\vspace{1cm}
{
\fontsize{0.85cm}{1cm}\selectfont
الإصدار الثاني
}
\vfill

{
\fontsize{0.5cm}{1.5cm}\selectfont

\begin{tabu} to 0.5\textwidth {>{\itshape}r r}
تَرْجَمَة & عدن بلواضح\\
مُرَاجَعَة وإعْدَاد & حمزة عبّاد\\
تَصميم الغلاف & أحمد زبوشي\\
\end{tabu}
}

\vfill
\includegraphics[height=0.2\textheight]{OpenClassrooms-logo}
\end{center}

  \clearpage
\thispagestyle{empty}
\oldsection*{\LARGE\color{section}
تنزيل المشروع}

{\large
تمّت إنشاء هذا الكتاب بلغة التوصيف
\LaTeX
و ترجمته بمترجم
\XeLaTeX.
يمكن الحصول على الشفرة المصدريّة الخاصة به عن طريق استنساخ مستودع
\textenglish{GitHub}
التالي~:

\url{https://github.com/Hamza5/Learn-to-program-with-C_AR}

يوجد في هذه الصفحة أيضا رابط لتنزيل النسخة الرقميّة بصيغة
\textenglish{PDF}،
و شرح لطريقة الترجمة و الاعتماديّات الواجب توفّرها، بالإضافة إلى الشفرة المصدريّة الخاصة به.

إذا كنت من مستخدمي
\textenglish{GitHub}،
يمكنك التبليغ عن الأخطاء الّتي قد تجدها في الكتاب عن طريق فتح
\textenglish{issue}
في هذا المستودع و كتابة تفاصيل الخطأ (الفصل، القسم، الفقرة، رقم الصفحة و التصحيح الموافق إن أمكن)؛ أو عن طريق القيام بـ\textenglish{fork}
لإنشاء نسخة مطابقة كمستودعك الخاص، ثم إدخال التعديلات المرادة. بعد ذلك، يمكنك القيام بـ\textenglish{pull request}
إلى المستودع الأصلي. في حالة ما كان التعديل جيّدا، سأقوم بدمجه في المستودع.
}

\oldsection*{\LARGE\color{section}
الترخيص
}
{\large
محتوى هذا الكتاب مرخّص تحت بنود رخصة
\textbf{المشاع الإبداعي، نسب المصنف - غير تجاري - الترخيص بالمثل، النسخة الثانية
(\textenglish{CC-BY-NC-SA 2.0})}،
تماما مثل ترخيص الدرس الأصلي المتوفّر في موقع
\textenglish{OpenClassrooms}.

\url{https://creativecommons.org/licenses/by-nc-sa/2.0/}

هذا يعني أنّه بإمكانك الاستفادة من هذا العمل، نسخه و إعادة توزيعه بأيّة وسيلة أو صيغة، و كذلك تعديله و استخدامه في أعمال أخرى. كلّ هذا بشرط أن تشير إلى العمل الأصلي، تعطي رابطا إلى هذه الرخصة، و تدلّ على التعديلات إن قمت بذلك. بالإضافة إلى ذلك، لا يمكنك استخدام عملك للأغراض التجاريّة من دون إذن صاحب العمل، كما يجب عليك ترخيص عملك بنفس الرخصة من دون فرض أيّة قيود إضافيّة على مستخدمي عملك.

  \chapter*{تقديم}
إن التحرّر الفكري في بداية القرن العشرين أدّى إلى توسّع في البحوث العلمية التي شملت كل الميادين لاسيّما التكنولوجية منها كعلوم الحاسوب. هذه الأخيرة أعقبتها ثورة في لغات البرمجة التي تعتبر ركيزة أساسية تقوم عليها البرامج. من بين هذه اللغات نجد لغة الـ\textenglish{C}،
إذ تعتبر من أقوى لغات البرمجة و أكثرها شيوعاً، فهي مستلهمة من طرف لغتي
 \textenglish{B}
 و
 \textenglish{BCPL}
حيث تمّ تطويرها في عام 1972 من طرف
\textenglish{Ken Thompson}
و
 \textenglish{Dennis Ritchie}،
و في ظرف سنة واحدة توسّعت لتكون عِـماد نظام التشغيل
\textenglish{UNIX}
بنسبة
90\%
ثم تم توزيعها في العام المـُوالي رسمياً عبر الجامعات لتصبح بذلك لغة برمجة عالمية. و اشتهرت لغة الـ\textenglish{C}
 كونـُها لغة عالية المستوى، لها مُترجم سريع و فعّال. كما أنها لغة برمجية نقّالة، هذا يعني أن أي برنامج يحترم المعيار
\textenglish{AINSI}
يمكن أن يتمّ تشغيله على أيّة منصّة تحتوي على مترجم
\textenglish{C}
 دون أيّة تخصيصات.

يعتبر هذا الكتاب بوابة سهلة لكلّ مبتدئ لتعلّم لغة الـ\textenglish{C}
خطوة بخطوة بدءً من الأساسيات وصولاً إلى تطوير ألعاب ثنائية الأبعاد و التحكّم في هياكل البيانات الأكثر تعقيداً. الكتاب مرفق بجملة من التمارين و الأعمال التطبيقية المحلولة التي تساعد على هضم المفاهيم المكتسبة و تطبيقها على أيّ مشكل برمجي مهما كان نوعه. و لأن الكثير من لغات البرمجة تعتمد أساساً على الـ\textenglish{C}
كالـ\textenglish{Java}
و الـ\textenglish{C++}
و الـ\textenglish{C\#}
(لغات برمجية غرضية التوجّه) و حتى
\textenglish{PHP}
(لغة لبرمجة المواقع) فإن تعلّم لغة الـ\textenglish{C}
 سيساعد على تعلّم أيّة لغة برمجية كانت. تبقى الإرادة و حبّ العمل و الشغف المفاتيح الرئيسية للنجاح و الوصول إلى الاحترافية.

\vfill

\hfill\parbox{0.3\textwidth}{\centering
عدن بلواضح

\vspace{1em}
الجزائر\\[0.5em]
في
24 ذو القعدة 1438\\[0.3em]
الموافق لـ17 أوت 2017
%\Hijritoday\\[0.3em]
%الموافق لـ\today

}


  \chapter*{مقدّمة}

\vspace{-0.6em}
تحبّ تعلّم البرمجة لكن لا تعرف من أين تبدأ ؟ هذه الدروس لتعليم لغة الـ\textenglish{C}
للمبتدئين قد جُعلت خصّيصاً من أجلك !

\vspace{-0.1em}
لغة الـ\textenglish{C}
هي لغة لا مفرّ منها، أُستلهمَت منها العديد من اللغات الأخرى. تمّ اختراعها في السبعينات و لا تزال مستعملة لحدّ الآن في البرمجة النظامية و عالم الروبوتات. تعتبر لغة الـ\textenglish{C}
لغة معقّدة، لكن إن استطعت تعلّمها ستكوّن لك قاعدة برمجية صلبة !

\vspace{-0.1em}
في هذه الدروس، ستبدأ باكتشاف مبدأ عمل الذاكرة، المتغيرات، الشروط و الحلقات. ثم ستقوم باستعمال كلّ ما تعلّمته في إنشاء واجهات رسومية بالاستعانة بالمكتبة
\textenglish{SDL}
 (ألعاب فيديو، تسجيلات صوتية \dots). أخيراً، ستتعلّم كيف تتعامل مع هياكل البيانات الأكثر شيوعاً من أجل تنظيم المعلومات في الذاكرة : قوائم متسلسلة، مكدّسات، طوابير، جداول تجزئة \dots

\vspace{-0.1em}
التحق بي في هذه الدروس من أجل اكتشاف البرمجة بلغة الـ\textenglish{C} !

\begin{figure}[H]
	\centering
	\includegraphics[height=0.3\textheight]{Introduction_original}\\
\small بعض الإنجازات الّتي سنقوم بها في هذا الكتاب
\end{figure}

\vfill
\hfill\parbox{0.3\textwidth}{\centering \textenglish{Mathieu Nebra}\\[0.2em]
مؤسس مشارك لموقع
\href{http://openclassrooms.com/}{\textenglish{OpenClassrooms}}
}

  \tableofcontents
  \part{أساسيّات البرمجة بلغة الـ\textenglish{C}}
  \chapter{قلت برمجة ؟}
\section{ما هي البرمجة؟}
\begin{question}
  ما الذي تعنيه كلمة "بَرْمَجَ"؟
\end{question}

لن أتعبك وأعطيك أصل كلمة "بَرْمَجَ"، لكنني سأختصر كل شيء في جملة: البرمجة تعني إنشاء برامج حاسوب. وهذه البرامج التي تنشئها تأمر الجهاز بالقيام بتعليمات وأفعال معيّنة.
حاسوبك الخاص يحتوي على كثير من هذه البرامج وبمختلف أنواعها:

\begin{itemize}
  \item الآلة الحاسبة تعتبر برنامجاً.
  \item معالج النصوص يعتبر برنامجاً أيضاً.
  \item وكذلك برنامج المحادثة.
  \item ألعاب الفيديو هي برامج كذلك.
\end{itemize}

\Picture[\caption{نسخة عن لعبة \textenglish{MetalSlug} الشهيرة تم إنشاؤها من طرف العضو \href{http://www.siteduzero.com/membres-294-176405.html}{\textenglish{joe87}}}]{Chapter_I-1_MetalSlug}
باختصار البرامج موجودة في كل جهاز، وهي التي تعطي الحاسوب قدرته على إنجاز مختلف المهام التي تُخوَّل إليه. يمكنك أن تنشئ برنامج تشفير أو لعبة ثنائية / ثلاثية الأبعاد باستخدام لغة برمجة مثل \textenglish{C}.

ملاحظة: لم أقل أن إنشاء لعبة يتم برمشة عين، لقد قلت فقط بأنه شيء ممكن، لكن كن متأكداً، سوف يتطلب ذلك جهدا كبيراً!

وبما أننا في بداية الطريق، فإّننا لن نقوم بإنشاء لعبة ثلاثية الأبعاد! لكنّنا سنبدأ بكيفية عرض نص على الشاشة، طبعا ستقول ما علاقة هذا بإنشاء الألعاب؟ لكن ثِق بي، هذا الأمر ليس بسيطا كما يبدو!

بالطبع هذا ليس شيئا مُبهراَ، ولكن يجب علينا أن نبدأ من هنا؛ وشيئا فشيئا يمكنك أن تنشئ برامج معقّدة أكثر. فالهدف من هذا الدرس هو أن أعرفك على كل ما يتعلق بهذه اللغة.

\section{البرمجة، بأي لغة يا ترى؟}
حاسوبك هو آلة غريبة جداً، هذا أقل ما يمكن أن نقوله عنه. يمكننا أن نخاطبه فقط بالصفر والواحد، فمثلا إذا طلبنا منه حساب 3+5 فيمكن لهذا أن يعطينا نتيجة كالتالي (هذه ليست ترجمة دقيقة ولكنها تشبه ما يحدث بالفعل):
\InlineCode{0010110110010011010011110}

ما تَرَاه هنا يسمى اللغة الثنائية
(\textenglish{Binary language})
أو لغة الآلة
(\textenglish{Machine language})،
وحاسوبك لا يفهم سوى هذه اللغة، وكما تلاحظ، هذه اللغة غير مفهومة على الإطلاق!

مشكلتنا الآن:
\begin{question}
  كيف يمكننا التعامل مع حاسوب لا يفهم سوى اللغة الثنائية؟
\end{question}

حاسوبك لا يتحدث الإنجليزية، ولا العربية، ولا أي لغة غير هذه اللغة، ولكنها صعبة جدا لدرجة أن حتى أكبر خبراء الحاسوب لا يستخدمونها.
لهذا قام بعض مهندسي الحواسيب باختراع لغات يمكن أن تُتَرجَمَ إلى اللغة الثنائية، لكن الشيء الأصعب هو إنشاء البرامج الّتي تقوم بهذه الترجمة. ولحسن الحظ فقد قاموا بهذا العمل نيابة عنا. هذه البرامج تقوم بترجمة الأوامر الّتي تكتبها (مثلا: "أُحسب 3+5") إلى شيء يشبه هذا:
\InlineCode{0010110110010011010011110}.

هذا المخطط يلخص ما كنت أشرح:

\Picture{Chapter_I-1_Translation}

\section{قليل من المفردات}
حتّى الآن كنت أتحدّث إليك بكلمات بسيطة، لكن يجب أن تعلم أنه في المعلوماتية توجد مصطلحات علمية لكل ما ذكرت. طوال هذا الدرس، سوف تتعلم استخدام المفردات المناسبة. هذا سيفيدك كثيرا خصوصا عندما تتحدث مع مبرمجين آخرين، حيث أنك سوف تتفاهم معهم بكل سهولة.

نعود إلى الحديث عن المخطط السابق في المستطيل الأول قلت أن "برنامجك مكتوب بلغة مُبَسَّطة"، في الواقع هذا النوع من اللغات يُعرف باسم لغات البرمجة عالية المستوى (\textenglish{High-level programming languages}). هناك مستويات عديدة من لغات البرمجة، وكلما كان مستوى اللغة أعلى كانت أقرب إلى اللغة الحقيقية وكان استخدامها أسهل. إذن، اللغات عالية المستوى سهلة الاستخدام لكنها تتضمن بعض السلبيّات سوف نتعرّف عليها لاحقا.

توجد العديد من لغات البرمجة، وهي متفاوتة المستوى، منها:
\begin{itemize}
  \item \textenglish{C}
  \item \textenglish{C++}
  \item \textenglish{Java}
  \item \textenglish{Visual Basic}
  \item \textenglish{Delphi}
  \item و العديد غيرها
\end{itemize}

كما تلاحظ، لم أرتبها حسب مستوياتها، لذلك لا تعتقد أن اللغة الأولى في القائمة هي الأسهل أو العكس. عموما، لائحة اللغات الموجودة طويلة جدا لدرجة أنه لا يمكنني كتابتها كلها هنا.

مصطلح آخر يجب تذكّره هو
\underline{الشفرة المصدرية}
(\textenglish{Source code})،
 وهي ببساطة الشفرة الخاصة ببرنامجك الذي تكتبه بلغة عالية المستوى والذي يتم ترجمته فيما بعد إلى اللغة الثنائية.

 ثم يأتي دور البرنامج الذي يحوّل هذه اللغة عالية المستوى إلى اللغة الثنائية، هذا النوع من البرامج يعرف باسم
 \underline{المترجم}
  أو
  \underline{المصنّف}،
 والعملية الّتي يقوم بها تسمى
 \underline{الترجمة}
 أو
 \underline{التصنيف}.

\begin{information}
  يوجد لكل لغة عالية المستوى مترجم خاص، وهذا شيء منطقي، فاللغات مختلفة فيما بينها، فلا يمكننا ترجمة لغة
\textenglish{C}
بنفس الطريقة الّتي نترجم بها
\textenglish{Delphi}
مثلا.
  بعض اللغات مثل
\textenglish{C}
تملك العديد من المترجمات، فمنها من هو مكتوب من طرف
\textenglish{Microsoft}
، و منها من
\textenglish{GNU}
، إلخ… سوف نتعرّف على كل هذا في الدرس القادم.
  لحسن الحظ، هذه المترجمات متطابقة تقريبا (رغم وجود اختلافات طفيفة بينها سوف نتعرف عليها لاحقا).
\end{information}

أخيرا، البرنامج الثنائي المنشئ بواسطة المترجم يسمى الملف
\underline{القابل للتنفيذ}
أو
\underline{التنفيذي}
(\textenglish{Executable}).
 لهذا السبب تملك البرامج
 (على الأقل برامج
 \textenglish{Windows})
 الامتداد
\textenglish{.exe}
 والذي هو اختصار كلمة
 \textenglish{EXEcutable}.

 نعود إلى مخططنا السابق، وهذه المرة سنستخدم المصطلحات الصحيحة:

 \Picture{Chapter_I-1_Compilation}

 \section{لماذا نختار تعلّم \textenglish{C}؟}
 كما قلت سابقا، يوجد كثير من اللغات عالية المستوى، فلماذا ينبغي علينا أن نبدأ بإحداها على وجه الخصوص؟ سؤال عظيم!

على أية حال يجب علينا أن نختار بأي لغة سنبدأ البرمجة عاجلا أم آجلا، وبالتالي لديك الخيار في البدء بـ:
\begin{itemize}
  \item \textbf{لغة ذات مستوى عالي جدّا}:
 وتكون سهلة جدّا أوعامة، نذكر من بينها
 \textenglish{Python}، \textenglish{Ruby}، \textenglish{Visual Basic}،
 وغيرها. هذه اللغات تسمح بكتابة برامج بشكل أسرع. عامّة تحتاج لأن تُرفق معها ملفات مُسَاعِدة لكي تعمل (كَمُفَسِّرٍ مثلا).
  \item \textbf{لغة ذات مستوى منخفض قليلا}:
هي أكثر صعوبة نوعا ما، ولكن مع لغة مثل
\textenglish{C}
 سوف تتعلم كثيرا عن البرمجة وحول طريقة عمل حاسوبك. ستكون بعد ذلك قادراً على تعلّم لغة برمجة أخرى إن أردت وبكل يُسْرٍ.
\end{itemize}

من ناحية أخرى،
\textenglish{C}
لغة برمجة واسعة الإنتشار، أُستخدمت في برمجة العديد من البرامج التي تعرفها. حتى أنها كثيرا ما تدرّس في الدراسات العليا في مجال المعلوماتية.
هذه هي الأسباب الّتي جعلتني أتحمّس لتعليمك لغة
\textenglish{C}
بالتحديد. لم أقل أنّه يجب عليك أن تبدأ بها، لكنّي قلت إنه خيار جيّد لكي أقدّم لك معرفة صلبة في هذا الدرس.

\begin{information}
  بعض لغات البرمجة موجّهة أكثر للشبكة العنكبوتية
 (\textenglish{Web})
 مثل
 \textenglish{PHP}
 أكثر منها من إنشاء البرامج المعلوماتية.
\end{information}

سوف أفترض في هذا الكتاب أنّ هذه هي لغة برمجتك الأولى وأنّه لم يسبق لكم أن برمجت من قبل. فإن كنت قد برمجت قليلا من قبل فلا مضرّة في أن تعيد من الصفر.

\begin{question}
  ما هو الفرق بين
  \textenglish{C}
  و
  \textenglish{C++}
  ؟
\end{question}

هاتان اللغتان قريبتان جدّا من بعضهما، وكلاهما مستخدمتان بكثرة. ولكي تعرف كيف نشأتا يجب عليك أن تدرس التاريخ قليلا:
\begin{itemize}
  \item في البداية، عندما كانت الحواسيب تَزِنُ أطنانا وتشغل مكانا قَدْرُهُ حجم منزلك، تمّ إختراع لغة برمجة تسمّى
\textenglish{Algol}.
  \item بعدها تطوّرت الأمور أكثر واختُرعَت لغة برمجة جديدة عُرِفَتْ باسْمِ
\textenglish{CPL}
 والّتي تطوّرت فيما بعد إلى لغة
\textenglish{BCPL}
 ثم أخذت إسم اللغة
\textenglish{B}.
  \item مع مضيّ الزمن توصّل الخبراء إلى ابتكار اللغة
\textenglish{C}
 وقد تمّ إدخال بعض التعديلات عليها إلّا أنها لا تزال من أحد اللغات الأكثر استخداما اليوم.
  \item وبعد زمن، أراد الخبراء أن يضيفوا بعض الأشياء إلى
\textenglish{C}
، يمكن اعتبارها نوعا من التحسينات. والنتيجة كانت بما يعرف بلغة
\textenglish{C++}
، وهي لغة
\textenglish{C}
 مع إضافات تمكّننا من البرمجة بطريقة مختلفة.
\end{itemize}

\begin{information}
  الـ
\textenglish{C++}
ليست أحسن من الـ
\textenglish{C}
، هي فقط تمكننا من البرمجة بطريقة مختلفة وتساعد المبرمج على تنظيم شفرة برنامجه. رغم ذلك هي تشبه الـ
\textenglish{C}
كثيرا. وإن كنت تنوي تعلّم الـ
\textenglish{C++}
فيما بعد فَسَوْفَ تجد ذلك سهلا.
\end{information}

ولو اعتُبرت
\textenglish{C++}
 تطويرا لـ
\textenglish{C}
 فإن هذا لا يعني أنه يجب استخدام
\textenglish{C++}
 فقط لإنشاء البرامج. لغة
\textenglish{C}
 ليست لغة عجوزا منسيّة، بالعكس هي مستخدمة بكثرة اليوم. بل إنها أساس أنظمة التشغيل الكبيرة مثل
\textenglish{Unix }
(ومنه
\textenglish{GNU/Linux}
 و
\textenglish{Mac OS}) و
\textenglish{Windows}.

\section{هل البرمجة صعبة؟}
هذا سؤال يعذّب روح كل من يريد تعلّم البرمجة! هل يجب أن تكون أستاذ رياضيات كبير درس 10 سنوات من التعليم العالي حتّى تبدأ البرمجة؟

الجواب هو لا بالطبع. كل ما تحتاج إليه هو معرفة العمليات الأربع الأساسية:
\begin{itemize}
  \item الجمع
  \item الطرح
  \item الضرب
  \item القسمة
\end{itemize}
هذا ليس مخيفا! سوف أشرح لك في درس لاحق كيف يقوم الحاسوب بهذه العمليات الأساسية في برامجك.

باختصار، لا توجد صعوبات غير قابلة للحلّ. في الواقع، هذا يعتمد على طبيعة برنامجك، فإذا كنت تريد إنشاء برنامج تشفير فيجب عليك معرفة بعض الأشياء في الرياضيات، وإن كان برنامجك يقوم بالرسم ثلاثي الأبعاد فيجب أن تكون لديك بعض المعرفة بالهندسة الفضائية.

كل حالة تعامل بطريقة خاصّة. ولكن لتعلّم لغة
\textenglish{C}
 نفسها لا تحتاج إلى أيّة معارف قبليّة.

\begin{question}
  إذن أين هو الفخ؟ وأين تكمن الصعوبة؟
\end{question}

يجب أن تعرف كيف يعمل الحاسوب، لتفهم ما الّذي نقوم به في C. من هذا المنطلق، كن متيقّنا أنّي سأعلّمك كلّ هذا شيئا فشيئا.

اعلم أن للمبرمج صفات أيضا مثل:
\begin{itemize}
  \item الصبر: البرنامج لا يعمل عادة من أوّل محاولة، يجب أن تكون مثابراً.
  \item حسّ المنطق: صحيح أنّك لست بحاجة إلى أن يكون لديك مستوى جيّد في الرياضيّات، لكنّ هذا لا يمنع من التفكير وتحليل المشكلات بالمنطق.
  \item الهدوء: فيجب عليك ألّا تضرب حاسوبك بالمطرقة، فهذا لن يجعل برنامجك يعمل!
\end{itemize}

  \chapter{الحصول على الأدوات اللازمة}

بعد تجاوزنا لدرس تمهيدي مليئ بالثرثرة سوف نبدأ بالدخول في صلب الموضوع. سوف نجيب عن السؤال التالي : "ما هي البرامج التي نحتاج إليها للبدء في البرمجة ؟".

لا يوجد شيء صعب في هذا الدرس، سوف نأخذ وقتنا للتأقلم على هذه البرامج الجديدة.

اغتنموا الفرصة ! في الفصل التالي سنبدأ حقّا في البرمجة و لن يكون هناك وقت للقيلولة !

\section{الأدوات اللازمة للمبرمج}

إذن ما هي الأدوات التي نحتاج إليها ؟
إذا تابعت الفصل السابق جيّدا، فستعرف واحدا على الأقل !

هل تعلم عمّا أتحدّث ؟ حقّا لا ؟

حسنا، نحن نتحدّث عن
\textbf{المترجم}
الذي يمكّن من ترجمة لغة الـ\textenglish{C}
إلى اللغة الثنائيّة !

كما قلت لكم في الفصل الأوّل، يوجد العديد من المترجمات للغة الـ\textenglish{C}.
سنرى أن اختيار المترجم ليس أمرا معقّدا في حالتنا هذه.

ما الذي نحتاج إليه أيضا ؟ لن أتركك تخمّن كثيرا و سأعطيك القائمة :

\begin{itemize}
  \item \textbf{محرّر نصوص }
(\textenglish{Text Editor})
لكتابة الشفرة المصدرية الخاصّة بالبرنامح. نظريّا برنامج تحرير نصوص بسيط مثل
\textenglish{Notepad}
على
\textenglish{Windows}
أو
\textenglish{vi}
على
\textenglish{Unix}
يكفي، لكن من الأحسن استخدام محرّر نصوص ذكيّ يقوم بتلوين الشفرة المصدرية لكي يسهّل عليك العمل.
  \item \textbf{مترجم}
  لتحويل الشفرة المصدرية إلى ملف ثنائي.
  \item \textbf{المنقّح}
(\textenglish{Debugger})
لمساعدك على كشف الأخطاء في برنامجك. لسوء الحظ، لم نتمكّن بعد من ابتكار "المصحّح" الّذي يصحّح أخطائك لوحده. لكن، إن أحسنت استخدام المنقّح، يمكنك ببساطة إيجاد الأخطاء.
\end{itemize}

وجود مكتشف الأخطاء لا يعنى أن تتصرف بتهوّر و تسرع في كتابة برنامج مليء بالأخطاء، بل تريّث و كن هادئاً.

من الآن لدينا خياران :

\begin{itemize}
  \item إمّا أن نحصل على البرامج الثلاثة متفرّقة و هذه هي الطريقة الأكثر تعقيدا، و لكنّها تعمل. على
\textenglish{GNU/Linux}
تحديدا، عدد كبير من المبرمجين يفضّلون استخدام كلّ برنامج على حدة. لن أشرح هذه الطريقة هنا، بل سأتحدّث عن الطريقة الأسهل.
  \item أو أن تحصل على برنامج "ثلاثة في واحد" يتضمّن محرّر النصوص و المترجم و المنقّح. هذا النوع من البرامج يعرف باسم "بيئات التطوير المتكاملة"
(\textenglish{Integrated Development Environments})
و تسمّى اختصارا
\textenglish{IDE}.
\end{itemize}

يوجد العديد من بيئات التطوير. بداية قد تواجه صعوبة في اختيار البيئة الملائمة لك. الشيء الأكيد هو : أي بيئة مهما كانت ستحقق لك العمل المطلوب.

\subsection{اختيار البيئة الخاصة بك}

بدا لي أنه من الأفضل أن أريك بعضا من البيئات الشهيرة و المجانيّة في نفس الوقت. شخصيّا، أنا أستخدمها جميعا و أختار في كل يوم  واحدا منها.

\begin{itemize}
  \item أحد هذه البيئات الّتي أفضّلها هو
\textenglish{Code::Blocks}.
هو مجّاني و يعمل على أغلب أنظمة التشغيل. أنصح كلّ مبتدئ أن يختاره للبدء (و في ما بعد أيضا إذا شعرت أنّه يلائمك جيّدا !).

يعمل على أنظمة التشغيل
\textenglish{Windows}،
\textenglish{Mac OS}
و
\textenglish{GNU/Linux}.
  \item الأكثر شهرة على
\textenglish{Windows}
هو الّذي أنشأته
\textenglish{Microsoft}،
إنّه
\textenglish{Visual C++}.
هو برنامج مدفوع (و باهظ الثمن) لكن لحسن الحظّ توجد نسخة مجانية منه تسمّى
\textenglish{Visual Studio Express}
(أنا أستخدم النسخة القديمة
\textenglish{Visual C++ Express}
في هذا الكتاب). و هي ممتازة جدّا (بينها و بين النسخة المدفوعة فوارق طفيفة). إنه برنامج كامل و يملك منقّحا قويّا.

يعمل على
\textenglish{Windows}
فقط.
  \item على
\textenglish{Mac OS X}
يمكنك استخدام
\textenglish{XCode}
الّذي يفترض أن يكون متوفّرا على قرص تثبيت النظام. يناسب كثيرا مبرمجي
\textenglish{Mac}.

يعمل على
\textenglish{Mac OS X}
فقط.
\end{itemize}

\begin{information}
  ملاحظة لمستخدمي
  \textenglish{GNU/Linux} :
  يوجد العديد من البيئات لهذا النظام، و لكن المبرمجين المحترفين قد يفضّلون تجاوز البيئات و القيام بالترجمة "يدويّا"، و هو شيء أصعب قليلا. نحن سنبدأ باستخدام بيئات التطويرية. لذلك أنصحك بثبيت
  \textenglish{Code::Blocks}
  إن كنت على
  \textenglish{GNU/Linux}
  لكي تتمكن من متابعة شروحاتي.
\end{information}

\begin{question}
من هي البيئة الأفضل من بين كلّ بيئات التطوير هذه؟
\end{question}


كل واحدة من هذه البيئات تمكنك من البرمجة و متابعة بقيّة الدرس من دون أيّة مشاكل. بعضها كامل أكثر من ناحية المميزات، و أخرى سهلة الإستخدام أكثر، ولكن في كلّ الأحوال البرامج الّتي تنشؤها تكون ذاتها أيّا كانت البيئة التي اِخْتَرْتها. فهذا الخيار ليس بالأهمّية الّتي تعتقدها.

في هذا الكتاب سوف أستخدم
\textenglish{Code::Blocks}.
فإن أردت الحصول على نفس لقطات الشاشة خاصّتي، خصوصا لكي لا تضيع في البداية، أنصحك بِدَايَةً بتثبيت
\textenglish{Code::Blocks}.

  \chapter{برنامجك الأوّل}

لقد قمنا بتحضير كلّ شيء إلى حد الآن ويمكننا أن نبدأ قليلا من البرمجة. مع نهاية هذا الفصل ستكون قد نجحت في إنشاء أوّل برنامج لك.

لكي أصدقك القول، سيظهر البرنامج بالأبيض والأسود ولن يقوم بشيء سوى إلقاء التحيّة. يبدو عديم الفائدة، لكنّه برنامجك الأوّل وأؤكّد لك أنّك ستكون فخورا به.

\section{كونسول أو نافذة ؟}

لقد تحدثنا سابقا عن فكرة برامج الكونسول وبرامج النوافذ في الفصل السابق. البيئة التطويرية تطلب منا تحديد أي نوع من البرامج نريد أن ننشئها. ولقد قلنا إننا سننشئ برامج من نوع كونسول.

يوجد نوعان من البرامج، لا أكثر :

\begin{itemize}
  \item ،برامج بنوافذ
  \item برامج تعمل في الكونسول.
\end{itemize}

\subsection{البرامج الّتي تملك نوافذ}

هي البرامج التي نعرفها جميعا. هذا مثال على برنامج من نوع نافذة، مثل الرسام.

\begin{figure}[H]
	\centering
	\includegraphics[width=0.6\textwidth]{Chapter_I-3_Paint}
\end{figure}

أعتقد أنّك تحب إنشاء برامج كهذه، لكنّ هذا ليس في مقدورك حاليا. في الواقع، إنشاء برامج بنوافذ هو أمر ممكن بلغة \textenglish{C}، لكنّ بالنسبة لمبتدئ، هذا أمر معقّد جدّا. كبداية، يستحسن إنشاء برامج الكونسول.

\begin{question}
  لكن ماذا يعنى برنامج
\textenglish{Console}
؟
\end{question}

\subsection{البرامج الّتي تعمل في الكونسول}

برامج الكونسول هي أول ما ظهر من برامج. في ذلك الوقت، شاشات الحواسيب لم تكن سوى بالأبيض والأسود، ولم تكن فعّالة لكي تتمكّن من رسم النوافذ كما هو الحال مع حواسيبنا حاليّا.

مرّ الزمن بسرعة وزادت شعبية الويندوز نظراً لبساطته إلى أن نسي كثير من الناس ما هي الكونسول.

لديّ خبر جيّد لك !
\textbf{الكونسول لم تمت بعد} !
 في الواقع،
\textenglish{GNU/Linux}
 قد أعاد الكونسول إلى الحياة. هذه صورة لكونسول على
\textenglish{GNU/Linux}.

\begin{figure}[H]
	\centering
	\includegraphics[width=0.6\textwidth]{Chapter_I-3_Console}
\end{figure}

مرعب ! صحيح ؟ لكن على الأقل عرفت ما هي الكونسول، وهذه بعض الملاحظات :

\begin{itemize}
  \item اليوم، يمكننا عرض الألوان في الكونسول. ليس كلّ شيء بالأبيض والأسود كما تتخيّل.
  \item الكونسول هو الأسهل من ناحية البرمجة بالنسبة للمبتدئين.
  \item أداة عالية الإمكانيّات إذا عرفنا كيف نستخدمه.
\end{itemize}

كما قلت لك، إنشاء برامج كونسول أمر سهل جدّا وملائم للمبتدئين (وهذا عكس برامج النوافذ). ليكن في علمك أيضا أنّ الكونسول قد تطوّرت وبإمكانها عرض الألوان، ولا شيء يمنعك من إضافة صورة خلفيّة لها.

\begin{question}
  وفي الويندوز ألا توجد
\textenglish{Console}
 ؟
\end{question}

بلى، لكنّها مخفيّة لو صح القول. يمكنك فتحها بالذهاب إلى "إبدأ"
(\InlineCode{Start})
 ثمّ "ملحقات"
(\InlineCode{Accessories})
 ثمّ "موجه الأوامر"
(\InlineCode{Command prompt})
 أو بالذهاب إلى "إبدأ" ثمّ "تشغيل"
(\InlineCode{Run})
 واكتب فيها
\InlineCode{cmd}
 واضغط على "موافق".

\begin{figure}[H]
	\centering
	\includegraphics[width=0.6\textwidth]{Chapter_I-3_Console-Windows}
\end{figure}

إذا كنت تستخدم نظام ويندوز، فاعلم بأن أولى برامجك ستكون في نوافذ شبيهة بهذه. أنا لم أختر البداية هكذا لجعلك تشعر بالملل، بل لتعليمك الأساسيّات اللازمة لكي تتمكّن لاحقا من إنشاء النوافذ.

إذن فلتكن متيقّناً، بمجرّد أن تصل إلى المستوى اللازم لإنشاء النوافذ، سوف أعلّمك كيف تفعل ذلك.

\section{الحدّ الأدنى من الشفرة المصدرية}

من أجل أي برنامج، يجب كتابة قدر معيّن من الشفرة المصدرية. هذه الشفرة لا تقوم بشيء خاصّ لكنّها ضروريّة. هذه الشفرة التي سنكتشفها الآن ستكون أساس أغلب برامجك الّتي ستكتبها بلغة \textenglish{C}.

\subsection{أطلب من البيئة التطويرية الخاصة بك تزويدك بالحد الأدنى من الشفرة المصدرية}

لقد لاحظت أن طريقة إنشاء مشروع جديد تختلف من بيئة تطويرية إلى أخرى. إليك تذكيراً بسيطا : في برنامج
\textenglish{Code::Blocks}
 (الذي سنستخدمه في هذا الكتاب)، عليك التوجه نحو
\InlineCode{File}
 ثمّ
\InlineCode{New}
 ثمّ
\InlineCode{Project}
 ثم تختار
\InlineCode{Console Application}
 وبعدها اللغة
\textenglish{C}.
سيولّد لك الحد الأدنى من الشفرة المصدرية
 \textenglish{C}
 التي تحتاجها. ها هي :

\begin{Csource}
#include <stdio.h>
#include <stdlib.h>

int main()
{
    printf("Hello world!\n");
    return 0;
}

\end{Csource}

\begin{information}
لاحظ أنّه يوجد سطر فارغ في نهاية الشفرة. يفترض أن ينتهي كل ملف مكتوب بلغة
\textenglish{C}
هكذا. إن لم تفعل ذلك، فهذه ليست بمشكلة، لكن توقّع أن يعرض لك المترجم تحذيراً
(\textenglish{Warning}).
\end{information}

علماً أنّ السطر :
\begin{Csource}
int main()
\end{Csource}
\dots
بإمكانه أن يُكتب كالتالي :

\begin{Csource}
int main(int argc, char *argv[])
\end{Csource}

كلتا العبارتين تحملان نفس المعنى لكن الثانية، الأكثر تعقيدا، هي الأكثر شيوعا، لذلك فإنّنا سنستخدمها في الفصول القادمة.\\
إستخدامنا للشكل الأوّل أو الثاني لا يغيّر شيئا بالنسبة لنا. لذلك لا داعي لإضاعة الوقت هنا، خصوصاً أنّك لا تملك المستوى اللازم لفهم ما تعنيه.

إذا كنت تستخدم بيئة تطويرية أخرى فقم بنسخ هذه الشفرة المصدرية وألصقها في الملف \InlineCode{main.c} ليكون لديكم نفس الشفرة.

أخيرا، قم بحفظ عملك في المشروع. أعلم أننا لم نقم بشيء حتّى الآن لكن من الجيّد التعوّد على الحفظ في كلّ مرّة.

\subsection{تحليل أسطر الشفرة المصدرية السابقة}
قد تبدو لك الشفرة المصدرية السابقة أنّها كاللغة الصينيّة، أنا أتخيّل ذلك ! في الواقع هي تسمح بإنشاء برنامج كونسول يعرض نصّا على الشاشة. يجب تعلّم كيفيّة قراءة كلّ هذا.

فلنبدأ بأوّل سطرين :
\begin{Csource}
#include <stdio.h>
#include <stdlib.h>
\end{Csource}

هذان السطران يبدآن بعلامة
\InlineCode{\#}.
وهي أسطر خاصّة تُعرف باسم
\textbf{توجيهات المعالج القبلي}
(\textenglish{Preprocessor directives}). اسم معقّد، أليس كذلك ؟ هذه الأسطر تتمّ قراءتها من طرف البرنامج المسمّى بالمعالج القبلي، وهو برنامج يتمّ تشغيله في بداية الترجمة.

ما رأيناه سابقا كان مخطّطا بسيطا لعمليّة الترجمة. لكنّ في الواقع، هناك الكثير من المراحل التي تحدث في هذه العمليّة. سنقوم بتفصيل هذا لاحقا. حاليّا عليك فقط تذكّر وضع هذين السطرين أعلى كلّ ملفّاتك.

\begin{question}
  حسنا لكن ماذا يعنيه هذان السطران ؟ أريد أن أعرف !
\end{question}

كلمة
 \InlineCode{include}
 بالإنجليزيّة تعني "تضمين". هذان السطران يقومان بتضمين ملفّات في المشروع، أي إضافة هذه الملفّات من أجل عمليّة الترجمة. هناك سطران وبالتالي هناك ملفان يتمّ تضمينهما في المشروع وهما بالترتيب :
\InlineCode{stdio.h}
 و
\InlineCode{stdlib.h}.
هذان الملفّان موجودان بالفعل على حاسوبك وهما ملفّان مصدريّان جاهزان، سوف تعرف مستقبلا أنّنا نسميها
\textbf{مكتبات}
(\textenglish{Libraries}).
 هذه الملفّات تحتوي الشفرة المصدرية اللازمة لعرض نصّ على الشاشة.

 بدون هذين الملفّين، كتابة نصّ على الشاشة سيكون أمرا مستحيلاً. فالحاسوب لا يعرف فعل أي شيء مبدئيا.

 باختصار، السطران الأول والثاني يقومان بتضمين المكتبات التي ستساعدنا في إظهار نصّ على الشاشة بكلّ سهولة.

 نمر للتالي، باقي الأسطر :
 
\begin{Csource}
int main()
{
    printf("Hello world!\n");
    return 0;
}
\end{Csource}

ما تراه هنا هو ما نسميه بـ\textbf{التابع}
أو
\textbf{الدالّة}
(\textenglish{Function}).
 البرنامج في لغة
\textenglish{C}
 يتكوّن من مجموعة دوال. حاليّا برنامجنا لا يحوي سوى دالّة واحدة.

الدالّة تمكّننا من تجميع مجموعة من الأوامر. الغرض من تجميع الأوامر هو جعلها تقوم بوظيفة ما. مثلا يمكننا إنشاء دالّة باسم
 \InlineCode{open\_file}
 وجعلها تحتوي التعليمات التي تشرح للحاسوب كيفيّة فتح ملف.

 دون الدخول في تفاصيل إنشاء الدالّة (الوقت مبكّر، سوف نتحدّث عن الدوال في وقت لاحق) لنحلّل رغم ذلك أجزائه الكبيرة. السطر الأوّل يحتوي اسم الدالّة، إنّه الكلمة الثانية.\\
 أجل، اسم دالّتنا هو
\InlineCode{main}
والذي يعني
"الرئيسية"
. وتشغيل البرنامج دائما يبدأ من الدالة
\InlineCode{main}.

للدالّة بداية ونهاية، وهي محدودة بالحاضنتين
\InlineCode{\{}
و
\InlineCode{\}}.
محتوى الدالّة موجود بين هاتين الحاضنتين. إن كنت قد تابعت جيداً فقد عرفت أنّ الدالّة مشكّلة من سطرين :

\begin{Csource}
printf("Hello world!\n");
return 0;
\end{Csource}

هاته الأسطر في الداخل نسميها
\textbf{التعليمات}
(\textenglish{Instructions})
 (هذه إحدى المصطلحات الّتي يجب عليك حفظها). كلّ تعليمة تمثّل أمراً بالنسبة للحاسوب. فكلّ واحدة منها تطلب منه فعل شيء محدّد.

 كما قلت لك، بتجميع ذكيّ للتعليمات في الدالّة يمكننا إنشاء أجزاء برنامج جاهزة للاستخدام. باستخدام التعليمات المناسبة يمكننا إنشاء دالّة
 \InlineCode{open\_file}
 كما شرحت لك قبل قليل، و أيضا دالّة
\InlineCode{move\_character}
 في لعبة فيديو، على سبيل المثال.

 البرنامج في الواقع ما هو إلّا تتابع لتعليمات : إفعل هذا و إفعل ذاك. أنت تعطي أوامر للحاسوب و هو يقوم بتنفيذها.

 \begin{critical}
هامّ جدّا : لا بدّ أن تنتهي كلّ تعليمة بفاصلة منقوطة
"\InlineCode{;}"
. بهذا يمكن التفريق بين ما إذا كانت هذه تعليمة أم لا. إذا نسيت وضع فاصلة منقوطة نهاية تعليمة ما، فلن تتمّ ترجمة برنامجك.
 \end{critical}

 السطر الأول :
 \InlineCode{printf("Hello world!\\n");}
 يطلب إظهار الرسالة
 "\textenglish{Hello world!}"
  على الشاشة. عندما يصل برنامجك إلى هذا السطر، فسوف يقوم بعرض هذه الرسالة ثمّ المرور إلى التعليمة التالية.

  التعليمة التالية هي
\InlineCode{return 0;}
 و هي تخبرنا أنّ الدالّة
\InlineCode{main}
 قد انتهت و تطلب منه إعادة 0.

 \begin{question}
   لماذا يقوم برنامجي بإعادة العدد 0 ؟
 \end{question}

 في الواقع، كلّ برنامج عندما ينتهي يُرجع قيمة معينة. على سبيل المثال، ليقول أنّ كلّ شيء سار على ما يرام. عمليّا، 0 يعني  أنّ كلّ شيء سار على ما يرام، و كلّ قيمة أخرى تدلّ على حدوث خطأ. في أغلب الأحيان هذه القيمة لا تُستخدم ، لكن يجب رغم ذلك استعمالها.\\
 كان يمكن أن يعمل برنامجك بدون
 \InlineCode{return 0}
، لكن يمكننا القول أن وضعها يعتبر أمراً أكثر نظافة و أكثر جدّية.

إلى هنا نكون قد فصّلنا قليلا في عمل هذه الشفرة المصدرية.

طبعا، نحن لم ندرس كلّ شيء بعمق، و قد تكون لديك بعض الأسئلة عالقة في ذهنك. كن على يقين بأنك ستجد لها أجوبة شيئا فشيئا مع تقدّمنا في الكتاب. لا يمكنني أن أطلعك على كلّ شيء من البداية، لأنّ هناك كثيراً من الأشياء لاستيعابها.

إليك ما يلي : بما أنني في حال جيّدة، سأقوم بوضع مخطّط يضمّ المصطلحات الّتي تعلّمناها في هذا الفصل.

\begin{figure}[H]
	\centering
	\includegraphics[width=0.8\textwidth]{Chapter_I-3_HelloWorld}
\end{figure}

\subsection{لنجرّب برنامجنا}

كلّ ما سنقوم به الآن هو ترجمة المشروع ثمّ تشغيله (اضغط على
\InlineCode{Build \& Run}
 إذا كنت على
\textenglish{Code::Blocks}).
سيطلب منك حفظ مشروعك إذا لم تقم بذلك من قبل.

\begin{critical}
  إن لم تنجح الترجمة و ظهر لك خطأ مثل :\\
\InlineCode{"My-program - Release" uses an invalid compiler. Skipping...}\\\InlineCode{Nothing to be done...}
فهذا يعني أنّك نزلت نسخة
\textenglish{Code::Blocks}
 دون
\InlineCode{mingw}
 (المترجم)، عد و نزّل النسخة التي تحتوي على
\InlineCode{mingw}.
\end{critical}

بعد بُرهة، يظهر برنامجك كما في الصورة :

\begin{figure}[H]
	\centering
	\includegraphics[width=0.8\textwidth]{Chapter_I-3_HelloWorld-run}
\end{figure}

البرنامج يُظهر
"\textenglish{Hello world!}"
 (في السطر الأوّل).\\
الأسطر الّتي أسفله تمّ توليدها من طرف
\textenglish{Code::Blocks}
 وتدلّ على أنّ البرنامج قد تمّ تشغيله بنجاح كما أنها تعطي الوقت الذي استغرقه البرنامج في التشغيل.

 سيطلب منك الضغط على إحدى المفاتيح لإغلاق النافذة. أعلم أن الأمر لم يكن ممتعا جدّا. لكنه برنامجك الأوّل، وهذه لحظة ستتذكرها طيلة حياتك ! ألا تعتقد ذلك ؟

\section{كتابة رسالة على الشاشة}

من الآن سنقوم بإدخال التعديلات على الشفرة المصدرية السابقة. مهمّتك، إن قبلتها : عرض رسالة
"\textenglish{Bonjour}"
 على الشاشة.

\begin{question}
  كيف يمكنني اختيار النص الّذي سيظهر على الشاشة ؟
\end{question}

الأمر بسيط جدا، إذا بدأت من الشفرة التي رأيناها سابقاً، فسيكون عليك استبدال
"\textenglish{Hello world!}"
 بـ"\textenglish{Bonjour}"
 في السطر الذي يستدعي
\InlineCode{printf}.

كما قلت من قبل،
\InlineCode{printf}
 هي
\textbf{تعليمة}
 وهي تعطي أمراً للحاسوب : "قم بعرض هذه الرسالة على الشاشة".\\
يجب أن تعرف أيضا أن
\InlineCode{printf}
 هي دالّة كُتِبَت من قبل من طرف مبرمجين قبلك.

\begin{question}
   أين توجد هذه الدالّة ؟ أنا لا أرى سوى الدالّة \InlineCode{main} !
\end{question}

هل تذكر هذين السطرين ؟

\begin{Csource}
#include <stdio.h>
#include <stdlib.h>
\end{Csource}

قلت لك من قبل أنهما يمكنان البرنامج من إضافة مكتبات. المكتبات في الحقيقة هي ملفّات تحوي أطنانا من الدوال جاهزة للإستخدام. هذه الملفات
(\InlineCode{stdio.h} و \InlineCode{stdlib.h})
 تحوي أغلب الدوال الأساسية التي قد نحتاجها في برنامج ما.
\InlineCode{stdio.h}
 بحد ذاته يحوي دوال تمكّن من عرض أشياء على الشاشة (مثل
 \InlineCode{printf})
 و أيضا الطلب من المستخدم إدخال شيء ما (هذه دوال سنتعرّف عليها لاحقا).

\subsection{لنقل مرحبا للسيّد}

في دالّتنا
\InlineCode{main}
نستدعي الدالّة
 \InlineCode{printf}.
 أي أن لدينا دالّة تستدعي أخرى (هنا
\InlineCode{main}
تستدعي
\InlineCode{printf}).
سترى أن هذا ما يحدث دائما في لغة
\textenglish{C}
: دالّة تحتوي تعليمات تستدعي دوال أخرى، وهكذا.

إذن، لاستدعاء دالّة يكفي كتابة اسمها متبوعا بقوسين، ثم فاصلة منقوطة.

\begin{Csource}
printf();
\end{Csource}

هذا جيد، لكنه غير كاف. يجب أن نُعلم البرنامج بما يجب أن يكتبه في الشاشة. لفعل هذا يجب أن نعطي
\InlineCode{printf}
النص المطلوب عرضه. لفعل هذا نقوم بوضع النص داخل علامات الإقتباس المزدوجة بين القوسين.\\
في حالتنا هذه سنكتب تماما :

\begin{Csource}
printf("Bonjour");
\end{Csource}

آمل ألا تكون قد نسيت رمز الفاصلة المنقوطة في النهاية، وأذكّرك أنّها مهمّة جدا لأنّها تدلّ على نهاية التعليمة.\\
هذه هي الشفرة المصدرية التي يجب أن تحصل عليها :

\begin{Csource}
#include <stdio.h>
#include <stdlib.h>

int main()
{
    printf("Bonjour");
    return 0;
}
\end{Csource}

لدينا إذن تعليمتان تطلبان من الحاسوب القيام بهذين الأمرين بهذا الترتيب :
\begin{enumerate}
  \item عرض
"\textenglish{Bonjour}"
على الشاشة.
  \item نهاية الدالّة
\InlineCode{main}
، إعادة 0. البرنامج يتوقّف.
\end{enumerate}

هذا ما يظهر على شاشتك :

\begin{figure}[H]
	\centering
	\includegraphics[width=0.8\textwidth]{Chapter_I-3_Good-Morning}
\end{figure}

كما ترى، السطر الذي يحتوي الرسالة يكون ملتصقاً قليلا بباقي النص، على خلاف ما رأيناه سابقا.\\
أحد الحلول الممكنة هو إضافة رمز للعودة إلى السطر بعد
 "\textenglish{Bonjour}"
 (كما لو أنّنا ضغطنا على المفتاح
\InlineCode{Enter}).

ولكن ضغط المفتاح
\InlineCode{Enter}
 في الشفرة المصدرية لن يعمل كما تتوقع، لهذا يجب استخدام المحارف الخاصّة
(\textenglish{Special characters}).

\subsection{المحارف الخاصّة}

المحارف أو الرموز الخاصّة هي محارف تمكّن من تعريف عودة إلى السطر، جدولة، إلخ.\\
من السهل التعرّف عليها، فهي مكوّنة من محرفين. الأوّل هو الشَرْطَةُ المائلة الخلفية 
(\textbackslash) (\textenglish{Backslash})
والثاني يكون رقما أو حرفا. إليك محرفين خاصّين قد تحتاجهما كثيرا :

\begin{itemize}
  \item \InlineCode{\textbackslash n} :
 العودة إلى السطر.
 \item \InlineCode{\textbackslash t} :
 الجدولة (فراغ كبير في نفس السطر).
\end{itemize}

في حالتنا هذه، يكفي أن نكتب
\InlineCode{\textbackslash n}
 لإنشاء العودة إلى السطر. إذن، إذا أردنا أن نضع عودة إلى السطر بعد
\textenglish{Bonjour}
، فيكفي أن نكتب :

\begin{Csource}
printf("Bonjour\n");
\end{Csource}

وسيفهم حاسوبك أنّ عليه كتابة
"\textenglish{Bonjour}"
 ويعود إلى السطر.

\begin{figure}[H]
	\centering
	\includegraphics[width=0.8\textwidth]{Chapter_I-3_Good-Morning-backslash-n}
\end{figure}

\begin{information}
  يمكنك الكتابة بعد
\InlineCode{\textbackslash n}
بدون أيّة مشكلة. كلّ ما تكتبه بعد
\InlineCode{\textbackslash n}
 سيوضع في السطر الجديد. يمكنك إذن التدرّب على كتابة :
\InlineCode{printf("Good morning\textbackslash nGood bye\textbackslash n");}\\
و سيتمّ عرض
"\textenglish{Good morning}"
على السطر الأوّل و
"\textenglish{Good bye}"
على السطر الثاني.
\end{information}

\subsection{متلازمة \textenglish{Gérard}}

\begin{question}
  مرحبا، اسمي
\textenglish{Gérard}
و قد حاولت تعديل برنامجك ليقول
"\textenglish{Bonjour Gérard}"،
و لكنّي ألاحظ أنّ حرف
\textenglish{é}
 لا يظهر بشكل جيّد
\dots
 مالّذي عليّ فعله ؟
\end{question}

أوّلا، مرحبا بك
\textenglish{Gérard}
. هذا سؤال جيّد. لكن لديّ خبر سيّء لك. الكونسول الخاصة بـ\textenglish{Windows}
لا تمكّن من عرض الحروف الّتي تحوي علامات النطق الصوتي مثل
\textenglish{é}
، خلافا لكونسول
\textenglish{GNU/Linux}
التي تفعل. لديّ حلّان لهذه المشكلة :

\begin{itemize}
  \item \textbf{استخدم
\textenglish{GNU/Linux}}
. هذا حلّ جذريّ بعض الشيء. أحتاج إلى درس كامل لأعلّمك كيف تعمل على
\textenglish{GNU/Linux}
. إذا لم يكن لديك المستوى، إنس هذا الخيار حاليّا.
  \item \textbf{لا تستخدم الحروف الّتي تحوي علامات النطق الصوتي}.
للأسف إنّه الحل الّذي قد يكون عليك اختياره. الكونسول الخاصة بـ\textenglish{Windows}
لها عيوبها. يجب عليك التعوّد على عدم كتابة مثل هذه الحروف. لكن مستقبلا قد تنشئ برامج بنوافذ ولن تعاني من هذا المشكل. لذلك أنصحك بالصبر على هذه المشكلة حاليّا، فبرامجك المستقبلية "الاحترافية" لن يكون فيها هذا المشكل.
\end{itemize}

لكيلا تنزعج، يمكنك الكتابة دون استخدام الحروف التي تملك علامات النطق الصوتي :

\begin{Csource}
printf("Bonjour Gerard\n");
\end{Csource}

نشكر صديقنا
\textenglish{Gérard}
لتنبيهنا على هذه المشكلة !

\section{التعليقات، مهمّة جدا !}

قبل ختم هذا الفصل الأوّل "الحقيقي" في البرمجة، يجب أن أعرّفك على
\textbf{التعليقات}
(\textenglish{Comments})
. أيّا كانت لغة البرمجة الّتي تستخدمها، ستكون لديك القدرة على إضافة التعليقات للشفرة المصدرية الخاصة بك.

ولكن ما الذي يعنيه "التعليق"؟\\
هذا يعني إمكانية وضع نصّ في وسط برنامجك لشرح دوره، مثلاً : ما الذي يفعله هذا السطر، إلخ. هذا بالفعل أمر ضروريّ، لأنّه حتّى لو كنت عبقرياً في البرمجة، ستكون بحاجة إلى وضع ملاحظات هنا وهناك. هذا يمكنك من :

\begin{itemize}
  \item العثور على ما تبحث عنه بسهولة في الشفرة المصدرية عندما تعود إليه بعد مدّة. من الطبيعيّ أن ننسى كيف تعمل البرامج الّتي كتبناها بعد مدّة. إن توقّفت عن البرمجة لأيّام ثمّ عدت فستكون بحاجة إلى التعليقات لإيجاد ما تريد في شفرة كبيرة جدّا.
  \item إذا أعطيت مشروعك لأحد غيرك (وهو لا يعرف شيئا عن الشفرة المصدرية الخاصة بك)، فالتعليقات تمكّنه من التآلف مع مشروعك بسرعة.
  \item وأخيرا، ستسمح لي بإضافة شروحات وملاحظات حول الشفرة المصدرية في هذه الدروس. وهذا سيفيدك في فهم ما الذي يعنيه كلّ سطر.
\end{itemize}

توجد طريقتان لإضافة تعليق. وهذا يعتمد على طول التعليق المراد إدراجه :

\begin{itemize}
  \item إذا كان تعليقك
\textbf{قصيرا}
: فيمكن كتابته على سطر واحد، ولا يحتوي سوى كلمات قليلة. في هذه الحالة، عليك كتابة شرطتين مائلتين
(\InlineCode{//})
متبوعين بتعليقك. على سبيل المثال :

\begin{Csource}
// This is a comment.
\end{Csource}

بإمكانك إضافة تعليق وحده على السطر، أو على يمين تعليمة معينة. وهذا أمر مهمّ جدّا، لأنّ بهذه الطريقة يمكننا تحديد ما الذي يعنيه السطر الّذي كُتب بجانبه. مثال :

\begin{Csource}
printf("Bonjour"); // This instruction displays 'Bonjour' on the screen
\end{Csource}

  \item  إذا كان تعليقك
\textbf{طويلا}:
لديك الكثير لتقوله، تريد كتابة الكثير من الجمل على كثير من الأسطر. في هذه الحالة، يجب عليك كتابة شفرة تشير إلى "بداية التعليق" وأخرى تشير إلى "نهاية التعليق":

  \begin{itemize}
    \item لبدء التعليق : أكتب شرطة مائلة متبوعة بنجمة 
    (\InlineCode{/*}).
    \item لإنهاء التعليق : أكتب نجمة متبوعة بشرطة مائلة 
    (\InlineCode{*/}).
  \end{itemize}

  يمكنك كتابة هذا على سبيل المثال :
  
  \begin{Csource}
/* This is
a comment
written on several lines */
  \end{Csource}
\end{itemize}
فلنعد إلى الشفرة المصدرية التي تُظهر
"\textenglish{Bonjour}"
على الشاشة ونضيف إليها بعض التعليقات للتدرّب :

\begin{Csource}
/*
Below, the directives of preprocessor.
These lines allow you to add files to your program,
files that we call libraries. Thanks to these libraries, we are ready to use functions for display.
for example, a message on screen.
*/

#include <stdio.h>
#include <stdlib.h>

/*
Following, you have the principal function of the program, called main.
All programs start with this function.
Here, all what does my function is displaying "Bonjour" on the screen.
*/

int main()
{
  printf("Bonjour"); // This instruction displays 'Bonjour' on the screen
  return 0;          // The program returns 0 then it stops.
}
\end{Csource}

هذا هو برنامجنا مع إضافة بعض التعليقات، نعم هو يبدو أكبر نوعا ما، لكنّه في الحقيقة مكافئ للبرنامج السابق. عند الترجمة، كلّ التعليقات يتمّ تجاهلها من طرف المترجم. هذه التعليقات لا تظهر في البرنامج النهائي، فهي تصلح فقط للمبرمجين.

عادة لا نقوم بوضع تعليق لكلّ سطر. لقد قلت وأكرر أنّه من المهم وضع التعليقات في الشفرة المصدرية، لكن يجب عليك معرفة القدر اللازم من التعليقات الواجب وضعه، وضع تعليق في كلّ سطر قد لا يفيد في شيء، بل يضيّع الوقت فقط. مثلا، أنت تعرف أن وظيفة
\InlineCode{printf}
هي عرض نصّ على الشاشة، فلا حاجة لوضع تعليق يشرح ذلك في كلّ مرّة.

من الأحسن التعليق عن عدد من الأسطر دفعة واحدة. هذا يفيد في ذكر وظيفة مجموعة من التعليمات المتتابعة. فيما بعد إن أراد المبرمج إضافة مزيد من التفاصيل في تعليماته، فسيكون بمستوى ذكاء يسمح له بفعل ذلك.

\textbf{تذكر إذن}:
يجب أن تكون التعليقات لإرشاد المبرمج في شفرته المصدرية. حاول التعليق عن مجموعة من الأسطر دفعة واحدة بدل التعليق عن كلّ سطر على حدة.

وإليك هذه المقولة من
\textenglish{IBM} :

\begin{center}
  \itshape\Large
  'إذا قرأت التعليقات الموجودة في برنامج و لم تفهم مبدأ عمله، قم برميه !'
\end{center}

\section*{ملخّص}

\begin{itemize}
  \item البرامج يمكنها التفاعل مع المستخدم عن طريق الكونسول أو عن طريق النافذة.
  \item من السهل على المبرمج في برامجه الأولى استخدام
\textbf{الكونسول}،
رغم أنّ هذه قد تكون غير محبوبة لدى المبتدئ، فهذا لا يمنع من استخدام النوافذ في الجزء الثالث من هذا الكتاب.
  \item البرنامج يتكوّن من
\textbf{تعليمات}
 تنتهي دائما بفاصلة منقوطة.
  \item الدالة
\InlineCode{main}
 (التي تعني الرئيسيّة) هي الدالة الّتي يبدأ بها تنفيذ البرنامج. إنّها الدالة الوحيدة الإجبارية في البرنامج، لا يمكن لأي برنامج أن يُترجم بدونها.
 \item \InlineCode{printf}
 هي دالة تمكننا من عرض رسالة على الشاشة.
 \item \InlineCode{printf}
موجودة في
\textbf{مكتبة}
 تحتوي على كثير من الدوال الأخرى الجاهزة للاستخدام.
\end{itemize}

  \chapter{عالم المتغيّرات}

تعلّمت كيفية إظهار نصّ على الشاشة. جيد، لكنّ هذا ليس شيئا مهماً. هذا لأنك لا تعرف بعد ما يدعى بـ
\underline{المتغيّرات}
(\textenglish{Variables})
في البرمجة.

فائدة هذه المتغيرات هي تمكين الحاسوب من حفظ أعداد في الذاكرة. سنبدأ ببعض الشرح حول ذاكرة الحاسوب وكيفيّة عملها. قد يبدو هذا بسيطا جدّا للبعض، لكنّي أفترض أنّك لا تعرف شيئا عن ذاكرة الحاسوب.

\section{أمر متعلق بالذاكرة}
ما سأعلمك في هذا الدرس هو أمر له علاقة مباشرة بذاكرة حاسوبك.

كل إنسان حيّ له ذاكرة. الأمر عينه بالنسبة للحاسوب، لكن الحاسوب له أنواع عديدة من الذاكرة.

\begin{question}
  لم يملك الحاسوب أنواع عديدة من الذاكرة، واحدة يمكنها أن تكفي، أليس الأمر كذلك؟
\end{question}
كلّا: المشكلة أننا نحتاج ذاكرة سريعة (لاسترجاع المعلومات بسرعة) وفي نفس الوقت كبيرة (لحفظ بيانات كثيرة) قد تضحك إن أخبرتك أننا حتى اليوم لم نتمكن من صنع ذاكرة بهذه المواصفات. أو بالأحرى الذاكرة السريعة باهظة الثمن لذلك لا يتم إنتاج الكثير منها.

لذلك نجد في الحواسيب الحديثة ذاكرة سريعة جدا لكنها ليس ذات سعة كبيرة، وأخرى ذات سعة كبيرة جدّا لكنها غير سريعة.

\subsection{الأنواع المختلفة من الذاكرة}
كي أوضح لك الصورة أكثر، إليك أنواع الذاكرة الموجودة في الحاسوب، من الأسرع إلى الأبطأ:
\begin{enumerate}
  \item السجلات (
\textenglish{Registers}
): ذاكرة سريعة جدّا، موجودة داخل المعالج.
  \item ذاكرة التخبئة (
\textenglish{Cache memory}
): تمثل همزة وصل بين السجلات والذاكرة الحية.
  \item ذاكرة الوصول العشوائي (
\textenglish{Random access memory}
): وهي الذاكرة التي نستخدمها كثيرا، وتدعى اختصارا
\textenglish{RAM}.
  \item القرص الصلب (
\textenglish{Hard disk}
): والذي تعرفه بالطبع، نستعمله لحفظ الملفات.
\end{enumerate}
كما قلت لك، لقد رتبتها من الأسرع (السجلات) إلى الأبطأ (القرص الصلب)، وإن كنت قد تابعت جيدا فقد فهمت أن الذاكرة الأصغر هي الأسرع والأبطأ هي الأكبر.\\
السجلات لا تسع إلا لحمل بضعة أعداد أما القرص الصلب فيمكنه تخزين ملفات ضخمة.

\begin{information}
   عندما أقول ذاكرة بطيئة فهذا بالنسبة لحاسوبك، ففي نظر الحاسوب استغراق 8 ميلي ثانية للوصول إلى القرص الصلب يعتبر زمنا طويلا جدّا!
\end{information}

ما الذي يجب أن أتذكره من كل هذا؟\\
أردت أن أخبرك أننا في الدروس القادمة سوف نستخدم ذاكرة الوصول العشوائي كثيرا. سنتعلم أيضا كيفية القراءة والكتابة في الملفات على القرص الصلب (ليس الآن، لا يزال الوقت مبكّرا على هذا). أمّا بخصوص السجلّات وذاكرة التخبئة فلن نتعامل معهما مطلقا، فالحاسوب هو من سيهتم بأمرهما.

\begin{information}
  في لغات البرمجة منخفضة المستوى، كلغة التجميع (
\textenglish{Assembly language}
) نتعامل مباشرة مع السجلّات، لقد درستها، ويمكنني أن أقول لك أن القيام بعملية ضرب بسيطة يتطلب مجهودا! لحسن الحظ ففي لغة
\textenglish{C}
 (وفي أغلب اللغات الأخرى) الأمر أسهل من ذلك بكثير.
\end{information}

يجب إضافة شيء مهمّ آخر: القرص الصلب هو الوحيد الذي يمكنه حفظ المعلومات بشكل دائم.
\textbf{كل أنواع الذاكرات الأخرى مؤقتة، فبمجرد إطفاء الحاسوب تفقد كل محتواها}!

لحسن الحظ فعند إعادة تشغيل الحاسوب يقوم القرص الصلب بتذكيرها بمحتواها.

\subsection{صورة لذاكرة الوصول العشوائي}
نظرا لأننا سنستعمل ذاكرة الوصول العشوائي خلال لحظات، فمن الأفضل أن أريها لكم (مؤطر بالأحمر):
\Picture{Chapter_I-4_Computer}
لا أطلب منك معرفة كيفية عملها، لكن أردت فقط أن أريك مكانها داخل جهازك. وهذه صورة مقربة لإحدى أشرطتها:
\Picture{Chapter_I-4_RAM}
وهي تدعى اختصارا
\textbf{\textenglish{RAM}}
، لذلك لا تحتر إن سميتها هكذا لاحقا. بالنسبة للذاكرات الأخرى (السجلات والتخبئة) فهي صغيرة لدرجة أنه لا يمكن رؤيتها بالعين المجرّدة.

\subsection{مخطط ذاكرة الوصول العشوائي}
عرض المزيد من الصور لن يفيدك كثيرا، لكن يجب عليك فهم كيف تعمل من الداخل، لذلك سأقدم لك هذا المخطط البسيط الذي يمثل هندسة ذاكرة الوصول العشوائي:
\Picture{Chapter_I-4_RAM-Schema}

كما ترى، يمكننا أن نميز عمودين:
\begin{itemize}
  \item هناك
\textbf{العناوين}
: هي أعداد تسمح للحاسوب بتحديد موضع القيم في الـ
\textenglish{RAM}
. نبدأ بالعنوان 0 وننتهي بالعنوان 3,448,765,900,126 وبعض الأجزاء. لا أعلم بالضبط كم عدد العناوين الموجودة في الـ
\textenglish{RAM}
، لكني أعرف أنها كثيرة جدا. إضافة إلى ذلك، هذا أمر يتعلق بكمية الذاكرة الموجودة في جهازك، فكلما زادت الذاكرة زادت معها العناوين وصار بإمكاننا تخزين معلومات أكثر.
  \item عند كل عنوان يمكننا تخزين
\textbf{قيمة}
(عدد). حاسوبك يقوم بتخزين هذه الأعداد في ذاكرة الوصول العشوائي لكي يتمكن من تذكرها. ولا يمكننا تخزين سوى عدد واحد عند كل عنوان.
\end{itemize}

لا يمكن للذاكرة الحية تخزين شيء سوى الأعداد.

\begin{question}
  لكن كيف يمكننا تخزين الكلمات؟
\end{question}

سؤال جيد. في الواقع حتى الحروف ليست سوى أعداد في نظر الحاسوب! الجملة هي مجرد تتابع لأعداد.\\
يوجد جدول يوافق بين الأعداد والحروف، جدول يقول مثلا بأن العدد 67 يوافق الحرف
\textenglish{Y}
. لن أدخل في التفاصيل أكثر، ستكون لنا فرصة للرجوع إلى هذا لاحقا.

فلنعد إلى مخططنا، الأمور بسيطة جدا: إذا أراد الحاسوب تذكر العدد 5 (الذي قد يمثل عدد الأرواح المتبقية لشخصية في لعبة) فسوف يضعه في مكان ما في الذاكرة أين يتوفر مكان شاغر ويحفظ العنوان الموافق (مثلا 3,062,199,902). لاحقا، عندما يريد معرفة هذا العدد فسيذهب إلى خانة الذاكرة التي تحمل العنوان رقم 3,062,199,902 وسيجد القيمة 5.

هذه آلية عمل الذاكرة بشكل عام. قد يكون الأمر لا زال غامضا في ذهنك حاليا (ما فائدة تخزين عدد إن كان علينا تذكر عنوانه بدلا من ذلك؟) لكن كل شيء سيتضح مع بقية الدروس، أنا أعدك!

\section{التصريح عن متغير}
صدّقني هذه المقدّمة القصيرة عن الذاكرة ستكون مهمّة أكثر مما تعتقد. الآن يمكننا العودة إلى البرمجة.

إذن، ما هو
\underline{المتغير}
(\textenglish{Variable}) ؟\\
إنه معلومة صغيرة نخزنها مؤقتا في الذاكرة الحية. ببساطة يمكننا القول إن المتغير هو قيمة يمكن أن تتغير أثناء اشتغال البرنامج. مثلا عددنا 5 الذي ذكرناه سابقا يمكن أن يتناقص بمرور الزمن. إذا وصل إلى العدد 0 فسنعرف أن اللاعب قد خسر.

في برامجنا سيكون هناك الكثير من المتغيرات. ستراها في كلّ مكان.

في لغة السي، المتغير يتميز بشيئين:
\begin{itemize}
  \item \underline{قيمة}
: هو العدد الذي يحويه، 5 مثلا.
  \item \underline{اسم}
: وهو الذي يمكننا من معرفة المتغيّر. في البرمجة لن يكون علينا تذكّر عناوين الذاكرة. بدلا من ذلك علينا فقط استخدام أسماء المتغيرات. المترجم هو من سيقوم بتحويل الأسماء إلى عناوين.
\end{itemize}

\subsection{إعطاء اسم للمتغير}
في لغة البرمجة
\textenglish{C}
كل متغير يجب أن يملك اسما خاصا به. ومن أجل متغيرنا الذي يحوي عدد الأرواح المتبقية للاعب يمكننا أن نسميه
"\textenglish{Number of lives}"
أو شيء من هذا القبيل.

للأسف توجد بعض الشروط، لا يمكنك تسمية المتغير كيفما شئت:
\begin{itemize}
  \item لا يجب أن يحتوي الاسم سوى على الحروف الصغيرة والكبيرة والأرقام
(\InlineCode{abcABC012}).
  \item يجب أن يبدأ الاسم بحرف.
  \item المسافات ممنوعة. بدلا من ذلك يمكننا استخدام الحرف المعروف باسم
\textenglish{underscore}
 (\InlineCode{\_}).
إنه الحرف الخاص الوحيد غير الحروف والأرقام الذي يمكن استعماله في اسم متغير.
  \item لا يمكنك استخدام حروف غير الحروف الإنجليزية.
\end{itemize}

وأخيرا يجب أن تعرف أن لغة
\textenglish{C}
 تفرّق بين الحروف الصغيرة والكبيرة. ولثقافتك، نقول إن
\textenglish{C}
 حساسة لحالة الأحرف
(\textenglish{Case sensitive}).
كمثال، الأسماء
\InlineCode{width}
 أو
\InlineCode{WIDTH}
 أو
\InlineCode{WiDth}
تعتبر أسماء متغيرات مختلفة، حتى لو كانت تعني لنا الأمر نفسه.

هذه أمثلة عن أسماء متغيرات صالحة:
\InlineCode{numberOfLives}،
\InlineCode{name}،
\InlineCode{surname}،
\InlineCode{phone\_number}،
\InlineCode{phoneNumber}.

لكل مبرمج طريقة خاصة في كتابة أسماء المتغيرات. خلال هذا الدرس سأريك طريقتي:
\begin{itemize}
  \item أبدأ دائما بحرف صغير.
  \item إن كان في الاسم أكثر من كلمة أضع حرف كبيرا في بداية كلّ كلمة.
\end{itemize}

أطلب منك كتابة أسماء متغيراتك بنفس الطريقة التي أتبعها، هذا لكي نكون على تفاهم.

\begin{critical}
  أيّا كان اختيارك، فعليك دائما إعطاء أسماء واضحة لمتغيراتك. كان بإمكاننا اختصار
\InlineCode{numberOfLives}
إلى
\InlineCode{nol}
مثلا. هذا أقصر في الكتابة، لكنه أقل وضوحا عندما تعيد قراءة الشفرة المصدرية. فأنصحك بإعطاء أسماء أطول لمتغيراتك إن كان ذلك يحسّن فهمها.
\end{critical}

\subsection{أنواع المتغيرات}
حاسوبنا كما نعلم ليس سوى آلة كبيرة جدا للحساب. لا يجيد التعامل سوى مع الأعداد. لكن يوجد أنواع كثيرة من الأعداد:
\begin{itemize}
  \item الأعداد الصحيحة الموجبة (الطبيعية) مثل :45، 398، 7650.
  \item الأعداد العشرية، أي التي تحوي فاصلة عشرية: 75.909، 1.7741، 9810.7.
  \item الأعداد الصحيحة السالبة: -87، -916.
  \item الأعداد العشرية السالبة: -76.9، -100.11.
\end{itemize}

حاسوبك المسكين بحاجة للمساعدة! عندما تطلب منه تخزين عدد، يجب أن تذكر له نوعه. هذا ليس لأنّه لا يمكنه التعرف عليه تلقائيّا، ولكن للتنظيم ولعدم أخذ كميات كبيرة من الذاكرة بدون فائدة.

عندما تصرّح عن متغيّر فسيكون عليك تحديد نوعه. إليك أنواع المتغيرات الأساسية في لغة \textenglish{C}:

\begin{Table}{3} % The number of columns is required (here is 3)
  النوع & الحد الأدنى & الحد الأقصى\\
  \LR{\ttfamily signed char} & $-128$ & $127$ \\
  \LR{\ttfamily int} & $-32768$ & $32767$ \\
  \LR{\ttfamily long} & $-2147483648$ & $2147483647$ \\
  \LR{\ttfamily float} & $-1 \times 10^{37}$ & $1 \times 10^{37}-1$\\
  \LR{\ttfamily double} & $-1 \times 10^{37}$ & $1 \times 10^{37}-1$\\
\end{Table}

\begin{warning}
  القيم المعروضة هنا تمثل الحد الأدنى المضمون من طرف اللغة. في الحقيقة قد تتمكن من تخزين أعداد أكبر من هذه. في كلّ الأحوال من المستحسن تذكّر هذه القيم عندما تختار نوع متغيراتك.
\end{warning}

\begin{information}
  للعلم أنّي لم أعرض جميع الأنواع هنا، بل الأساسية منها فقط.
\end{information}

الأنواع الثلاثة الأولى
(\InlineCode{char}، \InlineCode{int}، \InlineCode{long})
تسمح يتخزين الأعداد الصحيحة (1،2،3،4،...).\\
النوعان الأخيران
(\InlineCode{float}، \InlineCode{double})
يسمحان بتخزين الأعداد العشرية (13.8, 16.911…).

سترى أنّنا نتعامل من الأعداد الصحيحة معظم الوقت لأنّها سهلة الإستخدام.

\begin{critical}
  احذر في الأعداد العشرية من استخدام الفاصلة، حاسوبك لا يستخدم سوى النقطة. لذلك لا تكتب
$54,9$
 بدل
$54.9$!
\end{critical}

هذا ليس كلّ شيء، توجد أنواع أخرى تعرف بـ
\InlineCode{unsigned}
 (عديمة الإشارة) تصلح لتخزين الأعداد الموجبة فقط. يجب إضافة كلمة
\InlineCode{unsigned}
إلى النوع لاستخدامها.

\begin{Table}{2}
  النوع & المجال\\
  \LR{\ttfamily unsigned char} & من
$0$
 إلى
$255$ \\
  \LR{\ttfamily unsigned int} & من
$0$
إلى
$65535$ \\
  \LR{\ttfamily unsigned int} & من
$0$
إلى
$4294967295$\\
\end{Table}

كما ترى، مشكلة الأنواع عديمة الإشارة هي عدم القدرة على تخزين الأعداد السالبة، لكن الشيء الإيجابي هي أنّها توفّر لنا ضعف حجم التخزين لكلّ نوع موافق (مثلا
\InlineCode{signed char}
يتوقّف عند 127، بينما
\InlineCode{unsigned char}
يمتد إلى 255.

  \chapter{حسابات سهلة}

كما قلت لك في الفصل السابق: جهازك ماهو إلا آلة حاسبة كبيرة. سواء كنت تسمع الموسيقى، تشاهد فلمًا أو تلعب لعبة، فإن الحاسوب ينجز الحسابات طيلة الوقت.

هذا الفصل سيساعدك على التعرف على معظم الحسابات التي يقوم بها الجهاز. سنعيد استعمال ما نحن بصدد تعلّمه عن عالم المتغيرات. الفكرة هي أننا سنقوم بعمليات على المتغيرات: نجمعها، نضربها، نخزّن النتائج في متغيرات أخرى، إلخ.

حتى وإن لم تكن من هواة الرياضيات، فإن هذا الفصل إلزامي ولا مفرّ منه.

\section{الحسابات القاعدية}

بالرغم من قدرة الجهاز الواسعة إلا أنه في الأساس يتعمد في حساباته على عمليات بسيطة للغاية وهي:

\begin{itemize}
  \item الجمع،
  \item الطرح،
  \item القسمة،
  \item الضرب،
  \item الترديد
(\textenglish{Modulo})
(سأشرح لاحقا ما الّذي يعنيه إذا لم تكن تعرفه الآن).
\end{itemize}

إن كان بودك القيام بحسابات أكثر تعقيدا (كالأسس واللوغاريثم وماشابه)، يجب عليك إذا برمجتها أو بمعنى آخر:
\textbf{توضح للجهاز كيف يقوم بها}.\\
لحسن الحظ، سترى لاحقًا في هذا الفصل أنه توجد مكتبة في لغة \textenglish{C}،
تحتوي على دوال رياضية جاهزة. لن يكون عليك إعادة كتابتها إلا إذا أردت فعل ذلك تطوّعيا أو كنت أستاذ رياضيات.

لنبدأ بالعملية الأسهل وهي الجمع.\\
طبعا في الجمع نحتاج الرمز +.\\
يجب وضع نتيجة الجمع في متغير ولهذا سنقوم بإنشاء متغير اسمه مثلا
\InlineCode{result}
من نوع
\InlineCode{int}
و نقوم بالحساب:

\begin{Csource}
  int result = 0;
  result = 5 + 3;
\end{Csource}

لا يجب أن تكون محترفا في الحساب الذهني لتعرف أن النتيجة ستكون 8 بعد تشغيل البرنامج.\\
بالطبع البرنامج لن يظهر أية نتيجة باستعمال هذه الشفرة المصدرية. إذا أردت معرفة محتوى المتغير
\InlineCode{result}
عليك باستعمال الدالة
\InlineCode{printf}
التي تجيد كيفية استخدامها جيدًا الآن:

\begin{Csource}
  printf("5 + 3 =  %d", result);
\end{Csource}

و هذا ما سيظهر على الشاشة:

\begin{Console}
  5 + 3 = 8
\end{Console}

و هكذا ننهى عملية الجمع بسهولة.\\
الأمر مماثل بالنسبة للعمليات الأخرى، نحتاج تغيير الرمز ليس إلا:

\begin{Table}{2}
  العمليّة & الرمز\\
  الجمع & \texttt{+}\\
  الطرح & \texttt{-}\\
  الضرب & \texttt{*}\\
  القسمة & \texttt{/}\\
  الترديد & \texttt{\%}\\
\end{Table}

إذا كنت قد استعملت من قبل الآلة الحاسبة الخاصة بحاسوبك، فيفترض بك أن تكون متعوّدا على هذه الإشارات. لا يوجد أي شيء صعب بخصوصها باستثناء القسمة والترديد اللذان سأشرحهما فيما يلي بالتفصيل.

\subsection{القسمة}

ينجز الحاسوب عملية القسمة بشكل طبيعيّ عندما لا يوجد أي باق. مثلا العملية
\InlineCode{6 / 2}
تعطينا النتيجة 3، النتيجة صحيحة. حتّى الآن، لا مشكلة.

لكن لو نأخذ الآن عملية قسمة بباقٍ مثل
\InlineCode{5 / 2}
\dots
نتوقع أن النتيجة ستكون $ 2.5 $، ولكن أنظر إلى ما تعطيه الشفرة:

\begin{Csource}
int result = 0;
result = 5 / 2;
printf("5 / 2 =  %d", result);
\end{Csource}

\begin{Console}
  5 / 2 = 2
\end{Console}

هناك مشكل كبير، فنحن نتوقّع أن نحصل على القيمة $ 2.5 $، لكن الحاسوب أعطى القيمة $ 2 $!

هل يا ترى أجهزتنا غبية لهذه الدرجة؟\\
في الواقع، يقوم الجهاز بعملية قسمة صحيحة (إقليدية) أي أنه يحتفظ بالجزء الصحيح فقط الذي هو 2.

\begin{question}
  هه أنا أعرف السبب! لأن المتغير
  \InlineCode{result}
  الذي استخدمناه هو من نوع
  \InlineCode{int}!
  لو استخدمنا النوع
  \InlineCode{double}
  لاستطاع تخزين العدد العشري!
\end{question}

لا، ليس هذا هو السبب! جرب  نفس الشفرة بتغيير نوع النتيجة إلى
\InlineCode{double}
و ستجد بأننا نتحصّل على نفس النتيجة 2 لأن طرفا العملية من نوع
\InlineCode{int}
فإن الحاسوب سيعيد نتيجة من نوع
\InlineCode{int}.

إن أردنا أن يظهر لنا الجهاز القيمة الصحيحة، يجب أن نغير العددين $ 2 $ و$ 5 $ إلى عددين عشريين كالتالي: $ 2.0 $ و$ 5.0  $(قيمتهما هي نفسها لكن الجهاز سيعتبرهما عددين عشريين، وبالتالي هو يظن بأنه يقوم بقسمة عددين عشريين):

\begin{Csource}
  double result = 0;
  result = 5.0 / 2.0;
  printf("5 / 2 =  %f", result);
\end{Csource}

\begin{Console}
  5 / 2 = 2.500000
\end{Console}

هنا العدد صحيح بالرغم من وجود عدة أصفار في نهاية العدد، لكنّ القيمة تبقى نفسها.

فكرة القسمة الإقليدية التي يقوم بها الحاسوب مهمة، تذكّر أنه بالنسبة للحاسوب:

\begin{itemize}
  \item $ 5 / 2 = 2 $،
  \item $ 10 / 3 = 3 $،
  \item $ 4 / 5 = 0 $.
\end{itemize}

هذا مفاجئ بعض الشيء، لكنّها طريقته في التعامل مع الأعداد الصحيحة.

إن أردت الحصول على نتيجة عشريّة، فيجب أن يكون حدّا العملية عشريّين:

\begin{itemize}
  \item $ 5.0 / 2.0 = 2.5 $،
  \item $ 10.0 / 3.0 = 3.33333 $،
  \item $ 4.0 / 5.0 = 0.8 $.
\end{itemize}

يمكن القول أن الجهاز يطرح على نفسه السؤال: "كم يوجد من 2 في العدد 5؟" طبعا يوجد 2 فقط.

و لكن أين الباقي من العملية؟ لأنني لما أقول 5 هي أثنين من 2 ، يبقى 1 طبعا، كيف لنا أن نسترجعه؟\\
هنا يتدخل الترديد الذي كلمتك عنه.

\subsection{الترديد}

هو عبارة عن عملية حسابية تسمح بالحصول على باقي عملية القسمة، وهي عملية غير معروفة مقارنة بالعمليات الأربع الأخرى، لكن الجهاز يعتبرها من العمليات القاعدية، ويمكن اعتبارها حلا لمشكل قسمة الأعداد الطبيعية.

كما قلت لك الترديد يمثل بالرمز
\InlineCode{\%}.\\
إليكم بعض الأمثلة:

\begin{itemize}
  \item $ 5 \% 2 = 1 $،
  \item $ 14 \% 3 = 2 $،
  \item $ 4 \% 2 = 0 $.
\end{itemize}

الترديد
\InlineCode{5 \% 2}
هو باقي العملية
\InlineCode{5 / 2}
مما يعني أن الجهاز يقوم بالعملية
\InlineCode{5 = 2 * 2 + 1}
حيث أن 1 هو الباقي والذي يقوم بإرجاعه الترديد.

نفس الشيء بالنسبة للعملية
\InlineCode{14 \% 3}،
العملية هي
\InlineCode{14 = 3 * 4 + 2}
(الترديد يعطي القيمة  2). أخيرا، من أجل
\InlineCode{4 \% 2}،
القسمة تامة، فلا يوجد باقي، لهذا يعطي الترديد القيمة 0.

حسنا، لا يوجد ما يمكنني إضافته بخصوص عملية الترديد. كان هذا فقط شرحا لمن لا يعرفها.

لدي خبر جيد آخر، وهو أننا أتممنا كلّ عمليات الحساب القاعدية وتخلصنا من درس الرياضيات!

\subsection{عمليات على المتغيرات}

الشيء الجيد هو أنه بعد أن تعلمت كيف تستخدم العمليات القاعدية، يمكنك الآن أن تتعلّم كيفية القيام بهذه العمليات على المتغيرات.\\
لا شيء يمكنه منعك من كتابة الشفرة التالية:

\begin{Csource}
  result = number1 + number2;
\end{Csource}

هذا السطر يعمل على جمع المتغيرين
\InlineCode{number1}
و
\InlineCode{number2}
ثم يخزن النتيجة في المتغير
\InlineCode{result}.

هنا بدأت الأمور الممتعة تظهر، وحقيقة، مستواك الحالي يسمح لك ببرمجة آلة حاسبة بسيطة. نعم، نعم، أؤكّد لك ذلك!

تخيل وجود برنامج يطلب من المستخدم إدخال عددين، ثم يقوم بتخزينهما في متغيرين، ثم يجمع هذين المتغيرين ويخزن النتيجة في متغير اسمه
\InlineCode{result}.
لم يبق سوى إظهار النتيجة على الشاشة في وقت لا يتمكّن فيه المستخدم حتى من تخمين النتيجة.

حاول كتابة هذا البرنامج البسيط، إنّه سهل وسيكون تدريبا لك!

إليك الجواب:

\begin{Csource}
int main(int argc, char * argv[])
{
  int result = 0, number1 = 0, number2 = 0;

  // We request the two numbers from the user :

  printf("Enter the first number : ");
  scanf("%d", &number1);
  printf("Enter the second number : ");
  scanf("%d", &number2);

  // We calculate the result :

  result = number1 + number2;

  // We display the result on the screen :

  printf("%d + %d = %d\n", number1, number2, result);

  return 0;
}
\end{Csource}

\begin{Console}
  Enter the first number : 30
  Enter the second number : 25
  30 + 25 = 55
\end{Console}

بدون أن تشعر، لقد أنشأت أول برنامج لك ذو فائدة. إنّه قادر على جمع عددين وإظهارا النتيجة على الشاشة!

يمكنك التجريب باستخدام أعداد أخرى (يجب ألا تتجاوز الحد الأقصى لتحمّل نوع \InlineCode{int})
و سيقوم الحاسوب بالحساب بشكل سريع جدًا لا يتجاوز بعض أجزاء من المليار من الثانية!

أنصحك أيضًا بتجريب العمليات الأخرى (الطرح، القسمة والضرب) لكي تتدرب. لن يكون هذا متعبا إلّا بقدر تغيير إشارة أو اثنتين. يمكنك أيضًا إضافة متغير ثالث وجمع ثلاثة متغيرات دفعة واحدة. سيشتغل البرنامج دون مشاكل:

\begin{Csource}
result = number1 + number2 + number3;
\end{Csource}

\section{الاختصارات}

كما وعدتك، لا توجد عمليات أخرى لنتعلّمها اليوم لأن هذه هي كلّ العمليات الموجودة! بهذه العمليات البسيطة يمكنك برمجة أي شيء تريده. أعلم أنّه يصعب عليك التصديق لو قلت لك أن لعبة ثلاثية الأبعاد ليست في النهاية سوى مجموعة من عمليات الجمع والطرح\dots
لكنها الحقيقة.

توجد طرق في لغة \textenglish{C}
تسمح لنا باختصار كتابة بعض العمليات. لماذا نستعمل هذه الاختصارات؟ لأننا نحتاج في غالب الأحيان من كتابة عمليات مكرّرة. ستفهم ما أريد قوله حينما ترون ما نسمّيه بالزيادة.

\subsection{الزيادة (\textenglish{Incrementation})}

في غالب الأحيان ستضطر إلى إضافة 1 إلى محتوى متغير. وبالتقدّم في برنامجك، تكون لديك متغيرات يزيد محتواها في كلّ مرة بـ1.

نفترض أن لديك متغيرا يحمل اسم
\InlineCode{number}،
هل تعرف كيف تضيف له 1 دون أن تعرف محتواه؟

إليك ما يجب عليك فعله:

\begin{Csource}
number = number + 1;
\end{Csource}

ما الذي يحصل هنا؟ نقوم بالحساب
\InlineCode{number + 1}
ثم نخزن الناتج في المتغير
\InlineCode{number}!
و منه فإن كان المتغير يحمل القيمة 4 فهو بعد العملية يحمل القيمة 5. لو أنه كان يحمل القيمة 8، فهو الآن يحمل القيمة 9، إلخ.

هذه العملية تتكرر كثيرا. وبما أن المبرمج شخص كسول، سيتعبه أمر كتابة اسم المتغير مرتين في نفس التعليمة (نعم هذا أمر متعب!). لهذا تم اختراع اختصار لهذه العملية بما نسميه بـ\textbf{الزيادة }
(\textenglish{Incrementation})
التعليمة أسفله تعطي تماما نفس نتيجة التعليمة السابقة:

\begin{Csource}
  number++;
\end{Csource}

هذا السطر له نفس وظيفة السطر السابق الذي كتبناه قبل قليل، أليس مختصرا وقصيرا؟ إنه يعني "إضافة 1 لمتغير". يكفي إذا أن نرفق باسم المتغير
\InlineCode{number}
الاشارة + مرتين، مع عدم نسيان الفاصلة المنقوطة الخاصة بنهاية التعليمة.

هذه العملية ستساعدنا كثيرا مستقبلا لأننا سنضطر للقيام بعملية الزيادة كثيرا.

\begin{information}
إذا كنت دقيق الملاحظة، كنت لتلحظ أن إشارتي
++
متواجدتان أيضًا في اسم اللغة
\textenglish{C++}.
أنت الآن قادر على أن تفهم السر وراء ذلك! \textenglish{C++}
تعني أننا نتكلم عن لغة \textenglish{C}
"مع زيادة". عمليّا، يسمح لنا \textenglish{C++}
بالبرمجة بطريقة مختلفة، لكن لا يعني أنه "أفضل" من \textenglish{C}.
هو فقط مختلف.
\end{information}

\subsection{الإنقاص (\textenglish{Decrementation})}

إنّها بكلّ بساطة عكس عملية الزيادة، فهي تقوم بإنقاص 1 من متغير.\\
بالرغم من أن عملية الزيادة هي أكثر استعمالًا إلا أن عملية الإنقاص تبقى شائعة أيضا.

إليكم كيف ننقص 1 من متغير بالشكل "الطويل":

\begin{Csource}
  number = number - 1;
\end{Csource}

و في شكلها المختصر:

\begin{Csource}
  number--;
\end{Csource}

ربّما كان بإمكانك تخمين ذلك وحدك! بدل وضع إشارة
\InlineCode{++}
نضع إشارة
\InlineCode{{-}{-}}.
إذا كان محتوى المتغير هو 5 فسيصبح بعد الإنقاص يساوي 4.

\subsection{الاختصارات الأخرى}

توجد اختصارات أخرى تعمل بنفس المنطلق. هذه الاختصارات تصلح لكل العمليات القاعدية:
\InlineCode{+}
\InlineCode{-}
\InlineCode{*}
\InlineCode{/}
\InlineCode{\%}.
هي تساعدنا على تجنب تكرار نفس اسم المتغير في نفس التعليمة.\\
لضرب محتوى المتغير في 2 مثلا نقوم بالتالي:

\begin{Csource}
  number = number * 2;
\end{Csource}

و بالشكل المختصر:

\begin{Csource}
  number *= 2;
\end{Csource}

بالطبع إن كان للمتغير القيمة $ 5 $ قبل إجراء العملية فسيحمل الآن $ 10 $ بعد هذه التعليمة.\\
بالنسبة لباقي العمليات القاعدية، فالمبدأ نفسه. إليك برنامجا صغيرا كمثال:

\begin{Csource}
int number = 2;
number += 4; // number = 6 ...
number -= 3; // ... number = 3
number *= 5; // ... number = 15
number /= 3; // ... number = 5
number %= 3; // ... number = 2 (because 5 = 1 * 3 + 2)
\end{Csource}

\textit{(لا تتذمر فبعض الحسابات الذهنيّة لن تقتل شخصًا!)}

الشيء الجيد هنا أنه يمكننا استعمال اختصارات على كلّ العمليات القاعدية، فيمكننا أن نجمع، نطرح، نضرب أيّ عدد.\\
هي اختصارات عليك تعلّمها إن كان البرنامج الذي تكتبه يحتوي الكثير من التعليمات المكرّرة.

تذكّر أيضًا أن عملية الزيادة هي الاختصار الأكثر استعمالًا.

\section{المكتبة الرياضياتيّة}

في لغة \textenglish{C}
هناك دائما ما نسميه بالمكتبات القياسية
(\textenglish{Standard libraries})،
 وهي المكتبات التي تستخدم على الدوام. إنّها مكتبات قاعدية تستخدم كثيرا.

أذكرك بما قلت سابقا، المكتبة هي مجموعة دوال جاهزة. هذه الدوال تمت كتابتها من طرف مبرمجين قبلك، وهي تساعدك على تجنب إعادة اختراع العجلة في كلّ برنامج جديد.

في الواقع، العمليات الخمس القاعدية الّتي رأيناها هي أقلّ من أن تكون كافية. إذا لم تفهم هذا، فربّما قد تكون صغيرا في السنّ أو لم تتعلّم الكثير عن الرياضيّات في حياتك. المكتبة الرياضياتيّة تحوي العديد من الدوال الأخرى الّتي قد تحتاجها.


أعطيك مثالا، لغة \textenglish{C}
لا تحتوي على عملية الأس! كيف نحسب المربع؟ يمكنك كتابة العملية
\InlineCode{$5~\hat{}~2$}
في برنامجك لكن الجهاز لن يفهمها أبدًا لأنّه لا يعرف مالّذي تعنيه هذه، إلا إن قمت بشرح العملية له باستخدام المكتبة الرياضياتيّة!

يمكننا الاستعانة بالدوال الجاهزة في المكتبة الرياضياتيّة، لكن لا تنس كتابة توجيهات المعالج القبلي الخاصة بها في بداية كل برنامج:

\begin{Csource}
#include <math.h>
\end{Csource}

ما إن تكتب السطر السابق حتّى تصبح قادرًا على استخدام كل الدوال المتوفرة في هذه المكتبة.

لديّ نيّة في عرضها لك الآن.\\
حسنا، بما أنّه يوجد الكثير منها، فلا يمكنني إنشاء قائمة كاملة هنا. من جهة لأنّ هذا سيكون كثيرا للفهم، ومن ناحية أخرى، فأصابعي  المسكينة ستذوب قبل إنهاء كتابة هذا الفصل! سأريك إذن الدوال الرئيسيّة فقط، أي الّتي تبدو أكثر أهميّة.

\begin{information}
ربّما قد لا يكون لديك مستوى  الرياضيات اللازم لفهم ما تفعله هذه الدوال. إن كان هذا هو حالك، فلا تقلق. قم بالقراءة فقط، فهذا لن يضرّك فما يلي.\\
على الرغم من ذلك، أقدّم لك نصيحة مجّانيّة: كن منتبها في دروس الرياضيات، لا نقول هذا من دون سبب، هذا سيفيدك في النهاية.
\end{information}

\subsection{\texttt{fabs}}

هذه الدالة تحسب القيمة المطلقة للرقم، ونرمز لها بالشكل التالي:
\textenglish{|x|}.\\
القيمة المطلقة لعدد هو قيمته الموجبة:

\begin{itemize}
  \item إعطاء العدد $ -52 $ للدالة يجعلها ترجع القيمة $ 52 $،
  \item إعطاء العدد $ 52 $ للدالة يجعلها تعيد $ 52 $.
\end{itemize}

باختصار، تعيد دائما العدد الموجب الموافق لما أعطيته لها.

\begin{Console}
double absolute = 0, number = -27;
absolute = fabs(number); // absolute = 27
\end{Console}

هذه الدالّة تعيد
\InlineCode{double}،
لذا فالمتغيّر
\InlineCode{absolute}
يجب أن يكون من نوع
\InlineCode{double}.

\begin{information}
هناك دالة اخرى مماثلة لـ\InlineCode{fabs}
تسمى
\InlineCode{abs}،
نجدها في
\InlineCode{stdlib.h}.\\
إنّها تعمل بنفس طريقة الأولى إلا أنها تعمل مع الأعداد الصحيحة، فهي تعيد
\InlineCode{int}.
\end{information}

\subsection{\texttt{ceil}}

هذه الدالّة تعطي أول عدد طبيعي بعد العدد العشري الذي نعطيه لها. يمكن القول أنها تدوّر العدد دائما إلى العدد الذي يعلوه في الجزء الصحيح.\\
لو نعطيها مثلا العدد $ 26.512 $ فستعطينا العدد $ 27 $.

الدالة تعمل بنفس الطريقة وتعيد
\InlineCode{double}
أيضا:

\begin{Csource}
double above = 0, number = 52.71;
above = ceil(number); // Above = 53
\end{Csource}

\subsection{\texttt{floor}}
هذه عكس السابقة، تعيد العدد الأقل مباشرة في الجزء الصحيح.\\
إذا أعطيتها مثلا العدد $ 37.91 $ تعطيني العدد $ 37 $، يعني الجزء الصحيح.

\subsection{\texttt{pow}}

هذه خاصة بحساب قوى عدد (الأسس). يجب أن تعطيها قيمتين، الأولى هي العدد الذي تريد إجراء العملية عليه والثانية هي القوة الّتي يجب رفع العدد إليها. هذا مخطط الدالة:

\begin{Csource}
pow(number, power);
\end{Csource}

كمثال: "2 قوة 3" (الّتي نكتبها عادة
\InlineCode{$2~\hat{}~3$}
على الحاسوب) هو الحساب
$ 2 \times 2 \times 2 $
الّذي يعطي النتيجة 8:

\begin{Csource}
double result = 0, number = 2;
result = pow(number, 3); // result = 2^3 = 8
\end{Csource}

\subsection{\texttt{sqrt}}

هذه الدالة تحسب الجذر التربيعي لعدد، تعيد
\InlineCode{double}.

\begin{Csource}
double result = 0, number = 100;
result = sqrt(number); // Result = 10
\end{Csource}

\subsection{\texttt{sin}، \texttt{cos}، \texttt{tan}}

إنّها الدوال المثلثية الثلاث الشهيرة.\\
طريقة عملها هي نفسها، تعيد
\InlineCode{double}.

هذه الدوال تأخذ قيما بـ\textbf{الراديان}
(\textenglish{Radians}).

\subsection{\texttt{asin}، \texttt{acos}، \texttt{atan}}

و هي الدوال
\textenglish{arcsin}، \textenglish{arccos}، \textenglish{arctan}،
دوال مثلثية أخرى.\\
تُستخدم بنفس الطريقة وتعيد
\InlineCode{double}
أيضا.

\subsection{\texttt{exp}}

هذه الدالة تحسب قيمة الدالة الأسيّة ذات الأساس
\textenglish{e}
لعدد معين.\\
تعيد
\InlineCode{double}.

\subsection{\texttt{log}}

هذه الدالة تحسب اللوغاريتم النيبيري لعدد معين. (الّذي نرمز له أيضا بـ"\textenglish{ln}")

\subsection{\texttt{log10}}

هذه الدالة تحسب اللوغاريتم ذو الأساس 10 لعدد.

\section*{ملخّص}

\begin{itemize}
  \item الحاسوب ما هو سوى
\textbf{آلة حاسبة كبيرة}: كل ما يجيد فعله هو القيام بالعمليّات.
  \item العمليات التي يجيدها الحاسوب
\textbf{قاعدية جدا}: الضرب، القسمة، الجمع، الطرح والترديد (باقي القسمة).
  \item بالإمكان
\textbf{إجراء عمليات على المتغيرات}،
الحاسوب سريع جدًا في هذا النوع من العمليات.
  \item \textbf{الزيادة}
(\textenglish{Incrementation})
هي عملية إضافة الرقم 1 إلى متغير. نكتبها
\InlineCode{variable++}.
  \item \textbf{الإنقاص}
(\textenglish{Decrementation})
هي عملية طرح الرقم 1 من متغير. نكتبها
\InlineCode{variable{-}{-}}.
  \item لزيادة عدد العمليات التي يمكن للحاسوب القيام بها، نستعمل
\textbf{المكتبة الرياضيّاتيّة}
(أي\\
\InlineCode{\#include <math.h>})
  \item تحتوي هذه المكتبة على
\textbf{دوال رياضيّاتيّة متقدّمة}
كالأسّ والجذر واللوغاريثم وغيرها.
\end{itemize}

  \chapter{الشروط}

لقد رأينا فيما سبق بأن هناك العديد من لغات البرمجة. بعضها متشابه : الكثير منها مستلهم من الـ\textenglish{C}.

في الواقع، لغة  الـ\textenglish{C}
أُنشئت منذ زمن طويل، و هذا جعلها نموذجا للّغات الجديدة.

لغات البرمجة تختلف في بعض الأمور، لكن هناك مبادئ لا يمكن أن تخلو منها أية لغة برمجية. لقد رأينا كيف ننشؤ المتغيّرات، كيف نقوم بالحسابات، و الآن سنمرّ إلى 
\textbf{الشروط}.\\
من دون استعمال الشروط، برامجنا ستقوم دائما بنفس العمل !

\section{الشرط
\texttt{if \dots else}}

الشروط تسمح لنا بأن نقوم باختبارات على المتغيرات. مثلا يمكننا القول، "إن كان المتغيّر 
\InlineCode{machine}
يساوي 50، يجب أن نقوم بكذا و كذا. و لكن من المؤسف عدم إمكانية اختبار سوى المساواة ! يجب أيضا اختبار ما إن كان المتغيّر، أقل من 50، أقل أو يساوي 50، أكبر، أكبر أو يساوي
\dots
لا تقلق، في الـ\textenglish{C}
كل شيء مُعَد !

لدراسة الشرط 
\InlineCode{if \dots else}
يجب أن نتبع المخطط التالي :
\begin{itemize}
\item نتعلم بعض الرموز قبل البدأ،
\item الشرط 
\InlineCode{if}،
\item الشرط 
\InlineCode{else}،
\item الشرط
\InlineCode{else if}،
\item كثير من الشروط في مرة واحدة،
\item بعض الأخطاء لنتجنبها.
\end{itemize}

قبل أن نرى كيف نكتب شرطا من النوع
\InlineCode{if \dots else}
في الـ\textenglish{C}، يجب أن تعرف ثلاثة رموز أساسية. هذه الرموز ضرورية لإنشاء الشروط.

\subsection{بعض الرموز للتعلّم}

الجدول التالي يحوي رموز لغة الـ\textenglish{C} الّتي
\textbf{يجب حفظها عن ظهر قلب} :

\begin{Table}{2}
\texttt{{=}{=}} & يساوي\\
\texttt{>} & أكبر\\
\texttt{<} & أصغر\\
\texttt{<=} & أصغر أو يساوي\\
\texttt{>=} & أكبر أو يساوي\\
\texttt{!=} & لا يساوي\\
\end{Table}

\begin{critical}
انتبه جيدا، هناك رمزا مساواة
\InlineCode{{=}{=}}
لنقوم باختبار المساواة. فالمبتدؤون يقومون غالبا باقتراف خطأ وضع إشارة واحدة
\InlineCode{=}،
و هذا لديه معنى مختلف في الـ\textenglish{C}. سأذكّرك بهذا لاحقا.
\end{critical}

\subsection{\texttt{if} بسيط}

فلنبدأ، لنقم باختبار بسيط، يقول للحاسوب : إن كان المتغير يساوي كذا فلنقم بكذا.

في الإنجليزيّة، كلمة "إذا" تُتَرجم إلى
\InlineCode{if}.
و هذا ما الّذي نستخدمه في لغة الـ\textenglish{C} لإنشاء اختبار.\\
أكتب إذن
\InlineCode{if}
ثم الأقواس و في داخلها الشرط.

بعد ذلك افتح حاضنة
\InlineCode{\{}
و اغلقها لاحقا
\InlineCode{\}}.
كلّ ما يوجد بين الحاضنتين سيتمّ تشغيله فقط في حالة تحقق الشرط.

هذا يعطينا إذن :
\begin{Csource}
if (/* Your condition */)
{
	// Instructions to execute
}
\end{Csource}

في مكان التعليق
"\textenglish{Your condition}"
سنكتب شرطا للتحقّق من متغيّر.\\
كمثال سنحاول أن نقوم بختبار متغيّر
\InlineCode{age}
لاحتواء العمر. سنختبر ما إن كان المستعمل راشدا أم قاصرا استنادا إلى 
\textbf{إذا ما كان عمره يساوي أو أكبر من 18} :

\begin{Csource}
if (age >= 18)
{
	printf("You are major !");
}
\end{Csource}

\InlineCode{>=}
يعني "أكبر أو يساوي"، كما رأينا في الجدول السابق.

\begin{information}
عندما لا يكون هناك سوى تعليمة واحدة بين الحاضنتين، يمكننا أن نتخلى عنهما. لكني أفضل أن تضعوما دائما من اجل قراءة أفضل للشفرة.
\end{information}

\subsubsection{جرّب هذه الشفرة}
لكي نجرّب الشفرات السابقة و نفهم كيف يعمل الـ\InlineCode{if}، يجب أن نضعه داخل الدالة
\InlineCode{main}
و لا ننس التصريح بالمتغير 
\InlineCode{age}
و إعطاؤه قيمة ابتدائية.

هذا قد يبدو بديهيّا للبعض، و لكنّ كثيرا من القرّاء قد ضاعوا في هذه الأسطر و هذا ما دفعني لإضافة هذا الشرح. هذه شفرة كاملة يمكنك تجريبها :

\begin{Csource}
#include <stdio.h>
#include <stdlib.h>
int main(int argc, char *argv[])
{
	int age = 20;
	
	if (age >= 18)
	{
		printf("You are major !\n");
	}
	return 0;
}
\end{Csource}

هنا،
\InlineCode{age}
يساوي 20 إذن سيتم عرض
"\textenglish{You are major !}".\\
جرّب تغيير القيمة إلى 15 مثلا، سيصبح الشرط خاطئاً و بالتالي لن يُعرض شيء هذه المرّة.

استخدم هذه الشفرة لتجريب الأمثلة اللاحقة في هذا الفصل.

\subsubsection{مسألة نظافة}

الطريقة الّتي تفتح بها الحاضنات ليست مهمّة. سيعمل برنامجك إن كتبت كلّ شيء على نفس السطر. مثال :

\begin{Csource}
if (age >= 18) { printf("You are major !"); }
\end{Csource}

و على الرغم من أنّه ممكن، فهذا أمر
\textbf{غير منصوح به مطلقا}.\\
في الواقع، الكتابة على نفس السطر تجعل شفرتك صعبة القراءة. إذا لم تتعوّد من الآن على تهوية شفرتك، فلاحقا عندما تكتب شفرات كبيرة فستضيع بالتأكيد !

حاول عرض شفرتك بنفس طريقتي : حاضنة على سطر، ثمّ التعليمات (مسبوقة بجدولة 
(\textenglish{tabulation}))،
 ثمّ حاضنة الإغلاق على سطر آخر.

\begin{information}
توجد طرق عديدة لعرض الشفرة بشكل جيّد. هذا لا يغيّر في عمل البرنامج شيئا، لكنّها مسألة "نمط معلوماتيّ" إن أردت. إذا رأيت الشفرة شخص آخر معروضة بشكل مختلف قليلا، فهذا لأنّه يبرمج بنمط مختلف. الهدف من هذا كلّه هو أن تبقى الشفرة مهوّاة و مقروءة.
\end{information}

\subsection{الـ\texttt{else}
لكي نقول : و إلا}

الآن و بما أنّك تعرف كيف تقوم باختبار بسيط، سنذهب بعيدا : إن لم يعمل الشرط (كان خاطئا)، فسنطلب من الحاسوب تشغيل تعليمات أخرى.

هذا يشبه ما يلي : إن كان المتغير يساوي كذا فلنقم بكذا، و إلا فلنقم بكذا.

يكفي أن نضيف
\InlineCode{else}
بعد الحاضنة الأخيرة لأوامر الـ\InlineCode{if}،
مثلا :
\begin{Csource}
if (age >= 18)
{
	printf("You are major !");
}
else {
	printf("You are minor !");
}
\end{Csource}

الأمر سهل : إن كان المتغيّر
\InlineCode{age}
أكبر من أو يساوي 18 سنكتب على الشاشة 
"\textenglish{You are major !}"،
و إن لم يكن الأمر كذلك فسنكتب 
"\textenglish{You are minor !}".

\subsection{\texttt{else if}
لكي نقول : و إلا فإذا}

سنرى كيف نقوم بـ"إذا" و "إلّا". يمكن أيصا القيام بـ"و إلاّ فإذا" للقيام باختبار آخر في حالة ما إذا لم ينجح الأوّل. هذا الاختبار يوضع بين
\InlineCode{if}
و
\InlineCode{else}.

يمكن ترجمة ذلك بما يلي : إن كان المتغير يساوي كذا، فافعل كذا و إلا فإن كان يساوي كذا، فافعل كذا و إلا (جميع الحالات المتبقية)، افعل كذا.

الترجمة بلغة
\textenglish{C} :

\begin{Csource}
if (age >= 18)
{
	printf("You are major !");
}
else if (age > 4)
{
	printf("Well you're not too young anyway...");
}
else
{
	printf("Aga gaa aga gaaa");
	// Baby language, you can’t understand !
}
\end{Csource}

الجهاز يقوم بالاختبارات التالية بالترتيب :
\begin{itemize}
\item سيختبر الـ\InlineCode{if}
الأول، إن كان الشرط محققا، فسيقوم بتنفيذ التعليمات الموجودة بين الحاضنتين الأولتين.
\item إن لم يكن الشرط الأول محققا (يعني أننا في الـ\InlineCode{else if})
سيختبر الشرط الثاني، إن كان هذا الأخير محققا فإنه سينفذ التعليمات الموجودة بين الحاضنتين الثانيتين.
\item في حالة ما لم يكن الشرط الأول محققا و لا حتى الثاني، فإنه سينفذ التعليمات الموجودة بين الحاضنتين الأخيرتين.
\end{itemize}

\begin{information}
الـ\InlineCode{else}
و الـ\InlineCode{else if}
ليسلا ضروريّين. لإنشاء شرط، فقط
\InlineCode{if}
هو الضروري (هذا منطقي، و إلّا فكيف سيكون هناك شرط !).
\end{information}

لا حظ أنّه يمكننا أن نستعمل الكمّ الّذي نريده من الـ\InlineCode{else if}.

\subsection{عدّة شروط في مرّة واحدة}

قد يكون من المهم القيام بعدّة اختبارات في شرط واحد. مثلا، قد تريد اختبار ما إن كان العمر أكبر من 18 و العمر أصغر من 25.\\
لهذا علينا بتعلم رموز جديدة :

\begin{Table}{2}
الرمز & المعنى\\
\texttt{\&\&} & و \\
\texttt{||} & أو \\
\texttt{!} & لا (عكس الشرط)\\
\end{Table}

\subsubsection{الاختبار بالواو (\textenglish{and})}

لنختبر إن كان العمر في نفس الوقت أكبر من 18 و أقل من 25 يجب كتابة :

\begin{Csource}
if (age > 18 && age < 25)
\end{Csource}

\InlineCode{\&\&}
يعنيان "و". بالعربية هذا يعني : "إذا كان العمر أكبر من 18 و إذا كان العمر أصغر من 25".

\subsubsection{الاختبار بالأو (\textenglish{or})}

لاستخدام "أو"، يجب كتابة الرمزين
\InlineCode{||}.
لكتابة الرمز 
\InlineCode{|}،
نضغط في لوحة المفاتيح على
\InlineCode{Alt Gr} + \InlineCode{6}
(باعتبار أن تخطيط لوحة المفاتيح
\textenglish{AZERTY}
فرنسي). 

فلنعتبر برنامجا بسيطا، يختبر ما إن كان الشخص قادراً على فتح حساب بنكي او لا. فلكي يكون قادراً على ذلك يجب أن لا يكون شابّا كثيرا (فلنقل مثلا، ليس تحت 30 سنة) أو لديه الكثير من المال :

\begin{Csource}
if (age > 30 || argent > 100000)
{
	printf("Welcome to PicsouBank !");
}
else
{
	printf("Get out of my sight, miserable!");
}
\end{Csource}

الاختبار سيكون ناجحا فقط إذا كان الشخص بعمر أكبر من 30 سنة، أو على الأقل يملك مبلغاً أكبر من 100000 دينار مثلاً.

\subsubsection{الإختبار بـلا (\textenglish{not})}

الرمز الموافق لهذا الاختبار هو علامة التعجّب (!). في علوم الحاسوب، علامة التعجّب تعني "لا". يجب  وضع هذه العلامة من أجل عكس الشرط لقول  "إن لم يكن هذا صحيحا" :
\begin{Csource}
if (!(age < 18))
\end{Csource}

يمكن أن نترجم هذا إلى "إن كان الشخص غير قاصر". لو حذفنا
\InlineCode{!}
فسيعكس المعنى : "إن كان الشخص قاصرا".
\subsection{بعض الأخطاء التي يرتكبها المبتدؤون}

\subsubsection{لا تنس الإشارتين \texttt{==}}

مثلما قلت سابقا، لكي نختبر ما إن كان العمر يساوي 18 نكتب :

\begin{Csource}
if (age == 18)
{
	printf("You have just become major !");
}
\end{Csource}

\textbf{لا تنس}
وضع علامتي "يساوي" داخل
\InlineCode{if}،
هكذا :
\InlineCode{==}.

إن وضعت علامة 
\InlineCode{=}
واحدة فإن المتغير 
\InlineCode{age}
سيأخذ القيمة 18 (مثلما تعلمنا ذلك سابقا في درس المتغيرات). نحن نريد أن نختبر قيمة المتغير و ليس تغييرها فاحذر ! الكثير يقع في هذا الخطأ و بالتأكيد فبرامجهم لن تعمل بالشكل المطلوب !

\subsubsection{الفاصلة المنقوطة الزائدة}

بعض المبتدئين يقومون بإضافة فاصلة منقوطة في نهاية سطر الـ\InlineCode{if}، و لكن الـ\InlineCode{if} هو شرط و لا نضع فاصلة منقوطة إلّا في نهاية تعليمة. الشفرة التالية لن تعمل كما هو متوقّع لأنّه يوجد
\InlineCode{;}
في نهاية الشرط.

\begin{Csource}
if (age == 18); // Note the semicolon that mustn't be here
{
	printf("You are just major !");
}
\end{Csource}

  \chapter{الحلقات التكرارية}

بعدما تعلمنا كيف ننشؤ شروطا بلغة 
\textenglish{C}،
سنكتشف معاً 
\underline{الحلقات التكراريّة} (\textenglish{loops}).
ما هي الحلقة ؟ هي تقنية تسمح بتكرار نفس التعليمات عدة مرات. و ستساعدنا كثيرا من الآن و صاعدا خاصة في العمل التطبيقي الأوّل الذي ينتظرنا بعد هذا الفصل.

استرخ : هذا الفصل سيكون سهلاً. لقد تعرفنا سابقا على ما تعنيه المتغيّرات المنطقية
(\textenglish{booleans})
و الشروط 
(\textenglish{conditions})
في الفصل السابق، و بذلك كنا قد تخلصنا من عمل كبير. من الآن فصاعداً ستكون الأمور سلسة أكثر و لن يكون في العمل التطبيقي القادم الكثير من المشاكل.

فلننتهز الفرصة، لأننا لن نتأخر في الدخول في الجزء الثاني من الكتاب. سيكون من الجيّد لك أن تنتبه !

\section{ماهي الـحلقة ؟}

كما قلت سابقاً : هي عبارة عن تعليمة تسمح لنا بتكرار نفس التعليمات عدة مرات. 

تماما مثل الشروط، توجد طرق عديدة لإنشاء الحلقات. و لكن مهما اختلفت الطرائق فالهدف واحد : تكرار تعليمات لعدد معيّن من المرات. \\
لدينا في لغة 
\textenglish{C}
ثلاثة أنواع من الحلقات :
\begin{itemize}
	\item \InlineCode{while}
	\item \InlineCode{do \dots while}
	\item \InlineCode{for}
\end{itemize}
في جميع الحالات يبقى المخطط نفسه :

\Picture{Chapter_I-7_Loop}
و هذا ما سيحصل بالترتيب :

\begin{enumerate}
	\item الجهاز يقرأ التعليمات من الأعلى إلى الأسفل كالعادة.
	\item ما إن يصل لنهاية الحلقة يتوجه نحو التعليمة الأولى.
	\item يعيد بعدها قراءة التعليمات كلها من الأعلى إلى الأسفل.
	\item يصل لنهاية الحلقة و يعاود الرجوع للأول من جديد و هكذا \dots
\end{enumerate}

المشكلة في هذا النظام هو أننا إن لم نقم بإيقافه، فالجهاز قادر على تكرار نفس التعليمات إلى مالانهاية ! و لن يتذمّر، أنت تعرف : هو يفعل ما تأمره أنت بفعله \dots يمكنه أن يعلق في حلقة غير منتهية، و هذا النوع من الحالات يعتبر مصدر خوف  بالنسبة للمبرمجين.

و هنا نجد \dots الشروط ! فعندما ننشئ حلقة نقوم دائما بتعريف شرطها. هذا الشرط يعني "كرّر الحلقة دون توقف مادام هذا الشرط صحيحا".

كما قلت فهناك عدة طرق للقيام بذلك و سنبدأ من دون تأخير بإنشاء حلقة من نوع 
\InlineCode{while}
في الـ\textenglish{C}.

\section{الحلقة \texttt{while}}

هكذا نشكل حلقة 
\InlineCode{while} :

\begin{Csource}
while (/* Condition */)
{
	// The instructions that we want to repeat
}
\end{Csource}

لا يوجد أبسط من هذا. الكلمة 
\InlineCode{while}
تعني "مادام" ، لذا نقول للجهاز: مادام الشرط صحيحا، كرر التعليمات المتواجدة بين الحاضنتين.

أقترح عليك أن نقوم باختبار بسيط : سنطلب من المستعمل ادخال العدد 47، مادام لم يقم بإدخاله، نطلب منه إعادة إدخاله مجدداً \dots و لن يتوقف البرنامج حتى يقوم المستعمل بإدخال العدد 47 (نعم أعرف، إنه عمل شيطاني) :

\begin{Csource}
int entredNumber = 0;
while (entredNumber != 47)
{
	printf("Enter the number 47 ! ");
	scanf("%d", &entredNumber);
}
\end{Csource}

أنظر إلى الاختبار الذي قمت به، للعلم أنني تعمدت الخطأ ثلاث مرات :

\begin{Console}
Enter the number 47 ! 10
Enter the number 47 ! 27
Enter the number 47 ! 40
Enter the number 47 ! 47
\end{Console}

يتوقف البرنامج بعد إدخال العدد 47.\\
 هذه الحلقة 
\InlineCode{while}
 ستتكرر مادام المستعمل لم يدخل العدد 47، لا يوجد أسهل من هذا.
 
الآن لنجعل الأمر ممتعاً أكثر : نريد من الحلقة أن تتوقف بعد عدد معين من التكرارات.\\
لهذا سنستعين بمتغير
\InlineCode{counter}
الذي سيأخذ القيمة 0 في بداية البرنامج ثم نقوم 
\textbf{بزيادته}،
 هل تتذكر ما قلناه في الدرس السابق حول الزيادة 
(\textenglish{incrementation}) ؟
 التي تنص على إضافة 1 لمتغير حينما نكتب
\InlineCode{variable++}.

إقرأ جيدا الشفرة المصدرية التالية و حاول التمعن فيها و فهمها :

\begin{Csource}
int counter = 0;
while (counter < 10)
{
	printf("Hello !\n");
	counter++;
}
\end{Csource}

النتيجة :

\begin{Csource}
Hello !
Hello !
Hello !
Hello !
Hello !
Hello !
Hello !
Hello !
Hello !
Hello !
\end{Csource}

البرنامج يكرر عشر مرات العبارة
"\textenglish{Hello !}"

\begin{question}
كيف يعمل هذا بالتحديد ؟
\end{question}

\begin{enumerate}
	\item في البداية لدينا متغير 
	\InlineCode{counter}
	مهيّأ على القيمة الإبتدائية 0.
	\item الحلقة 
	\InlineCode{while}
تأمر بالتكرار مادامت قيمة المتغير
\InlineCode{counter}
أصغر من 10. بما أن قيمة المتغير 
\InlineCode{counter}
هي 0 في البداية، فإننا ندخل في الحلقة لأن الشرط محقق.
	\item نقوم بإظهار الرسالة 
	"\textenglish{Hello !}"
على الشاشة باستخدام الدالة 
\InlineCode{printf}.
	\item نقوم بزيادة قيمة المتغير 
	\InlineCode{counter}
بفضل التعليمة 
	\InlineCode{counter++;}.
كان المتغير
\InlineCode{counter}
يحمل القيمة 0، أمّا الآن فهو يحمل القيمة 1.
	\item نصل لنهاية الحلقة (حاضنة الإغلاق) : نعيد العملية من جديد و نتأكد ما إن كان الشرط محققاً أي ما إن كانت قيمة المتغير أصغر من 10 ؟ في هذه الحالة نعم لأن المتغير
	\InlineCode{counter}
	يحمل القيمة 1 و هي أصغر من 10 إذا سنمرّ بنفس التعليمة داخل الحاضنتين.
\end{enumerate}

و هكذا دواليك،
\InlineCode{counter}
تصبح 1، 2، 3، \dots، 8، 9 ثمّ 10. حينها يصيح الشرط 
\InlineCode{counter < 10}
غير محقق، و عندها نخرج من الحلقة.

و يمكننا ملاحظة أن قيمة المتغير
\InlineCode{counter}
 تزيد في كلّ مرة بواحد. يمكننا التأكّد بـ\InlineCode{printf} :

\begin{Csource}
int counter = 0;
while (counter < 10)
{
	printf("counter = %d\n", counter);
	counter++;
}
\end{Csource}

\begin{Csource}
counter = 0
counter = 1
counter = 2
counter = 3
counter = 4
counter = 5
counter = 6
counter = 7
counter = 8
counter = 9
\end{Csource}

إن كنت قد فهمت المثال السابق فقد فهمت كلّ شيء !\\
يمكنك الاستمتاع بتجربة أعداد أكبر من 10 ( مثلا 100 أو أي عدد آخر). كان هذا سيفيدني كثيرا في صغري لكتابة العقوبات الّتي كان يجب عليّ تكرارها مائة مرّة.

\subsection{احذر من الحلقات غير المنتهية ! }

عندما تنشئ حلقة فـ\textbf{تأكّد دائما من جعلها قادرة على التوقف في لحظة معينة !}
إن كان الشرط محققاً دائماً، فلن يتوقف البرنامج أبداً ! و هذا مثال على حلقة غير منتهية :

\begin{Csource}
while (1)
{
	printf("Infinite loop\n");
}
\end{Csource}

تذكر ما قلناه بخصوص القيم المنطقية 
(\textenglish{booleans}) :
فصحيح = 1 و خاطئ = 0. هنا، الشرط محقق دائماً و بهذا فإن البرنامج سيستمرّ في كتابة العبارة 
"\textenglish{Infinite loop}"
بدون توقّف !

\begin{information}
لإيقاف برنامج كهذا في الويندوز، ليس هناك حلّ سوى الضغط على الزر 
\InlineCode{X}
الملوّن بالأحمر في أعلى النافذة بينما على مستخدمي اللينكس الضغط على
\InlineCode{Ctrl} + \InlineCode{C}
للخروج من البرنامج.
\end{information}
\textbf{\textbf{\textbf{}}}
لكن توخ الحذر. تجنب دائما الوقوع في الحلقات غير المنتهية، بالرغم من أنها قد تكون مفيدة في بعض الحالات، خصوصا في برمجة ألعاب الفيديو، هذا ما سنراه لاحقاً.
  \chapter{عمل تطبيقي : "أكثر أو أقل"، لعبتك الأولى}

نصل اليوم إلى أول عمل تطبيقي. الهدف هو أن أريك أنك قادر على برمجة الكثير من الأشياء بما علّمتك إياه. لأنه في الواقع، الجانب النظري للغة أمر جيّد لكننا إن كنا لا نعرف كيف نطبّق ما تعلّمناه بشكل سلس فلا داعي لإهدار وقتنا بتعلم المزيد.

صدّق أو لا تصدّق، يسمح لك مستواك الآن ببرمجة أول برنامج ممتع. إنّه لعبة كونسول (أذكّرك بأننا سنصل للبرامج بنافذة لاحقا). مبدأ عمل اللعبة سهل للغاية، وسهل البرمجة، و لهذا اخترتها لتكون موضوع أول عمل تطبيقي لك.

  \chapter{الدوال}

ننتهي من الجزء الأول من الكتاب (المبادئ) بهذها المفهوم المهم الذي يتكلّم عن الدوال في لغة
\textenglish{C}.
كل البرامج في لغة  
\textenglish{C}
ترتكز على المبدأ الذي سأشرحه في هذا الفصل.

سوف نتعلم هيكلة برامجنا وتقسيمها لعدة أجزاء، تقريباً كما لو كنّا نلعب لعبة
\textenglish{Lego}.\\
البرامج الكبيرة في لغة 
\textenglish{C}
ماهي إلا تجميعات لأجزاء صغيرة من الشفرات المصدرية، و هذه الأجزاء هي بالضبط ما نسمّيه \dots بالدوال !

\section{إنشاء و استدعاء دالة}

رأينا في الدروس السابقة بأن برامج الـ\textenglish{C}
تبدأ بدالة تدعى 
\InlineCode{main}.
لقد أعطيتك مخططاً تلخيصيّاً لتذكيرك ببعض الكلمات المفتاحية :

\Picture{Chapter_I-9_main}

في الأعلى، نجد توجيهات المعالج القبلي (اسم معقد سأرجع لشرحه في وقت لاحق). من السهل التعرّف على هذه التوجيهات : هي تبدأ بإشارة 
\InlineCode{\#}
و هي غالباً توضع في أعلى الملف المصدري.

لقد قلت لك بأن البرنامج في الـ\textenglish{C}
يبدأ بالدالة
\InlineCode{main}.
 أؤكد لك، هذا صحيح ! إلا أننا في هذه الحالة بقينا داخل الدالة 
\InlineCode{main}
و لم نخرج أبداً منها. أعد قراءة الشفرات المصدرية في الدروس السابقة و سترى : لقد بقينا دائما داخل الحاضنتين الخاصتين بالدالة الرئيسية.

\begin{question}
حسنا، هل من السيء فعل ذلك ؟
\end{question}

لا، هذا ليس "سيئا". لكن ذلك خلاف ما يفعله المبرمجون بلغة
\textenglish{C}
حقيقة.\\
بل و إن كل البرامج تقريباً لا تكون مكتوبة فقط داخل  حاضنتي الدالة 
\InlineCode{main}.
لحدّ الآن كانت البرامج التي نكتبها صغيرة و لهذا فهي لم تطرح أي مشكل، لكن تخيّل معي برامج ضخمة تحتوي على آلاف الأسطر من الشفرة المصدرية ! لو كانت كلّ الأسطر مكتوبة داخل حاضنتي الدالة الرئيسية لأصبحنا في السوق.

سنتعلّم الآن كيف ننظّم عملنا. سنبدأ بتقسيم برامجنا إلى قطع صغيرة (تذكّر فكرة
\textenglish{Lego}
الّتي حدّثتك عنها قبل قليل). كلّ قطعة تسمّى
\underline{دالّة}.

الدالة تقوم بتنفيذ تعليمات و تقوم بإرجاع نتيجة. هي عبارة عن 
\textbf{قطعة من الشفرة المصدرية} 
تعمل على القيام بمهمة معينة.\\
نقول بأن الدالة تملك قيمة الإدخال و قيمة الإخراج. المخطط التالي يمثل المبدأ الذي تعمل به الدالة :

\Picture{Chapter_I-9_function-io}

حينما نقوم باستدعاء دالة، نمرّ بثلاثة خطوات.

\begin{enumerate}
	\item \textbf{الإدخال}
	: نقوم بإدخال المعلومات إلى الدالة (بإعطائها معلومات تعمل عليها).
	\item \textbf{الحسابات}
	: تقوم الدالة بعمل حسابات على المعلومات التي تم ادخالها.
	\item \textbf{الإخراج}
	: بعد أن تنتهي الدالة من الحسابات تعطينا النتيجة على شكل قيمة الإخراج أو الإرجاع.
\end{enumerate}

يمكن أن نتصوّر دالة تسمى
\InlineCode{triple}
تضرب العدد في 3. بالطبع الدوال في الغالب هي أكثر تعقيداً من هذا.

\Picture{Chapter_I-9_triple-io}

هدف الدوال إذا هو تبسيط الشفرة المصدرية، لكي لا نضطر إلى إعادة كتابة نفس الشفرة المصدرية عدّة مرات على التوالي.

أُحْلُم قليلاً : لاحقاً، سننشئ مثلا دالة اسمها
\InlineCode{showWindow}
تقوم بفتح نافذة في الشاشة. ما إن نكتب الدالة (المرحلة الأصعب)، لن يتبقّ لنا سوى القول "أيتها الدالة
\InlineCode{showWindow}،
أظهري لي النافذة !". يمكننا أيضاً كتابة دالة
\InlineCode{moveCharacter}
تهدف إلى تحريك شخصية ما في اللعبة، الخ.

\subsection{مخطط دالة}

لقد تكوّنت لديك فكرة على الطريقة التي تعمل بها الدالة 
\InlineCode{main}.\\
و مع ذلك يجب أن أريك كيف نقوم بإنشاء دالة. 

الشفرة المصدرية التالية تمثّل دالة تخطيطياً. هذا نموذج للحفظ :

\begin{Csource}
type functionName(parameters)
{
	// We write the instructions here
}
\end{Csource}

أنت تعرف شكل الدالة
\InlineCode{main}.\\
إليك ما عليك فهمه بخصوص المخطط.

\begin{itemize}
	\item \InlineCode{type}
	 (نوع قيمة الإخراج) : 
	هو نوع الدالة. مثل المتغيرات، للدوال أنواعها الخاصة. هذا النوع يعتمد على القيمة التي ترجعها الدالة : إن كانت الدالة ترجع عدداً عشرياً، فسنضع بالتأكيد الكلمة المفتاحية 
	\InlineCode{double}،
	أما إن كانت ترجع عدداً صحيحاً، سنضع النوع 
	\InlineCode{int}
	أو 
	\InlineCode{long}
	مثلا. و لكن يمكن أيضاً إنشاء دوال لا ترجع أي شيء !
	هناك إذا نوعان من الدوال :
	\begin{itemize}
		\item دوال ترجع قيمة : تعطيها أحد الأنواع الّتي نعرفها كـ\InlineCode{int}
		أو 
		\InlineCode{char}
		أو 
		\InlineCode{double}،
		إلخ.
		\item دوال لا ترجع أية قيمة : نعطيها نوعا خاصا يدعى 
		\InlineCode{void}
		(و الذي يعني الفراغ).
		\item \InlineCode{functionName} :
		هو اسم الدالة. يمكنك أن تسمي الدالة مثلما تريد لطالما تحترم القواعد التي تتبعها في تسمية المتغيرات (لا للأحرف التي تحتوي على العلامات الصوتية
		(\textenglish{accents})،
		لا فراغات، إلخ).
		\item \InlineCode{parameters} :
		(هي قيم الإدخال) : داخل قوسين، يمكنك أن تبعث معاملات للدالة. هي القيم التي ستعمل بها الدالة.
		\begin{information}
			يمكنك أن تبعث القدر الذي تريد من المعاملات. كما يمكنك ألا تبعث أية معامل، و لكن نادراً ما يُستخدم هذا.
		\end{information}
		
		مثلاً، بالنسبة للدالة 
		\InlineCode{triple}،
		أنت تبعث عدداً كمعامل. الدالة "تسترجع" العدد و تضربه في العدد 3. تقوم بعد ذلك بإرجاع نتيجة حساباتها.
		\item بعد ذلك، نجد 
		\textbf{الحاضنتين}
		اللّتان تشيران إلى بداية الدالة و نهايتها. داخل الحاضنتين، تضع التعليمات التي تريدها. بالنسبة للدالة 
		\InlineCode{triple}،
		يجب أن تكتب التعليمات التي توافق ضرب قيمة الإدخال في العدد 3.
	\end{itemize}
\end{itemize}
الدالة إذا هي عبارة عن آلية تتلقّى قيم إدخال (المعاملات) و ترجع قيمة إخراج.

\subsection{إنشاء دالة}

فلنرى مثالا تطبيقيّا دون مزيد من التأخير : الدالة 
\InlineCode{triple}
التي حدثتك عنها منذ قليل. فلنقل أن هذه الدالة تتلقّى عدداً صحيحاً من نوع 
\InlineCode{int}
و أنها تُرجع عدداً صحيحاً أيضاً من نوع 
\InlineCode{int}.
هذه الدالة تضرب العدد الذي نعطيها في 3 :

\begin{Csource}
int triple (int number)
{
	int result=0;
	result=number*3; // We multiply the input number by 3
	return result; // We return the result as an output value
}
\end{Csource}

هاهي أول دالة لك ! شيء مهمّ للغاية : كما ترى، الدالة من نوع 
\InlineCode{int}.
فهي مجبرة على  أن ترجع قيمة من نوع
\InlineCode{int}.

داخل القوسين، نجد المتغيرات التي تتلقّاها الدالة. الدالة 
\InlineCode{triple}
تتلقّى متغيرا من نوع 
\InlineCode{int}
يسمى 
\InlineCode{number}.

السطر الذي يشير إلى أن الدالة تقوم بـ"إرجاع قيمة" هو السطر الذي يحتوي على الكلمة المفتاحية 
\InlineCode{return}.
هذا السطر يوجد في العادة في نهاية الدالة، بعد الحسابات.

\begin{Csource}
return result;
\end{Csource}

هذه الشفرة المصدرية تعني للدالة : "توقّفي و أرجعي العدد 
\InlineCode{result}".
\underline{يجب أن}
يكون هذا المتغير
\InlineCode{result}
من نوع
\InlineCode{int}،
لأن الدالة تقوم بإرجاع قيمة من نوع
\InlineCode{int}
كما قلت في الأعلى.

صرّحت عن (= أنشأت) المتغير 
\InlineCode{result}
في الدالة 
\InlineCode{triple}.
هذا يعني أنه لا يستخدم إلا داخل هذه الدالة و ليس داخل أخرى كالـدالة
\InlineCode{main}
مثلا. أي أنه متغير خاص بالدالة 
\InlineCode{triple}.

لكن هل هذه هي الطريقة الأقصر لكتابة الدالة
\InlineCode{triple} ؟\\
لا، يمكننا كتابة محتوى الدالة في سطر واحد كالتالي :

\begin{Csource}
int triple (int number)
{
	return number*3;
}
\end{Csource}

هذه الدالة تقوم بنفس المهمة التي تقوم بها الدالة السابقة، هي فقط أسرع من ناحية كتابتها. عموماً، الدوال التي تكتبها تحتوي الكثير من المتغيرات من أجل إجراء الحسابات عليها، نادرة هي الدوال القصيرة مثل
\InlineCode{triple}.

\subsection{العديد من المعاملات، لا معاملات}

\subsubsection{العديد من المعاملات}

الدالة
\InlineCode{triple}
تحوي معاملاً واحد، لكن من الممكن إنشاء دوال تقبل العديد من المعاملات.\\
مثلا، دالة 
\InlineCode{addition}
لجمع عددين
\InlineCode{a}
و 
\InlineCode{b} :

\begin{Csource}
int addition (int a, int b)
{
	return a+b;
}
\end{Csource}

يكفي تفريق المعاملات بفاصلة كما ترى.

\subsubsection{لا معاملات}

بعض الدوال، نادرة أكثر، لا تأخذ أية معامل كقيمة إدخال. هذه الدوال تقوم بنفس الشيء في غالب الأحيان. في الواقع، إذا لم تكن لديها أعداد تعمل عليها، فمهمة هذه الدوال هي القيام بوظائف معينة، كإظهار رسالة على الشاشة. و أيضاً، سيكون نفس النص الذي تظهره في كلّ مرة لأن الدالة لا تتلقّى أيّ معامل قد يكون قادرا على تغيير سلوكها !

تخيل دالة
\InlineCode{hello}
تقوم بإظهار الرسالة
"\textenglish{Hello}"
على الشاشة:

\begin{Csource}
void hello ()
{
	printf("Hello");
}
\end{Csource}

لم أضع أي شيء داخل الأقواس لأن الدالة لا تتلقّى أي معامل.\\
بالإضافة إلى ذلك، استعملت النوع 
\InlineCode{void}
الذي كلّمتك عنه أعلاه.

بالفعل، كما ترى فالدالة لا تحتاج إلى التعليمة 
\InlineCode{return}
لأنها لا ترجع أي شيء. الدالة التي لا ترجع أي شيء هي دالة من النوع
\InlineCode{void}.

\subsection{استدعاء دالة}

سنقوم الآن بتجريب الشفرة المصدرية للتمرّن قليلاً مع ما تعلّمناه.\\
سنستعمل الدالة 
\InlineCode{triple}
لضرب عدد في 3.

لحد الآن، أطلب منك كتابة الدالة 
\InlineCode{triple}
\underline{قبل}
الدالة 
\InlineCode{main}.
فإذا وضعتها بعدها، فلن يشتغل البرنامج. سأشرح لك هذا لاحقاً.

إليك الشفرة المصدرية التالية :

\begin{Csource}
#include <stdio.h>
#include <stdlib.h>
int triple(int number)
{
	return 3 * number;
}   
int main(int argc, char *argv[])
{
	int inputNumber = 0, tripleNumber = 0;
	
	printf("Enter a number... ");
	scanf("%d", &inputNumber);
	
	tripleNumber = triple(inputNumber);
	printf("The number's triple = %d\n", tripleNumber);
	
	return 0;
}
\end{Csource}

يبدأ البرنامج بالدالة 
\InlineCode{main}
كما تعلم.\\
نطلب من المستعمل إدخال عدد. نبعث هذا العدد كإدخال للدالة 
\InlineCode{triple}،
ثم نسترجع النتيجة في المتغير 
\InlineCode{tripleNumber}.
أنظر بشكل خاص السطر التالي:

\begin{Csource}
tripleNumber = triple(inputNumber);
\end{Csource}

داخل القوسين، نبعث المتغير كمدخل للدالة 
\InlineCode{triple}،
إنه العدد الذي ستعمل به الدالة. هذه الدالة تقوم بإرجاع قيمة، هذه القيمة نسترجعها في المتغير 
\InlineCode{tripleNumber}.
نأمر إذا الحاسوب في هذا السطر : "اُطلب من الدالة 
\InlineCode{triple}
ضرب العدد
\InlineCode{inputNumber}
في 3 و تخزين النتيجة في المتغير 
\InlineCode{tripleNumber}".

\subsubsection{نفس الشرح على شكل مخطط}

ألازالت لديك صعوبات في فهم المبدأ الذي تعمل به الدالة ؟\\
لا تقلق ! أنا متأكد أنك ستفهم بالمخططات.

هذه الشفرة المصدرية التي تحتوي على تعليقات سَتُريك في أي ترتيب يتم تنفيذ التعليمات. إبدأ إذا بقراءة السطر المرقّم 1 ثم 2، إلخ (أعتقد أنك فهمت) :

\begin{Csource}
#include <stdio.h>
#include <stdlib.h>
int triple(int number) // 6
{
	return 3 * number; // 7
}   
int main(int argc, char *argv[]) // 1
{
	int inputNumber = 0, tripleNumber = 0; // 2
	
	printf("Enter a number... "); // 3
	scanf("%d", &inputNumber); // 4
	
	tripleNumber = triple(inputNumber); // 5
	printf("The number's triple = %d\n", tripleNumber); // 8
	
	return 0; // 9
}
\end{Csource}

إليك ما يحدث سطراً بسطر :

\begin{enumerate}
	\item يبدأ البرنامج من الدالة
	\InlineCode{main}.
	\item يقرأ التعليمات في الدالة واحدة تلو الأخرى بالترتيب.
	\item يقرأ التعليمة الخاصة بالدالة
	\InlineCode{printf}.
	\item يقرأ أيضاً التعليمة الخاصة بالدالة 
	\InlineCode{scanf}.
	\item يقرأ التعليمة
	\dots
	آه نحن نستدعي الدالة 
	\InlineCode{triple}،
	يجب إذا أن نقفز إلى أول سطر من محتوى هذه الدالة في الأعلى.
	\item نقفز إلى الدالة
	\InlineCode{triple}
	 ثم نقوم باسترجاع المعامل
	\InlineCode{number}.
	\item نقوم بالحسابات و ننتهي من الدالة. الكلمة المفتاحية 
	\InlineCode{return}
	تعني أن الدالة قد انتهت و تسمح بتحديد النتيجة التي سترجعها الدالة.
	\item نرجع للدالة 
	\InlineCode{main}
	إلى التعليمة الموالية.
	\item نصل إلى 
	\InlineCode{return}
	و منه تنتهي الدالة
	\InlineCode{main}
	و ينتهي البرنامج.
\end{enumerate}

إذا فهمت في أي ترتيب يتم تنفيذ التعليمات فيه، فقد فهمت المبدأ. الآن يجب أن تفهم بأن الدالة تستقبل معاملات كمداخل و ترجع قيمة كمخرج.

\Picture{Chapter_I-9_triple-call}

\paragraph{ملاحظة :}

ليس الأمر نفسه بالنسبة لكل الدوال. أحياناً، لا تأخذ دالة أية معامل كإدخال، و بالعكس أحياناً تأخذ الكثير من المعاملات كإدخال (لقد شرحت لك هذا سابقاً).\\
أيضاً، يمكن لدالة أن ترجع قيمة كما يمكنها ألا ترجع أي شيء (و في هذه الحالة لا يكون هناك
\InlineCode{return}).

\subsubsection{فلنجرب هذا البرنامج}

هذا مثال عن تنفيذ البرنامج :

\begin{Console}
Enter a number... 10
The number's triple = 30
\end{Console}

\begin{information}
لست مضطراً أن تخزن النتيجة في متغير ! يمكنك أن تعطي النتيجة المُرجعة من طرف الدالة 
\InlineCode{triple}
إلى دالة أخرى و كأن التعليمة
\InlineCode{triple(inputNumber)}
في حدّ ذاتها متغير.
\end{information}
لاحظ هذا جيداً، هي نفس الشفرة المصدرية لكن هناك تغيير آخر على مستوى 
\InlineCode{printf}.
بالإضافة إلى ذلك، لم نقم بالتصريح عن المتغير 
\InlineCode{tripleNumber}
و الذي لا يفيدنا في أي شيء الآن :

\begin{Csource}
#include <stdio.h>
#include <stdlib.h>
int triple(int number)
{
	return 3 * nombre;
}
int main(int argc, char *argv[])
{
	int inputNumber = 0, tripleNumber = 0و
	
	printf("Enter a number... ");
	scanf("%d", &inputNumber);
	// The result returned by the function is directly sent to printf without being saved in a variable
	printf("The number's triple = %d\n", triple(inputNumber));
	
	return 0;
}
\end{Csource}

كما ترى، استدعاء الدالة 
\InlineCode{triple}
يتم داخل الدالة 
\InlineCode{printf}.\\
ماذا يفعل الجهاز حينما يصل إلى هذا السطر من الشفرة المصدرية ؟

الأمر سهل، يجد أن السطر يبدؤ بـ\InlineCode{printf}،
فسيقوم إذا باستدعاء الدالة 
\InlineCode{printf}.
يبعث إلى هذه الأخيرة كل المعاملات التي كتبناها. أول معامل هو النص الذي نريد طباعته و الثاني هو عدد.\\
يجد الجهاز بأنه قبل أن يبعث عددا إلى الدالة 
\InlineCode{printf}
عليه أولا استدعاء الدالة 
\InlineCode{triple}.
هذا ما يقوم به~: يستدعي 
\InlineCode{triple}،
يقوم بالحسابات و ما إن يتلقّ النتيجة حتّى يبعثها للدالة 
\InlineCode{printf} !

هذه الحالة تمثّل نوعاً ما تداخل الدوال. الشيء الذي نستنتجه من هذا هو أنه بإمكان دالة أن تستدعي دالة أخرى !\\
هذا هو مبدأ البرمجة بلغة
\textenglish{C} !
كلّ شيء مركّب مع الأشياء الأخرى، كما في لعبة
\textenglish{Lego}.

في النهاية، سيبقى الشيء الأصعب هو كتابة الدوال. ما إن تكتبها، لن يبق عليك سوى استدعائها دون أن تلقي بالا على العمليات التي تجري بداخلها. هذا سيسمح لك بتبسيط كتابة برامجك بشكل كبير. و صدّقني ستحتاج إلى هذه المبادئ كثيراً !

  \part{تقنيات متقدّمة في لغة الـ\textenglish{C}}
  \chapter{البرمجة المجزّأة
(\textenglish{Modular programming})}
في هذه المرحلة الثانية، سنكتشف مبادئ متقدّمة في لغة الـ
\textenglish{C}
 لن أخفي عليك، هذه المرحلة صعبة الفهم و تحتاج منك التركيز. في نهاية المرحلة، ستكون قادراً على تدبّر أمرك في معظم البرامج المكتوبة بلغة السي. في المرحلة التي تليها نتعلّم كيف نفتح نافذة، كيف ننشئ لعبة ثنائية الأبعاد... الخ

لحدّ الآن عملنا في ملف واحد سمّيناه
\InlineCode{main.c}
. كان أمراً مقبولاً لحدّ الآن لأن برامجنا كانت صغيرة، لكنها ستصبح في القريب العاجل مركّبة من عشرات، لن أقول من مئات الدوال، و إن كنت تريد وضعها كلّها في نفس الملف، فإن هذا الأخير سيصبح ضخماً جداً. لهذا السبب تم اختراع ما نسمّيه بالبرمجة المجزّأة. المبدأ سهل: بدل أن نضع كل الشفرة المصدرية في ملف واحد
\InlineCode{main.c}
، سنقوم بتفريقها إلى عدة ملفات.

\section{النماذج (\textenglish{prototypes})}
لحدّ الآن، كنت عندما تنشئ دالة، أطلب منك وضعها قبل الدالة الرئيسية
\InlineCode{main}
. لماذا؟

لأن للترتيب أهمية حقيقية هنا: فإن قمت بوضع الدالة قبل الـ
\InlineCode{main}
في الشفرة المصدرية، سيقرؤها الجهاز و يتعرف عليها. حينما تقوم باستدعاء الدالة داخل الـ
\InlineCode{main}
، سيعرفها الجهاز و يعرف أيضاً أين يبحث عليها.\\
بالعكس، لو تضع الدالة بعد الـ
\InlineCode{main}
، لن يعمل البرنامج لأن الجهاز لم يتعرّف بعد على الدالة. جرّب ذلك و سترى.
\begin{question}
  لكنه تصميم سيّء نوعاً ما، أليس كذلك ؟
\end{question}
أنا متفق معك! لكن انتبه المبرمجون لهذه النقطة قبلك و عملوا على حلّ المشكل.

بفضل ما سأعلمك إياه الآن، ستتمكن من الدوال في أي ترتيب كان في الشفرة المصدرية، هكذا لن تقلق من هذه الناحية.

\subsection{استعمال النموذج للتصريح عن دالة}
سنقوم بتصريح دوالنا للحاسوب، و هذا بكتابة ما نسميه بـ
\textbf{النماذج}
.لا تنبهر بهذا الاسم، إنه يخبّئ معلومة بسيطة جداً.

تأمل في السطر الأول من دالتنا
\InlineCode{rectangleSurface}
\begin{Csource}
double rectangleSurface(double width, double height)
{
	return width * height;
}
\end{Csource}
قم بنسخ السطر الأول
(\InlineCode{double rectangleSurface...})
المتواجد أعلى الشفرة المصدرية (مباشرة بعد تعليمات التضمين
\InlineCode{\#include}
). أضف
\textbf{فاصلة منقوطة}
في نهاية هذا السطر.\\
و هكذا يمكنك أن تضع الدالة الخاصة بك
\InlineCode{rectangleSurface}
بعد الدالة
\InlineCode{main}
ان أردت !

هذا ما يجب أن تكون عليه الشفرة المصدرية :
\begin{Csource}
#include <stdio.h>
#include <stdlib.h>
// The next line represents the prototype of the function rectangleSurface :
double rectangleSurface(double width, double height);
int main(int argc, char *argv[])
{
	printf("width = 5 and height = 10. Surface = %f\n", rectangleSurface(5, 10));
	printf("width = 2.5 and height = 3.5. Surface = %f\n", rectangleSurface(2.5, 3.5));
	printf("width = 4.2 and height = 9.7. Surface = %f\n", rectangleSurface(4.2, 9.7));

	return 0;
}
// Now, we can put our function wherever we want in the source code:
double rectangleSurface(double width , double height )
{
	return width * height ;
}
\end{Csource}
الشيء الذي تغيّر هنا هو إضافة النموذج أعلى الشفرة المصدرية.\\
النموذج هو عبارة عن إشارة للجهاز، يوحي إليه بوجود دالة تسمى
\InlineCode{rectangleSurface}
و التي تأخذ معاملات إدخال معينة و تُرجِع مخرج من نوع أنت من تحدده.  هذا يساعد الجهاز على تنظيم نفسه.

بفضل ذلك السطر، يمكنك الآن وضع دوالك في أي ترتيب كان دون أي تفكير زائد.

أكتب دائما النموذج الخاص بدوالك. البرامج التي ستكتبها من الآن و صاعداً ستصبح أكثر تعقيداً و تستعمل الكثير من الدوال: من الأحسن أن تتعلّم منذ الآن العادة الجيدة  بوضع نموذج لكل دالة في الشفرة المصدرية.

كما ترى، الدالة
\InlineCode{main}
لا تملك أي نموذج، و كمعلومة فهي الوحيدة التي لا تملك نموذجاً ! لأن الجهاز يعرفها (فهي نفسها مكررة في جميع البرامج).

عليك أن تعرف أنه في سطر النموذج، لست مضطراً إلى تحديد المعاملات التي تتلقاها الدالة كمدخل. الجهاز يحتاج أن يتعرّف إلى نوع المداخل فقط.

يمكننا أن نكتب ببساطة :
\begin{Csource}
double rectangleSurface (double, double);
\end{Csource}
و مع ذلك، فالطريقة التي أريتك إياها أعلاه تعمل أيضاً. الشيء الجيد فيها هو أن كلّ ما عليك فعله هو نسخ و لصق السطر الأول الخاص بالدالة مع إضافة فاصلة منقوطة (طريقة سهلة و سريعة).
\begin{critical}
  لا تنس
\underline{أبدا}
وضع فاصلة منقوطة بعد النموذج، هذا يمكّن الحاسوب من التفريق بين النموذج و بداية الدالة.\\
إن لم تفعل، ستعترضك أخطاء غير مفهومة أثناء عملية الترجمة.
\end{critical}

\section{الملفات الرأسية
(\textenglish{headers})}
لحدّ الآن لا نملك غير ملف مصدري واحد في مشروعنا و هو الذي كنّا نسمّيه
\InlineCode{main.c}.

\subsection{عدة ملفات في مشروع واحد}
تطبيقياً، برامجك لن تكون مكتوبة في ملف واحد
\InlineCode{main.c}.
بالطبع يمكن فعل ذلك، لكن لن يكون من الممتع أن تتجوّل في ملف به 10000 سطر (شخصياً أعتقد هذا). و لهذا فإنه في العادة ننشئ العديد من الملفات في المشروع الواحد.
\begin{question}
  عفوا ... ماهو المشروع ؟
\end{question}
لا ! هل نسيت بسرعة ؟ سأعيد الشرح لأنه من اللازم أن نتّفق على هذا المصطلح.

المشروع هو مجموع الملفات المصدرية الخاصة ببرنامجك. لحد الآن برنامجنا لم تتكون إلا من ملف واحد. و يمكنك التحقق من هذا بالنظر في البيئة التطويرية الخاصة بك، غالبا ما يظهر المشروع في القائمة على اليسار (الصورة الموالية):
\Picture{Chapter_II-1_Project}
كما يمكنك رؤيته في يسار الصورة، هذا المشروع ليس مكوّنا إلا من الملف
\InlineCode{main.c}.

إسمح لي الآن أن أُرِيَكَ صورة لمشروع حقيقي ستقوم به في وقت لاحق من الدروس : لعبة سوكوبان (الصورة الموالية) :
\Picture{Chapter_II-1_Project-Sokoban}
كما ترى، هناك ملفات عديدة. هذا ما يكون عليه المشروع الحقيقي، أي تتواجد به ملفات عديدة في القائمة اليسارية  يمكن التعرّف على الملف
\InlineCode{main.c}
من بين القائمة و الذي يحتوي الدالة
\InlineCode{main}.
بصورة عامة في برامجي، لا أضع إلّا الدالة
\InlineCode{main}
في الـملف
\InlineCode{main.c}.
لمعلوماتك، هذا ليس أمراً إجبارياً، كل واحد ينظّم ملفاته بالشكل الذي يريد. لكن لكي تتبعني جيّداً أنصحك بفعل ذلك.
\begin{question}
  لكن لم يجب عليّ إنشاء ملفات عديدة ؟ و كم من ملف يجب علىّ أن أنشئ في مشروعي ؟
\end{question}
هذا يبقى اختيارك أنت، في الغالب نجمع في نفس الملف المصدري الدوال التي تشترك في الموضوع الذي تعالجه. و هكذا ففي الملف
\InlineCode{editeur.c}
جمعت كلّ الدوال الخاصة ببناء المستوى، و في الملف
\InlineCode{jeu.c}
قمت بتجميع الدوال الخاصة باللعبة نفسها و هكذا ...

\subsection{الملفات
\texttt{\textenglish{.c}}
و
\texttt{\textenglish{.h}}}
كما يمكنك أن تلاحظ، يوجد نوعان مختلفان من الملفات في الصورة السابقة.
\begin{itemize}
  \item \textbf{ملفات ذات الإمتداد
\texttt{\textenglish{.c}}}
: الملفات المصدرية، تحتوي الدوال نفسها.
  \item \textbf{ملفات ذات الإمتداد
\texttt{\textenglish{.h}}}
: تسمى الملفات الرأسية و هي تحتوي النماذج الخاصة بالدوال.
\end{itemize}
عموما، انه لمن النادر وضع نماذج في الملفات من صيغة
\InlineCode{.c}
مثلما فعلنا للتوّ في الـملف
\InlineCode{main.c}
(إلا إذا كان برنامجك صغيرا).

من أجل كل ملف
.\InlineCode{c}
هناك ملف مكافئ له، و الذي يحتوي نماذجا للدوال الموجودة في الملف
\InlineCode{.c}
، تمعّن في الصورة السابقة.
\begin{itemize}
  \item هناك
\InlineCode{editeur.c}
(الشفرة الخاصة بالدوال) و
\InlineCode{editeur.h}
(ملف النماذج الخاصة بالدوال).
  \item هناك
\InlineCode{jeu.c}
و
\InlineCode{jeu.h}.
  \item إلخ...
\end{itemize}
\begin{question}
  لكن كيف يعرف الحاسوب بأن نماذج الدوال موجودة في ملف آخر خارج الملف
\InlineCode{.c}
؟
\end{question}
يجب عليك تضمين الملف الرأسي
\InlineCode{.h}.
مستعيناً بتوجهات المعالج القبلي.\\
كن مستعداً لأنّي سأعطيك الكثير من المعلومات في وقت قصير.

كيف نقوم بتضمين ملف رأسي ؟ أنت تجيد فعل ذلك لأنك قمت بذلك من قبل.

أنظر مثالاً من بداية الملف
\InlineCode{jeu.c} :
\begin{Csource}
#include <stdlib.h>
#include <stdio.h>
#include "jeu.h"
void play(SDL_Surface* screen)
{
// ...
\end{Csource}
التضمين يتم عن طريق توجيهات المعالج القبلي
\InlineCode{\#include}
التي يجدر بك أن تكون قد تعلّمتها من قبل.\\
تمعن في التالي :
\begin{Csource}
#include <stdlib.h>
#include <stdio.h>
#include "jeu.h" // We include jeu.h
\end{Csource}
قمنا بتضمين ثلاثة ملفات من صيغة
\InlineCode{.h}
و هي :
\InlineCode{stdio}، \InlineCode{stdlib} و \InlineCode{jeu}.\\
لاحظ الفرق : الملفات التي قمت بإنشاءها ووضعها في الـمجلّد الخاص بمشروعك يجب أن تكون مضمّنة مع اشارات الاقتباس
(\InlineCode{"jeu.h"})
بينما ملفات المكتبات (التي توجد عادة في البيئة التطويرية الخاصة بك) تكون مضمّنة بعلامات الترتيب
(\InlineCode{<stdio.h>}).

تستعمل إذا :
\begin{itemize}
  \item علامتي الترتيب
\InlineCode{< >}
: لتضمين الملفات المتواجدة في المجلّد
\InlineCode{include}
الخاص بالبيئة التطويرية.
  \item علامتي الاقتباس
\InlineCode{" "}
 : لتضمين  الملفات المتواجدة في مجلّد المشروع (و غالبا بجانب الملفات
\InlineCode{.c}).
\end{itemize}
الأمر
\InlineCode{\#include}
يطلب إدخال محتوى ملف معيّن في الملف
\InlineCode{.c}
فهي تعليمة تقول :"أدخل الملف
\InlineCode{jeu.h}
هنا" مثلا .

\begin{question}
  و في الملف
\InlineCode{jeu.h}
ماذا نجد ؟
\end{question}
لا نجد إلا نماذج خاصة بدوال الملف
\InlineCode{jeu.c}
!
\begin{Csource}
void play(SDL_Surface* screen);
void movePlayer(int map[][NB_BLOCS_HEIGHT], SDL_Rect *pos, int direction);
void moveBox(int *firstBox, int *secondeBox);
\end{Csource}
هكذا يعمل المشروع الحقيقي !
\begin{question}
  ما الهدف من وضع نماذج في ملفات من نوع
  \InlineCode{.h}
  ؟
\end{question}
السبب بسيط للغاية، عندما تستدعي دالة في الشفرة المصدرية الخاصة بك، ينبغى لجهازك أن يكون متعرفا عليها من قبل، و يعرف كم من المعاملات تستعمل...الخ. إن هذا هو الهدف وراء وجود النماذج، انه دليل الاستخدام الخاص بالدالة بالنسبة للجهاز.

كلّ هذا هو مسألة تنظيم، عندما تضع نماذجك في ملفات
\InlineCode{.h}
(ملفات رأسية) مضمّنة في أعلى الملفات
\InlineCode{.c}
، سيعرف جهازكم طريقة استخدام الدوال الموجودة في الملف ما إن يبدأ في قراءته.

عند القيام بهذا، لن يكون عليك القلق حيال الترتيب الذي ستكون عليه دوالك في الملفات
\InlineCode{.c}.
اذا كنت قمت الآن بإنشاء برنامج صغير يحتوي على دالتين أو ثلاث يمكنك أن تفكّر أنه من الممكن للبرنامج أن يتشغل دون وجود النماذج، لكن هذا لن يستمر ذلك طويلا ! فما إن يكبر البرنامج و إن لم تنظّم النماذج في ملفات رأسيّة فستفشل الترجمة دون أدنى شك.
\begin{information}
  عندما تستدعي دالة متواجدة في الملف
  \InlineCode{functions.c}
  إنطلاقا من الملف
  \InlineCode{main.c}
  سيكون عليك تضمين النماذج الخاصة بالملف
  \InlineCode{functions.c}
  في الملف
  \InlineCode{main.c}
  يجب إذن وضع
  \InlineCode{\#include "functions.h"}
  في أعلى الـملف
  \InlineCode{main.c}.\\
  تذكر هذه القاعدة : "في كلّ مرة تستدعي الدالة
  \textenglish{X}
  في ملف، يجب عليك إدراج نموذج هذه الدالة في ملفكم" هذا ما يسمح للـمترجم بمعرفة ما إن كنت قد استدعيتها بشكل صحيح.
\end{information}
\begin{question}
  كيف أقوم بإضافة ملفات
\InlineCode{.c}
 و
\InlineCode{.h}
 إلى مشروعي ؟
\end{question}
هذا راجع للـبيئة التطويرية التي تستخدمها. لكن المبدأ هو نفسه في جميع البرامج :
\InlineCode{File} / \InlineCode{New} / \InlineCode{Source File}\\
هذا يسمح بإنشاء ملف جديد فارغ. هذا الملف ليس حاليا من النوع
\InlineCode{.c}
ولا
\InlineCode{.h}
أنت من يحدد ذلك أثناء عملية حفظ الملف. قم إذن بحفظه (حتّى و إن كان لا يزال فارغا !) و هنا يطلب منكم إدخال اسم للملف، يمكنك هنا اختيار صيغة الملف :
\begin{itemize}
  \item إذا سميته
\InlineCode{file.c}
فسيكون بامتداد
\InlineCode{.c}.
  \item إذا سميته
\InlineCode{file.h}
فسيكون بامتداد
\InlineCode{.h}.
\end{itemize}
هذا سهل. قم بحفظ الملف في المجلّد أين تتواجد باقي الملفات الخاصة بمشروعك (نفس المجلّد أين يتواجد الملف
\InlineCode{main.c}). عموما كل ملفات المشروع تقوم بحفظها في نفس المجلّد سواء كانت ذات صيغة
\InlineCode{.c}
أو
\InlineCode{.h}.

مجلّد المشروع في النهاية سيكون مثل هذا :
\Picture{Chapter_II-1_Project-Sokoban-Folder}
الملف الذي أنشأته محفوظ لكن لم تتم إضافته إلى مشروعك بعد !\\
لإضافته قم بالنقر يمينا على القائمة أيسر الشاشة (الخاصة بملفات المشروع) و اختر
\InlineCode{Add files}
كالتالي :
\Picture{Chapter_II-1_Project-Add-File}
ستظهر لك نافذة تطلب منك اختيار الملفات التي تريد أن تدخلها للمشروع، اختر الملف الذي قمت بإنشاءه، للتو، و سيتم إدخاله أخيرا في المشروع.  ستجده حاضراً في القائمة اليسارية !

\subsection{الـ
\texttt{include}
الخاصّة بالمكتبات النموذجية}
يفترض أنّ لديكم سؤالا يدور في رؤوسكم الآن...\\
إذا ضمّنا الملفات
\InlineCode{stdio.h}
و
\InlineCode{stdlib.h}
فهذا يعني أنهما موجودان في مكان ما و يمكننا البحث عنهما، أليس كذلك ؟

نعم بالطبع !\\
يفترض أنهما مسطبان في المكان الذي تتواجد به البيئة التطويرية الخاصة بك، بالنسبة للبيئة
\textenglish{Code::Blocks}
أجدهم هنا :

\InlineCode{C:\textbackslash Program Files\textbackslash CodeBlocks\textbackslash MinGW\textbackslash include}

على العموم يجب البحث عن مجلد يحمل اسم
\InlineCode{include}.\\
بداخله تجد كمّا هائلا من الملفات، و هي ملفات رأسية
(\InlineCode{.h})
خاصة بمكتبات نموذجية أي مكتبات متوفرة في كل مكان (سواء في الويندوز أو الماك أو اللينكس...)، و ستجد داخلها الملفات
\InlineCode{stdio.h}
و
\InlineCode{stdlib.h}
مع ملفّات أخرى.

يمكنك فتحها إذا أردت، لكن ستتفاجئ بالعديد من الأشياء التي لم أدرّسها لك من قبل خاصة بما يتعلق ببعض توجيهات المعالج القبلي. يمكنك أن تلاحظ بأن الملف مليء بنماذج لدوال نموذجية مثل
\InlineCode{printf}.
\begin{question}
  حسناً، الآن عرفت أين أجد نماذج الدوال النموذجية لكن ألا يمكنني رؤية الشفرة المصدرية الخاصة بالدوال ؟ أين هي الملفات
\InlineCode{.c}
؟
\end{question}
إنها غير موجودة أساسا ، لأنها مترجمة (إلى ملفات ثنائية
(\textenglish{binary files})
يعني إلى لغة الحاسوب). و لهذا فإنه من المستحيل أن تقرأها.

يمكنك إيجاد الملفات المترجمة في المجلّد المسمى
\InlineCode{lib}
(و الذي هو اختصار لكلمة
\InlineCode{library}
أي مكتبة)، بالنسبة لي هي موجودة في المسار :

\InlineCode{C:\textbackslash Program Files\textbackslash CodeBlocks\textbackslash MinGW\textbackslash lib}

ملفات المكتبات المترجمة لها الصيغة
\InlineCode{.a}
في البيئة
\textenglish{Code::Blocks}
و التي تستخدم
\InlineCode{mingw}
كمترجم. و لها صيغة
\InlineCode{.lib}.
في برنامج
\textenglish{Visual C++}
الذي يستخدم المترجم
\InlineCode{Visual}.
لا تحاولوا قراءتها لأنها غير قابلة للقراءة من طرف إنسان عادي.

باختصار، يجب عليك أن تضمّن الملفات الرأسية
\InlineCode{.h}
في الملفات
\InlineCode{.c}
تتمكن من استخدام الدوال النموذجية مثل
\InlineCode{printf}
و كما تعرف فالجهاز على اطلاع على النماذج فهو يعرف ما إن كنت قد طلبت الدوال بشكل صحيح (إن لم تنس أحد المعاملات مثلا).

\section{الـترجمة المنفصلة}
الآن و بعدما عرفت أن المشروع مبنى على أساس ملفات مصدرية عديدة، يمكننا الدخول الآن في تفاصيل عملية الترجمة فلحد الآن لم نر سوى مخطط مبسط عنها.

سأعطيك الآن مخططا مفصلا عنها و من المستحسن أن تحفظه عن ظهر قلب :
\Picture{Chapter_II-1_Compilation-Schema}
هذا مخطط حقيقي عمّا يجري بالضبط أثناء التجميع و سأشرحه لكم :
\begin{enumerate}
  \item \textbf{المعالج القبلي} :
المعالج القبلي هو برنامج ينطلق قبل عملية الترجمة و هو مخصص للقيام بتشغيل تعليمات نطلبها منه عن طريق ما سميناه بتوجيهات المعالج القبلي، و هي الأسطر الشهيرة التي تبدأ يإشارة
\InlineCode{\#}.

لحد الآن توجيهة المعالج القبلي الوحيدة الّتي نعرفها هي
\InlineCode{\#include}
و الّتي تسمح بإدراج ملف في ملف آخر. طبعا للمعالج القبلي مهام اخرى سنتعرف إليها لاحقا لكن ما يهمنا الآن هو ما أعطيتك إياه.
لمعالج القبلي يقوم إذن بـ"استبدال" أسطر
\InlineCode{\#include}
بملفات أخرى نحددها، فهو يضمّن داخل كل الملفات
\InlineCode{.c}
الملفات
\InlineCode{.h}
التي نعينها و نطلب منه تضمينها في السابقة.
  \item \textbf{الترجمة} : هذه الخطوة المهمة التي تسمح بتحويل ملفاتك إلى شفرات ثنائية مفهومة للحاسوب. فالمترجم ييقوم بتجميع الملفات
\InlineCode{.c}
واحدا بواحدا حتى ينهيها جميعها، و لضمان ذلك يجب أن تكون كل الملفات موجزدة في المشروع (بحيث تظهر في القائمة اليسارية).

سيقوم المترجم بتوليد ملف
\InlineCode{.o}
أو
\InlineCode{.obj}
و هذا راجع لنوع المترجم و هي ملفات ثنائية مؤقتة، و على أي حال تحذف هذه الملفات في نهاية الـترجمة و لكن بتعديل الخيارات يمكنك الإبقاء عليها لكن لكن يكون هناك من داع.
  \item \textbf{إنشاء الروابط} :
محرر الروابط
(\textenglish{Linker})
هو برنامج يعمل على جمع الملفات الثنائية من نوع
\InlineCode{.o}
في ملف واحد كبير : الملف التنفيذي النهائي ! هذا الملف يحمل الصيغة
\InlineCode{.exe}
في الويندوز. إن كنت تملك نظام تشغيل آخر فسيأخذ الصيغة المناسبة له.
\end{enumerate}
و هكذا تكون قد تعرفت على الطريقة الحقيقية لعمل الترجمة. كما قلتها و أكررها المخطط أعلاه مهم للغاية، فهو يفرق بين مبرمج يقوم بجمع و نسخ الشفرة المصدرية دون فهم و بين مبرمج يعرف تماما ما عليه فعله !

معظم الأخطاء تحدث في الـترجمة و قد تأتي من محرر الروابط و هذا يعني أنه لم يتمكن من تجميع كل الملفات
\InlineCode{.o}
بطريقة صحيحة (ربمّا لفقدان إحداها).

لا يزال المخطط أعلاه غير كامل، إذ أن المكتبات لم تظهر فيه ! إذن كيف تحدث العملية عندما نستخدم مكتبات برمجية ؟

تبقى بداية المخطط هي نفسها، لكن يقوم محرر الروابط بأعمال اخرى، سيقوم بتجميع ملفاتك
\InlineCode{.o}
(المؤقتة) مع مكتبات جاهزة تحتاجها (
\InlineCode{.a}
أو
\InlineCode{.lib}
وفقا للمترجم) :
\Picture{Chapter_II-1_Compilation-Schema-Libraries}
هكذا ننتهي و يكون مخططنا هذه المرة كاملا، ملفاتك من المكتبات
\InlineCode{.a}
(أو
\InlineCode{.lib})
يتم تجميعها في الملف التنفيذي مع الملفات
\InlineCode{.o}.

فبهذه الطريقة نتحصل في النهاية على برنامج كامل
100\%
و الذي يحتوي كل التعليمات اللازمة للجهاز لتشرح له كيف يعرض نصّا !\\
كمثال، الدالة
\InlineCode{printf}
توجد في ملف
\InlineCode{.a}
و طبعا سيتم تجميعها مع الشفرة المصدرية الخاصّة بنا في الملف التنفيذي.

لاحقا سنتعلم كيف نستخدم المكتبات الرسومية التي نجدها أيضا في ملفات
\InlineCode{.a}
و تعطي للجهاز تعليمات خاصة بكيفية إظهار نافذة على الشاشة كمثال. لكن طبعا، لن ندرسها الآن ، كلّ شيء في وقته.

\section{نطاق الدوال و المتغيرات}
لننهي هذا الدرس، يجب أن أطلعكم عما يسمى بـ
\textbf{نطاق}
المتغيرات و الدوال، سنعرف إمكانية الوصول للدوال و المتغيرات، يعني متى يمكننا استدعاؤها.

\subsection{المتغيرات الخاصة بدالة}
عندما تصرّح عن متغير في داخل دالة يتم حذف هذا المتغير من الذاكرة مع نهاية الدالة.
\begin{Csource}
int triple(int number)
{
	int result = 0; // The variable result is created in the memory
	result = 3 * number;
	return result;
} // The function finished, the variable result is destroyed
\end{Csource}
كلّ متغير تم التصريح عنه في دالة، لا يكون موجودا سوى حينما تكون الدالة في طور الإشتغال.\\
لكن ماذا يعني هذا تحديدا ؟ أنه لا يمكن الوصول إليه من خلال  دالة اخرى !
\begin{Csource}
int triple(int number);
int main(int argc, char *argv[])
{
	printf("The triple of 15 = %d\n", triple(15));
	printf("The triple of 15 = %d",  result); // Error
	return 0;
}

int triple(int number)
{
	int result = 0;
	result = 3 * number;
	return result;
}
\end{Csource}

في الدالة الرئيسية أحاول الوصول إلى المتغير
\InlineCode{result}
و بما أن هذا المتغير تم التصريح عنه داخل الدالة
\InlineCode{triple}
فطبعا لا يمكنني الوصول إليه من خلال الدالة
\InlineCode{main} !

\textbf{تذكّر جيّدا}
: كل متغير تم التصريح عنه داخل دالة، لا يسرى مفعوله إلا في داخل هذه الدالة نفسها ! و نقول أن المتغير محلّي
(\textenglish{local}).

\subsection{المتغيرات الشاملة
(\textenglish{global variables})
: فلتتجنّبها}
\subsubsection{متغير شامل قابل للوصول إليه من خلال كلّ الملفات}
إنه من الممكن التصريح عن متغير يمكن الوصول إليه من خلال كل الدوال من ملفات المشروع.
سأريك كيفية فعل ذلك كي تعرف بأنه أمر موجود، لكن عموما تجنب القيام بذلك.
د يظهر أنها ستسهل لك التعامل مع الشفرة المصدرية لكن قد يؤدي بك هذا لوجود العديد من المتغيرات التي يمكننا الوصول إليها من كلّ مكان مما سيصعّب عليك عملية إدارتها.

للتصريح عن متغير
\underline{شامل}
(\textenglish{global})،
يجب أن تقوم بذلك خارج كلّ الدوال، يعني في أعلى الملف، و عموما بعد أسطر الـ
\InlineCode{\#include}.
\begin{Csource}
#include <stdio.h>
#include <stdlib.h>

int result = 0; // Declaration of a global variable
void triple(int number ); // Prototype of the function
int main(int argc, char *argv[])
{
	triple(15); // We call the function triple which is going to modify the variable result
	printf("The triple of 15 = %d\n", result); // We can access to the variable result
	return 0;
}

void triple(int number)
{
	result = 3 * number;
}
\end{Csource}
في هذا المثال، الدالة
\InlineCode{triple}
لا تُرجع أي شيء
(\InlineCode{void}).
إنها تقوم بتعديل قيمة المتغير الشامل
\InlineCode{result}
التي يمكن للدالة
\InlineCode{main}
أن تسترجعه.

المتغير
\InlineCode{result}
يمكن الوصول إليه من خلال كل الملفات في المشروع و منه يمكننا مناداتها من خلال
\underline{كلّ}
دوال البرنامج.
\begin{warning}
  هذا شيء يجب ألا يتواجد في برامج الـ
\textenglish{C}
الخاصة بك. من المستحسن استعمال التعليمة
\InlineCode{return}
لإرجاع النتيجة بدل التعديل عليه كمتغير شامل.
\end{warning}

\subsubsection{متغير شامل قابل للوصول إليه من خلال ملف واحد}
المتغير الشامل الذي أريتك إياه قبل قليل يمكن الوصول إليه من خلال كل الملفات الخاصة بالمشروع.\\
يمكننا جعل متغيّر شامل مرئيا فقط في الملف الذي تتواجد به. و لكنه يبقى متغيرا شاملا على أية حال حتى و إن كنا نقول أنه ليس كذلك إلا على الدوال المتواجدة في ذات الملف و ليس على كل دوال البرنامج.

لإنشاء متغير شامل مرئي في ملف واحد نستعمل الكلمة المفتاحية
\InlineCode{static}
قبله :
\begin{Csource}
static int result = 0;
\end{Csource}

\subsubsection{متغير ساكن
(\textenglish{static})
بالنسبة لدالة}
حذار: الأمر حساس هنا قليلاً. إن استعملت الكلمة المفتاحية
\InlineCode{static}
عند التصريح عن متغير في داخل دالة، فهذا معنى آخر غير الخاص بالمتغيرات الشاملة.\\
في هذه الحالة، لا يتم حذف المتغير الساكن مع نهاية الدالة، بل حينما نستدعي الدالة مرّة أخرى، سيحفظ المتغير قيمته مثلا :
\begin{Csource}
int triple(int number)
{
	static int result = 0; // The first time when the variable is created
	result = 3 * number;
	return result;
} // when we exit the function, the variable is not destroyed
\end{Csource}
ماذا يعني هذا بالضبط ؟\\
يعني أنه يمكننا استدعاء الدالة لاحقا و يبقى المتغير
\InlineCode{result}
محتفظا بنفس القيمة الاخيرة.

و هذا مثال آخر للفهم أكثر :
\begin{Csource}
int increment();

int main(int argc, char *argv[])
{
	printf("%d\n", increment());
	printf("%d\n", increment());
	printf("%d\n", increment());
	printf("%d\n", increment());

	return 0;
}

int increment()
{
	static int number= 0;

	number++;
	return number;
}
\end{Csource}
\begin{Console}
1
2
3
4
\end{Console}
هنا، في المرة الأولى التي نطلب فيها الدالة
\InlineCode{increment}،
يتم انشاء المتغير
\InlineCode{number}.
ثم نقوم بزيادة 1 إلى قيمته. و ما إن تنتهي الدالة لا يمسح المتغير.

عندما نطلب الدالة للمرة الثانية، يتم ببساطة قفز السطر الخاص بالتصريح بالمتغير، و لا نقوم بإعادة إنشاء المتغير بل فقط نعيد استعمال المتغير الذي أنشأناه سابقا. عندما يأخذ المتغير القيمة 1، تصبح قيمته 2 ثم 3 ثم 4 ... الخ.

هذا النوع من المتغيرات ليس مستعملا بكثرة، لكن يمكنه مساعدتك في بعض الأحيان و لهذا ذكرته في الدرس.

  \chapter{المؤشّرات}
لقد حان الوقت لنكتشف المؤشرات. خذ نفسا عميقا قبل أن تقرر قراءة هذا الدرس لأنه لن يكون درساً للهو و المرح. تمثل المؤشرات واحداً من المبادئ الأكثر أهمية و حساسية في لغة الـ
\textenglish{C}
. و إن كنت أصرّ على أهميتها فهذا  لأنه لا يمكنك إالبرمجة بـ
\textenglish{C}
دون معرفتها و فهمها جيّدا. المؤشرات موجودة في كلّ مكان، و لقد استعملتها من قبل دون أن تعلم بذلك.

كثير من المتعلّمين يصلون إلى المؤشرات و يواجهون صعوبات في فهمها. سنعمل على ألا يكون الأمر مماثلا بالنسبة لك. ضاعف التركيز و خذ الوقت اللازم لفهم المخططات التي سأقدمها لك في هذا الفصل.

\section{مشكل مضجر بالفعل}
واحد من أكبر المشاكل مع المؤشرات هي أنّه بالإضافة إلى أنها صعبة الاستيعاب قليلا بالنسبة للمبتدئين، فإن المتعلّم لا يعرف ما هي أهميتها و فيما يمكننا استعمالها.

يمكنني أن أقول لك بأن "المؤشرات لا يمكن الاستغناء عنها في أي برنامج
\textenglish{C}
، صدقني !"، لكنّي أعرف أن هذه الحجّة ليست كافية لك.

سأطرح عليك مشكلاً لا يمكنك حلّه إلا باستخدام المؤشرات. سيكون هذا مقدّمتنا في هذا الفصل. سنعود إليه في نهاية هذا الفصل و سنترون حلّه باستعمال ما تعلّمتموه في هذا الفصل.

إليك المشكل: أريد كتابة دالة تقوم بإرجاع قيمتين مختلفتين. ستجيبني "هذا مستحيل !". بالفعل، الدالة لا يمكنها ارجاع سوى قيمة واحدة.
\begin{Csource}
int function()
{
	return value;
}
\end{Csource}
اذا استخدمنا
\InlineCode{int}
ترجع لنا قيمة من نوع
\InlineCode{int}
(بفضل التعليمة
\InlineCode{return}).

يمكننا ايضا كتابة دالة لا تُرجع اية قيمة باستخدام الكلمة المفتاحية
\InlineCode{void}.
\begin{Csource}
void function()
{

}
\end{Csource}
لكن إرجاع قيمتين مختلفتين في نفس الوقت... هذا أمر مستحيل لأنه لا يمكننا استعمال تعليمتي
\InlineCode{return}.

لنفرض أنني أريد كتابة دالة أعطيها كمدخل عددا من الدقائق. تقوم الدالة بإرجاع عدد الساعات و الدقائق المواقفة لها.
\begin{itemize}
  \item إذا أعطيت القيمة 45  الدالة ترجع 0 ساعة و 45 دقيقة.
  \item إذا أعطيت القيمة 60 الدالة ترجع القيمة  1 ساعة و 0 دقائق.
	\item إذا أعطيت القيمة 90 الدالة ترجع القيمة  1 ساعة و 30 دقائق.
\end{itemize}
نكن مجانين و لنجرّب ذلك :
\begin{Csource}
#include <stdio.h>
#include <stdlib.h>

/* I put the prototype at the top.
Because it's a short code, I don't put it in a .h file.
In a real program I would have put the prototype
in a separate .h file of course */

void minutesDevision(int hours, int minutes);

int main(int argc, char *argv[])
{
	int hours = 0, minutes = 90;

/* We have a variable "minutes" equals to 90.
   after calling the function, I want from the variable
“hours" to take the value 1 and from my variable
"minutes" to take the value 30 */

	minutesDivision(hours, minutes);
	printf("%d hours and %d minutes", hours, minutes);
	return 0;
}

void minutesDevision(int hours, int minutes)
{
	hours = minutes / 60;  // 90 / 60 = 1
	minutes = minutes % 60; // 90 % 60 = 30
}
\end{Csource}
النتيجة :
\begin{Console}
0 hours and 90 minutes
\end{Console}
لم تشتغل ! ما الأمر يا ترى ؟
في الواقع، عندما نبعث متغيرا إلى دالة، يتم إنشاء نسخة من المتغير، و لهذا فالمتغير
\InlineCode{hours}
في الدالة
\InlineCode{minutesDevision}
هو ليس نفسه الذي في الدالة
\InlineCode{main} !
إنه فقط نسخة !

الدالة
\InlineCode{minutesDevision}
تقوم بعملها. ففي داخلها المتغيران
\InlineCode{hours}
و
\InlineCode{minutes}
يحملان القيمتين الصحيحتين : 1 و 30.

لكن بعد ذلك، تتوقف الدالة مباشرة عند الوصول إلى الحاضنة الغالقة، مثلما تعلمنا سابقا: المتغيرات الخاصة بدالة يتم حذفها مباشرة عند انتهاء الدالة. اذن النسخ عن المتغيرات
\InlineCode{minutes}
و
\InlineCode{hours}
تُمسح.
نرجع بعد ذلك للدالة
\InlineCode{main}.
و التي فيها متغيراتنا
\InlineCode{minutes}
و
\InlineCode{hours}
تحملان القيمتين 0 و 90. لقد فشلنا !
\begin{information}
	لاحظ إذن،بما أن الدالة تقوم بنسخ المتغيرات التي نعطيها لها، لست مضطراً لستمية متغيراتك بنفس الأسماء التي تحملها في الـدالة الرئيسية
\InlineCode{main}.
و بالتالي يمكنك ببساطة كتابة :

\InlineCode{void minutesDivision(int h, int m)}\\
\InlineCode{h}
للساعات و
\InlineCode{m}
للدقائق.\\
إن كانت متغيراتك لم تسمّى بنفس الطريقة في الدالة و في
\InlineCode{main}
فهذا لا يطرح أيّ مشكل !
\end{information}


باختصار، يمكنك إعادة المشكل في كلّ الاتجاهات. يمكنك محاولة بعث قيمة باستخدام الدالة (باستخدام
\InlineCode{return}
و باستخدام النوع
\InlineCode{int}
للدالة) ، لكن لا يمكنك إعادة أكثر من قيمة واحدة من بين القيمتين، هذا مشكل مطروح إذن. كما لا يمكنك استعمال متغيرات شاملة لأن هذا أمر غير مستحسن إطلاقا.

حسناً المشكل لازال مطروحاً، كيف يمكننا حلّه باستخدام المؤشرات ؟

\section{الذاكرة، مسألة عنوان}
\subsection{تذكير بالمكتسبات القبليّة}
سأعود بك قليلاً إلى الوراء، هل تتذكر درس المتغيرات ؟

أيّا كانت إجابتك، أنصحك بأن تعود إلى ذلك الفصل و تقرأ منه الجزء الذي يحمل عنوان (مسألة ذاكرة). هناك مخطط مهم جداً سأقترحه عليك من جديد (الصورة الموالية) :
\Picture{Chapter_II-2_RAM-Schema}
هكذا نقوم تقريباً بتمثيل الذاكرة الحية (الرام) الخاصة بالحاسوب.

يجب قراءة المخطط سطراً بسطر، السطر الأول يمثل "الخانة" الخاصة بأول الذاكرة. لكل خانة رقم، هذا الرقم يمثل
\textbf{عنوانها}
(تذكّر هذا المصطلح جيدا). تحتوي الذاكرة على عدد كبير جداً من العناوين تبدأ من الرقم 0 و تنتهي بالرقم
\textit{(ضع رقماً كبيراً جداً هنا)}.
عدد العناوين التي تتوفر عليها تعتمد على حجم الذاكرة التي يحتوي عليها الجهاز الخاص بك.

في كلّ عنوان يمكننا تخزين عدد واحد فقط. لا يمكننا تخزين عددين في نفس العنوان.

الذاكرة ليست مصنوعة سوى لتخزين الأعداد. لا يمكنها تخزين لا حروف و لا جُمل. و للتخلص من هذا المشكل تم اختراع جدول يقوم بالربط بين الحروف و الأعداد. يقول الجدول مثلاً :"العدد 89 يمثّل الحرف
\textenglish{Y}".
سنعود في درس لاحق إلى عملية التحكّم في الحروف. حاليّا، سنتكفي بالتكلّم عن عمل الذاكرة.

\subsection{عنوان و قيمة}
حينما تنشئ متغيراً
\InlineCode{age}
من نوع
\InlineCode{int}
مثلا، بكتابة :
\begin{Csource}
int age = 10;
\end{Csource}
يطلب البرنامج من نظام التشغيل (الويندوز مثلا) الإذن لاستعمال جزء من الذاكرة. نظام التشغيل يجيب بالإشارة إلى أي عنوان سيسمح لك بتخزين العدد.

هنا تكمن أحد أهم وظائف نظام التشغيل : نقول أنه يحجز الذاكرة للبرامج. يمكننا القول أنه هو القائد، يتحكم في كلّ برنامج و يتأكد من أن هذا الأخير له الإذن لاستعمال الذاكرة في المكان الذي يطلبه.
\begin{information}
إن هذا هو السبب الرئيسي في توقف البرامج عن العمل : إذا حاول برنامجك الوصول إلى مكان غير مسموح له بالوصول إليه في الذاكرة، سيرفض نظام التشغيل و يوقف تشغيله بشكل عنيف (لأنه القائد هنا). بينما يتلقّى المستعمل نافذة خطأ تحتوي على رسالة تشير بأن البرنامج يحاول القيام بعملية غير لائقة.
\end{information}
لنعد للمتغير
\InlineCode{age}
. تم تخزين القيمة 10 في مكان ما من الذاكرة، لنقل مثلاً في العنوان رقم 4655.\\
ما يحدث (و هذا دور المترجم) هو أن الكلمة
\InlineCode{age}
يتم تعويضها بالعنوان 4655 لحظة التنفيذ. مما يعني أنه في كلّ مرة قمت فيها بكتابة الكلمة
\InlineCode{age}
في الشفرة المصدرية، يتم تعوضيها بـ4655، و بهذا يرى الجهاز إلى أي عنوان في الذاكرة عليه الذهاب. و منه يجيب بكلّ فخر بأن المتغير
\InlineCode{age}
يحتوي القيمة 10.

نحن نعرف إذا كيف نسترجع قيمة متغير، يكفي بكلّ بساطة أن نكتب الكلمة
\InlineCode{age}
في الشفرة المصدرية. إذا أردنا إظهار السنّ، يمكننا استعمال الدالة
\InlineCode{printf} :
\begin{Csource}
printf("The value of variable age is : %d", age);
\end{Csource}
النتيجة على الشاشة :
\begin{Console}
The value of variable age is : 10
\end{Console}
لا شيء جديد لحدّ الآن.

\subsection{الخبر المثير لليوم}
أنت تعرف كيف تظهر قيمة متغير، لكن هل تعرف أنه بإمكاننا أيضا إظهار عنوانه ؟

لكي نُظهر عنوان متغير، نستعمل الإشارة
\InlineCode{\%p}
(الحرف
\textenglish{p}
مأخوذ من الكلمة
\textenglish{pointer})
في الدالة
\InlineCode{printf}.
أي أننا لن نبعث للدالة
\InlineCode{printf}
المتغير في حدّ ذاته لكن نبعث لها عنوانه. و لفعل هذا، يجب عليك استعمال الإشارة
\InlineCode{\&}
أمام المتغير
\InlineCode{age}
كما طلبت منك أن تفعل مع الدالة
\InlineCode{scanf}
من قبل دون أن أشرح لك لماذا.

أكتب إذا:
\begin{Csource}
printf("The address of the variable age is  : %p", &age);
\end{Csource}
النتيجة :
\begin{Console}
The address of the variable age is : 0023FF74
\end{Console}
ما تراه هنا هو عنوان المتغير
\InlineCode{age}
في اللحظة التي طلبتُ فيها تنفيذ البرنامج من طرف حاسوبي. نعم نعم 0023FF74 هو رقم، هو فقط مكتوب في النظام الست عشري
(\textenglish{hexadecimal})
عوض النظام العشري الذي تعوّدنا عليه. لو تقوم بتعويض
\InlineCode{\%p}
بـ
\InlineCode{\%d}
فإنه سيظهر لك رقما عشريا كما تعوّدنا.
\begin{information}
	إذا شغّلت البرنامج على حاسوبك فمن المؤكّد أن تحصل على عنوان آخر. الأمر يعتمد على المكان في الذاكرة، البرامج المشتغلة، إلخ.
فيستحيل أن تتوقع العنوان الّذي سيتم تخزين المتغيّر فيه.
إذا قمت بتشغيل البرنامج عدة مرات الواحدة تلو الأخرى قد تتحصل على نفس النتيجة كون الذاكرة لم تتغير في ذلك الزمن القصير.
لكن بالمقابل إن أعدت تشغيل الحاسوب فستتحصل بكل تأكيد على نتائج مختلفة.
\end{information}
إلى أين أريد الوصول بكلّ هذا ؟ أريدك أن تتذكّر التالي:
\begin{itemize}
	\item \InlineCode{age} : تعني قيمة المتغير.
	\item \InlineCode{\&age} : تعني عنوان المتغير.
\end{itemize}
عند استخدام
\InlineCode{age}
سيقرأ الحاسوب قيمة المتغيّر في الذاكرة. أمّا عند استخدام
\InlineCode{\&age}
فسيعيد العنوان الذي يوجد فيه المتغيّر.

\section{إستعمال المؤشرات}
لحدّ الآن، قمنا فقط بإنشاء متغيرات تحتوي على أعداد. الآن سنتعلّم كيف ننشئ متغيرات تحتوي على عناوين: هذا ما نسميه بالمؤشرات.
\begin{question}
	لكن … العناوين هي أعداد أيضاً، أليس كذلك ؟ هذا يعني أننا سنخزن أعداداً دائما !
\end{question}
هذا صحيح، لكن لهذه الأعداد معنى آخر : هي تشير إلى عنوان متغير آخر في الذاكرة.

\subsection{إنشاء مؤشّر}
لإنشاء متغير من نوع مؤشّر، يجب علينا أن نضيف الرمز
\InlineCode{*}
أمام إسم المتغير :
\begin{Csource}
int *myPointer;
\end{Csource}
\begin{information}
	لاحظ أنه يمكننا أيضا أن نكتب
\InlineCode{int* myPointer;}

لهذا نفس المعنى. لكن الطريقة الأولى هي المفضّلة. في الواقع، إن كنت تريد التصريح عن العديد من المؤشرات في نفس السطر، سيكون عليك أن تعيد كتابة النجمة أمام كل اسم :\\

\InlineCode{int *pointer1, *pointer2, *pointer3;}
\end{information}
كما قلت لك، من المهمّ أن تقوم بإعطاء قيم إبتدائية للمتغيرات منذ البداية، و ذلك بإعطائها القيمة 0 مثلا ! إنه من المهم أكثر أن تفعل نفس الشيء مع المؤشرات.\\
لتهيئة مؤشّر، نعطيه قيمة افتراضية، لا نستعمل غالبا القيمة 0 و لكن الكلمة المفتاحية
\InlineCode{NULL}
(أكتبها بأحرف كبيرة).
\begin{Csource}
int *myPointer = NULL;
\end{Csource}
هنا لدينا مؤشر يحمل القيمة الإبتدائية
\InlineCode{NULL}.
هكذا ستعرف لاحقاً في البرنامج أن المؤشر لا يحتوي على أي عنوان.

ما الذي يحصل؟ ستقوم هذه الشفرة المصدرية بحجز خانة في الذاكرة كما لو أننا أنشأنا متغيراً عادياً. الشيء الذي يتغير هو أن المؤشر سيحتوي عنوانا. عنوان متغير آخر.

لم لا عنوان المتغير
\InlineCode{age}
؟ أنت تعرف الآن كيف تشير إلى عنوان متغير في مكان قيمته (باستعمال الرمز
\InlineCode{\&})،
هيا بنا إذن ! هذا ما عليك كتابته :
\begin{Csource}
int age = 10;
int *PointerOnAge = &age;
\end{Csource}
السطر الأول يعني : "أنشئ متغيرا من نوع
\InlineCode{int}
يحمل القيمة 10". السطر الثاني يعني "أنشئ متغيراً من نوع مؤشّر قيمته هي عنوان المتغير
\InlineCode{age}".

يقوم إذا السطر الثاني بمهمّتين معاً. لكي لا تختلط عليك الأمور، إعلم أنه يمكننا تقسيم السطر إلى سطرين :
\begin{Csource}
int age = 10;
int *PointerOnAge; // 1) means “I create the pointer”
PointerOnAge = &age; // 2) means the pointer “PointerOnAge contains the address of age”
\end{Csource}
يمكنك الملاحظة أنه لا يوجد في لغة الـ
\textenglish{C}
نوع نسميه
"\textenglish{pointer}"
كالنوع
\InlineCode{int}
و
\InlineCode{double}.
أي أنه لا يمكننا أن نكتب :
\begin{Csource}
pointer PointerOnAge;
\end{Csource}
في مكان هذا، نستعمل الرمز
\InlineCode{*}
، و لكن نستمر في كتابة
\InlineCode{int}.
ماذا يعني هذا ؟ في الواقع يجب أن نشير إلى نوع المتغير الذي سيحوي عنوانه المؤشر. بما أن المؤشر
\InlineCode{PointerOnAge}
سيحتوي عنوان المتغير
\InlineCode{age}
(الذي هو من نوع
\InlineCode{int})،
إذا فالمؤشر يجب أن يكون من نوع
\InlineCode{int*}
! إذا كان المتغير من نوع
\InlineCode{double}
فإنه يجب عليّ أن أكتب
\InlineCode{double *myPointer}.

\textbf{إصطلاح}
: نقول بأن المؤشّر
\InlineCode{PointerOnAge}
يؤشّر على المتغير
\InlineCode{age}.

المخطط التالي يلخّص ما يحصل في الذاكرة :
\Picture{Chapter_II-2_RAM-Schema-Pointer}
في هذا المخطط، تم تعويض المتغير
\InlineCode{age}
بالعنوان 177450 (أنت ترى بأن قيمته هي 10)، و المؤشّر
\InlineCode{PointerOnAge}
تم تعويضه بالعنوان 3 (هذه محض صدفة).

حينما يتم إنشاء المؤشر، يقوم نظام التشغيل بحجز خانة في الذاكرة كما فعل مع المتغير
\InlineCode{age}.
الشيء المختلف هنا هو أن المتغير
\InlineCode{PointerOnAge}
له معنى آخر. أنظر للمخطط جيداً : قيمته هي عنوان المتغير
\InlineCode{age}.

هذا، عزيزي القارئ، هو السرّ المطلق من وراء كتابة البرامج في لغة الـ
\textenglish{C}.
بهذا نحن ندخل في عالم المؤشرات العجيب !
\begin{question}
	و ... ما هي فائدة هذا ؟
\end{question}
هذا لا يقوم بتحويل الحاسوب إلى آلة صنع القهوة، طبعا. لكن الآن لدينا المؤشر
\InlineCode{PointerOnAge}
يحتوي عنوان المتغير
\InlineCode{age}.

فلنحاول رؤية ما يحتويه المؤشر بالإستعانة بالدالة
\InlineCode{printf} :
\begin{Csource}
int age = 10;
int *PointerOnAge = &age;
printf("%d", PointerOnAge);
\end{Csource}
\begin{Console}
177450
\end{Console}
هذا ليس مفاجئاً، نحن نطلب قيمة
\InlineCode{PointerOnAge}
و قيمته هي عنوان المتغير
\InlineCode{age}
(أي 177450).\\
ماذا نفعل لكي نطلب قيمة المتغير المتواجدة في العنوان الذي يشير إليه المؤشّر
\InlineCode{PointerOnAge}
؟ يجب أن نضع الرمز
\InlineCode{*}
أمام إسم المؤشّر :
\begin{Csource}
int age = 10;
int *PointerOnAge = &age;
printf("%d, *PointerOnAge);
\end{Csource}
\begin{Console}
10
\end{Console}
ها قد وصلنا ! بوضع الرمز
\InlineCode{*}
أمام إسم المؤشّر، يمكننا الوصول إلى قيمة المتغير
\InlineCode{age}.

لو استعملنا الرمز
\InlineCode{\&}
أمام اسم المؤشّر، سنتحصل على العنوان الذي يتواجد به المؤشّر (هنا الرقم 3).
\begin{question}
	ماذا نربح هنا ؟ لقد نجحنا في تعقيد الأمور لا أكثر. لم نكن نحتاج إلى مؤشّر لنظهر قيمة المتغير
\InlineCode{age} !
\end{question}
هذا السؤال (الذي لا مفر من طرحه) شرعي، حاليّا الهدف ليس واضحا، لكن قليلاً بقليل، و مع تقدّم الدروس، ستفهم بأن كلّ هذه المبادئ لم يتم اختراعها من أجل تعقيد الأمور بكلّ سذاجة.

المهم هو أن تفهم المبدأ الآن و بعده ستتوضح الأمور لوحدها رويداً رويداً.

  \chapter{الجداول}
هذا الدرس هو ملحق مباشر للدرس المتعلق بالمؤشرات، و سيعلّمك أهميتها أكثر. إن كنت تعتقد بأنك قادر على تفادي المؤشرات فأنت مخطئ ! هي في كلّ مكان في لغة الـ\textenglish{C}. لقد حذّرتك !

سنتعلم في هذا الدرس كيف ننشئ متغيرات من نوع "جداول". الجدوال مهمّة للغاية في لغة الـ\textenglish{C} لأنها تساعد في تنظيم سلسلة من القيم.

نبدأ هذا الدرس ببعض الشروحات و التفسيرات حول كيفية عمل الجداول في الذاكرة (سأقدم لك الكثير من المخططات التفسيرية). هذه المقدمات حول الذاكرة مهمة جداً : ستساعدك في في معرفة عمل الجداول. فمن المستحسن أن يعرف المبرمج ما يقوم به كي يتحكم في برامجه أكثر، أليس كذلك ؟

  \chapter{السلاسل المحرفيّة}
السلاسل المحرفيّة هي اسم صحيح
\textit{برمجيّا}
لتسمية ... النصّ، ببساطة !\\
السلسلة المحرفيّة هي إذن نصّ يمكننا حفظه على شكل متغيّر في الذاكرة. بهذه الطريقة يمكننا تخزين اسم المستخدم.

كنت قد قلت من قبل أن الحاسوب لا يفهم إلا الأعداد، فما هو السحر الذي يفعله المبرمجون للتعامل مع النصوص ؟ إنّعم ماكرون، سوف ترى !

  \chapter{المعالج القبلي}
بعد كل المعلومات المتعبة التي تلقّيتها في الدروس حول الجداول، النصوص و المؤشرات، فسنقوم بالتوقف قليلا. لقد تعلّمت أشياء جديدة كثيرة في الفصول السابقة، لن يكون لديّ مانع من نسترجع أنفاسنا قليلا.

هذا الفصل يتحدّث عن المعالج القبلي، هذا البرنامج الّذي يعمل مباشرة قبل الترجمة.\\
لا تخطئ : المعلومات التي به ستكون مهمّة لك. لكنّها ستكون أقل تعقيداً من الّتي تعلّمتها مؤخّراً.

\section{الـ\texttt{include}}
كما شرحت لكم في الفصول الأولى من الكتاب، نجد في الشفرات المصدريّة سطورا خاصّة تسمّى بـ\textbf{توجيهات المعالج القبلي} (\textenglish{Preprocessor directives}).\\
هذه السطور لديها الخاصيّة التالية : تبدأ دائما بالرمز
\InlineCode{\#}.
لذا فمن السهل التعرّف عليها.

التوجيهة الوحيدة التي رأيناها لحدّ الآن هي
\InlineCode{\#include}.\\
هذه التوجيهة تسمح لنا بتضمين محتوى ملف في آخر. قلت لكم هذا من قبل.\\
نحن نحتاجها في تضمين الملفات ذات الصيغة
\InlineCode{.h}
كملفات
\InlineCode{.h}
الخاصّة بالمكتبات
(\InlineCode{stdlib.h}، \InlineCode{stdio.h}، \InlineCode{string.h}، \InlineCode{math.h}...)،
و أيضاً ملفات
\InlineCode{.h}
الخاصّة بنا.

لنضمّن ملفاً ذو صيغة
\InlineCode{.h}
موجوداً في نفس المجلّد الذي ثبتنا فيه الـ\textenglish{IDE}
(أي البيئة التطويرية كالـ\textenglish{Code::Blocks}
مثلا)، نستعمل علامات الترتيب
\InlineCode{< >}
كالتالي :
\begin{Csource}
#include <stdlib.h>
\end{Csource}
بينما لتضمين ملفّ
\InlineCode{.h}
موجود في المجلّد الذي به مشروعنا، فسنقوم يذلك باستخدام علامتي الترتيب كالتالي :
\begin{Csource}
#include "myfile.h"
\end{Csource}
في الحقيقة، المعالج القبلي يتمّ تشغيله قبل الترجمة. يبحث في كلّ ملفاتك عن توجيهات المعالج القبلي، تلك الأسطر المشهورة التي تبدأ بـ\InlineCode{\#}.\\
عندما يجد التوجيهة
\InlineCode{\#include}،
يقوم بإدراج محتوى الملفّ في مكان وجود
\InlineCode{\#include}.


افترض أن لديّ ملفّا
\InlineCode{file.c}
يحتوي الشفرة الخاصة بالدوال التي كتبتها، و لدي ملف
\InlineCode{file.h}
يحتوي نماذج الدوال التي هي موجودة بالملف
\InlineCode{file.c}،
يمكن تلخيص ذلك بالمخطط التالي.
\Picture{Chapter_II-5_file-c-file-h}
كل محتوى الملف
\InlineCode{file.h}
سيتم وضعه داخل الملف
\InlineCode{file.c}
في مكان التوجيهة
\InlineCode{\#include "file.h"}.

تخيّل أن لدينا في الملف
\InlineCode{file.c}
التالي :
\begin{Csource}
#include "file.h"

int myFunction(int something, double stupid)
{
  /* The code of the function */
}
void anotherFunction(int value)
{
  /* The code of the function */
}
\end{Csource}
و في الملف
\InlineCode{file.h} :
\begin{Csource}
int myFunction(int something, double stupid);
void anotherFunction(int value);
\end{Csource}
عندما يمر المعالج القبلي بهذه الشفرة، قبل أن تتم ترجمة الملف
\InlineCode{file.c}،
سيضع كما قلت محتوى الملف
\InlineCode{file.h}
 في الملف
\InlineCode{file.c}.
في النهاية، يعني أن الملف
\InlineCode{file.c}
\textit{قُبَيْل}
الترجمة سيحتوي التالي :
\begin{Csource}
int myFunction(int something, double stupid);
void anotherFunction(int value);

int myFunction(int something, double stupid)
{
  /* The code of the function */
}
void anotherFunction(int value)
{
  /* The code of the function */
}
\end{Csource}
محتوى
\InlineCode{.h}
تمّ ادخاله مكان
\InlineCode{\#include}.

هذا ليس بالأمر المعقد لفهمه، و لعلّ بعض القراء يشكك في أن الأمر يحصل بهذه الطريقة.\\
مع هذه الشروحات الإضافيّة، أتمنّى أنّ الجميع يوافقني.
الـ\InlineCode{\#include}
لا تفعل أي شيء سوى إحضار محتوى ملف و تضمينه في آخر، من المهمّ فهم هذا الأمر جيّدا.
\begin{information}
  إن كنّا قد قررنا وضع النماذج في ملفّات
\InlineCode{.h}
بدل ملفّات
\InlineCode{.c}،
فهذا من المبدأ.
بالطبع، كان بإمكاننا وضع نماذج الدوال في أعلى الملفات
\InlineCode{.c}
بأنفسنا (قد نفعل هذا أحيانا في بعض البرامج الصغيرة)، لكن لأسباب تنظيميّة، من المنصوح به جدّا وضع النماذج في ملفّات
\InlineCode{.h}.
 عندما يكبر برنامجك و يصبح لديك الكثير من ملفّات
\InlineCode{.c}
يعتمدون على نفس
\InlineCode{.h}،
ستكون سعيدا لأنّك لن تظطرّ إلى نسخ و لصق النماذج الخاصّة بنفس الدوال عدّة مرّات !
\end{information}

\section{الـ\texttt{define}}
سنتعرف الآن على توجيهة معالج جديدة و هي
\InlineCode{\#define}.

هذه التوجيهة تسمح بالتصريح عن
\textbf{ثابت معالج قبلي}.
هذا يسمح بإرفاق قيمة بعبارة.\\
إليكم مثالا :
\begin{Csource}
#define INITIAL_NUMBER_OF_LIVES 3
\end{Csource}
يجب أن تكتب بالترتيب :
\begin{itemize}
  \item الـ\InlineCode{define}.
  \item الكلمة التي تريد ربط القيمة بها.
  \item قيمة الكلمة.
\end{itemize}
احذر : رغم التشابه (خصوصا في الاسم الذي اعتدنا كتابته بحروف كبيرة)، فهذه مختلفة كثيرا عن الثوابت التي تعلّمناها حتّى الآن، مثل :
\begin{Csource}
const int INITIAL_NUMBER_OF_LIVES = 3;
\end{Csource}
الثوابت تأخذ حيّزا في الذاكرة. حتى و إن لم تتغير قيمتها فإن العدد 3 مخزّن في مكان ما من الذاكرة. هذا ليس هو الحال مع ثوابت المعالج القبلي !

كيف تعمل ؟ في الواقع، الـ\InlineCode{\#define}
تستبدل في شفرتك المصدريّة كلّ الكلمات بقيمتهم الموافقة. هذا تقريبا مثل عمليّة البحث و الاستبدال
(\textenglish{Search/Replace})
الموجودة في برنامج
\textenglish{Word}
مثلا. إذن السطر :
\begin{Csource}
#define INITIAL_NUMBER_OF_LIVES 3
\end{Csource}
يستبدل في الملف كلّ
\InlineCode{INITIAL\_NUMBER\_OF\_LIVES}
بالعدد 3.

هذا مثال على ملف
\InlineCode{.c}
قبل مرور المعالج القبلي :
\begin{Csource}
#define INITIAL_NUMBER_OF_LIVES 3
int main(int argc, char *argv[])
{
	int lives = INITIAL_NUMBER_OF_LIVES;
  /* Code ... */
\end{Csource}
و بعدما يمرّ المعالج القبلي :
\begin{Csource}
int main(int argc, char *argv[])
{
	int lives = 3;
  /* Code ... */
\end{Csource}
قبل الترجمة، كلّ
\InlineCode{\#define}
يتمّ استبدالها بالقيمة الموافقة. المترجم "يرى" الملفّ بعد مرور المعالج القبلي، حيث تكون الاستبدالات قد تمت.
\begin{question}
ما الفائدة بالنسبة للثوابت التي رأيناها حتّى الآن ؟
\end{question}
كما قلت لك، هي لا تأخذ مكانا في الذاكرة. هذا منطقيّ، نظرا لأنّه عند الترجمة لا يتبقّى سوى الأعداد في الشفرة المصدريّة.

توجد فائدة أخرى و هي أنّ الاستبدال يتمّ في كامل الملف حيث توجد
\InlineCode{\#define}.
إن قمت بتعريف ثابت في الذاكرة داخل دالّة، فلن يكون صالحا إلّا داخل تلك الدالّة، ثمّ يتمّ حذفه بعد نهايتها.
بينما بالـنسبة لـ\InlineCode{\#define}
فإنها تُطبّق على كلّ دوال الملف، و هذا قد يكون عمليّا جدّا في بعض الحالات.

هل من مثال واقعيّ لاستخدام
\InlineCode{\#define} ؟\\
هذا ما لن تتأخّر هن فعله. عندما تفتح نافذة في
\textenglish{C}، قد تحتاج إلى تعريف ثوابت المعالج القبلي لتحديد أبعاد النافذة :
\begin{Csource}
#define WINDOW_WIDTH 800
#define WINDOW_HEIGTH 600
\end{Csource}
الفائدة هي أنّه إن أردت تغيير حجم الواجهة (لأنّها تبدو لك صغيرة جدّا)، فيكفي أن تغيّر
\InlineCode{\#define}
و تعيد ترجمة الشفرة.

لاحظ أنّ
\InlineCode{\#define}
تكون عادة في ملفات
\InlineCode{.h}
مع نماذج الدوال (بامكانك أن ترى
\InlineCode{.h}
الخاصّة بالمكتبات مثل
\InlineCode{stdlib.h}،
ستجد الكثير من
\InlineCode{\#define}).\\
\InlineCode{\#define}
إذن هي "مسهّلات وصول"، يمكنك تعديل حجم نافذه عن طريق تعديل
\InlineCode{\#define}
بدل الذهاب للبحث في الدوال عن الموضع الذي تفتح فيه النافذة لتعديل الأبعاد. هذا ربح وقت للمبرمج.

كملخّص، ثوابت المعالج القبلي تسمح بـ"إعداد" برنامجك قبل ترجمته. إنّها أشبه بطريقة إعدادات صغيرة.

\subsection{الـ\texttt{define} من أجل حجم جدول}
نستخدم كثيرا
\InlineCode{\#define}
من أجل تعريف حجم الجداول. نكتب مثلا :
\begin{Csource}
#define MAX_SIZE 1000
int main(int argc, char *argv[])
{
	char string1[MAX_SIZE], string2[MAX_SIZE];
	// ...
\end{Csource}
\begin{question}
  و لكن... كنت أعتقد أنّه لا يمكننا وضع متغيّر أو ثابت بين القوسين المربعين أثناء تعريف جدول ؟
\end{question}
نعم هذا صحيح، لكنّ
\InlineCode{MAX\_SIZE}
ليس متغيّرا و لا ثابتا. في الواقع لقد قلت لك، المعالج القبلي يحوّل الملف قبل الترجمة إلى :
\begin{Csource}
int main(int argc, char *argv[])
{
	char string1[1000], string2[1000];
	// ...
\end{Csource}
و هذا شيء صحيح.

بتعريف
\InlineCode{MAX\_SIZE}
بهذه الطريقة، يمكنك استخدامها لإنشاء جداول ذات حجوم محدّدة. إذا صارت في المستقبل غير كافية، فليس عليك سوى تعديل سطر
\InlineCode{\#define}،
إعادة الترجمة، و جداول
\InlineCode{char}
تأخذ القيمة الجديدة الّتي حددتها.

\subsection{الحسابات في الـ\texttt{define}}
من الممكن القيام بحسابات صغيرة في الـ\InlineCode{\#define}.\\
مثلا، هذه الشفرة تنشئ ثابتا
\InlineCode{WINDOW\_WIDTH}،
و آخر
\InlineCode{WINDOW\_HEIGHT}،
ثمّ ثالثا
\InlineCode{PIXELS\_NUMBER}،
الذي يحوي عدد البيكسلز المعروضة داخل النافذة (الحساب بسيط : العرض $\times$ الطول).
\begin{Csource}
#define WINDOW_WIDTH 800
#define WINDOW_HEIGHT 600
#define PIXELS_NUMBER (WINDOW_WIDTH * WINDOW_HEIGHT)
\end{Csource}
قيمة
\InlineCode{PIXELS\_NUMBER}
يتمّ استبدالها قبل الترجمة بالشفرة التالية :
\InlineCode{(WINDOW\_WIDTH * WINDOW\_HEIGHT)}،
أي (800*600)، و تعطينا 480000.
ضع دائما حساباتك بين قوسين كما أفعل من باب الاحتياط لكي تعزل العمليّة.

يمكنك القيام بكل العمليات القاعدية التي تعرفها : جمع (+)، طرح (-)، ضرب (*)، قسمة (/)، ترديد (\%).

\subsection{الثوابت مسبقة التعريف}
بالإضافة إلى الثوابت التي أنت عرّفتها، فإنه توجد ثوابت معرّفة من قِبَل المعالج القبلي.

كل من هذه الثوابت تبدأ و تنتهي برمزين
\textenglish{underscore} \InlineCode{\_}
(تجده في لوحة المفاتيح تحت الرقم 8 أعلى اللوحة بالنسبة للتخطيط
\textenglish{AZERTY}).
\begin{itemize}
  \item \InlineCode{\_\_LINE\_\_} : يعطي رقم السطر الحالي من الشفرة.
  \item \InlineCode{\_\_FILE\_\_} : يعطي إسم الملف الحالي.
  \item \InlineCode{\_\_DATE\_\_} : يعطي تاريخ ترجمة الشفرة.
  \item \InlineCode{\_\_TIME\_\_} : تعطي وقت ترجمة الشفرة.
\end{itemize}
قد تكون هذه الثوابت مفيدة للتحكم في الأخطاء، مثال :
\begin{Csource}
printf("Error in the line n° %d of the file %s\n", __LINE__, __FILE__);
printf("This file has been compiled on %s at %s\n", __DATE__, __TIME__);
\end{Csource}
\begin{Console}
Error in the line n° 9 of the file main.c
This file has been compiled on 13 Jan 2006 at 19:21:10
\end{Console}

\subsection{المعرّفات البسيطة}
إنه من الممكن أن نكتب بكل بساطة :
\begin{Csource}
#define CONSTANT
\end{Csource}
دون إعطاء القيمة.\\
هذا يعني للمعالج القبلي أنّ الكلمة
\InlineCode{CONSTANT}
معرّفة، بكلّ بساطة. ليست لها قيمة لكنّها "موجودة".
\begin{question}
  ما الفائدة من ذلك ؟
\end{question}
القائدة قد لا تبدو واضحة كما كان الأمر في السابق، لكن لهذا فائدة و سنكتشفها بسرعة.

\section{الماكرو (\textenglish{Macro})}
كنا قد رأينا بانه باستعمال الـ\InlineCode{\#define}،
بامكاننا أن نطلب من المعالج القبلي استبدال كلمة بقيمتها في الشفرة بأكملها. مثال :
\begin{Csource}
#define NUMBER 9
\end{Csource}
و الذي يعني أنّ جميع
\InlineCode{NUMBER}
في الشفرة يتمّ استبدالها بـ9. لقد رأينا أنّها تعمل كوظيفة بحث و استبدال يقوم بها المعالج القبلي قبل الترجمة.

لديّ خبر جديد ! في الواقع
\InlineCode{\#define}
أقوى من هذا بكثير. فهي قادرة على الاستبدال بـ... شفرة مصدرية بأكملها ! عندما نستخدم
\InlineCode{\#define}
للبحث و استبدال كلمة بشفرة مصدرية نقول أننا أنشأنا
\textbf{ماكرو
(\textenglish{Macro})}.

\subsection{ماكرو بدون معاملات}
هذا مثال عن ماكرو بسيطة :
\begin{Csource}
#define COUCOU() printf("Coucou");
\end{Csource}
الشيء الذي تغيّر هو القوسين الذين أضفناهما بعد الكلمة المفتاحيّة (هنا
\InlineCode{COUCOU()}).
سنرى فائدتهما بعد قليل.

فلنجرب الماكرو داخل الشفرة المصدرية:
\begin{Csource}
#define COUCOU() printf("Coucou");
int main(int argc, char *argv[])
{
  COUCOU()
  return 0;
}
\end{Csource}
\begin{Console}
Coucou
\end{Console}
أعلم أنّ هذا ليس شيئا جديدا حاليّا. لكنّ الّذي عليك فهمه، هو أن الماكرو عبارة عن بضعة أسطر من الشفرة التي يتم استبدالها مباشرة في الشفرة قبل الترجمة.\\
الشفرة التي كتبناها تصبح هكذا قبل الترجمة :
\begin{Csource}
int main(int argc, char *argv[])
{
	printf("Coucou");
	return 0;
}
\end{Csource}
إذا فهمت هذا فقد فهمت مبدأ عمل الماكرو.
\begin{question}
  لكن، هل يمكننا أن نضع سطراً واحدا فقط من الشفرة في كلّ ماكرو ؟
\end{question}
لا، لحسن الحظ يمكنك وضع عدّة أسطر من الشفرة في المرّة. يكفي وضع
\InlineCode{\textbackslash}
قبل كلّ سطر جديد، مثل هذا :
\begin{Csource}
#define TELL_YOUR_STORY() printf("Hello, my name is Brice\n"); \
                          printf("I live at Nice\n"); \
                          printf("I love rice\n");
int main(int argc, char *argv[])
{
	TELL_YOUR_STORY()
	return 0;
}
\end{Csource}
\begin{Console}
Hello, my name is Brice
I live at Nice
I love rice
\end{Console}
كما تلاحظ في
\InlineCode{main}،
أنّ نداء الماكرو لا يوضع بعده فاصلة منقوطة في النهاية. في الواقع، لأنها توجيهة خاصة بالمعالج القبلي و لا تحتاج إلى  أن تنتهي بفاصلة منقوطة.

\subsection{ماكرو بالمعاملات}
لحدّ الآن، رأينا كيف نقوم بإنشاء ماكرو بدون معاملات، أي بقوسين فارغين. الفائدة من هذا النوع من الماكرو أنّه يفيد في "اختصار" شفرة طويلة، خاصّة إذا كانت ستتكرّ كثيرا في شفرتك المصدريّة.

لكن الماكرو تصبح مفيدة أكثر عندما نضع لها الأقواس. هذا يعمل تقريبا مثل الدوال :
\begin{Csource}
#define ADULT(age) if (age >= 18) \
                    printf("You are adult\n");
int main(int argc, char *argv[])
{
	ADULT(22)
	return 0;
}
\end{Csource}
\begin{Console}
You are adult
\end{Console}
\begin{information}
يمكننا مثلا إضافة الـ\InlineCode{else}
لكي نُظهر على الشاشة : أنت لست بالغاً
"\textenglish{You are not adult}".
حاول القيام بذلك، الأمر ليس صعبا.لا تنس وضع الشرطة الخلفيّة
\InlineCode{\textbackslash}
قبل السطر الجديد.
\end{information}
مبدأ الماكرو بسيط جدّا :
\begin{Csource}
#define ADULT(age) if (age >= 18) \
                    printf("You are adult\n");
\end{Csource}
نقوم بوضع اسم "متغير" بين القوسين، و الّذي نسميه
\InlineCode{age}.
في كلّ شفرة الماكرو،
\InlineCode{age}
سيتم استبداله بالعدد المحدد عند النداء (هنا 22).

أي أن الشفرة المصدريّة السابقة بعد مرور المعالج القبلي مباشرة تصبح هكذا :
\begin{Csource}
int main(int argc, char *argv[])
{
	if (22 >= 18)
		printf("You are adult\n");
	return 0;
}
\end{Csource}
تم استبدال السطر الذي ينادي الماكرو بالشفرة التي تحتويه الماكرو، و تم تعويض "المتغير"
\InlineCode{age}
بقيمته مباشرة في الشفرة المصدريّة للاستبدال.

يمكننا انشاء ماكرو بعدة معاملات :
\begin{Csource}
#define ADULT(age, name) if (age >= 18)
printf("You are adult %s\n", name);
int main(int argc, char *argv[])
{
	ADULT(22, "Maxime")
	return 0;
}
\end{Csource}
هذا كلّ ما يمكننا أن نقوله حول الماكرو و المميزات التي تقدّمها لنا. يمكنك تذكّر أنّه مجرّد استبدال للشفرة المصدرية يمكنه استخدام المعاملات.
\begin{information}
في الواقع، أنت لست بحاجة أن تتعامل كثيراً مع الماكرو، لكن اعلم أن مكتبات معقدة كالـ\textenglish{wxWidgets}
و الـ\textenglish{Qt}
(مكتبات لإنشاء الواجهات الرسوميّة) تستعملان بكثرة الماكرو. لهذا من المستحسن أن تتعلّم كيف تعمل الأمور من الآن كي لا تضيع لاحقا.
\end{information}

\section{الشروط}
أجل : يمكننا أن نستعمل الشروط في لغة المعالج القبلي ! لاحظ كيف تعمل :
\begin{Csource}
#if condition
  /* Code to compile if the condition is true */
#elif condition2
  /* Else, compile this code if the condition2 is true */
#endif
\end{Csource}
الكلمة المفتاحية
\InlineCode{\#if}
تسمح بإدراج شرط معالج قبلي،
\InlineCode{\#elif}
تعني
\InlineCode{else if}.
الأمر يتوقف عندما نضع
\InlineCode{\#endif}،
تلاحظ أنه لا توجد حاضنتان في لغة المعالج القبلي.

الفائدة هي أننا سنتمكن من إجراء
\textbf{ترجمة شرطية
(\textenglish{Conditional compilation})}.\\
في الواقع، إن كان الشرط محققا فإن الشفرة التالية ستتم ترجمتها، و إلّا فسيتم حذفه و لن يكون جزءً من البرنامج النهائي.

\subsection{
\texttt{\#ifdef}
و
\texttt{\#ifndef}
}
سنرى الآن الفائدة من استعمال
\InlineCode{\#define}
لتعريف ثابت دون إعطائه أيّ قيمة، مثلما علّمتك من قبل :
\begin{Csource}
#define CONSTANT
\end{Csource}
في الواقع، يمكننا استعمال الشرط
\InlineCode{\#ifdef}
لنقول "إن كان الثابت معرّفا".
بالنسبة لـ\InlineCode{\#ifndef}،
فهذا يعني "إن كان الثابت غير معرّف".

يمكننا أن نتخيل هذا :
\begin{Csource}
#define WINDOWS
#ifdef WINDOWS
  /* Source code for Windows */
#endif
#ifdef LINUX
  /* Source code for Linux */
#endif
#ifdef MAC
  /* Source code for Mac */
#endif
\end{Csource}
هذا مثال عن برنامج متعدد المنصات
(\textenglish{multi-platform})
للتلاؤم مع النظام مثلا.\\
إذن ،يجب من أجل كلّ نظام إعادة ترجمة الشفرة (هذا ليس أمرا سحريّا).
إن كنت في
\textenglish{Windows}
فستكتب
\InlineCode{\#define WINDOWS}
في الأعلى و تعيد الترجمة.\\
إن أردت الترجمة لـ\textenglish{Linux}
فسيكون عليك تغيير
\InlineCode{\#define}
لوضع
\InlineCode{\#define LINUX}
و تعيد الترجمة. هذه المرّة الجزء الخاصّ بـ\textenglish{Linux}
الّذي ستتمّ ترجمته أمّا باقي الشروط فلن تكون محققة يعني أنه سيتم تجاهلها.

  \chapter{أنشئ أنواع متغيرات خاصّة بك}
تسمح لغة الـ\textenglish{C}
بالقيام بشيء يعتبر قوياً جداً : و هو أن ننشئ أنواعاً خاصة بنا، "أنواع متغيّرات مخصّصة". سنرى نمطين : الـهياكل
(\textenglish{Structures})
و التعدادات
(\textenglish{Enumerations}).

 إن إنشاء أنواع خاصّة بنا يعتبر أمراً ضروريا خاصة إذا أردنا إنشاء برامج أكثر تعقيداً.

الأمر ليس  (لحسن الحظّ) بالصعب، لكن ركّز جيّدا لأننا سنستعمل الهياكل كل الوقت انطلاقا من الفصل القادم.\\
يجب أن تعلم أنّ المكتبات تنشئ غالبا أنواعها الخاصّة. لن يمرّ وقت كثير حتّى تستخدم نوعا يدعى "ملف"، و بعده بقليل، أنواع أخرى مثل "نافذة"، "صوت"، "لوحة مفاتيح"، إلخ.

  \chapter{قراءة و كتابة الملفات}
المشكل مع استعمال المتغيّرات، هو أنها موجودة فقط في الذاكرة العشوائية
\textenglish{RAM}.
بخروجنا من البرنامج، كلّ المتغيّرات يتم حذفها من الذاكرة و لن يصبح ممكنا إستعادة قيمها. كيف يمكننا إذن أن نحتفظ بأحسن العلامات التي تحصّلنا عليها في لعبة ؟ كيف يمكننا إنشاء محرر نصوص إذا كان كلّ النصّ  المكتوب يختفي بمجرّد إيقاف البرنامج ؟

لحسن الحظّ يمكننا القراءة من الملفاّت و كذا الكتابة فيها في لغة
\textenglish{C}.
هذه الملفّات مُخزّنة في القرص الصلب
(\textenglish{Hard disk})
الخاص بالحاسوب : الشيء الإيجابيّ إذن هو أنها تبقى محفوظة، حتّى عند إيقاف البرنامج أو الحاسوب.

للقراءة من الملفات و الكتابة فيها، سنحتاج إلى استعمال كلّ ما درسناه حتّى الآن : المؤشرات، الهياكل، السلاسل المحرفيّة، الخ.

\section{فتح و غلق ملف}
للقراءة و الكتابة في الملفّات، سنستعمل دوالاً معرّفة في المكتبة
\InlineCode{stdio}
التي استعملناها سابقاً.\\
نعم، هذه المكتبة تحتوي على الدالتين
\InlineCode{scanf}
و
\InlineCode{printf}
اللتان نعرفهما جيّدا ! لكن ليس هذا فحسب : يوجد بها الكثير من الدوال الأخرى، خصوصا التي تعمل على الملفات.

\begin{information}
  كل المكتبات التي استعملناها حتّى الآن
(\InlineCode{stdlib.h}، \InlineCode{stdio.h}، \InlineCode{math.h}، \InlineCode{string.h}...)
تشكّل ما نسميه بالمكتبات القياسية
(\textenglish{standard libraries})،
و هي مكتبات تأتي تلقائيا مع البيئة التطويرية التي تستخدمها و لديها الميزة في أنّها تعمل على كل أنظمة التشغيل. بالإمكان استعمالها في أيّ مكان، سواء كنت في
\textenglish{Windows}،
أو
\textenglish{GNU/Linux}
أو
\textenglish{Mac}
أو غير ذلك.
المكتبات القياسيّة ليست كثيرة و لا تمكّننا من القيام بأكثر من بعض الأمور الأساسيّة، كما فعلنا لغاية الآن. للحصول على وظائف أكثر تقدّما، كفتح النوافذ، يجب تحميل و تثبيت مكتبات جديدة. سنرى ذلك قريبا !
\end{information}

تأكّد إذن، للبدأ، أن تقوم بتضمين المكتبتين
\InlineCode{stdio.h}
و
\InlineCode{stdlib.h}
على الأقل أعلى ملفكم
\InlineCode{.c} :

\begin{Csource}
#include <stdlib.h>
#include <stdio.h>
\end{Csource}

هاتان المكتبتان ضروريتان و أساسيّتان لدرجة أنّي أنصحك بتضمينهما في كلّ البرامج التي تكتبها في المستقبل، أيّا كانت.

حسناً و بعدما قمنا بتضمين المكتبتين، يمكننا أن ننطلق في بالأمور الجدّيّة. إليك الخطوات التي يجب إتّباعها دائماً حينما تريد العمل على ملف، سواء للقراءة منه أو للكتابة فيه :
\begin{itemize}
  \item نقوم بمناداة دالة
\textbf{فتح الملف}
\InlineCode{fopen}
التي تقوم بإرجاع مؤشّر نحو هذا الملف.
  \item \textbf{نتأكّد من نجاح عمليّة الفتح}
(أي إن كان الملفّ موجودا) باختبار قيمة المؤشر الذي أرجعته الدالة. فإن كان المؤشر يساوي
\InlineCode{NULL}،
فهذا يعني أنّ فتح الملف لم ينجح، في هذه الحالة لا يمكننا الإكمال (يجب أن نظهر رسالة خطا).
  \item إذا تم الفتح بنجاح (أي أن قيمة المؤشر تختلف عن
\InlineCode{NULL})،
سنستمتع
\textbf{بالكتابة على الملف أو القراءة منه}،
و ذلك باستخدام دوال سنراها لاحقاً.
  \item بمجرّد أن
\textbf{ننهي العمل على الملف}،
يجب تذكّر "غلقه" باستعمال الدالة
\InlineCode{fclose}.
\end{itemize}
سنتعلّم كخطوة أولى كيف نستخدم
\InlineCode{fopen}
و
\InlineCode{fclose}،
حينما تتعلّم هذا، سنتعلّم كيف نقرأ محتواه و نكتب نصّا فيه.

\subsection{\texttt{fopen} : فتح ملف}
في فصل السلاسل المحرفيّة، كنا نستعين بنماذج الدوال مثل "دليل استخدام". هذا ما يفعله المبرمجون غالبا : يقرؤون نموذج دالة و يفهمون كيف يستخدمونها. مع ذلك، أعلم أنّنا بحاجة إلى بعض الشروحات البسيطة !

لهذا فلنرى قليلاً نموذج
\InlineCode{fopen} :

\begin{Csource}
FILE* fopen(const char* fileName, const char* openMode);
\end{Csource}

هذه الدالة تنتظر معاملين :
\begin{itemize}
  \item اسم الملف الذي نريد فتحه.
  \item أسلوب فتح الملف، أي دلالة تذكر ما الّذي تريد فعله : القراءة من الملف، أو الكتابة فيه، أو كليهما.
\end{itemize}

هذه الدالة ترجع... مؤشّرا على
\InlineCode{FILE} !
إنّه مؤشّر على هيكل من نوع
\InlineCode{FILE}.
هذا الهيكل متواجد في المكتبة
\InlineCode{stdio.h}.
يمكنك فتح الملف لترى مما يتكوّن النوع
\InlineCode{FILE}،
لكن هذا ليس ما يهمّنا.

\begin{question}
  لكن لِمَ اسم الهيكل كله بأحرف كبيرة؟ اعتقدت أن الأسماء بالأحرف الكبيرة حجزناها للثوابت و لـ\InlineCode{\#define} ؟
\end{question}

هذه "القاعدة"، أنا من قمت بتحديدها (و كثير من المبرمجين يتيعونها)، و لكنّها لم تكن أبدا مفروضة. و يبدو أنّ من برمجوا
\InlineCode{stdio.h}
لا يتبعون نفس القواعد !\\
هذا لا يجب أن يشوّشك كثيرا. سوف ترى أنّ المكتبات الّتي سندرسها لاحقا تتبّع نفس القواعد التي أتّبعها، أي أن اسم الهيكل يبتدئ فقط بحرف واحد كبير.

لنعد إلى دالتنا
\InlineCode{fopen}،
إنها تقوم بارجاع
\InlineCode{FILE*}.
إنه من المهم جدّا استرجاع هذا المؤشّر كي نتمكّن لاحقاً من القراءة و الكتابة في الملف.  و لهذا سنقوم بإنشاء مؤشّر على
\InlineCode{FILE}،
في بداية دالتنا
(\InlineCode{main}
مثلا) :

\begin{Csource}
int main(int argc, char *argv[])
{
	FILE* file = NULL;
	return 0;
}
\end{Csource}

لقد هيّأنا المؤشّر على
\InlineCode{NULL}
من البداية. أذكّرك بأنّ هذه قاعدة أساسيّة أن تهيّأ كلّ المؤشّرات على
\InlineCode{NULL}
إنّ لم تكن لديك قيمة أخرى لإعطائها. إن لم تفعل ذلك، فأنت تزيد كثيرا خطر وجود أخطاء لاحقا.

\begin{information}
  إنه ليس ضرورياً أن تكتب
\InlineCode{struct FILE* file = NULL}،
لأن منشئي
\InlineCode{stdio.h}
قد وضعوا
\InlineCode{typedef}
كما علّمتك منذ مدّة قصيرة.
لاحظ أن شكل الهيكل قد يتغيّر من نظام تشغيل إلى آخر (لا تملك بالضرورة نفس المركّبات في كل الأنظمة). لهذا فلن نعدّل محتوى
\InlineCode{FILE}
مباشرة (لا نقوم بـ\InlineCode{file.element}
مثلا). بل سنكتفي باستدعاء دوال، تتعامل مع
\InlineCode{FILE}
نيابة عناً.
\end{information}

الآن سنقوم بمناداة الدالة
\InlineCode{fopen}،
و استرجاع القيمة الّتي تعيدها في المؤشر
\InlineCode{file}.
و لكن قبل هذا يجب أن اشرح لك كيف تستخدم المعامل الثاني
\InlineCode{openMode}.
في الواقع، هناك شفرة تدلّ للحاسوب على أنك تريد أن تفتح الملف بوضع القراءة فقط، الكتابة فقط أو الاثنين معاً.\\
هذه هي أوضاع فتح الملف المختلفة :
\begin{itemize}
  \item \textbf{\textenglish{"r"} :
قراءة فقط
(\textenglish{read only})}.
يمكنك قراءة محتوى الملف، و لكن لا يمكنك الكتابة فيه.
\textit{يجب أن يكون الملف موجوداً من قبل}.
  \item \textbf{\textenglish{"w"} :
كتابة فقط
(\textenglish{write only})}.
يمكنك الكتابة في الملف، لكن لا يمكنك قراءة محتواه.
\textit{إذا لم يكن الملف موجوداً من قبل، فإنه سيتم إنشاؤه}.
  \item \textbf{\textenglish{"a"} :
إلحاق
(\textenglish{append})}.
يمكنك الكتابة في الملف، إنطلاقا من نهايته.
\textit{إن لم يكن الملف موجوداً، فسيتم إنشاؤه}.
  \item \textbf{\textenglish{"r+"} :
قراءة و كتابة
(\textenglish{read and write})}.
يمكنك القراءة من الملف و الكتابة فيه.
\textit{يجب أن يكون الملف موجوداً من قبل}.
  \item \textbf{\textenglish{"w+"} :
قراءة و كتابة مع مسح المحتوى أوّلا}.
سيتم تفريغ الملف من محتواه أولاً، ثم بإمكانك الكتابة فيه و قراءة محتواه بعد ذلك.
\textit{إن لم يكن الملف موجوداً من قبل، سيتم إنشاؤه}.
  \item \textbf{\textenglish{"a+"}
إلحاق مع القراءة / الكتابة في آخر الملف}.
يمكنك القراءة و الكتابة إنطلاقا من نهاية الملف.
\textit{إن لم يكن موجوداً، سيتم إنشاؤه}.
\end{itemize}

لمعلوماتك، أنا عرضت لك بعضا من أوضاع فتح ملف. في الحقيقة، يوجد ضعفها !
من أجل كل وضع رأيناه هنا، إن أضفت
\InlineCode{"b"}
بعد المحرف الأول
(\InlineCode{"rb"}، \InlineCode{"wb"}، \InlineCode{"ab"}، \InlineCode{"rb+"}، \InlineCode{"wb+"}، \InlineCode{"ab+"})،
فإن الملف سيتم فتحه بالوضع الثنائي
(\textenglish{binary}).
هذا وضع خاص قليلاً فلن ندرسه هنا. في الواقع وضع النص يختصّ بتخزين... النص، تماما كما يوحي الاسم (فقط المحارف القابلة للعرض). أما الوضع الثنائي، يسمح بتخزين المعلومات
بايتا بايتا
(\textenglish{Byte by byte})
(أرقام بشكل أساسي). هذا مختلف كثيرا. على أي حال فطريقة العمل هي تقريبا نفس الّتي سنراها هنا.

شخصياً، أستعمل كثيراً الأوضاع :
\InlineCode{"r"}
(قراءة)،
\InlineCode{"w"}
(كتابة)،
\InlineCode{"r+"}
(قراءة و كتابة في آن واحد). وضع
\InlineCode{"w+"}
خطر قليلاً لأنه يقوم بمسح محتوى الملف مباشرة، بدون أن يطلب التأكيد قبل القيام بذلك. إن هذا الوضع ليس مفيداً إلا إذا أردنا أن نعيد تهيئة الملف أوّلا.
وضع الإلحاق
(\InlineCode{"a"})
يمكنه أن يفيد في بعض الحالات، إذا كنت تريد إضافة معلومات إلى نهاية الملف.

\begin{information}
  إن كنت تريد قراءة ملفّ، فمن المستحسن وضع
\InlineCode{"r"}.
بالطبع، الوضع
\InlineCode{"r+"}
يعمل أيضا، لكن بوضع
\InlineCode{"r"}
فأنت تضمن أنّ الملفّ لا يمكن تعديله، هذا نوع من الحماية.
\end{information}

  \chapter{الحجز الحيّ للذاكرة
(\textenglish{Dynamic memory allocation})}

كل المتغيّرات التي أنشأناها لحد الآن تمّ إنشاؤها تلقائيّا من طرف المترجم الخاصّ بلغة
\textenglish{C}.
لقد كانت الطريقة البسيطة. رغم ذلك، توجد طريقة يدوية أكثر لإنشاء متغيّرات و نسمّيها بالحجز الحيّ
(\textenglish{Dynamic allocation}).

من بين فوائد الحجز الحيّ هو السماح لبرنامج بحجز مكان لازم لتخزين جدول في الذاكرة لا يُعرف حجمه قبل بداية الترجمة. في الواقع، حتّى الآن، كان حجم جداولنا ثابتاً في الشفرة المصدريّة. بعد قراءة هذا الفصل، ستستطيع إنشاء جداول بطريقة أكثر مرونة !

من الضروري أن تتقن التعامل مع المؤشرات لتتمكّن من قراءة هذا الفصل ! إن كانت لديك بعض الشكوك حول المؤشرات، أنصحك بالذهاب لإعادة قراءة الفصل الموافق قبل البدأ.

عندما نقوم بالتصريح عن متغيّر، فإننا نقول أننا
\textbf{طلبنا حجز مكان في الذاكرة} :

\begin{Csource}
int myNumber = 0;
\end{Csource}

عندما يصل المترجم إلى سطر مشابه للسطر السابق، يقوم بالأمور التالية :
\begin{itemize}
  \item يقوم البرنامج بطلب إذن من نظام التشغيل
(\textenglish{Windows}، \textenglish{GNU/Linux}، \textenglish{Mac OS}...)
ليحجز شيئا من الذاكرة.
  \item يستجيب نظام التشغيل بإعطاء البرنامج عنوان الخانة حيث يمكنه تخزين المتغيّر (يعطيه العنوان الّذي حجزه له).
  \item عندما تنتهي الدالّة، المتغيّر يتم حذفه من الذاكرة. برنامجك يقول لنظام التشغيل : "أنا لم أعد بحاجة إلى المكان في الذاكرة الّذي حجزته في ذلك العنوان، شكرا ! التاريخ لا يحدّد إن كان البرنامج قد قال فعلا "شكرا" لنظام التشغيل، لكنّ هذا في مصلحته لأنّ نظام التشغيل هو الّذي يتحكم في الذاكرة !
\end{itemize}

لحد الآن كل الأمور كانت تلقائيّة. عندما نصرّح عن متغير فإن نظام التشغيل يتمّ استدعاءه تلقائياً من طرف البرنامج.
ما رأيك إذا بفعل هذا بطريقة يدوية ؟ ليس لأننا نريد أن نستمتع بفعل شيء معقّد، بل لأننا أحيانا نظطرّ لفعل ذلك !

في هذا الفصل سنقوم بـ :
\begin{itemize}
  \item دراسة كيف تعمل الذاكرة (نعم، مرّة أخرى !) لنعرف ما الحجم الذي يحجزه كل متغيّر حسب نوعه.
  \item ثمّ ندخل في موضوعنا الأساسي : سنرى كيف نطلب من نظام التشغيل يدويّا أن يحجز لنا مكانا في الذاكرة. هذا ما سنسميه الحجز الحيّ للذاكرة.
  \item و أخيراً، سنكتشف الفائدة من القيام بالحجز الحيّ بتعلّم إنشاء جدول ذو حجم غير معروف إلّا عند اشتغال البرنامج.
\end{itemize}

\section{حجم المتغيرات}
بحسب نوع المتغير التي نريد إنشاءه
(\InlineCode{int}،
\InlineCode{char}،
\InlineCode{float}...)
فنحن نحتاج إلى حجم معيّن من الذاكرة.

في الواقع، لتخزين عدد من
$-128$
إلى
$127$
(\InlineCode{char})
لن نحتاج إلا إلى بايت واحد من الذاكرة. هذا حجم صغير للغاية.\\
بالمقابل،
\InlineCode{int}
يحجز عادة حوالي 4 بايتات من الذاكرة. بينما
\InlineCode{double}
يحجز 8 بايتات.

المشكل هو ... أن هذا ليس دائما صحيحا. هذا يعتمد على الأجهزة : فقد يكون
\InlineCode{int}
يحجز 8 بايتات. من يعلم ؟\\
هدفنا هنا أن نتعرّف كم يحجز كلّ نوع من حجم في الذاكرة على حاسوبك.

توجد وسيلة سهلة جدّا لمعرفة هذا : استعمال العامل
\InlineCode{sizeof()}.\\
على عكس الظاهر، فهو ليس دالة، بل عبارة عن إحدى الوظائف الأساسية من لغة الـ\textenglish{C}،
يجب عليك فقط أن تضع بين القوسين النوع الذي تريد تحليله.\\
لمعرفة حجم
\InlineCode{int}،
يجب كتابة التالي :

\begin{Csource}
sizeof(int)
\end{Csource}

عند الترجمة، سيتم استبدال هذه الشفرة بعدد : عدد البايتات الّتي يحجزها
\InlineCode{int}
في الذاكرة. بالنسبة لي،
\InlineCode{sizeof(int)}
تساوي 4، و هذا يعني أنّ
\InlineCode{int}
يأخذ 4 بايتات. بالنسبة لك، ستكون نفس القيمة على الأرجح، لكنّها ليست قاعدة. جرّب لترى، بعرض القيمة عن طريق
\InlineCode{printf}
مثلا :

\begin{Csource}
printf("char : %d bytes\n", sizeof(char));
printf("int : %d bytes\n", sizeof(int));
printf("long : %d bytes\n", sizeof(long));
printf("double : %d bytes\n", sizeof(double))
\end{Csource}

بالنسبة لي ، هذا يظهر على الشاشة :

\begin{Csource}
char : 1 bytes
int : 4 bytes
long : 4 bytes
double : 8 bytes
\end{Csource}

لم أختبر كل الأنواع الّتي نعرفها، أتركك لتجرّب أحجام الأنواع الأخرى.

أنت تلاحظ أن
\InlineCode{int}
و
\InlineCode{long}
يحجزان نفس الحجم من الذاكرة. إنشاء
\InlineCode{long}
يعود تماما إلى إنشاء
\InlineCode{int}،
هذا يأخذ 4 بايتات من الذاكرة.

\begin{information}
في الواقع، النوع
\InlineCode{long}
هو مكافئ لنوع نسميه
\InlineCode{long int}،
و الذي هو مكافئ لنوع...
\InlineCode{int}
نفسه. باختصار، فإن هذه أسماء كثيرة مختلفة لأجل أشياء ليست بالكبيرة، في النهاية ! امتلاك أنواع مختلفة كثيرة كان أمرا مهمّا  في الوقت الذي لمّ تكن الحواسيب تملك كثيرا من ذاكرة. كنا نبحث دائما لاستخدام الحدّ الأدنى من الذاكرة باستخدام النوع المناسب.\\
اليوم، هذا لم يعد مفيدا كثيرا لأنّ ذاكرة الحاسوب صارت كبيرة جدّا. بالمقابل، هذه الأنواع لا تزال مفيدة إذا كنت تنشئ برامج للأنظمة المضمّنة
(\textenglish{Embedded systems})
حيث الذاكرة المتوفّرة أقل. أظن مثلا في البرامج الموجّهة للهواتف المحمولة، الأليّات، إلخ.
\end{information}

\begin{question}
هل بإمكاننا أن نُظهر حجم نوع مخصّص قمنا نحن بإنشائه (هيكل) ؟
\end{question}

نعم !
\InlineCode{sizeof()}
تعمل مع الهياكل أيضا !

\begin{Csource}
typedef struct ِCoordinates ِCoordinates ;
struct ِCoordinates
{
	int x;
	int y;
};
int main(int argc, char *argv[])
{
	printf("ِCoordinates  : %d bytes\n", sizeof(ِCoordinates));
	return 0;
}
\end{Csource}

\begin{Console}
Coordinates : 8 bytes
\end{Console}

كلما احتوى الهيكل من مركّبات كلّما أخذ حجما أكثر من الذاكرة. الأمر منطقي تماما، أليس كذلك ؟

\subsection{طريقة أخرى للنظر إلى الذاكرة}
لحد الآن، كل المخططات التي قدّمتها لك عن الذاكرة لم تكن دقيقة. سنجعلها أخيرا دقيقة حقا و صحيحة بما أننا تعلّمنا الآن كم يأخذ كل نوع من حجم بالذاكرة.

إن صرّحنا عن متغير من نوع
\InlineCode{int} :

\begin{Csource}
int number = 18;
\end{Csource}

و
\InlineCode{sizeof(int)}
يعطينا 4 بايت على حاسوبنا، هذا يعني أن المتغير يحجز 4 بايت في الذاكرة !

لنفترض أن المتغير
\InlineCode{number}
محجوز بالعنوان
$1600$
من الذاكرة. سيكون لدينا إذا المخطط التالي للذاكرة :

\Picture{Chapter_II-8_RAM-Schema-int}

هنا، يمكننا فعلاً أن نرى بأن المتغير
\InlineCode{number}
من النوع
\InlineCode{int}
يحجز 4 بايت من الذاكرة.
فهو يبدأ من العنوان
$1600$
و ينتهي عند العنوان
$1603$،
المتغير القادم لن يتم تخزينه إلا إبتداءً من العنوان
$1604$ !

إن جربنا نفس الشيء مع
\InlineCode{char}،
فالمتغير لن يأخذ سوى بايت واحد في الذاكرة (الشكل التالي) :

\Picture{Chapter_II-8_RAM-Schema-char}

تخيّل الآن جدولا من
\InlineCode{int} !\\
كل "خانة" من الجدول ستحجز 4 بايت. إن كان الجدول يحوي مثلاً  100 خانة :

\begin{Csource}
int table[100];
\end{Csource}

سنحجز إذن
$100 * 4 = 400$
بايت في الذاكرة.

\begin{question}
ماذا لو كان الجدول فارغاً، هل سيحجز 400 بايت ؟
\end{question}

نعم بالطبع ! فالمكان  في الذاكرة قد تمّ حجزه، و لا يملك أي برنامج الحقّ في استخدام هذه الخانات (غير هذا البرنامج). بمجرّد التصريح عن متغيّر، سيأخذ مكانه مباشرة المكان في الذاكرة.

لاحظ لو أننا ننشئ جدولا من نوع
\InlineCode{Coordinates} :

\begin{Csource}
Coordinates table[100];
\end{Csource}

سيستخدم هذه المرّة
$8 * 100 = 800$
بايت.

من المهمّ الفهم الجيّد لهذه الحسابات البسيطة لنواصل بقيّة الفصل.

\section{الحجز الحيّ للذاكرة}
فلندخل إلى صلب الموضوع. سأذكّرك بهدفنا : تعلّم كيفيّة طلب الذاكرة يدوياً.

سنحتاج إلى تضمين المكتبة
\InlineCode{stdlib.h}.
إن كنت قد اتّبعت نصائحي، فقد ضمّنتها في كلّ برامجك. هذه المكتبة تحتوي على دالّتين سنحتاج إليهما :
\begin{itemize}
  \item \InlineCode{malloc} ("\textenglish{Memory ALLOcation}"
بمعنى "حجز الذاكرة") : تطلب الإذن من نظام التشغيل لاستخدام الذاكرة.
  \item \InlineCode{free}
(تحرير) : تسمح للإشارة لنظام التشغيل بأننا لم نعد بحاجة إلى الذاكرة الّتي طلبناها. المكان في الذاكرة تمّ تحريره، يستطيع برنامج آخر الآن استخدامها عند الحاجة.
\end{itemize}

عندما تقوم بحجز يدوي للذاكرة، فعليك اتباع الخطوات التالية :
\begin{enumerate}
  \item استدعاء
\InlineCode{malloc}
من أجل طلب الذاكرة.
  \item اختبار القيمة التي تم ارجاعها من طرف
\InlineCode{malloc}
لمعرفة ما إن نجح نظام التشغيل في حجز الذاكرة.
  \item ما إن ننتهي من استخدام الذاكرة، يجب علينا تحريرها باستعمال
\InlineCode{free}.
إن لم نفعل هذا، فسنتعرّض لتسريبات ذاكرة، أي أنّ البرنامج يخاطر بحجز كثير من الذاكرة مع أنّه ليس بحاجة إلى كلّ هذا المكان.
\end{enumerate}

هل تذكّرك هذه الخطوات الثلاث بفصل الملفات ؟ نعم يجب أن تفعل ! المبدأ واحد تماما : نحجز، نختبر إن نجح الحجز، ثمّ نحرر عندما ننتهي من الاستعمال.

\subsection{\texttt{malloc}
لنطلب الإذن لحجز الذاكرة}
نموذج الدالة
\InlineCode{malloc}
هزليّ جدّا، سترى :

\begin{Csource}
void* malloc(size_t numberOfNecessaryBytes);
\end{Csource}

الدالة تأخذ معاملا واحدا : عدد البايتات الّتي يجب حجزها. هكذا، يكفي أن كتابة
\InlineCode{sizeof(int)}
لحجز مكان من أجل تخزين
\InlineCode{int}.

و لكنّ الشيء الذي يثير الفضول، هو القيمة التي ترجعها الدالة : إنّها تعيد ...
\InlineCode{void*} !
إذا لازلت تتذكّر فصل الدوال، كنت قد قلت لك بأن الكلمة
\InlineCode{void}
تعني "الفراغ" و نستعملها لنشير إلى أن الدالة لا تُعيد أية قيمة.

إذن هنا، لدينا دالة تُعيد ... "مؤشّراً نحو فراغ" ؟ هذه نكتة جيدة !\\
يبدو أن هؤلاء المبرمجين لديهم حسّ فكاهي متطوّر.

كن متأكّدا، يوجد سبب. في الحقيقة، هذه الدالة تعيد عنوان الخانة التي حجزها نظام التشغيل من أجل متغيّرك. إن استطاع النظام إيجاد مكان لك في العنوان
$1600$،
فالدالة ستعيد مؤشّرا يحوي العنوان
$1600$.

المشكل هو أن الدالة
\InlineCode{malloc}
لا تعرف نوع المتغير التي نريد إنشاءه. في الواقع، أنت لا تعطيها سوى معامل واحد : عدد البايتات في الذاكرة الّتي تحتاجها. فإذا طلبت 4 بايت، فهذا يمكن أن يعني
\InlineCode{int}
أو ربما
\InlineCode{long}
مثلا !

بما أنّ
\InlineCode{malloc}
لا تعرف أيّ نوع يجب عليها أن تعيد، فهي تعيد النوع
\InlineCode{void*}.
سيكون مؤشّرا نحو
\textit{أيّ نوع كان}.
يمكننا أن نقول أنّه مؤشّر جامع.

لننتقل إلى التطبيق.\\
إذا كنت أريد الاستمتاع بإنشاء متغير من نوع
\InlineCode{int}
يدويّا في الذاكرة، يجب أن أشير للـ\InlineCode{malloc}
أنني أحتاج إلى
\InlineCode{sizeof(int)}
بايت في الذاكرة.\\
أسترجع قيمة
\InlineCode{malloc}
في مؤشر على
\InlineCode{int} :

\begin{Csource}
int* allocatedMemory = NULL; // Create a pointer on int
allocatedMemory = malloc(sizeof(int)); // The function malloc puts the allocated address in the pointer.
\end{Csource}

في نهاية هذه الشفرة،
\InlineCode{allocatedMemory}
هو مؤشّر يحتوي على عنوان حجزه نظام التشغيل لك، لنقل مثلا القيمة
$1600$
للاكمال من مخططاتي السابقة.

\subsubsection{اختبار المؤشّر}
الدالة
\InlineCode{malloc}
أعادت في المتغير
\InlineCode{allocatedMemory}
عنوان الخانة التي تم حجزها بالذاكرة. هناك احتمالان :
\begin{itemize}
  \item إذا نجح الحجز، فالمؤشّر سيحتوي عنوانا.
  \item إذا فشل الحجز، فالمؤشّر سيحتوي العنوان
\InlineCode{NULL}.
\end{itemize}

إنه من النادر أن تفشل عملية حجز الذاكرة، لكن هذا ممكن. تخيّل أنك تطلب حجز 34
\textenglish{Go}
من الذاكرة العشوائية، في هذه الحالة، ستفشل عملية الحجز على أغلب الظن.

من المستحسن دائماً أن نختبر ما إن تمت العملية بنجاح. سنفعل هذا : إن فشل الحجز، فهذا يعني أن المساحة الحرّة من الذاكرة العشوائية لم تكن كافية (هذه حالة حرجة). في حالة كهذه، يجب إيقاف البرنامج فورا لأنّه، على أية حال، لن يكون قادراً على الاستمرار بشكل عاديّ.

سنستعمل دالة قياسيّة لم يسبق لنا رؤيتها حتّى الآن :
\InlineCode{exit()}.
هذه الأخيرة توقف البرنامج فورا. إنّها تأخذ معاملا : القيمة الّتي يجب إعادتها من طرف البرنامج (هذا في الحقيقة يوافق الـ\InlineCode{return}
الخاص بالـ\InlineCode{main}).

\begin{Csource}
int main(int argc, char *argv[])
{
	int* allocatedMemory = NULL;
	allocatedMemory = malloc(sizeof(int));
	if (allocatedMemory == NULL) // If the allocation has failed
	{
    exit(0); // Stop the program
	}
	// Else, we can continue the program normally.
	return 0;
}
\end{Csource}

إذا كان المؤشر مختلفا عن
\InlineCode{NULL}،
يمكن للبرنامج أن يواصل العمل، و إلا فيجب إظهار رسالة خطأ أو حتّى إنهاء البرنامج لأنّه لن يتمكّن من الاستمرار بشكل صحيح إن لّم يكن هناك مكان في الذاكرة.

  \chapter{برمجة لعبة
الـ\textenglish{Pendu}}
أكرر دائما : التطبيق شيء ضروريّ. هو ضروريّ لك لأنك اكتشفت كثيرا من المفاهيم النظرية و، أيّا كان ما تقول، لن تفهمها حقّا بدون تطبيق.

في هذا العمل التطبيقي، أقترح عليك إنشاء لعبة الـ\textenglish{Pendu}.
و هي لعبة حروف تقليديّة يتمّ فيها تخمين كلمة سريّة حرفا بحرف. و الـ\textenglish{Pendu}
سيكون إذن لعبة في الكونسول بلغة
\textenglish{C}.

الهدف هو جعلك تستخدم كلّ ما تعلّمته حتّى الآن : المؤشرات، السلاسل المحرفيّة، الملفات، الجداول... باختصار، الأشياء الجيّدة فقط !

\section{التعليمات}
سأقوم بشرح قواعد الـ\textenglish{Pendu}
الواجب إنشاءه. سأعطيك هنا التعليمات، أي سأشرح لك بدقّة كيف يجب أن تعمل اللعبة التي ستُنشئها.

أعتقد أن الجميع يعرف
الـ\textenglish{Pendu}،
أليس كذلك ؟ هيّا، تذكير صغير لا يمكن أن يحدث ضررا : هدف الـ\textenglish{Pendu}
هو إيجاد الكلمة المخبّأة في أقلّ من عشر محاولات (يمكنك تغيير العدد الأقصى لتغيير صعوبة اللعبة، بالطبع !).

\subsection{سريان الجولة}
فلنفترض أن الكلمة المخبّأة هي \textenglish{RED}.\\
ستقوم باقتراح حرف على الحاسوب، مثلا الحرف
\textenglish{A}.
سيتأكّد الحاسوب ما إن كان هذا الحرف موجوداً في الكلمة المخفيّة.

\begin{information}
تذكّر : هناك دالة جاهزة في
\InlineCode{string.h}
تقوم بالبحث عن حرف في كلمة ! و بالطبع أنت لست مجبراً على استخدامها (شخصيّا، أنا لم أفعل).
\end{information}

إنطلاقاً من هنا، يوجد احتمالان :
\begin{itemize}
  \item الحرف موجود بالفعل في الكلمة : سنكشف مكان الحرف في الكلمة.
  \item الحرف غير موجود في الكلمة (هذا هو الحال هنا، لأن
\textenglish{A}
ليس موجوداً في الكلمة
\textenglish{RED}) :
سنخبر اللاعب بأن الحرف هذا غير موجود في الكلمة، و سننقص عدد المحاولات المتبقّية. عندما لا تتبق أية محاولة (0 محاولة)، ستنتهي اللعبة و سنخسر.
\end{itemize}

\begin{information}
في لعبة
\textenglish{Pendu}
"حقيقة"، يفترض وجود شخص يتأسّف في كلّ مرّه نخطئ فيها. في الكونسول، سيكون من الصعب كثيرا رسم شخص يتأسّف بواسطة لاشيء غير النص،  لذا سنكتفي بعرض جملة بسيطة مثل "بقي لك
\textenglish{X}
محاولات قبل الموت الأكيد".
\end{information}

فلنفرض الآن أن اللاعب أدخل الحرف
\textenglish{D}.
هذا الحرف موجود في الكلمة المخفيّة، لهذا لن نقوم بإنقاص عدد المحاولات المتبقّية للاعب. سنقوم بإظهار الكلمة مع الحروف الّتي تم إيجادها، أي شيء كهذا :

\begin{Console}
Secret word : **D
\end{Console}

إذا أدخل اللاعب فيما بعد الحرف
\textenglish{R}،
و بما أنّه موجود في الكلمة، سنضيف الحرف إلى قائمة الحروف التي تم إيجادها و يتم إظهار الكلمة مع الحروف الّتي تمّ اكتشافها :

\begin{Console}
Secret word : R*D
\end{Console}

\subsubsection{حالة وجود حرف مكرر}
في بعض الكلمات، يمكن أن نجد حرفاً مكرراً مرتين أو ثلاث، أو ربّما أكثر !\\
مثلا : يوجد إثنان من
\textenglish{Z}
في كلمة
\textenglish{PUZZLE}،
و كذلك يوجد ثلاثة
\textenglish{E}
في كلمة
\textenglish{ELEMENT}.

ماذا علينا أن نفعل في حالة كهذه ؟ قواعد
\textenglish{Pendu}
واضحة : إذا أدخل اللاعب الحرف
\textenglish{E}،
كلّ حروف
\textenglish{E}
في كلمة
\textenglish{ELEMENT}
يجب أن تظهر دفعة واحدة :

\begin{Console}
Secret word : E*E*E**
\end{Console}

يعني أنه ليس على اللاعب أن يدخل 3 مرات الحرف
\textenglish{E}
ليتم إكتشاف كل تكرار له في الكلمة.

\subsubsection{مثال عن جولة كاملة}
هذا ما ستبدو عليه جولة كاملة في الكونسول عند انتهاء البرنامج :

\begin{Console}
Welcome !
You have 10 remaining tries
What's the secret word ? ****
Suggest a letter : B
You have 9 remaining tries
What's the secret word ? ****
Suggest a letter : F
You have 9 remaining tries
What's the secret word ? F***
Suggest a letter : D
You have 9 remaining tries
What's the secret word ? F**D
Suggest a letter : O
You win ! The secret word is  : FOOD
\end{Console}

\subsubsection{قراءة حرف من الكونسول}
قراءة حرف من الكونسول هي أكثر تعقيداً ممّا تبدو.\\
بديهيّا، لاسترجاع محرف، يفترض أنّك تفكّر في :

\begin{Csource}
scanf("%c", &myLetter);
\end{Csource}

و تماما، هذا جيّد.
\InlineCode{\%c}
تعني أننا ننتظر محرفاً، و الذي سنقوم بتخزينه في
\InlineCode{myLetter}
(متغيّر من نوع
\InlineCode{char}).


كل شيء يعمل جيداً... ما دمنا لم نقم بـ\InlineCode{scanf}
مرّة اخرى. يمكنك تجريب الشفرة التالية :

\begin{Csource}
int main(int argc, char* argv[])
{
 	char myLetter = 0;
 	scanf("%c", &myLetter);
 	printf("%c", myLetter);
 	scanf("%c", &myLetter);
 	printf("%c", myLetter);
 	return 0;
}
\end{Csource}

يفترض بهذه الشفرة أن تطلب حرفاً و تظهره، و ذلك لمرّتين.\\
جرّب. ما الذي يحصل ؟ تدخل حرفا، نعم، و لكن... البرنامج يتوقّف مباشرة بعدها، فهو لا يطلب منك المحرف الثاني ! و كأنه تم تجاهل
\InlineCode{scanf}
الثانية.

\begin{question}
ما الذي حصل ؟
\end{question}

في الواقع، حينما تدخل نصاً في الكونسول، فإن كل ما قمت بإدخاله يتمّ تخزينه في الذاكرة، بما في ذلك الزر
\texttt{Enter}
(\InlineCode{\textbackslash n}).

لذلك، في أوّل مرّة تدخل فيها حرفا
(\textenglish{A}
مثلاً) ثمّ تضغط على
\textit{\textenglish{Enter}}
فإن الحرف
\textenglish{A}
هو من يتم إعادته من طرف
\InlineCode{scanf}.
بينما في المرّة الثانية،
\InlineCode{scanf}
سيعيد
\InlineCode{\textbackslash n}
الموافق لـ\textit{\textenglish{Enter}}
الّذي أدخلته سابقا !

لتجنب هذا، من الأحسن أن نكتب بأنفسنا دالتنا الخاصّة الصغيرة
\InlineCode{readCharacter()} :

\begin{Csource}
char readCharacter()
{
  char character = 0;
  character = getchar(); // Read the first character
  character = toupper(character); // Convert the character to uppercase
  // Read other characters until reaching \n (to erase them)
  while (getchar() != '\n') ;
  return character; // Return the first character that have been read
}
\end{Csource}

هذه الدالة تستخدم
\InlineCode{getchar()}
الّتي هي دالة من
\InlineCode{stdio}
و هذا يعود تماماً إلى كتابة\\
\InlineCode{scanf("\%c", \&letter);}.
الدالة
\InlineCode{getchar()}
تقوم بإرجاع المحرف الذي قام اللاعب بإدخاله.

بعد ذلك، أستعمل أيضاً الدالة القياسيّة التي لم تسنح لنا فرصة تعلّمها في كتابنا :
\InlineCode{toupper()}.
هذه الدالّة تحوّل الحرف المعطى إلى كبير
(\textenglish{Uppercase}).
هكّذا، اللعبة ستعمل حتى إن أدخل اللاعب حروفاً صغيرة. يجب تضمين
\InlineCode{ctype.h}
لتستطيع استخدام هذه الدالة (لا تنس ذلك !).

تأتي بعد ذلك المرحلة الأكثر أهمية : و هي أن نقوم بمسح المحارف التي يمكن أن نكون قد أدخلناها. في الواقع، بإعادة استدعاء
\InlineCode{getchar}
نحصل على المحرف الثاني الّذي تمّ إدخاله (مثلا
\InlineCode{\textbackslash n}).\\
ما أقوم به بسيط و يأخذ سطرا واحدا : أستدعي الدالة
\InlineCode{getchar}
في حلقة تكرارية حتى الوصول إلى
\InlineCode{\textbackslash n}.
تتوقف الحلقة إذن، و هذا يعني أننا "قرأنا" كلّ المحارف الأخرى، سيتمّ إذن إفراغها من الذاكرة. نقول أنّنا
\textbf{نفرغ المتغير المؤقت
(\textenglish{Buffer})}.

\begin{question}
لماذا توجد فاصلة منقوطة في نهاية الـ\InlineCode{while}
و لماذا لا نرى أية حاضنة ؟
\end{question}

في الواقع، استعملت حلقة تكرارية لا تحتوي على تعليمات (التعليمة الوحيدة، هي
\InlineCode{getchar}
داخل القوسين). الحاضنتان ليستا ضروريّتين نظرا لأنه ليس لدينا ما نفعله غير
\InlineCode{getchar}.
لهذا أضع فاصلة منقوطة لتعويض الحاضنتين. هذه الفاصلة المنقوطة تعني "لا تفعل شيئاً في كلّ دورة للحلقة". هذا أمر غريب قليلا، لكنها تقنيّة يجب معرفتها، تقنيّة يستعملها المبرمجون لانشاء حلقات بسيطة و قصيرة.

اعلم أنّ الـ\InlineCode{while}
كان بالإمكان كتابتها هكذا :

\begin{Csource}
while (getchar() != '\n')
{

}
\end{Csource}

لا يوجد شيء داخل الحاضنتين، إنّها تطوّعيّة، نظرا لأنّه ليس هناك شيء آخر لفعله. تقنيّتي الّتي تقتضي وضع فاصلة منقوطة فقط أبسط من تلك الخاصّة بالحاضنتين.

أخيرا، تقوم الدالة
\InlineCode{readCharacter}
بإرجاع المحرف الأوّل الذي قمنا بقراءته : المتغيّر
\InlineCode{character}.

خلاصة القول، في شفرتك، لا تستعمل :

\begin{Csource}
scanf("%c", &myLetter);
\end{Csource}

و إنما استعمل بدل ذلك دالّتنا الرائعة :

\begin{Csource}
myLetter = readCharacter();
\end{Csource}

\subsection{قاموس الكلمات}
لتجربة أولية للشفرة الخاصة بك، أطلب منك أن تقوم بتثبيت الكلمة السريّة مباشرة في الشفرة. أكتب مثلا :

\begin{Csource}
char secretWord[] = "RED";
\end{Csource}

طبعا ستبقى الكلمة السريّة نفسها دائما إن تركناها هكذا، هذا ليس ممتعا. لكني طلبت منك فعل ذلك لكي لا تخلط المشاكل. في الواقع، عندما تعمل لعبة
\textenglish{Pendu}
جيّدا (و فقط ابتداء من هذه اللحظة)، يمكنك البدء بالطور الثاني : إنشاء قاموس الكلمات.

\begin{question}
ما هو هذا "قاموس الكلمات" ؟
\end{question}

هو ملف يحتوي كثيرا من الكلمات للعبتك
\textenglish{Pendu}.
يجب أن تكون كل كلمة على سطر. مثلا :

\begin{Console}
HOUSE
BLUE
AIRPLANE
XYLOPHONE
BEE
BUILDING
WEIGHT
SNOW
ZERO
\end{Console}

في كل جولة جديدة، يجب على برنامجك أن يفتح الملف، و يأخذ كلمة عشوائية من القائمة. بفضل هذه الطريقة، سيكون لديك ملف يمكنك التعديل عليه كلّما أردت من أجل إضافة كلمات سريّة ممكنة من أجل
\textenglish{Pendu}.

\begin{information}
ستلاحظ أنني منذ البداية تعمّدت كتابة كلّ الكلمات بالحروف الكبيرة. في الواقع، في الـ\textenglish{Pendu}
لا يتم التمييز بين الحروف الكبيرة و الحروف الصغيرة، و لهذا فمن المستحسن أن نقول منذ البداية : "كل حروف كلمات اللعبة كبيرة". عليك أن تنبّه اللاعب، في دليل استخدام اللعبة مثلا، أنه يفترض به إدخال حروف كبيرة لا صغيرة.\\
بالمقابل، نتعمّد تجنب العلامات الصوتية
(\textenglish{accents})
لتبسيط اللعبة (إن بدأنا اختبار \textenglish{é}، \textenglish{è}، \textenglish{ê}، \textenglish{ë}... فلن ننتهي أبداً !). عليك إذن أن تكتب كلماتك كلّها بحروف كبيرة و بدون علامات صوتيّة.
\end{information}

المشكل الذي سيحدث لك سريعا هو أنه عليك معرفة عدد الكلمات الموجودة في القاموس. في الواقع، إن أردت إختيار كلمة عشوائية، يجب أن يتم أخذ عدد بين 0 و
\textenglish{X}،
و أنت لا تعرف في بادئ الأمر كم من الكلمات يحتوي الملف.

لحلّ هذا المشكل، يوجد حلّان. يمكنك أن تشير في السطر الأول من الملفّ إلى عدد الكلمات الّتي يحويها :

\begin{Console}
3
HOUSE
BLUE
AIRPLANE
\end{Console}

إلا أن هذه الطريقة مملة، لأنه يجب إعادة حساب عدد الكلمات يدويا في كلّ مرّة تضيف فيها كلمة (أو إضافة 1 إلى هذا العدد إن كنت ماكرا بدل إعادة الحساب، لكنّها تبقى طريقة بدائيّة قليلا). لهذا، أقترح عليك أن تعدّ تلقائيّا عدد الكلمات عن طريق قراءة الملف مرّة أولى باستخدام برنامجك. معرفة كم يوجد من كلمات أمر بسيط : عليك عدّ الـ\InlineCode{\textbackslash n}
(العودة إلى السطر) في الملف.

حينما تقرأ الملفّ في مرّة أولى لعدّ
\InlineCode{\textbackslash n}،
فعليك القيام بـ\InlineCode{rewind}
للعودة إلى البداية. لن يكون عليك إذن سوى أخذ عدد عشوائيّ بين عدد الكلمات الّتي عددتها، ثمّ عليك تخزين هذه الكلمة في سلسلة محرفيّة في الذاكرة.

سأتركك قليلا لتفكّر في كلّ هذا، لن أساعدك أكثر، و إلّا فلن يكون عملا تطبيقيا ! و اعلم بأن  كلّ المعارف الّتي تحتاجها موجودة في الفصول السابقة، فأنت قادر تماما على إنشاء هذه اللعبة. إنه يتطلّب منك بعض الوقت و هو أقلّ سهولة ممّا يبدو عليه، و لكن إذا نظّمت الأمور جيّدا (بإنشاء قدر كاف من الدوال) سوف تصل.

بالتوفيق !

\section{التصحيح (1 : شفرة اللعبة)}
بقراءتك لهذه السطور، يعني أنك قد أكملت البرنامج، أو أنك لم تستطع إكماله.

لقد استغرقت شخصيّا وقتا أكبر ممّا كنت أعتقد في إنشاء هذه اللعبة البسيطة للغاية. هكذا دائما : نقول "هذا بسيط"، لكن في الحقيقة توجد الكثير من الحالات لدراستها.

رغم ذلك أصرّ على القول بأنك قادر على فعل هذا. يلزمك فقط بعض الوقت (بضع دقائق، بضع ساعات بضع أيام ؟)، لكنّنا لم نكن أبدا في سباق. أنا أفضّل أن تأخذ كثيرا من الوقت للوصول إلى الحل على ألّا تجرّب سوى 5 دقائق و ترى التصحيح.

لا تعتقد أنّي كتبت البرنامج من المحاولة الأولى. أنا أيضا، كنت أعمل خطوة بخطوة. بدأت بشيء بسيط جدّا، ثمّ شيئا فشيئا حسّنت الشفرة للوصول إلى النتيجة النهائيّة.\\
قمت بعدّة أخطاء أثناء كتابة الشفرة : نسيت في لحظة ما تهيئة متغير بشكل صحيح، نسيت كتابة نموذج دالة و كذلك حذف متغير لم يعد مفيدا في شفرتي. و حتى أنّي -أعترف- نسيت فاصلة منقوطة سخيفة في لحظة ما عند نهاية تعليمة.

لماذا أقول كل هذا ؟ لكي أخبرك أنّني لست معصوما من الأخطاء و أنّي أواجه تقريبا نفس المشاكل مثلك
("\textit{أيّها البرنامج البائس، هل ستعمل أم لا !؟}").

سأعرض عليك الحلّ على جزئين.
\begin{itemize}
  \item أوّلا سأريك كيف أنشأت شفرة اللعبة نفسها، بتثبيت الكلمة المخفيّة مباشرة في الشفرة. إخترت الكلمة
\textenglish{YELLOW}
لأنّها تسمح باختبار ما إن كنت تعاملت جيّدا مع المحارف المتكرّرة.
  \item يعد ذلك، سأريك كيف أضفت العمل بقاموس الكلمات لأخذ كلمة سرّية عشوائيّة لللاعب.
\end{itemize}

بالطبع، يمكنني أن أريك الشفرة دفعة واحدة و لكن... سيكون هذا كثيرا في مرّة واحدة، و البعض لن تكون لديه الشجاعة لمحاولة فهم الشفرة.

سأحاول أن أشرح لك خطوة بخطوة طريقة عملي. تذكّر أنّ ما يهم، ليس النتيجة، و إنّما طريقة التفكير.

  \chapter{إدخال نصّ بشكل أكثر أمانا}

إدخال النصوص في لغة الـ\textenglish{C}
هي من أكثر الأمور حساسية. أنت تعرف الدالة
\InlineCode{scanf}
التي تعرّفنا عليها في الدروس الأولى. ستقول : و أيّ الأدوات ستكون أكثر سهولة و طبيعية منها ؟ لكن جهّز نفسك، بعد هذا الدرس ستقول عنها أي شيء باستثناء "بسيطة".

الذين سيستعملون برنامجك هم بطبيعة الحال بشر. فهناك منهم من يخطئ في كتابة شيء، بينما هناك من يتعمّدون إرباك برنامجك بمعلومات غير منتظرة. فإن طلبت من المستعمل : ما هو عُمرك ؟ من يضمن لك بأنه لن يجيبك بـ:" إسمي فلان و أنا من البلد فلان" ؟

الهدف من هذا الدرس هو تعريفك إلى بعض المشاكل التي يمكن أن نواجهها أثناء استعمالنا للدالة
\InlineCode{scanf}،
و تقديم دالة بديلة أكثر أماناً و هي
\InlineCode{fgets}.

\section{حدود الدالة \texttt{scanf}}

هذه الدالة التي نستعملها جميعاً من الدروس الأولى في الكتاب، هي سلاح ذو حدين :

\begin{itemize}
  \item سهلة الاستعمال حينما نكون في مستوى "مبتدئ" ، و لهذا السبب عرّفتك بها.
  \item لكن الطريقة التي تعمل بها معقّدة و يمكن أن تكون خطيرة في بعض الحالات.
\end{itemize}

ألا يبدو الأمر متناقضا ؟ فإن الدالة
\InlineCode{scanf}
سهلة الاستعمال و في نفس الوقت أكثر تعقيداً مما نتصور، سأريك الحدود التي يمكن لهذه الدالة أن تصل إليها و ذلك بتقديم مثالين واقعيين.

\subsection{إدخال سلسلة محارف تحتوي على فراغات }

لنفرض أننا طلبنا من المستعمل أن يقوم بإدخال سلسلة محارف في الكونسول، و هو يقوم بكتابة فراغ في سلسلته~:

\begin{Csource}
  #include <stdio.h>
  #include <stdlib.h>
  int main(int argc, char *argv[])
  {
  	char name[20] = {0};
  	printf("What's your name ? ");
  	scanf("%s", name);
  	printf("Ah ! Your name is %s !\n\n", name);
  	return 0;
  }
\end{Csource}

\begin{Console}
  What's your name ? Mathieu Nebra
  Ah ! Your name is Mathieu !
\end{Console}

\begin{question}
لماذا اختفت الكلمة
"\textenglish{Nebra}"
؟
\end{question}

ذلك لأن الدالة
\InlineCode{scanf}
تتوقف عن القراءة حينما تصل إلى فراغ، أو رجوع إلى السطر أو محرف جدولة
(\textenglish{tabulation}).
يعني أنك غير قادر على قراءة سلسلة محرفيّة تحتوي على فراغات.

\begin{information}
  في الواقع، الكلمة
  "\textenglish{Nebra}"
  لازالت مخزّنة في الذاكرة، في  شيء نسميه بالمتغير المؤقّت
  (\textenglish{buffer})،
  المرة القادمة عندما نستدعي الدالة
  \InlineCode{scanf}
  فهي ستقوم بقراءة الكلمة
  "\textenglish{Nebra}"
    وحدها الموجودة في المتغير المؤقت.
\end{information}

يمكننا استعمال الدالة
\InlineCode{scanf}
بشكل يسمح لها بقراءة الفراغات، لكن الأمر معقّد جدّا. لمن يصرّ على ذلك، يمكنك إيجاد دروس مفصّلة على الويب، مثل الدرس الأجنبي المتوفّر على هذا الرابط :

\url{http://xrenault.developpez.com/tutoriels/c/scanf/}

\subsection{إدخال سلسلة محارف طويلة للغاية}

يوجد مشكل آخر، أكثر خطورة، و هو
\textbf{تجاوز الذاكرة}.

في الشفرة التي رأيناها، يوجد السطر التالي :

\begin{Csource}
  char name[5] = {0};
\end{Csource}

ترى أنني قمت بحجز 5 خانات من أجل الجدول المسمّى
\InlineCode{name}
الذي هو من نوع
\InlineCode{char}.
يعني أننا قادرون على تخزين كلمة من 4 محارف، بينما الحرف الأخير فهو محجوز لعلامة نهاية السلسلة
\InlineCode{\textbackslash 0}.\\
إذا نسيت كلّ هذا فراجع درس السلاسل المحرفية.

المخطط التالي يمثل المكان الذي هو محجوز للكلمة التي عرّفناها :

\Picture{Chapter_II-10_array}

ماذا لو كتبنا عددا كبيرا من المحارف بالنسبة للمساحة المتوقّعة لتخزين المتغير ؟

\begin{Console}
What's your name ? Patrice
Ah ! Your name is Patrice !
\end{Console}

ستقول أن كل شيء على ما يرام لكن الواقع أنك بصدد مواجهة أكبر كابوس لدى المبرمجين !

لقد قمنا بـ\emph{تجاوز في الذاكرة}،
 هذا ما نسميه بـ\emph{\textenglish{buffer overflow}}
بالإنجليزية.

كما ترى في المخطط التالي، لقد حجزت 5 خانات لكي تقوم باستعمال 8، ما الذي قامت به الدالة
\InlineCode{scanf}~؟
 لقد قامت بمواصلة الكتابة في الذاكرة وكأن شيئاً لم يحدث ! فلقد استغلّت خانات ليس لها الحق في الكتابة فيها.

\Picture{Chapter_II-10_array_patrice}

الذي جرى في الحقيقة، هو أن المحارف الزائدة تسببت في مسح معلومات من الذاكرة و استبدالها بهذه المحارف. هذا ما نسميه بالـ\emph{\textenglish{buffer overflow}}.

\Picture{Chapter_II-10_array_overflow}

\begin{question}
  لم الأمر خطير ؟
\end{question}

دون الدخول في التفاصيل، لأنه بإمكاننا البدء في محادثة  قدر 50 صفحة و لا نتوقف أبداً، فلنقل بأنه إن لم يقم البرنامج بالتحكم في حالات كهذه، فالمستعمل سيقوم بكتابة ما يحلو له و تخريب المعلومات المتواجدة في الخانات التالية من الذاكرة. أي أنه قادر على كتابة شفرة في تلك الخانات و برنامجك سيقوم بتشغيل تلك الشفرات و كأنها تابعة له، و هذا ما نسميه بالهجوم عبر المتغير المؤقت
\textenglish{buffer overflow attack}،
نوع من الهجومات المعروفة عند القراصنة، و لكنه صعب التحقيق.\\
إذا كنت مهتماً بهذا الموضوع، يمكنك قراءة المقال التالي من ويكيبيديا ( حذار، إنّه مع ذلك معقد جدا ) :

\url{http://fr.wikipedia.org/wiki/D%C3%A9passement_de_tampon}

الهدف من هذا الفصل هو تأمين قراءة البيانات و ذلك بمنع المستعمل من تجاوز الذاكرة و إحداث
\textenglish{buffer overflow}.
بالطبع كان بإمكاننا تعريف جدول كبير للغاية ( 10.000 خانة ) لكن هذا لا يحلّ المشكل فالشخص الذي يريد الوصول إلى الذاكرة ما عليه سوى إدخال سلسلة يتجاوز طولها 10.000 محرف و سيعمل هجومه كما يريد.

الشيء المحزن هو أن معظم المبرمجين لا ينتبهون دائما لهذه الأخطاء، و لو أنهم قاموا بكتابة الشفرة من المرة الأولى بشكل نظيف و صحيح، لما ظهرت كثير من الثغرات من التي نتحدّث عنها اليوم.

  \part{إنشاء ألعاب \textenglish{2D} في \textenglish{SDL}}
  \chapter{تثبيت \textenglish{SDL}}

ابتداءً من الآن، انتهت الدروس النظرية! لأننا سنمرّ إلى مرحلة مهمّة، وسنستمتع بالتطبيق بالاستعانة بمكتبة نسميها
\underline{\textenglish{SDL}}.

في الفصول السابقة كنا قد تطرّقنا تقريبًا لكلّ أساسيات اللغة
\textenglish{C}،
لكن تبقى هناك دائمًا بعض التفاصيل الصعبة نوعًا ما لنكتشفها. سأقول لك بأنه يُمكن لهذا الكتاب أن يتوقّف هنا مخبرا إيّاك: "نعم لقد تعلّمت البرمجة بلغة 
\textenglish{C}"،
لكني متأكّد بأن الجميع سيشاركني الرأي لو قلت بأن المُبرمج سيحسّ نفسه دائمًا مبتدئًا مادام لم "يخرج" من الكونسول!

\textenglish{SDL}
هي مكتبة تُستخدم خاصّة لإنشاء ألعاب ثنائية الأبعاد. سنتعرّف في هذا الفصل على هذه المكتبة ونتعلّم كيف نقوم بتثبيتها.

نسمي هذا النوع من المكتبات بمكتبات الطرف الثالث 
(\textenglish{Third party libraries}).
يجب أن تعرف أنه يوجد نوعان من المكتبات:

\begin{itemize}
	\item \textbf{المكتبة القياسية}
	(\textenglish{Standard library}):
	وهي المكتبة القاعدية التي تعمل على كلّ أنظمة التشغيل (من هنا تم استنباط الكلمة 
	\textenglish{standard})
	وهي تسمح بالقيام بأمور بسيطة كـ\InlineCode{printf}.
	هذه المكتبات يتمّ تسطيبها تلقائيّا عند تثبيتك للبيئة التطويرية والمترجم.
	
	خلال الجزأين الأوّلين من هذا الكتاب، كناّ قد استعملنا المكتبة القياسيّة فقط
(\InlineCode{stdlib.h}، \InlineCode{stdio.h}، \InlineCode{string.h}، \InlineCode{time.h}\dots).
	لم نقم بدراستها بالتفصيل لكنّا جرّبنا منها جزءً كبيرًا. إن كنت تريد معرفة المزيد عن هذا النوع من المكتبات أجْرِ بحثًا في 
	\textenglish{Google}،
	مثلًا بكتابة
	"\textenglish{C standard library}"،
	وستجد نماذج الدوال في هذه المكتبة، بالإضافة إلى شرح قصير حول دور كلّ دالة.
	\item \textbf{مكتبات الطرف الثالث}
	(\textenglish{Third party libraries}):
	هي مكتبات لا يتم تثبيتها تلقائيا. وإنّما يجب عليك تنزيلها من الأنترنت وتثبيتها بنفسك على حاسوبك.

	على عكس المكتبات القياسية، التي تكون بسيطة نسبيّا وتحتوي على عدد قليل من الدوال، فإنه توجد الآلاف من مكتبات الطرف الثالث، والتي تمت كتابتها من طرف مبرمجين آخرين. بعضها جيّدة، وأخرى أقل، بعضها مدفوع، وبعضها الآخر مجاني، إلخ. الأمر المثالي هو إيجاد مكتبة جيّدة ومجانية في نفس الوقت!
\end{itemize}

إنه لمن المستحيل أن أضع لك درسًا يشرح كل المكتبات الموجودة. حتّى لو أمضيت حياتي كلّها 24 ساعة / 24، لن أستطيع!\\
لذا سأقدّم لك مكتبة واحدة فقط مكتوبة بالـ\textenglish{C} ومُستعملة من طرف مبرمجين مثلك. 

هذه المكتبة تدعى 
\textit{\textenglish{SDL}}.
السؤال المطروح هو لماذا اخترت هذه المكتبة بالضبط؟ ما الذي يميّزها عن باقي المكتبات؟\\
هذه أسئلة سأبدأ في الإجابة عليها انطلاقًا من الآن.

\section{لماذا نختار \textenglish{SDL}؟}

\subsection{اختيار مكتبة ليس بالأمر السهل!}

كما قلت لك الآن، توجد الآلاف من المكتبات للتنزيل.\\
بعضها بسيط، وبعضها كبير جدًا لدرجة أن درسًا كهذا لا يكفي أن يشرحها كلّها!

الاختيار صعب. لكنّي اخترت هذه المكتبة، التي هي نوعًا ما سهلة الاستعمال، كبداية. ستكون هذه إذا أوّل مكتبة تقوم باستعمالها (إذا لم نحسب المكتبة القياسية).

إنه من الواضح أن أغلب القرّاء يريدون معرفة كيفية فتح نوافذ، إنشاء لعبة، إلخ. ولكن إن كنت تحب الكونسول فيمكننا الاستمرار فيها لوقت أطول، إذا أردت، لا؟ إذا لدينا هنا بعض الفضول! \\
أودّ كثيرًا أن أريك كيف تعمل كلّ هذه الأمور، لكننا سنحاول أن نتطرّق إليها خطوة بخطوة، وبالنسبة للأعمال التطبيقية، فلدينا عملان تطبيقيان لهذا الجزء من الكتاب!

لقد اخترت لك مكتبة سهلة وقوية، ستكون كبداية لك في تحقيق (تقريبا) أحلامك المتعلّقة بالواجهة الرسومية، ومن دون تعب (حسنًا، كلّ شيء نسبيّ بالطبع!).
\subsection{\textenglish{SDL}، اختيار جيّد!}

سنقوم الآن بدراسة هذه المكتبة. لماذا اخترتها هي وليس أخرى؟

\begin{figure}[H]
	\centering
	\includegraphics[width=0.2\textwidth]{Chapter_III-1_SDL}
\end{figure}

\begin{itemize}
	\item \textbf{هي مكتبة مكتوبة بلغة
	\textenglish{C}}:
	 أي أنه بإمكان المبرمجين أن يستعملوها في برامجهم المكتوبة بـ\textenglish{C}.
	 وكما هو الحال بالنسبة لأغلب المكتبات المكتوبة بـ\textenglish{C}،
	 يمكن استعمالها في لغة \textenglish{C++}
	 بالإضافة إلى لغات برمجية أخرى.
	 \item \textbf{هي مكتبة حُرّة ومجانية}:
و هذا كي لا تضطرّ لدفع أي ثمن مقابل استعمالك ما سأقدّمه لك في بقيّة الكتاب. على عكس ما قد نعتقد، إيجاد مكتبة جيدة ومجانية ليس أمرًا صعبًا كثيرًا، فقد انتشرت كثيرًا في أيامنا هذه. المكتبة الحرة هي ببساطة مكتبة يمكنك الحصول على الشفرة المصدرية الخاصة بها. في حالتنا هذه، رؤية الشفرة ليس مُهمّا بالنسبة لنا. لكن كونها حرة يفتح لنا الباب من أجل ميزات أخرى أهمّها المداومة (أي أنه إن توقف صاحب المكتبة عن تطويرها، يُمكن لمبرمجين آخرين أن يكملوا عمله)، بالإضافة إلى مجّانيّتها غالبا. هذا يعني عدم إمكانيّة اختفاء المكتبة في يوم من الأيام.
	 \item \textbf{يُمكنك إنشاء برامج تجارية ذات ملكية خاصة بفضل هذه المكتبة}.
	 قد أكون قد تسرّعت بهذا الكلام، لكنّه يجب اختيار مكتبة حرّة تمنحك الحريّة الأقصى. الحقيقة أنه يوجد نوعان من المكتبات الحُرة:
	 
	 \begin{itemize}
	 	\item المكتبات تحت 
	 	\textbf{رخصة \textenglish{GPL}}:
	 	 مكتبات مجانية، ويمكنك رؤية الشفرة المصدرية الخاصة بها، لكن بشرط أن تقوم أنت كذلك بنشر الشفرة المصدرية الخاصة بالبرنامج الذي أنشأته باستخدامها.
	 	\item المكتبات تحت 
	 	\textbf{رخصة \textenglish{LGPL}}:
	 	مثل سابقتها، لكن ليس عليك أن تنشر الشفرة المصدرية الخاصة بالبرنامج. أي أنه يمكنك بها إنشاء برامج مملوكة.
	 \end{itemize}
 	
 	\begin{information}
بالرغم من أنه يمكنك قانونيّا عدم نشر الشفرة المصدرية الخاصة بالبرنامج، إلا أنني أنصحك بذلك. فبهذا يمكنك أن تأخذ رأي المبرمجين الأكثر تمرّسًا منك. وهذا يسمح لك بالتحسّن. بعد هذا، فإن إنشاء برنامج حُر أو ذو ملكية خاصة، يرجع لطبيعة تفكير كل شخص. لن أدخل في نقاش بخصوص هذا الموضوع، لكن فلتعلم أن كلّ النوعين له مميزاته ومساوئه.
 	\end{information}

 	\item هي مكتبة
 	\textbf{متعددة المنصّات} 
 	(\textenglish{Multi-platform}):
 	سواء كنت على
 	\textenglish{Windows}،
 	\textenglish{\mbox{Mac OS X}}
 	أو
 	\textenglish{\mbox{GNU/Linux}}،
 	ستعمل لديك هذه المكتبة. والحقيقة أن هذه نقطة قوّة يراها المبرمجون بالمكتبة: يمكنها أن تعمل على عدد كبير جدًا من أنظمة التشغيل، فعلى غرار 
 	 	\textenglish{Windows}،
 	\textenglish{\mbox{Mac OS X}}
 	و
 	\textenglish{\mbox{GNU/Linux}}،
 	فهي تشتغل أيضًا على 
 	\textenglish{Atari}، \textenglish{Amiga}، \textenglish{Symbian}، \textenglish{Dreamcast}\dots
 	إلخ. أي أنه بالإمكان لبرامجك أن تعمل حتى على أجهزة 
 	\textenglish{Atari}
 	القديمة! مع ذلك يجب القيام ببعض التعديلات وربّما استخدام مترجم خاص. لن أدخل في التفاصيل هنا.
 	\item أخيرا، فإن هذه المكتبة تسمح لك بالقيام بالكثير من 
 	\textbf{الأمور الممتعة}
 	التي سنتعرّف إليها من خلال الفصول القادمة. لا أقول أنّ مكتبة رياضيّاتيّة قادرة على حلّ معادلات من الدرجة الرابعة ليست ممتعة، لكنّي سأركّز على أن يكون هذا الدرس سهلا قدر الإمكان لكيّ يحثّك على البرمجة.
\end{itemize}

هذه المكتبة ليست مخصصة فقط لإنشاء ألعاب الفيديو. سأعترف بأن معظم البرامج التي تمت كتابتها بهذه المكتبة، هي عبارة عن ألعاب، لكن هذا لا يعني أنك مجبر لاستعمالها من أجل ذلك. كما نعلم، كلّ شيء ممكن بالعمل والاجتهاد. كنت قد رأيت من قبل محرر نصوص تمت برمجته بالـ\textenglish{SDL}،
على الرغم من أنّه هناك مكتبات أخرى أحسن لهذا الغرض. إن كنت تريد برمجة واجهة رسومية تقليديّة تسمح بإظهار نافذة، زر، قائمة، إلخ. فأنا أنصحك إذا بالتوجّه إلى المكتبة 
\textenglish{GTK+}.

\subsection{الإمكانيات المتاحة بـ\textenglish{SDL}}

المكتبة
\textenglish{SDL}
هي مكتبة منخفضة المستوى. هل تتذكر أول الكتاب حينما حدّثتك عن لغات البرمجة عالية المستوى ولغات البرمجة منخفضة المستوى؟ هذا ينطبق على المكتبات أيضًا.

\begin{itemize}
	\item \textbf{المكتبات منخفضة المستوى}: 
	تحتوي على دوال قاعدية جدّا. يوجد عدد قليل من هذه الدوال لأنّه يمكننا القيام بكلّ شيء بها. وهذه الدوال لبساطتها تكون سريعة جدّا. لهذا فالبرامج المنشأة بهذا النوع من المكتبات تكون عادة الأسرع.
	\item \textbf{المكتبات عالية المستوى}: 
	تحتوي على الكثير من الدوال التي تسمح بالقيام بالكثير من المهام. هذا يجعلها أبسط من ناحية الاستخدام.
	
	لكن هذا النوع من المكتبات يكون عادة "كبيرا"، وليس من السهل دراستها ومعرفتها بأكملها. كما أنها قد تكون أثقل من المكتبات منخفضة المستوى (لكنّ هذا قد لا يكون واضحا).
\end{itemize}

على العموم، لا يمكننا القول بأن "مكتبة منخفضة المستوى هي أحسن من مكتبة عالية المستوى" أو العكس. فكلّ منهما لها مميزات ومساوئ. الـ\textenglish{SDL}
التي سنقوم بدراستها، تنتمي إلى المكتبات منخفضة المستوى.

يجب إذا أن تتذكر بأن الـ\textenglish{SDL}
تقدّم دوالا قاعدية. يمكنك إذا الرسم بيكسلا ببيكسل، رسم مستطيل أو إظهار صور. هذا كلّ شيء، وصدّقني أنّ هذا كافٍ.

\begin{itemize}
	\item بتحريك صورة، يمكنك أن تقوم بتحريك شخصيّة.
	\item بإظهار العديد من الصور الواحدة تلو الأخرى بسرعة، يمكنك إنشاء تحريك
	(\textenglish{Animation}).
	\item بوضع العديد من الصور، الواحدة بجنب الأخرى، يكون باستطاعتك إنشاء لعبة حقيقيّة.
\end{itemize}

كمثال عن لعبة تم صنعها بالـ\textenglish{SDL}،
اعلم أنّ اللعبة الشهيرة
"\textenglish{Civilisation: Call to power}"،
تم دعمها في نظام 
\textenglish{GNU/Linux}
لاحقًا باستخدام \textenglish{SDL}.

\begin{figure}[H]
	\centering
	\includegraphics[width=0.5\textwidth]{Chapter_III-1_Game}
\end{figure}

يجب أن تعلم أنّ جودة اللعبة تعود إليك وإلى الفريق الذي تعمل معه. إن كان لديك مصمم موهوب، فيمكنك صنع لعبة أجمل.

الشيء الوحيد الذي يحدّ \textenglish{SDL}
هو أنها تقتصر على الألعاب ثنائية الأبعاد، ولم تُنشأ من أجل الألعاب ثلاثية الأبعاد. هذه أمثلة على ألعاب يمكن تحقيقها بـ\textenglish{SDL}
(ليست سوى قائمة صغيرة، كلّ شيء ممكن مادام ثنائيّ الأبعاد):

\begin{itemize}
	\item \textenglish{Breakout}
	\item \textenglish{Bomberman}
	\item \textenglish{Tetris}
	\item ألعاب المنصّات:
	\textenglish{Super Mario Bros}، \textenglish{Sonic}، \textenglish{Rayman}،\dots
	\item \textenglish{RPG} ثنائية الأبعاد:
	\textenglish{Zelda}،
	الأجزاء الأولى للعبة
	\textenglish{Final Fantasy}،
	إلخ.
\end{itemize}
لا يمكن وضع لائحة كاملة، الأمر يعود فقط للقدرة على التخيّل. وصدّقني بأنك قادر على برمجة ألعاب فائقة الروعة. فلقد رأيت أحد القرّاء ينشئ تهجينا بين
\textenglish{Breakout}
و
\textenglish{Tetris}.

فلنعد إلى الأرض ولنمسك خيط هذا الفصل. سنقوم الآن بتسطيب المكتبة، لنتمكّن من التقدّم في العمل.

\section{تنزيل \textenglish{SDL}}

الموقع الرسمي للمكتبة 
\textenglish{SDL}
سيصبح قريبا الوجهة الّتي نقصدها كثيرا. هناك، يوجد كلّ ما تحتاجه، بدءً من المكتبة نفسها ومرورا إلى التوثيق 
(\textenglish{Documentation})
الخاص بها.

\url{http://www.libsdl.org/}

اذهب إلى اللائحة
\textenglish{Download}
المتواجدة على يسار الصفحة الرئيسية للموقع.\\
اختر النسخة الأحدث الّتي تجدها
(\textenglish{SDL 1.2}
عندما كتبت هذه السطور).

صفحة التنزيل مجزّأة إلى عدة أجزاء. 

\begin{itemize}
	\item \textbf{الشفرة المصدرية}
	(\textenglish{Source code}):
هنا يمكنك تحميل الشفرة المصدرية الخاصة بالمكتبة. أدري أن القراء فضوليون ليعرفوا كيف تعمل المكتبة من الداخل، لكنّ هذا لن يفيدنا. الأسوأ هو أنّه سيقوم بإلهائك عن هدفنا الرئيسي.
	\item \textbf{مكتبات وقت التشغيل}
	(\textenglish{Runtime libraries}):
	هي الملفات التي تحتاج إلى تقديمها مع الملف التنفيذي حين تريد أن تعطي برنامجك لشخص آخر. بالنسبة لـ\textenglish{Windows}، أنا أتكلم عن الملف 
	\InlineCode{SDL.dll}.
	هذا الأخير يجدر به أن يتواجد إما:
	
	\begin{itemize}
		\item بنفس المجلّد الذي يحتوي الملف التنفيذي
		(أنا أنصحك بهذا). الأحسن دائما هو أن تعطي الـ\textenglish{DLL}
		مع الملف التنفيذي وتبقيهم في نفس المجلد. إذا وضعت الـ\textenglish{DLL}
		في المجلّد الخاص بـ\textenglish{Windows}، لن يكون عليك إلحاق الـ\textenglish{DLL}
		مع كلّ مجلّد يحتوي البرنامج
		\textenglish{SDL}.
		ومع ذلك قد تحدث بعض المشاكل في حال ما قمت بمسح نسخة أحدث من الـ\textenglish{DLL}.
		\item في المجلّد
		\InlineCode{C:\textbackslash Windows}.
	\end{itemize}
	
	\item \textbf{مكتبات التطوير}
	(\textenglish{Development libraries}):
	هي الملفات
	\InlineCode{.a}
	(أو
	\InlineCode{.lib}
	بالنسبة لـ\textenglish{Visual})
	والملفات 
	\InlineCode{.h}
	التي تسمح بإنشاء برامجك
	\textenglish{SDL}.
	هذه الملفات ليست مفيدة إلا بالنسبة إليك أنت فقط المبرمج. أي أنه ليس عليك تقديمها مع ملفات البرنامج حين تنتهي من هذا الأخير.
\end{itemize}

إذا كنت تعمل في 
\textenglish{Windows}،
فسأعطيك ثلاثة نسخ، وذلك حسب المترجم الخاص بك:

\begin{itemize}
	\item \textenglish{VC6}:
	بالنسبة للذين يستخدمون النسخ القديمة غير المجانية من
	\textenglish{Visual studio}
	(لا أعتقد أن هناك من القرّاء من لازال يستعمل هذه النسخ). ستجد فيها على أي حال الملفات
	\InlineCode{.lib}.
	\item \textenglish{VC8}:
	 بالنسبة للذين يستعملون 
	\textenglish{Visual Studio 2005 Express}
	أو نسخة أحدث، ستجد فيها الملفات
	\InlineCode{.lib}.
	\item \textenglish{mingw32}:
	بالنسبة للذين يستعملون 
	\textenglish{Code::Blocks}
	(ستجدون فيها إذا الملفات
	\InlineCode{.a}).
\end{itemize}

الشيء الخاص هنا، هو أن "مكتبات التطوير" تحتوي كل الملفات
\InlineCode{.h}
و الملفات
\InlineCode{.a}
(أو
\InlineCode{.lib})
بالطبع، لكنها تحتوي أيضًا الملف 
\InlineCode{SDL.dll}
و ملفات التوثيق الخاصة بـ\textenglish{SDL}!\\
باختصار، كلّ ما عليك تنزيله هو "مكتبات التطوير"، فكلّ ما تحتاجه يتواجد بداخلها.

\begin{critical}
لا تخطئ في الرابط! قم باختيار \textenglish{SDL}
في قسم
"\textenglish{Development libraries}"
و ليس من قسم
"\textenglish{Source code}"!
\end{critical}

\begin{question}
ما هو التوثيق 
(\textenglish{Documentation})؟
\end{question}

التوثيق هو قائمة تحوي اللائحة الكاملة للدوال الخاصة بمكتبة معيّنة. وكل هذه الملفات تكون مكتوبة بالانجليزية (حتى لو كان كاتبوها مبرمجين فرنسيين). هذا سبب آخر يدفعك للتقدّم في لغة
\textenglish{Shakespeare}!

محتوى ملفات التوثيق  ليس عبارة عن درس، بل هي عادة موجزة. الشيء الإيجابي بالنسبة لدرس، هي أنّها تحتوي قائمة لكل الدوال الموجودة،  فهي إذا المرجع للمبرمج.\\
في كثير من الأحيان ستجد مكتبات بدون دروس تشرح كيفية عملها. وهنا لا يبقى لك سوى التوثيق الّذي نسميه عادة
"\textenglish{doc}"،
و يجب عليك تدبّر أمرك بهذا فقط (حتّى لو كان هذا صعبا أحيانا عندما تبدأ من دون أيّة مساعدة). المبرمج الحقيقيّ هو من يتمكّن من إيجاد ضالّته في الـ"\textenglish{doc}".

لحدّ الآن، أنت لست بحاجة إلى التوثيق الخاص بـ\textenglish{SDL}
لأنني أنا من سيشرح لك كيفية عملها. لكن بما أنني غير قادر على أن أشرح لك كل الدوال التي بها، ستحتاج إلى قراءة التوثيق لاحقا. 

ملفّات التوثيق توجد أصلًا في الحزمة
"\textenglish{Development libraries}"
كما سبق وذكرت، لكن بإمكانك تنزيلها وحدها من القائمة
\InlineCode{Documentation} / \InlineCode{Downloadable}.\\
أنصحك أن تجمع ملفات
\textenglish{HTML}
الخاصة بالتوثيق في مجلّد خاصّ (اسمه مثلًا
\InlineCode{Doc SDL})
ثم إنشاء اختصار إلى الفهرس
\InlineCode{index.html}.
و الهدف من هذا هو الوصول إلى هذه الملفات بشكل أسرع حينما تحتاج إليها.

\section{إنشاء مشروع \textenglish{SDL}: \textenglish{Windows}}

تثبيت مكتبة قد يكون أكثر صعوبة قليلًا مما تعوّد عليه الجميع. هنا لا يوجد تثبيت تلقائيّ يطلب منك أن تنقر "التالي"، "التالي"، "التالي"، "انتهى".

الحقيقة أن تثبيت مكتبة أمر صعب على المبتدئين. لكن لأقوم برفع المعنويات فإن تسطيب مكتبة 
\textenglish{SDL}
أمر سهل جدًا مقارنة بتسطيب مكتبات أخرى أتيحت ليّ فرصة استخدامها من قبل (هناك من يتم إعطاؤك منها الشفرة المصدرية فقط، بينما أنت تتولى أمر الترجمة!).

و الحقيقة أن كلمة "تثبيت" ليست الملائمة هنا. لن نقوم بتثبيت أي شيء، فقط نريد أن نصل إلى الكيفية التي ننشئ فيها مشروع 
\textenglish{SDL}
في البيئة التطويرية الخاصة بنا.\\
ستختلف كيفية التعامل حسب البيئة التطويرية التي تستعملها. سأقوم بتقديم الطريقة الخاصة بكلّ بيئة من بيئات التطوير التي قدّمتها في بداية الكتاب، وهكذا كي يستطيع الجميع المتابعة.

سأعرض الآن كيف ننشئ مشروع
\textenglish{SDL}
في كلّ واحد من البيئات الثلاث السابقة.

\subsection{تسطيب \textenglish{SDL} في \textenglish{Code::Blocks}}

\subsubsection{استخراج ملفات \textenglish{SDL}}

افتح الملف المضغوط
"\textenglish{Development Libraries}"
الذي قمت بتنزيله.\\
هذا الملف هو بامتداد
\InlineCode{.zip}
بالنسبة لـ\textenglish{Visual}
و 
\InlineCode{.tar.gz}
بالنسبة لـ\textenglish{mingw32}
(يلزمك برنامج مثل 
\textenglish{Winrar}
أو
\textenglish{7-Zip}
لكي تقوم بفك الضغط عن الملفات ذات الصيغة 
\InlineCode{.tar.gz}).

الملف المضغوط يحتوي العديد من المجلّدات الداخلية، وهذه هي الملفات التي تهمّنا:

\begin{itemize}
	\item \InlineCode{bin}:
	يحتوي الملف 
	\InlineCode{.dll}
	الخاص بـ\textenglish{SDL}.
	\item \InlineCode{docs}:
	يحتوي الملفات التوثيقيّة الخاصة بـ\textenglish{SDL}.
	\item \InlineCode{include}:
	يحتوي الملفات الرأسية
	\InlineCode{.h}.
	\item \InlineCode{lib}:
	يحتوي الملفات 
	\InlineCode{.lib}
	(أو
	\InlineCode{.a}
	بالنسبة لـ\textenglish{Code::Blocks})
\end{itemize}

يجب عليك استخراج كل الملفات والمجلّدات الداخلية ووضعها في مكان ما بالقرص الصلب لحاسوبك، يمكنك مثلا وضعها في مجلد خاص بـ\textenglish{SDL}
داخل مجلّد \textenglish{Code::Blocks}.

\begin{figure}[H]
	\centering
	\includegraphics[width=0.6\textwidth]{Chapter_III-1_SDL-folder}
\end{figure}


بالنسبة لي فإن المسار هو التالي:

\InlineCode{C:\textbackslash Program Files (x86)\textbackslash CodeBlocks\textbackslash SDL-1.2.13}

احفظ المسار الذي به البرنامج، ستحتاج إليه عندما تريد تعديل إعدادات 
\textenglish{Code::Blocks}
لاحقا.

والآن، علينا بالقيام بخطوة بسيطة، لتسهيل الأمور علينا، توجّه إلى المسار
\InlineCode{include/SDL}
(في حالتي، هو متواجد\\
بـ\InlineCode{C:\textbackslash Program Files (x86)\textbackslash CodeBlocks\textbackslash SDL-1.2.13\textbackslash include\textbackslash SDL})،
قم بنسخ الملفات الرأسية
\InlineCode{.h}
 في المجلّد الأب، (أي في:
\InlineCode{C:\textbackslash Program Files (x86)\textbackslash CodeBlocks\textbackslash SDL-1.2.13\textbackslash include}).

\begin{figure}[H]
	\centering
	\includegraphics[width=0.8\textwidth]{Chapter_III-1_SDL-h-copy}
\end{figure}
\begin{figure}[H]
	\centering
	\includegraphics[width=0.8\textwidth]{Chapter_III-1_SDL-h-past}
\end{figure}

ها قد تم تسطيب المكتبة، فلنقم الآن بتعديل إعدادات
\textenglish{Code::Blocks}.

\subsubsection{إنشاء مشروع \textenglish{SDL}}

افتح
\textenglish{Code::Blocks}
و قم بإنشاء مشروع جديد.

\begin{figure}[H]
	\centering
	\includegraphics[width=0.6\textwidth]{Chapter_III-1_CodeBlocks-New-Project}
\end{figure}


عوض أن تقوم باختيار 
\textenglish{Console Application}
كما جرت العادة، اختر 
\textenglish{SDL project}.

\begin{figure}[H]
	\centering
	\includegraphics[width=0.6\textwidth]{Chapter_III-1_SDL-Project}
\end{figure}

النافذة الأولى لا جدوى منها، قم بتجاوزها بالضغط على "التالي"
(\textenglish{Next}).

\begin{figure}[H]
	\centering
	\includegraphics[width=0.6\textwidth]{Chapter_III-1_SDL-Project-Welcome}
\end{figure}

سيُطلب منك أن تقوم بإدخال اسم المشروع، قم بذلك كالعادة:

\begin{figure}[H]
	\centering
	\includegraphics[width=0.6\textwidth]{Chapter_III-1_SDL-Project-path}
\end{figure}

الآن يجب اختيار المسار الذي ثبتنا فيه المكتبة:

\begin{figure}[H]
	\centering
	\includegraphics[width=0.6\textwidth]{Chapter_III-1_SDL-path}
\end{figure}

اضغط على الزر الذي يأخذ شكل مربّع به ثلاث نقاط، ستظهر لك النافذة التالية:

\begin{figure}[H]
	\centering
	\includegraphics[width=0.3\textwidth]{Chapter_III-1_SDL-path-select}
\end{figure}

قم باختيار المسار (بالنسبة لي هو
\InlineCode{C:\textbackslash Program Files (x86)\textbackslash CodeBlocks\textbackslash SDL-1.2.13}).

قد تظهر لك في مكان النافذة السابقة هذه النافذة:

\begin{figure}[H]
	\centering
	\includegraphics[width=0.55\textwidth]{Chapter_III-1_SDL-path-select-2}
\end{figure}

املأ الحقل
\textenglish{base}
بنفس الطريقة السابقة، ثمّ اضغط على زر الخروج، وستلاحظ أن المسار قد تم تسجيله كالتالي:

\begin{figure}[H]
	\centering
	\includegraphics[width=0.6\textwidth]{Chapter_III-1_SDL-path-filled}
\end{figure}

اضغط على
\textenglish{Next}،
 ستظهر لك نافذة اختيار المترجم، قم باختيار الأوضاع
\textenglish{Realease}
أو
\textenglish{Debug}
(هذا لا يهم).

أخيرا اضغط على "إنهاء"
(\textenglish{Finish}).
 سيتم إنشاء المشروع التجريبي:
 
 \begin{figure}[H]
 	\centering
 	\includegraphics[width=0.8\textwidth]{Chapter_III-1_SDL-test-project}
 \end{figure}


يحتوي المشروع على ملفين 
\InlineCode{main.cpp}
و ملف
\InlineCode{.bmp}
قبل أن تحاول القيام بالترجمة. يجب القيام بخطوة أخيرة (عليك القيام بها دائما)، وهي نسخ الملف 
\InlineCode{SDL.dll}
من ملفات المكتبة (الّذي يفترض أن يكون في المسار 
\InlineCode{C:\textbackslash Program Files (x86)\textbackslash CodeBlocks\textbackslash SDL-1.2.13\textbackslash bin\textbackslash SDL.dll})
و ضعه في المجلّد الخاص بالمشروع.

 \begin{figure}[H]
	\centering
	\includegraphics[width=0.7\textwidth]{Chapter_III-1_SDL-DLL-copy}
\end{figure}
 \begin{figure}[H]
	\centering
	\includegraphics[width=0.7\textwidth]{Chapter_III-1_SDL-DLL-past}
\end{figure}

أخرج من 
\textenglish{Code::Blocks}
و أعد الدخول إليه، ثم قم بترجمة البرنامج المُقترح مسبقًا. يفترض أن تظهر النافذة التالية:

 \begin{figure}[H]
	\centering
	\includegraphics[width=0.8\textwidth]{Chapter_III-1_SDL-test-window}
\end{figure}


إذا ظهرت لك النافذة السابقة، فهنيئا لك، المكتبة مثبتة بشكل جيد!

\begin{information}
إن ظهرت لك الرسالة 
"\textenglish{The application can't start because the file SDL.dll is missing}"
أي أنه لا يمكن تشغيل البرنامج، لأن ملف 
\InlineCode{SDL.dll}
غير موجود، فهذا يعني أنك لم تقم بنسخ الملف الأخير في ملفات المشروع كما طلبت منك!\\
و كما قلت، إن كنت تريد تسليم المشروع إلى أصدقائك، عليك بإرفاق الملف التنفيذي 
\InlineCode{.exe}
بالملف 
\InlineCode{SDL.dll}،
بينما أنت لست بحاجة إلى إعطائهم الملفات 
\InlineCode{.h}
و
\InlineCode{.a}
الّتي لا تهم أحدا سواك.
\end{information}

\begin{tcolorbox}[breakable,title=ملاحظات مترجمة الكتاب, colback=orange!20, colframe=orange!70, fontupper=\footnotesize, coltitle=white, fonttitle=\normalsize, attach title]
كلّ الشرح الموجود في هذا الجزء يعطي الطريقة الّتي استخدمتها مترجمة الكتاب لتشغيل البرنامج في حالة ما لم تعمل الطريقة السابقة (الأصليّة). هذا يعني أنّ هذه الفقرات التالية ليست موجودة في الكتاب الفرنسي الأصلي، وإنّما هي مساهمة شخصيّة من المترجمة.
\tcblower

إن لم يشتغل البرنامج ولازال المترجم يشير دائما إلى عدم وجود الملف 
\InlineCode{SDL.dll}،
فجرب نسخ هذا الأخير ولصقه في المجلدين 
\textenglish{Debug}
و 
\textenglish{Release}
من مجلّد المشروع، أي في نفس المكان الذي يتواجد به الملف التنفيذي.

أريد أن أنوّهك بأن المشروع الذي تم إنشاءه هو خاص باللغة
\textenglish{C++}
(لأنها اللغة التي يتم اختيارها تلقائيا من بيئة التطوير، كون أن هذه الأخيرة تم تطويرها للعمل بلغة \textenglish{C++})،
سنقوم إذا بتحويل هذا المشروع من
\textenglish{C++}
إلى 
\textenglish{C}
ببساطة.\\
توجّه إلى ملفات المشروع، ستجد الملف
\InlineCode{main.cpp}
قم بتغيير اسمه إلى
\InlineCode{main.c}.

 \begin{figure}[H]
	\centering
	\includegraphics[width=0.8\textwidth]{Chapter_III-1_SDL-rename}
\end{figure}

أدخل الآن إلى
\textenglish{Code::Blocks}،
ستظهر لك على الأرجح النافذة التالية:

 \begin{figure}[H]
	\centering
	\includegraphics[width=0.5\textwidth]{Chapter_III-1_SDL-keep-open}
\end{figure}


اضغط على الزر "لا"
(\textenglish{No}).

توجه إلى القائمة اليسارية، وقم بالنقر باليمين على الملف
\InlineCode{main.cpp}
و اختر حذفه من المشروع:

\begin{figure}[H]
	\centering
	\includegraphics[width=0.3\textwidth]{Chapter_III-1_SDL-remove-cpp}
\end{figure}

اضغط على اسم المشروع، واطلب إضافة ملف جديد:

\begin{figure}[H]
	\centering
	\includegraphics[width=0.3\textwidth]{Chapter_III-1_SDL-add}
\end{figure}

اختر الملف
\InlineCode{main.c}
من ملفات المشروع:

\begin{figure}[H]
	\centering
	\includegraphics[width=0.6\textwidth]{Chapter_III-1_SDL-mainc}
\end{figure}

ستظهر لك نافذة أخرى، قم باختيار
\textenglish{Next}،
ثم انقر على "موافق"
(\textenglish{OK}).

\begin{figure}[H]
	\centering
	\includegraphics[width=0.4\textwidth]{Chapter_III-1_SDL-selection}
\end{figure}

أنقر باليمين مجددا على الملف 
\InlineCode{main.c}
و اختر "خصائص" 
(\textenglish{Properties}):

\begin{figure}[H]
	\centering
	\includegraphics[width=0.4\textwidth]{Chapter_III-1_SDL-file-properties}
\end{figure}

توجه إلى القائمة
\textenglish{Advanced}،
ستجد
\textenglish{Compiler variable}،
غيّرها من
\textenglish{CPP}
إلى
\textenglish{CC}
كالتالي:

\begin{figure}[H]
	\centering
	\includegraphics[width=0.5\textwidth]{Chapter_III-1_SDL-file-properties-advanced}
\end{figure}

اضغط بعد ذلك على
\textenglish{OK}
هذا ما سيبدو عليه المشروع الجديد:

\begin{figure}[H]
	\centering
	\includegraphics[width=0.8\textwidth]{Chapter_III-1_SDL-new-file}
\end{figure}

ضع الشفرة التالية بدل الشفرة السابقة:

\begin{Csource}
#include <stdlib.h> 
#include <stdio.h> 
#include <SDL/SDL.h>
int main(int argc, char *argv[]) 
{ 
	SDL_Init(SDL_INIT_VIDEO); 
	SDL_SetVideoMode(640, 480, 32, SDL_HWSURFACE); 
	return EXIT_SUCCESS; 
}
\end{Csource}

و أخيرًا، من القائمة العلوية، اختر هدف البناء 
\InlineCode{Release}.

\begin{figure}[H]
	\centering
	\includegraphics[width=0.3\textwidth]{Chapter_III-1_SDL-build-target}
\end{figure}


يمكنك ترجمة البرنامج، ستظهر لك نافذة وتختفي فجأة، لا تقلق، سنعالج ذلك لاحقًا، أنا أهنّئك، كل شيء يعمل بشكل جيد جدا.

\end{tcolorbox}

يمكنك مسح الملف 
\InlineCode{.bmp}
لأننا لسنا بحاجة إليه. بالنسبة للملف 
\InlineCode{main.c}، 
يمكنك الآن استبدال محتواه بالشفرة التالية:

\begin{Csource}
#include <stdlib.h>
#include <stdio.h>
#include <SDL/SDL.h>
int main(int argc, char *argv[])
{
	return 0;
}
\end{Csource}
إنها شفرة مبدئية، تشبه الشفرات التي تعوّدنا عليها (تضمين
\InlineCode{stdlib.h}
و
\InlineCode{stdio.h}
ثم 
\InlineCode{main}).
الشيء الوحيد الذي تغيّر هو تضمين الملف
\InlineCode{SDL.h}.
إنّه ملف رأسيّ يتكفل بتضمين كل الملفات الرأسية الخاصة بالمكتبة 
\textenglish{SDL}.

\subsection{إنشاء مشروع \textenglish{SDL} في \textenglish{Visual C++}}

\subsubsection{استخراج ملفات \textenglish{SDL}}

من الموقع الرسمي، قم بتنزيل آخر نسخة من المكتبة من قسم
"\textenglish{Development Libraries}"
و اختر نسخة
\textenglish{Visual C++ 2005 Service Pack 1}.

افتح الملف
\InlineCode{.zip}.\\
إنه يحتوي  على التوثيق (في المجلد
\InlineCode{docs})،
 الملفات
\InlineCode{.h}
(في المجلد
\InlineCode{include})، 
و الملفات
\InlineCode{.lib}
(في المجّلد
\InlineCode{lib})
المكافئة للملفات
\InlineCode{.a}
بالنسبة لمترجم
\textenglish{Visual}.
ستجد أيضًا الملف 
\InlineCode{SDL.dll}
في المجلّد
\InlineCode{lib}.

\begin{itemize}
	\item انسخ الملف
	\InlineCode{SDL.dll}
	إلى مجلّد المشروع.
	\item انسخ الملفات
	\InlineCode{.lib}
	إلى المجلّد 
	\InlineCode{lib}
	الخاص بـ\textenglish{Visual C++}.
	بالنسبة لي، أنا أتكلم عن المجلّد
	
	\InlineCode{C:\textbackslash Program Files (x86)\textbackslash Microsoft Visual Studio 8\textbackslash VC\textbackslash lib}.
	
	\item انسخ الملفات
	\InlineCode{.h}
	إلى المجلّد
	\InlineCode{includes}
	الخاص بـ\textenglish{Visual C++}.
	أنشئ مجلّدا
	\InlineCode{SDL}
	في المجلد
	\InlineCode{includes}
	لجمع الملفات
	\InlineCode{.h}
	الخاصة بـ\textenglish{SDL}
	فيه. بالنسبة لي، سأضع تلك الملفات في المجلد:
	
	\InlineCode{C:\textbackslash Program Files (x86)\textbackslash Microsoft Visual Studio 8\textbackslash VC\textbackslash include\textbackslash SDL}.

\end{itemize}

\subsubsection{إنشاء \textenglish{SDL} مشروع جديد}

في \textenglish{Visual C++}
أنشئ مشروعًا من نوع 
\textenglish{Application console Win32}.
سمّه مثلًا
\InlineCode{testsdl}
ثم اضغط على "موافق".

ستفتح نافذة مساعدة. توجه إلى 
\textenglish{Application parameters}
و تأكد من أن الخانة
\InlineCode{Empty project}
مختارة.

\begin{figure}[H]
	\centering
	\includegraphics[width=0.6\textwidth]{Chapter_III-1_VisualCpp-Assistant}
\end{figure}

لقد تمّ إنشاء المشروع إذا. إنّه فارغ. أضف إليه ملفًا جديدًا وذلك بالنقر على
\InlineCode{Source files}
ثم
\InlineCode{Add}
ثم
\InlineCode{New element}:

\begin{figure}[H]
	\centering
	\includegraphics[width=0.4\textwidth]{Chapter_III-1_VisualCpp-Add}
\end{figure}

حينما تفتح نافذة جديدة، أطلب إنشاء ملف جديد من نوع
\InlineCode{C++ File (.cpp)}،
قم بتسميته
\InlineCode{main.c}.
بوضعك للامتداد
\InlineCode{.c}
فإن \textenglish{Visual}
سيقوم بإنشاء ملف
\textenglish{C}
و ليس
\textenglish{C++}.

أكتب (أو انسخ/ألصِق) الشفرة المبدئية الذي وضعتها أعلاه، في الملف الجديد الذي أنشأته.

\subsubsection{تخصيص مشروع \textenglish{SDL} في \textenglish{Visual C++}}

التعديل على المشروع أصعب قليلًا مما هو الحال مع \textenglish{Code::Blocks}،
لكن بقليل من التركيز، ستتمكن من فعله. توجّه إلى خصائص المشروع من
\InlineCode{Project} / \InlineCode{testsdl properties}.

\begin{itemize}
	\item في القسم
	\InlineCode{C / C++} / \InlineCode{Code generation}
	عدّل قيمة \InlineCode{Runtime Libraries}
	إلى\\
	\InlineCode{DLL multithread (/MD)}.
	\item في القسم 
	\InlineCode{C / C++} / \InlineCode{Advanced}
	اختر
	\InlineCode{Compilation as}
	وضع القيمة\\
	\InlineCode{Compile as C code (/TC)}
	(وإلا فإن
	\textenglish{Visual}
	سيترجم البرنامج كأنه ملف
	\textenglish{C++}
	وليس كملف
	\textenglish{C}).
	\item في القسم  
	\InlineCode{Link editor} / \InlineCode{Input}
	عدّل قيمة الـ\InlineCode{Additional dependencies}
	لكي تضيف
	\InlineCode{SDL.lib}
	و
	\InlineCode{SDLmain.lib}.
	\item في القسم
	\InlineCode{Link editor} / \InlineCode{System}
	عدّل قيمة الـ\InlineCode{Sub-System}
	إلى \InlineCode{Windows}.
\end{itemize}

\begin{figure}[H]
	\centering
	\includegraphics[width=0.6\textwidth]{Chapter_III-1_VisualCpp-project-properties}
\end{figure}


قم بالضغط على "موافق" لحفظ التغييرات.\\
يمكنك الآن الترجمة وذلك بالذهاب إلى
\InlineCode{Generate}
ثم
\InlineCode{Generate solution}.

ستجد الملف التنفيذي الذي يتواجد بمجلد المشروع (أو بمجلد داخلي يسمى
\InlineCode{Debug})
و لا تنس أنه على الملف
\InlineCode{SDL.dll}
أن يتواجد في نفس المجلّد الذي يتواجد به الملف التنفيذي. أنقر مرّتين على
\InlineCode{.exe}،
 إذا سار كلّ شيء على ما يرام، فلن يحصل أيّ شيء، وإلّا فسيحدث خطأ إذا لم يكن الملف
\InlineCode{SDL.dll}
في نفس المجلّد.

\section{إنشاء مشروع \textenglish{SDL}: \textenglish{Mac OS} (\textenglish{Xcode})}

فلتقم بتنزيل النسخة 1.2 من \textenglish{SDL}،
و ذلك من خلال الجزء
"\textenglish{Download}"
أسفل يسار الموقع، كالتالي:

\begin{figure}[H]
	\centering
	\includegraphics[width=0.2\textwidth]{Chapter_III-1_SDL-Mac-Download}
\end{figure}

في أسفل الصفحة ستجد قسمًا يدعى 
"\textenglish{Runtime Libraries}".
نزّل الملف الذي يتناسب مع هندسة معالج جهازك
(\textenglish{Intel} أو \textenglish{PowerPC})،
 هذا ما ستوضحه الصورة الموالية. إن كنت تريد معرفة هندسة المعالج، يمكنك الذهاب إلى القائمة 
"\textenglish{Apple}"
في أعلى اليسار، والنقر على 
"\textenglish{About this Mac}".
في السطر 
"\textenglish{Processor}"
ستجد إما
\textenglish{Intel} أو \textenglish{PowerPC}.

\begin{figure}[H]
	\centering
	\includegraphics[width=0.4\textwidth]{Chapter_III-1_SDL-Mac-Download-version}
\end{figure}

حينما يتم تنزيل الملف، انقر عليه مرتين، يفترض أن يفتح لوحده. ستجد بهذا المجلّد مجلّدا
\InlineCode{SDL.framework}
قم بنسخه ولصقه في المجلد 
\InlineCode{/Library/Frameworks}.

انتهى، المكتبة مسطّبة الآن!\\
ستجد مجلّدا آخر اسمه 
\InlineCode{devel-lite}
أتركه مفتوحا، سنعود إليه لاحقًا.

الآن قم بإنشاء مشروع جديد
"\textenglish{Cocoa Application}"،
اضغط على
"\textenglish{Next}".
في 
\InlineCode{Product Name}
قم بتسمية المشروع (كـ"\textenglish{SDL}"
مثلًا). وفي 
\InlineCode{Company Identifier}،
ضع ما تريد (كاسم مستعار لك مثلا). اترك الباقي كما هو ثم اضغط على 
"\textenglish{Next}".
اختر أين تريد وضع المشروع. سيتم إنشاء مجلّد بطريقة تلقائية وليس عليك إنشاء واحد بنفسك ووضع ملفات المشروع بداخله.

ما إن يتم إنشاء المشروع، قم بالتخلص من الملفات التي لا تحتاجها:
\InlineCode{AppDelegate.h}، \InlineCode{AppDelegate.m}، \InlineCode{MainMenu.xib}، \InlineCode{InfoPlist.strings}، \InlineCode{main.m} و\InlineCode{Credits.rtf}:

\begin{figure}[H]
	\centering
	\includegraphics[width=0.4\textwidth]{Chapter_III-1_SDL-project-unused-files}
\end{figure}

اختر المشروع من التفرع الشجري اليساري (القسم
\InlineCode{Install SDL}
من الصورة الموالية) في الشجرة الثانية اختر اسم مشروعك من قسم 
\InlineCode{PROJECT}
و ليس من
\InlineCode{TARGETS}:

\begin{figure}[H]
	\centering
	\includegraphics[width=0.2\textwidth]{Chapter_III-1_Xcode-SDL-project}
\end{figure}

يمكنك أيضًا تغيير الـ\textenglish{localisation}
من 
\InlineCode{English}
إلى 
\InlineCode{French}.
اختر 
\InlineCode{English}،
أنقر على
\InlineCode{-}
للمسح وعلى 
\InlineCode{+}
لإضافة 
\InlineCode{French}،
هذا يعود إليك ولست مضطرًا للقيام بذلك.

سنقوم الآن بتخصيص المشروع على نظام
\textenglish{32 bits}
( لأن المكتبة لا تشتغل على أنظمة
\textenglish{64 bits})،
و سنقوم بإضافة المسارات من أجل الـ\textenglish{frameworks}،
و للملفات الرأسية أيضًا. اضغط على
\InlineCode{Build Settings}
ثم 
\InlineCode{All}
ثم في 
\InlineCode{Architectures}
انقر على
\InlineCode{64-bit Intel}
 واختر
\InlineCode{32-bit Intel}:

\begin{figure}[H]
	\centering
	\includegraphics[width=0.8\textwidth]{Chapter_III-1_Xcode-SDL-build-setting}
\end{figure}


ما إن تفعل ذلك، اختر
\InlineCode{LLVM GCC 4.2}
من السطر
\InlineCode{Compiler for C/C++/Objective-C}.

\begin{figure}[H]
	\centering
	\includegraphics[width=0.8\textwidth]{Chapter_III-1_Xcode-gcc}
\end{figure}

اذهب إلى منطقة البحث في أعلى اليمين، واكتب
"\textenglish{search paths}"،
يجدر بك أن تجد سطرين مهمّين بالنسبة لنا وهما
\InlineCode{Header search paths}
و 
\InlineCode{Framework search paths}.
انقر مرتين على الجهة اليمنى للسطر 
\InlineCode{Framework search paths}
أنقر على علامة
\InlineCode{+}
 وأضف المسار
\InlineCode{/Library/Frameworks}.
بالنسبة للسطر 
\InlineCode{Header Search paths}
أضف المسار\\
\InlineCode{/Library/Frameworks/SDL.framework/Headers}.

\begin{figure}[H]
	\centering
	\includegraphics[width=0.8\textwidth]{Chapter_III-1_Xcode-SDL-framework}
\end{figure}

اختر الآن "هدفك"، وهذه المرة من قسم 
\InlineCode{TARGETS}:

\begin{figure}[H]
	\centering
	\includegraphics[width=0.2\textwidth]{Chapter_III-1_Xcode-SDL-target}
\end{figure}

توجه إلى 
\InlineCode{Summary}،
في المنطقة
\InlineCode{Application Category}
يمكنك وضع ما تشاء، لن يغير هذا شيئا كبيرًا، لأنه ينفع من أجل \textenglish{AppStore}
فقط. عدّل السطر 
\InlineCode{Main Interface}
و ضع 
"\textenglish{SDLMain}".
بالنسبة للـ\InlineCode{App Icon}
فاسمه يدل عليه، فهو يسمح لك بتحديد أيقونة لبرنامجك. يكفي سحب ثم تحرير الصورة المُراد استعمالها كأيقونة. بالنسبة للمنطقة 
\InlineCode{Linked Frameworks and Libraries}
سنقوم بإضافة الـ\textenglish{framework}
الخاص بنا
\InlineCode{SDL.framework}،
أنقر على
\InlineCode{+}
في منطقة البحث ، أكتب 
"\textenglish{SDL}"،
حين تجده في القائمة، أنقر على 
\InlineCode{Add}،
إن لم تجده فهذا يعني أنك لم تقم بوضعه في المجلّد المناسب 
(\InlineCode{/Library/Frameworks}).

\begin{information}
الأيقونات في
\textenglish{Mac OS}
هي بصيغة
\InlineCode{.icns}.
إن استعملت صيغة أخرى فستلاحظ أن الأيقونة لا تظهر. إن أردت تحويل صيغة صورة عادية إلى أيقونة استعمل برنامج 
\textbf{\textenglish{Icon Composer}}.
المتواجد في المجلد
\InlineCode{/Developer/Applications/Utilities}،
يكفي أن تسحب الأيقونة إلى المربع المخصص لها ثم تحفظها.
\end{information}

في المنطقة 
\InlineCode{Info}
يمكننا أن نشير إلى العديد من المعلومات في البرنامج، يمكنك الإطلاع عليها أكثر من خلال قراءة الملفات التوثيقية الخاصة بـ\textenglish{Apple}.
الشيء الوحيد الذي يمكنك التعديل عليه هو 
\InlineCode{Localization}
من 
\InlineCode{en}
إلى 
\InlineCode{fr}،
كما يمكنك تعديل الـ\InlineCode{Copyright}
و وضع ما تريد.

توجّه الآن إلى 
\InlineCode{Build Phases}
و انقر على 
\InlineCode{Add Build Phase} / \InlineCode{Copy Files}
في أسفل يمين النافذة، اضغط على 
\InlineCode{Copy Files}
و غيّره إلى 
\InlineCode{Copy frameworks into app}.
في 
\InlineCode{Destination}
اختر
\InlineCode{Frameworks}
لتضيف الخاصة بك، قم بسحبها من التفرع الشجري اليساري وإفلاتها في المنطقة
\InlineCode{Build phase}،
كما يظهر بالصورة:

\begin{figure}[H]
	\centering
	\includegraphics[width=0.8\textwidth]{Chapter_III-1_Xcode-drop-framework}
\end{figure}

أنصحك بترتيب كل الـ\textenglish{frameworks}
الخاص بك في مجلد 
\InlineCode{Framework}
و هذا لكي يسهل عليك إيجادها.\\
و أيضًا بالنسبة للشفرات المصدرية، أنصحك بترتيبها في مجلدات ليسهل الوصول إليها. لإنشاء مجلد انقر باليمين على الشجرة اليسارية واختر
\InlineCode{New Group}
ثم اسحب الملفات إلى داخلها.

سنقوم الآن بإضافة الملفات
\InlineCode{SDLMain.h}
و 
\InlineCode{SDLMain.m}،
توجه إلى المجلد 
\InlineCode{devel-lite}
المفتوح مسبقا وقم بإضافة الملفين إلى المشروع. إذا ظهرت لك نافذة تحديد خصائص النسخ، قم باختيار\\
\InlineCode{Copy items into destination group's folder (if needed)}.

آخر شيء: أنشئ ملفا
\InlineCode{main.c}.
توجه إلى القائمة 
\InlineCode{File} / \InlineCode{New} / \InlineCode{New File}
ثم إلى 
\InlineCode{C and C++}، 
اختر
\InlineCode{C File}
ثم
"\textenglish{Next}".
قم بتسمية الملف وها قد أكملت.

\section{إنشاء مشروع \textenglish{SDL}:  \textenglish{GNU/Linux}}

لمن يستعملون بيئة تطويرية من أجل الترجمة، فعليهم بتغيير خواص المشروع (فالعملية مشابهة لما كنت قد شرحت). بالنسبة لمن يستعمل
\textenglish{Code::Blocks}،
فالطريقة هي نفسها التي شرحتها سابقًا.

\begin{question}
 ماذا عن الذين يقومون بترجمة الشفرات يدويا؟
\end{question}

قد يوجد بين القراء من اعتاد على ترجمة الشفرات يدويا بالاستعانة بـ\textenglish{Makefile}
(ملف يساعد على عملية الترجمة).\\
إذا كانت هذه حالتك، فأدعوك لتحميل
\textenglish{Makefile}
الّذي يُمكن أن يُستخدم لترجمة مشاريع \textenglish{SDL}.

\url{http://www.siteduzero.com/uploads/fr/ftp/mateo21/makefile_sdl}

الشيء الوحيد المختلف، هو إضافة المكتبة
\textenglish{SDL}
إلى محرّر الروابط
(\InlineCode{LDFLAGS}).
يجدر بك أن تكون قد نزّلت المكتبة وثبّتها في مجلّد ملفات المترجم، بنفس طريقة
\textenglish{Windows}
(المجلدان
\InlineCode{include/SDL}
و\InlineCode{lib}).

بعد ذلك يجب عليك أن تكتب الأوامر التالية في الكونسول:

\begin{Console}
make      	# To compile the project
make clean	# To delete compilation files (useless .o files)
make mrproper	# To delete all files except source ones
\end{Console}

\section*{ملخّص}

\begin{itemize}
	\item الـ\textenglish{SDL}
	مكتبة منخفضة المستوى، تسمح بإنشاء نوافذ والتعامل مع الرسوميات
	\textenglish{2D}.
	\item المكتبة ليست مسطّبة تلقائيا في الحاسوب، يجب عليك أن تنزّلها بنفسك وتقوم بتخصيص البيئة التطويرية لتعمل معها.
	\item المكتبة حرة ومجّانية، مما يسمح باستعمالها السريع والدائم.
	\item توجد آلاف المكتبات الأخرى، وكثير منها ذو جودة عالية جدًا. وقد تم اختيار المكتبة 
	\textenglish{SDL}
	لبقيّة هذا الكتاب لأجل سهولتها. لمن يريد بناء واجهات رسومية بنوافذ، أزرار وقوائم، فأنا أنصحه بالمكتبة 
	\textenglish{GTK+}
	مثلا.
\end{itemize}

  \chapter{إنشاء نافذة و مساحات}

في الدرس السابق، قمنا بالإلمام حول أهم المميزات التي تمنحها المكتبة 
\textenglish{SDL}.
يجدر بك أن تكون قد ثبتّ المكتبة، و تعلّمت كيفية إنشاء مشروع جديد يشتغل بشكل جيد.  على الرغم من أنّه كان فارغا.

سندخل في مضمون موضوعنا في هذا الفصل. سنقوم بتطبيق أساسيات لغة الـ\textenglish{C}
مع
\textenglish{SDL}.
كيف يتم تحميل الـ\textenglish{SDL} ؟
كيف يتم فتح نافذة بالأبعاد التي نريد ؟ كيف نرسم داخل النافذة ؟

لدينا أمور كثيرة لنعرفها، فهيّا بنا !

\section{تحميل و إيقاف الـ\textenglish{SDL}}

العديد من المكتبات المكتوبة بلغة الـ\textenglish{C}،
تستلزم أن يتم تحميلها ثم غلقها حين ننتهي منها، و ذلك لاستعمال دوال محددة. المكتبة
\textenglish{SDL}
من بين هذه المكتبات. 

بالفعل، فالمكتبة تحتاج أن يتم تحميل عدد معيّن من المعلومات إلى الذاكرة العشوائية لتستطيع أن تشتغل بشكل صحيح. يتم هذا التحميل بشكل حيّ باستعمال الدالة 
\InlineCode{malloc}
(إنّها مهمّة جدّا هنا !). و كما تعلم فإن قلت
\InlineCode{malloc}،
سأقول كذلك
\InlineCode{free} !\\
يجب عليك تحرير الذاكرة التي حجزتها و لم تعد بحاجة إليها. إن لم تفعل، فالبرنامج يمكن أن يأخذ حيّزاً كبيراً من الذاكرة بدون فائدة، و يمكن لذلك أحيانا أن يدرّ بنتائج كارثية. تخيّل القيام بحلقة غير منتهية من
\InlineCode{malloc}
دون قصد، في بضع ثوان ستسدّ كلّ الذاكرة !

هاهما الدالتان الأولتان الخاصّتان بالـ\InlineCode{SDL}
اللتان يجب عليك أن تعرفهما :
\begin{itemize}
	\item \InlineCode{SDL\_Init} :
	تحميل المكتبة في الذاكرة العشوائية (باستخدام الـ\InlineCode{malloc}).
	\item \InlineCode{SDL\_Quit} :
	تحرير المكتبة من الذاكرة (باستعمال الـ\InlineCode{free}).
\end{itemize}

أي أن أوّل شيء يجب أن تقوم به في البرنامج هو استدعاء
\InlineCode{SDL\_Init}،
و آخر شيء هو استدعاء
\InlineCode{SDL\_Quit}.

\subsection{\texttt{SDL\_Init} : تحميل المكتبة \textenglish{SDL}}

الدالة
\InlineCode{SDL\_Init}
تستقبل معاملا. إذ يجب أن يتم تحديد أي جزء من المكتبة نريد تحميله.

\begin{question}
آه حقّا ! هل الـ\textenglish{SDL}
تتكون من كثير من الأجزاء ؟
\end{question}

نعم بالطبع ! فهناك جزء من المكتبة يتعامل مع الشاشة، و آخر يتعامل مع الصوت، إلخ.

توفّر لنا المكتبة عدداً من الثوابت التي تسمح لنا بتحديد اسم الجزء الذي نريد تحميله من المكتبة.

\begin{Table}{2}
الثابت & الشرح \\
\texttt{SDL\_INIT\_VIDEO} &
تحميل الجزء الخاص بالعرض (الفيديو)، إنه الجزء الذي نحمله غالباً.\\
\texttt{SDL\_INIT\_AUDIO} &
تحميل الجزء الخاص بالصوت، هذا ما يسمح لك مثلا بتشغيل الموسيقى مثلا.\\
\texttt{SDL\_INIT\_CDROM} &
تحميل الجزء الخاص بقارئ القرص المضغوط، و ذلك للتحكم به.\\
\texttt{SDL\_INIT\_JOYSTICK} &
تحميل الجزء الخاص بجهاز التحكم 
\textenglish{Joystick}.\\
\texttt{SDL\_INIT\_EVERYTHING} &
تحميل كل الأجزاء التي ذكرتها سابقا.\\
\end{Table}

إذا استدعيت الدالة بهذا الشكل

\begin{Csource}
SDL_Init(SDL_INIT_VIDEO);
\end{Csource}

فإن نظام العرض سيتم تحميله في الذاكرة، فيمكنك أن تفتح نافذة و ترسم فيها، إلخ.\\
كل ما قمنا به هو إعطاء عدد إلى الدالة 
\InlineCode{SDL\_Init}
بالاستعانة بثابت. أنت لا تعرف أي عدد هو، و هذا أمر جيد. إذ أنك غير مجبر على حفظ العدد، بل التعبير عنه باسم الثابت فقط. 

الدالة 
\InlineCode{SDL\_Init}
تقرأ العدد و هكذا تحدد الأنظمة الواجب تحميلها.

الآن لو تكتب :

\begin{Csource}
SDL_Init(SDL_INIT_EVERYTHING);
\end{Csource}

ستقوم بتحميل كل أنظمة الـ\textenglish{SDL}،
لا تقم بهذا إلا في حالة كنت بالفعل تحتاج إلى كلّ شيء، ليس جيداً إثقال الحاسوب بوحدات لا فائدة منها.

\begin{question}
ماذا لو أردت تحميل الصوت و الفيديو فقط. هل يجدر بي استخدام
\InlineCode{SDL\_INIT\_EVERYTHING} ؟
\end{question}

لن تستعمل
\InlineCode{SDL\_INIT\_EVERYTHING}،
من أجل تحميل وحدتين، هذا جنون ! لحسن الحظ، يمكننا تجميع الخيارات بواسطة الرمز
\InlineCode{|}.

\begin{Csource}
// Loading the video and the audio
SDL_Init(SDL_INIT_VIDEO | SDL_INIT_AUDIO);
\end{Csource}

كما يمكنك وضع ثلاثة دون مشاكل :

\begin{Csource}
// Loading the video, the audio and the timer
SDL_Init(SDL_INIT_VIDEO | SDL_INIT_AUDIO | SDL_INIT_TIMER);
\end{Csource}

\begin{information}
هذه "الخيارات" التي نبعثها للدالة 
\InlineCode{SDL\_Init}
نسميها بـ\textit{الأعلام}
(\textenglish{flags}).
هذه الكلمة نستعملها كثيراً في علوم الحاسوب.\\
تذكّر إذا أن الإشارة
\InlineCode{|}
خاصة بدمج الخيارات. إنها تشبه الإضافة إلى حدّ ما.
\end{information}

\subsection{\texttt{SDL\_Quit} : إيقاف المكتبة \textenglish{SDL}}

هذه الدالة سهلة الاستعمال لأنها لا تحتاج إلى أي معامل :

\begin{Csource}
SDL_Quit();
\end{Csource}

كل الأنظمة سيتم إيقافها و يتم تحرير الذاكرة.\\
باختصار، هذه الدالة أداة للخروج من المكتبة بشكل نظيف، و لنقل للخروج من برنامجك.

\subsection{نموذج عن برنامج \textenglish{SDL}}
باختصار، هذا ما يبدو عليه برنامج
\textenglish{SDL}
في نسخته الأبسط :

\begin{Csource}
#include <stdlib.h>
#include <stdio.h>
#include <SDL/SDL.h>
int main(int argc, char *argv[])
{
	SDL_Init(SDL_INIT_VIDEO); // Starting the SDL (Here we load the video system)
	SDL_Quit(); // Stopping the SDL (Freeing the memory).
	return 0;
}
\end{Csource}

هذا نموذج عن برنامج بسيط، عبارة عن مخطط لبرامج
\textenglish{SDL}
التي نكتبها. في الواقع، البرنامج الحقيقي يكون ممتلأ كثيرا إذ يحتوي عدّة استدعاءات لدوال، تقوم بدورها بمزيد من الاستدعاءات.\\
الأمر المهمّ في النهاية، هو أنّ
\textenglish{SDL}
يجب أن تُحمّل في البداية و تُغلق عندما لا تصبح بحاجة إليها.

\subsection{معالجة الأخطاء}

الدالة 
\InlineCode{SDL\_Init}
تقوم بإرجاع قيمة :

\begin{itemize}
	\item -1 : في حال وجود خطأ.
	\item 0 : في حالة عدم وجود أي خطأ.
\end{itemize}

لست مجبراً، لكن يمكنك اختبار القيمة المُرجعة. قد تكون طريقة جيّدة لمعالجة الأخطاء في برنامجك، و هذا ما سيساعدك على حلّها.

\begin{question}
لكن كيف أقوم بإظهار الخطأ الحادث ؟
\end{question}

سؤال وجيه ! ليس لدينا كونسول الآن، كيف نخزّن و نعرض رسائل الخطأ ؟

هناك حلّان :
\begin{itemize}
	\item يمكننا التعديل على خاصيات المشروع، لكي نسمح له باستعمال الكونسول أيضاً. سنتمكّن في هذه الحالة من استخدام الدالة 
	\InlineCode{printf})،
	\item أو نكتب الأخطاء في ملف. تستخدم الدالة
	\InlineCode{fprintf}.
\end{itemize}

لقد اخترت أن نكتب في ملف. و بهذا فإن العمل على ملف يحتاج إلى فتح هذا الأخير بـ\InlineCode{fopen}
و غلقه بـ\InlineCode{fclose}،
و الأمر أقلّ سهولة من استعمال الـ\InlineCode{printf}.\\
لحسن الحظ، هناك طريقة أسهل و هي استعمال مخرج الأخطاء القياسي.

يوجد متغير 
\InlineCode{stderr}
معرّف في 
\InlineCode{stdio.h}
يقوم بالتأشير نحو المنطقة التي يُمكن أن يُكتب فيها الخطأ. غالبا في الويندوز، هذه المنطقة عبارة عن ملف يحمل الاسم 
\InlineCode{stderr.txt}.
بينما في اللينكس فإن الأخطاء غالباً ما يتم إظهارها على الكونسول. هذا المتغير يتم إنشاؤه تلقائيّا في بداية البرنامج و يتم حذفه في نهايته، أي أنك لست مجبراً على استعمال 
\InlineCode{fopen}
و
\InlineCode{fclose}.\\
يمكنك استعمال الدالة
\InlineCode{fprintf}
على 
\InlineCode{stderr}
بدون استعمال  
\InlineCode{fopen}
و 
\InlineCode{fclose} :

\begin{Csource}
#include <stdlib.h>
#include <stdio.h>
#include <SDL/SDL.h>
int main(int argc, char *argv[])
{
	if (SDL_Init(SDL_INIT_VIDEO) == -1) // Starting the SDL, if there's an error :
	{
		fprintf(stderr, "Error while initializing SDL : %s\n",
		SDL_GetError()); // Writing the error
		exit(EXIT_FAILURE); // We exit the program
	}
	SDL_Quit();
	return EXIT_SUCCESS;
}
\end{Csource}

ما الجديد في هذه الشفرة المصدرية ؟

\begin{itemize}
	\item لقد كتبنا الخطأ الذي وجدناه في
	\InlineCode{stderr}.
	الرمز 
	\InlineCode{\%s}
	يسمح للـ\textenglish{SDL} 
	بالإشارة إلى تفاصيل الخطأ : الدالة 
	\InlineCode{SDL\_GetError}
	في الحقيقة تقوم بإرجاع آخر خطأ
	\textenglish{SDL}.
	\item نخرج باستعمال الـ\InlineCode{exit()}.
	لحدّ الآن لا يوجد شيء جديد مقارنة بما جرت العادة، ستلاحظ أنني استعمل الثابت 
	\InlineCode{EXIT\_FAILURE}
	كقيمة يقوم البرنامج الرئيسي بإرجاعها، بينما استعملت في النهاية الثابت 
	\InlineCode{EXIT\_SUCCESS}
	في مكان الـ0.\\
	ما الذي قمت به ؟ لقد قمت بتحسين الطريقة التي تعودنا أن نكتب بها الشفرة. لقد استخدمت اسم الثابت الّذي يعني "خطأ" و الذي هو نفسه بالنسبة لجميع أنظمة التشغيل. بينما الأعداد تختلف من نظام إلى آخر.\\
	لهذا فإن الملف
	\InlineCode{stdlib.h}
	تسمح باستعمال ثابتتين (معرّفي 
	\InlineCode{\#define}) :
	
	\begin{itemize}
		\item \InlineCode{EXIT\_FAILURE} :
		قيمة يتم إرجاعها في حالة وجود خطأ ما في البرنامج.
		\item \InlineCode{EXIT\_SUCCESS} :
		قيمة يتم إرجاعها في حالة عدم وجود أي خطأ.
	\end{itemize}
	
	باستعمال أسماء الثوابت بدلاً من قيمها، ستضمن بأنك قد بعثت القيمة الصحيحة.\\
	لماذا ؟ لأن الملف
	\InlineCode{stdlib.h}
	يتغيّر حسب نظام التشغيل الّذي أنت عليه، لذا فقيم الثوابت ستتأقلم مع النظام من دون أنّ نحتاج إلى تغيير شيء !  و هذا ما يجعل لغة الـ\textenglish{C}
	متوافقة مع كلّ أنظمة التشغيل (بافتراض أنّك تبرمج بالطريقة الصحيحة باستخدام الأدوات المتوفّرة، كما فعلنا هنا).

\begin{information}
	استعمال أسماء الثوابت لا يعود علينا بكثير من النفع الآن، لكن من الأحسن استعمالها. سنقوم بذلك انطلاقاً من الآن.
\end{information}
\end{itemize}

  \chapter{إظهار صور}

لقد تعلّمنا كيف نقوم بتحميل الـ\textenglish{SDL}،
فتح نافذة و التعامل مع المساحات. إنها بالفعل من المبادئ التي تجب معرفتها عن هذه المكتبة. لكن لحدّ الآن لا يمكننا سوى إنشاء مساحات موحّدة اللون، و هذا الأمر بدائي قليلاً.

في هذا الفصل، سنتعلّم كيف نقوم بتحميل صور على مساحات، مهما كانت صيغتها
\textenglish{BMP}،
\textenglish{PNG}
أو حتى 
\textenglish{GIF}
أو 
\textenglish{JPG}.
التحكم في الصور أمر مهم للغاية لأنه بتجميع الصور (نسميها أيضاً 
"\textenglish{sprites}")
نضع اللبنات الأولى في بناء لعبة فيديو.

  \chapter{معالجة الأحداث}

معالجة الأحداث هو من أهم الأساسيات في الـ\textenglish{SDL}.\\
و ربّما قد يكون الشطر الأكثر شغفاً لاكتشافه. لأنه انطلاقا من هنا ستبدأ فعلاً في التحكّم في تطبيقك.

كلّ من مرفقات الحاسوب (فأرة، لوحة مفاتيح، \dots) قادرة على إنتاج حدث. سنتعلّم كيف نستقبل كل حدث و نتعامل معه. تطبيقك سيصبح أخيراً تفاعليّا !

فعلياً، ما هو الحدث ؟ الحدث هو عبارة عن إشارة
(\textenglish{signal})
يتم إرسالها عن طريق إحدى مرفقات الحاسوب 
(\textenglish{peripherals})
(أو عن طريق نظام التشغيل بذاته) إلى التطبيق. هذه أمثلة عن بعض الأحداث المألوفة :

\begin{itemize}
	\item حينما يضغط المُستعمل على زر من لوحة المفاتيح.
	\item و أيضاً حينما ينقر بالفأرة.
	\item حينما يحرّك الفأرة.
	\item حينما يقوم بتصغير النافذة.
	\item حينما يطلب إغلاق النافذة.
	\item إلى آخره.
\end{itemize}

الهدف من هذا الدرس هو تعلّم كيفية معالجة الأحداث. يمكنك أخيراً القول للحاسوب : "إذا نقر المستعمل في هذا المكان، قم بفعل كذا، و إن لم يفعل، قم بكذا. إذا حرّك الفأرة، قم بكذا. إذا ضغط على الزر
\InlineCode{Q}،
أوقف البرنامج. إلخ".

  \chapter{عمل تطبيقي : \textenglish{Mario Sokoban}}

المكتبة
\textenglish{SDL}
تقدّم، مثلما رأينا، عددا كبيراً من الدوال الجاهزة للاستعمال. يمكن ألا نستطيع التعوّد عليها في البداية لقلّة التطبيق.

هذا العمل التطبيقي الأول في هذا الجزء من الكتاب سيعطيك فرصة التطبيق و اختبار أشياء لم تسنح لك فرصة تجريبها.
أعتقد أنه بإمكانك التخمين، فهذه المرة لن يكون التطبيق عبارة عن كونسول و إنما سيتحتوي على واجهة رسومية !

ماذا سيكون موضوع هذا العمل التطبيقي ؟ لعبة السوكوبان !\\
قد لا يعني لك هذا العنوان شيئاً، لكن هذه هي لعبة ذكاء تقليديّة. إنّها تنصّ على دفع صناديق لوضعها في أماكن محددة في متاهة.

\section{مواصفات \textenglish{Sokoban}}

\subsection{بخصوص \textenglish{Sokoban}}

الكلمة
"\textenglish{Sokoban}"
هي كلمة يابانية تعني "صاحب محلّ".\\
إنّها عبارة عن لعبة ذكاء تم اختراعها في الثمانينات بواسطة
\textenglish{Hiroyuki Imabayashi}.
و قد مثّلت برمجة هذه اللعبة تحدّيا كبيراً في ذلك الزمن.

\subsubsection{الهدف من اللعبة}

المبدأ بسيط : تقوم بتحريك شخصية في متاهة. يجدر بالشخصية أن تقوم بدفع صناديق إلى مواقع محددة. لا يمكن للاعب أن يدفع صندوقين في آن واحد.

حتى و إن كان المبدأ مفهوماً و بسيطاً، فهذا لا يعني أن اللعبة في حدّ ذاتها سهلة ! إذ أنه يجب عليك أحيانا تكسير رأسك بالتفكير لحلّ اللغز.

الصورة الموالية تُريك كيف تبدو اللعبة التي سنقوم ببرمجتها :

\Picture{Chapter_III-5_Mario-Sokoban}

\subsubsection{لماذا اخترتُ هذه اللعبة بالذات ؟}

لأنها لعبة شعبية، جيدة لأن تكون موضوعاً برمجياً و يمكننا إنشاؤها بواسطة ما تعلّمناه من الدروس السابقة.\\
يجب هنا أن نكون منظّمين. إذ أن الصعوبة لا تكمُن في برمجة اللعبة في حدّ ذاتها لكن في ما إن نظّمنا العمل. و لهذا فسنقوم بتقسيم البرنامج إلى عدّة ملفات
\InlineCode{.c}
بطريقة ذكيّة و نحاول إنشاء الدوال المناسبة.

من أجل هذا الأمر، قررت تغيير الطريقة بالنسبة لهذا العمل  التطبيقي : لن أقدّم لك توجيهات و أقدّم التصحيح في النهاية. بالعكس، سأريك كيف نقوم ببناء المشروع كلّه من الألف إلى الياء.

\begin{question}
ماذا لو كنتُ أريد التدرّب لوحدي ؟
\end{question}

حسناً إذا فلتنطلق لوحدك، هذا أمر جيد !\\
ستحتاج ربّما وقتاً أكثر : لقد استغرقت شخصيا يوماً كاملاً لبرمجة اللعبة، هذا ليس بالوقت الكثير ربّما لأنه جرت العادة أن أقوم بالبرمجة و و أن أتحاشى الوقوع في بعض الأفخاخ المتداولة (لكنّ هذا لم يمنعني من إعادة التفكير عدّة مرّات).

اِعلم بأنه توجد الكثير من الطرق التي يمكن بها برمجة هذه اللعبة. سأعطيك طريقتي في برمجتها : ليست أحسن طريقة و لكنها بالتأكيد ليست أسوء واحدة.\\
سننتهي من هذا التطبيق بقائمة من الإقتراحات لتحسين اللعبة، كما أنني سأعطيك الرابط لتحميل اللعبة و الشفرة المصدرية الكاملة.

أنصحك مجدداً أن تحاول برمجة اللعبة لوحدك، حتى لو استغرقت 3 أو 4 أيام. إفعل أحسن ما لديك. من المهم جدّا أن تقوم بالتطبيق.

\subsection{المواصفات}

المواصفات هي عبارة عن وثيقة نكتب فيها كل ما يجب على البرنامج أن يستطيع فعله.

هذا ما أقترحه :

\begin{itemize}
	\item يجب أن يتمكن اللاعب من التحرّك في المتاهة و دفع الصناديق.
	\item لا يمكنه أن يدفع صندوقين معاً.
	\item تُربح الجولة إذا تواجدت كلّ الصناديق في الأماكن المخصصة لها.
	\item سيتم حفظ كلّ مستويات اللعبة في ملف، (ليكن مثلا 
	\InlineCode{levels.lvl}).
	\item يجب أن يتم دمج مـُنشئ المستويات 
	(\textenglish{Levels editor})
	في البرنامج ليتمكن أي شخص كان من صنع مستويات خاصة به (هذا ليس أمراً ضرورياً لكنه يعتبر إضافة مميزة !).
\end{itemize}

هذا كافٍ لنعمل كثيراً.

يجب أن تعرف أنه هناك أشياء لا يجيد البرنامج القيام بها، و يجب ذِكرُ هذا الأمر أيضاً.

\begin{itemize}
	\item برنامجنا قادر على التحكّم في مرحلة واحدة في المرّة الواحدة. إن أردت أن تكون اللعبة عبارة عن تتالي جولات، فما عليك سوى برمجة ذلك بنفسك في نهاية هذا العمل التطبيقي.
	\item البرنامج لا يقوم بحساب الوقت المٌستغرق في كلّ جولة (نحن لا نجيد فعل ذلك بعد) و لا يمكنه حساب النقاط.
\end{itemize}

على أي حال، فكلّ الأشياء التي نريد القيام بها (خاصة مـُنشِئ المراحل) تأخذ منا وقتاً لابأس به.

\begin{information}
سأعطيك في نهاية العمل التطبيقي، جملة التحسينات التي تُمكن إضافتها إلى اللعبة. و هذه ليست كلمات في الهواء، لأنّها أفكار طبّقتها أنا شخصيّا في نسخة كاملة من اللعبة  سأقترح عليك تنزيلها.\\
بالمقابل، لن أعطيك الشفرة المصدرية الخاصة بالنسخة الكاملة لأنني أريدك أن تعمل بنفسك و تتدرّب (لن أعطيك كلّ شيء على طبق من فضّة !).
\end{information}

\subsection{الحصول على  الصور اللازمة لللعبة}

في معظم الألعاب ثنائية الأبعاد، أيّا كان نوعها، نسمّي الصور التي تشكّل اللعبة 
\textenglish{\textit{Sprites}}.\\
 في حالتنا، قرّرت إنشاء
\textenglish{Sokoban}
 و وضع الشخصية 
\textenglish{Mario}
لتكون اللاعب الرئيسي فيها (من هنا جاء اسم اللعبة 
\textenglish{Mario Sokoban}).
بما أن 
\textenglish{Mario}
شخصية لها شعبية كبيرة في عالم الألعاب 
\textenglish{2D}،
لن نتعب في الحصول على الـ\textenglish{sprites}
الخاصّة بهذه الشخصيّة. سنحتاج أيضا إلى
\textenglish{sprites}
خاصة بالجدران، الصناديق، الأماكن المستهدفة، إلخ.

إذا بحثت في
\textenglish{Google}
عن
"\textenglish{sprites}"
فستحصل على عدّة نتائج. توجد العديد من المواقع التي توفّر
\textenglish{sprites}
خاصّة بألعاب
\textenglish{2D}
قد تكون لعبتها في السابق.

و هذه هي الّتي سنحتاج إليها :

\begin{Table}{2}
\textenglish{Sprite} & الشرح\\
\includegraphics{Chapter_III-5_Wall}&
جدار\\
\includegraphics{Chapter_III-5_Box}&
صندوق\\
\includegraphics{Chapter_III-5_Box2}&
 صندوق متموضع فوق منطقة مستهدفة\\
\includegraphics{Chapter_III-5_Mario-down}&
بطل اللعبة
(\textenglish{Mario})
باتجاه الأسفل\\
\includegraphics{Chapter_III-5_Mario-right}&
بطل اللعبة باتجاه اليمين\\
\includegraphics{Chapter_III-5_Mario-left}&
بطل اللعبة باتجاه اليسار\\
\includegraphics{Chapter_III-5_Mario-up}&
بطل اللعبة باتجاه الأعلى\\
\end{Table}

الأسهل هو أن تقوم بتحميل الحزمة التي أعددتها لك.

\url{https://openclassrooms.com/uploads/fr/ftp/mateo21/sprites_mario_sokoban.zip}

\begin{information}
كان من الممكن أن أستعمل 
\textenglish{sprite}
واحداً خاصاً باللاعب. كان بإمكاني جعله موجّهاً إلى الأسفل فقط، لكن إضافة امكانية توجيهه في الإتجاهات الأربعة تضيف القليل من الواقعية. و هذا يشكّل تحدّيا آخر لنا !
\end{information}

قمت أيضاً بإنشاء صورة أخرى لتكون عبارة عن الواجهة الأساسية للعبة حين تبدأ، لقد أرفقت لك الصورة بالحزمة الّتي يفترض بك تنزيلها. لاحظ الصورة التالية :

\Picture{Chapter_III-5_Home}

ستلاحظ بأن الصور تأخذ صيغا مختلفة. يوجد منها ماهو
\textenglish{GIF}،
ماهو
\textenglish{PNG}
و حتى ماهو
\textenglish{JPEG}.
و لهذا فنحن بحاجة إلى استعمال المكتبة
\textenglish{SDL\_Image}.\\
فكّر في جعل مشروعك يعمل مع الـ\textenglish{SDL}
و الـ\textenglish{SDL\_Image}.
إذا نسيت كيف تفعل ذلك، فراجع الفصول السابقة. إذا لم تقم بتخصيص المشروع بشكل صحيح، سيشير المُترجم بأن الدوال التي تستعملها (مثل
\InlineCode{IMG\_Load})
غير موجودة !

  \chapter{تحكّم في الوقت !}

لهذا الفصل أهمية كبيرة : سيعلّمك كيف تتحكم في الوقت بالـ\textenglish{SDL}.
 إنه لمن النادر أن نقوم بإنشاء برنامج
\textenglish{SDL}
لا تحتاج إلى دوال خاصة بالتحكم في الوقت، بالرغم من أن لعبة
\textenglish{Mario Sokoban}
كانت حالة خاصة. رغم ذلك، في معظم الألعاب، إدارة الوقت هي شيء أساسيّ.

مثلاً، كيف لك أن تٌنشئ لعبة 
\textenglish{Tetris}
أو
\textenglish{Snake} ؟
يجب فعلاً على الكُتل أن تتحرّك كل
\textenglish{X}
ثانية، و هذا ما لا تجيد فعله. على الأقل، قبل أن تقرأ هذا الفصل.

  \chapter{كتابة نصوص باستعمال \textenglish{SDL\_ttf}}

يمكنني التكهّن بأن معظم القرّاء قد طرح هذا السؤال من قبل: "ولكن، ألا توجد أي دالة لكي تكتب نصًا على نافذة
\textenglish{SDL}؟"
حان الوقت لأجيبك: الجواب هو لا.

رغم ذلك، توجد طرق لفعل هذا. يمكننا فقط\dots وضع صور للحروف بجانب بعضها البعض على الشاشة. هذا الأمر يعمل لكنّه ليس عمليا.

لحسن الحظ، يوجد ماهو أبسط: يمكننا استعمال المكتبة
\textenglish{SDL\_ttf}.
إنها مكتبة تتم إضافتها إلى
\textenglish{SDL}
تمامًا مثل
\textenglish{SDL\_image}.
دورها هو إنشاء مساحة
\InlineCode{SDL\_Surface}
انطلاقا من النص الذي نبعثه لها.

\section{تسطيب \textenglish{SDL\_ttf}}

يجب أن تعرف أنه، مثل
\textenglish{SDL\_image}، \textenglish{SDL\_ttf}
هي مكتبة تحتاج إلى أن تكون المكتبة
\textenglish{SDL}
مثبّتة من قبل. حسنًا: إذا كنت إلى حدّ الآن لم تتمكّن من تسطيب المكتبة
\textenglish{SDL}
فهذا أمر شنيع ولهذا فسأعتبر أنك قمت بذلك!

تماما مثل 
\textenglish{SDL\_image}،
فإن المكتبة 
\textenglish{SDL\_ttf}
هي واحدة من المكتبات المُرتبطة بـ\textenglish{SDL}
الأكثر شعبية (أي أنه يتم تنزيلها بكثرة). كما ستُلاحظ، هذه المكتبة مُبرمجة بشكل جيد. ما إن تجيد استعمالها لن يمكنك أن تتوقّف عن ذلك!

\subsection{كيف تعمل \textenglish{SDL\_ttf}؟}

\textenglish{SDL\_ttf}
لا تقوم بإظهار صور 
\textenglish{bitmap}
لتولّد نصا في مساحات. في الحقيقة، هي طريقة ثقيلة لفعلها ولن يتاح لنا استعمال سوى خط واحد. \\
في الواقع، تستدعي المكتبة
\textenglish{SDL\_ttf}
مكتبةَ أخرى: 
\textenglish{FreeType}.
هي مكتبة قادرة على قراءة ملفات خطوط بصيغة
\InlineCode{.ttf}
لتُخرج منها صورة. تقوم
\textenglish{SDL\_ttf}
باسترجاع هذه الصورة وتحوّلها لـ\textenglish{SDL}
و ذلك بإنشاء مساحة
\InlineCode{SDL\_Surface}.

و بهذا فإن
\textenglish{SDL\_ttf}
تحتاج المكتبة
\textenglish{FreeType}
لكي تشتغل، وإلا فلن تكون قادرة على قراءة ملفات الخطوط
\InlineCode{.ttf}.

إذا كنت تعمل بـ\textbf{\textenglish{Windows}}
و تستعمل، مثلما أفعل، النسخة المُترجمَة للمكتبة، لن تحتاج إلى تنزيل أي شيء لأن
\textenglish{FreeType}
مضمّنة من قبل في المكتبة الحيّة
\InlineCode{SDL\_ttf.dll}
و لهذا فليس عليك القيام بأي شيء.

إذا كنت تعمل بـ\textbf{\textenglish{GNU/Linux}}
أو
\textbf{\textenglish{Mac OS X}}
فمن اللازم أن تعيد ترجمة المكتبة، فتلزمك
\textenglish{FreeType}
لتتم الترجمة. اذهب إذن إلى صفحة تنزيل
\textenglish{FreeType}:

\url{http://www.freetype.org/download.html#stable}

لتنزيل الملفات الخاصة بالمطورين.

\subsection{تثبيت \textenglish{SDL\_ttf}}

اذهب إلى  صفحة تنزيل 
\textenglish{SDL\_ttf}:

\url{http://www.libsdl.org/projects/SDL_ttf/}

هنا، اختر الملف اللازم من القسم
"\textit{\textenglish{Binary}}".

\begin{information}
في
\textenglish{Windows}،
 لاحظ أنه لا يوجد سوى ملفان بصيغة 
\InlineCode{.zip}
يحملان في نهاية اسميهما اللاحقتين
\InlineCode{win32}
و
\InlineCode{VC6}.
الأولى 
(\InlineCode{win32})
تحتوي الـ\textenglish{DLL}
التي تحتاج إلى تقديمها مع الملف التنفيذي. يجب عليك أيضًا وضع هذه الـ\textenglish{DLL}
في مجلّد المشروع لتستطيع تجريب البرنامج، طبعا.

الثانية 
(\InlineCode{VC6})
تحتوي الملفات 
\InlineCode{.h}
و الملفات
\InlineCode{.lib}
التي تحتاجها للبرمجة. يمكننا أن نفكّر من خلال الاسم أن هذه الملفّات تخص
\textenglish{Visual C++}
فقط، لكن في الحقيقة، وبشكل خاص، الملف
\InlineCode{.lib}
يعمل أيضًا مع
\textenglish{mingw32}،
سيشتغل إذن في
\textenglish{Code::Blocks}.
\end{information}

الملف
\InlineCode{.zip}
يحتوي كالعادة مجلد
\InlineCode{include}
و مجلد
\InlineCode{lib}.
قم بوضع محتوى المجلد
\InlineCode{include}
في المسار
\InlineCode{mingw32/include/SDL}،
و محتوى المجلد
\InlineCode{lib}
في المسار 
\InlineCode{mingw32/lib}.

\begin{warning}
 يجدر بك نسخ الملف
\InlineCode{SDL\_ttf.h}
في المجلد
\InlineCode{mingw32/include/SDL}
و ليس في المجلد 
\InlineCode{mingw32/include}
فقط. احذر الخطأ!
\end{warning}

\subsection{تخصيص مشروع من أجل \textenglish{SDL\_ttf}}

بقيت لنا مرحلة واحدة أخيرة: تخصيص المشروع لكي يكون قادرًا على استعمال
\textenglish{SDL\_ttf}
بشكل جيد. يجب أن يتم التعديل على خصائص محرّر الروابط لكي يُترجم البرنامج بشكل جيد وذلك باستعمال
\textenglish{SDL\_ttf}.

لقد تعلّمت من قبل هذه العملية بالنسبة لـ\textenglish{SDL\_image}،
و لهذا سأسرع قليلًا. \\
بما أنني أعمل في \textenglish{Code::Blocks}
سأعطيك العملية الخاصة بهذه البيئة التطويرية. بالنسبة لباقي البيئات، فالطريقة لا تختلف كثيرًا عن هذه:

\begin{itemize}
	\item توجّه نحو القائمة
	\InlineCode{Project} / \InlineCode{Build Options}.
	\item في القسم
	\InlineCode{Linker}
	أنقر على الزر الصغير
	\InlineCode{Add}.
	\item أشر إلى المسار الذي يوجد به الملف
	\InlineCode{SDL\_ttf.lib}
	(بالنسبة لي هو في\\
	\InlineCode{C:\textbackslash Program Files\textbackslash CodeBlocks\textbackslash mingw32\textbackslash lib}).
	\item ستظهر لك هذه الرسالة:
	"\textenglish{Keep this as a relative path ?}"
	لا يهمّ ما تختاره لأن الأمر سيشتغل في كلتا الحالتين. أنصحك أن تجيب بالسلب لأن المشروع لن يشتغل لو وضعته في مسار آخر غير المتواجد به لو أنك أجبت بالإيجاب.
	\item وافق على التغييرات بالنقر على 
	\InlineCode{OK}.
\end{itemize}

\begin{question}
ألا نحتاج إلى ربط المكتبة
\textenglish{FreeType}
أيضًا؟
\end{question}

كلا، مثلما قلتُ فـ\textenglish{FreeType}
مضمّنة في الـ\textenglish{DLL}
الخاصة بـ\textenglish{SDL\_ttf}.
لهذا فلن يكون عليك الاهتمام بها، لأن
\textenglish{SDL\_ttf}
تفعل ذلك الآن.
\subsection{الملفات التوثيقية}

و الآن بما أنك أصبحت مبرمجًا محنّكًا تقريبًا، يجدر بك أن تطرح التساؤل التالي: "لكن أين هو التوثيق؟" إن لم تطرح هذا السؤال فهذا يعني أنّك لازلت لم تصبح بعد مبرمجًا محنّكًا.

يوجد بالطبع دروس تفصّل في كيفية عمل المكتبات، مثل هذا الكتاب، ولكن:

\begin{itemize}
	\item لن أستطيع أن أضع لك فصلًا حول كل المكتبات الموجودة (حتى لو أمضيت حياتي كلّها في ذلك، لن يكفيني الوقت!). ولهذا يجب عاجلًا أم آجلا قراءة التوثيق ويجدر بك أن تتعوّد على ذلك من الآن!
	\item من جهة أخرى، في غالب الأحيان تكون المكتبة معقّدة نوعًا ما وتحتوي كثيرًا من الدوال. لن أتمكن من تقديم كلّ هذه الدوال في هذا الفصل لأنه سيكون بذلك طويلًا جدًا!
\end{itemize}

من الواضح جدًا أن التوثيق يكون كاملا ويلمّ بكل خفايا المكتبة، ولهذا أفضّل أن أعطيك من الآن رابط صفحة التوثيق الخاصة بـ\textenglish{SDL\_ttf}:

\url{http://sdl.beuc.net/sdl.wiki/SDL_ttf}

التوثيق متوفّر بصيغ مختلفة: 
\textenglish{HTML}
على الشبكة، 
\textenglish{HTML}
مضغوطة،
\textenglish{PDF}،
إلخ. خذ النسخة التي تناسبك.

ستجد بأن
\textenglish{SDL\_ttf}
مكتبة بسيطة جدًا: يوجد بها قليل من الدوال (حوالي 40 - 50، نعم إنها قليلة!). يجدر بهذا أن تكون إشارة (للمبرمجين المحنّكين من ضمن القرّاء) إلى أن هذه المكتبة سهلة وستستطيع التعامل معها سريعًا.

هيا، حان الوقت لنتعلّم كيف نستخدم
\textenglish{SDL\_ttf}
الآن!

\section{تحميل \textenglish{SDL\_ttf}}

\subsection{التضمين}

قبل كلّ شيء، يجب تضمين الملف الرأسي التالي قبل كلّ استعمال لهذه المكتبة:

\begin{Csource}
#include <SDL/SDL_ttf.h>
\end{Csource}

إذا صادفت أخطاء ترجمة الآن، تأكد بأنك وضعت الملف
\InlineCode{SDL\_ttf.h}
في المجلّد
\InlineCode{mingw32/include/SDL}
و ليس في 
\InlineCode{mingw32/include}
فقط.

\subsection{تشغيل \textenglish{SDL\_ttf}}

تماما مثل
\textenglish{SDL}،
تحتاج
\textenglish{SDL\_ttf}
أن تُشغّل في بداية الشفرة وتُوقّف في نهايتها.\\
توجد دالتان تشبهان كثيرًا الدالتين الخاصتين بـ\textenglish{SDL}:

\begin{itemize}
	\item \InlineCode{TTF\_Init}:
	تقوم ببدء تشغيل
	\textenglish{SDL\_ttf}.
	\item \InlineCode{TTF\_Quit}:
	توقّف
	\textenglish{SDL\_ttf}.
\end{itemize}

\begin{information}
ليس واجبًا أن يتم بدء تشغيل
\textenglish{SDL}
قبل
\textenglish{SDL\_ttf}.
\end{information}

 لكي تقوم ببدء تشغيل
\textenglish{SDL\_ttf}
(نقول أيضًا تهيئة)، يجب أن نستدعي الدالة
\InlineCode{TTF\_Init}.
هذه الأخيرة لا تحتاج إلى أن تستقبل أي معامل وهي تقوم بإرجاع القيمة
$-1$
إن حدث أي خطأ.

يمكنك البدء في تشغيل
\textenglish{SDL\_ttf}
ببساطة كالتالي:

\begin{Csource}
TTF_Init();
\end{Csource}

إذا أردت أن تتأكد ما إن كان قد حدث خطأ أم لا، جرّب الشفرة التالية:

\begin{Csource}
if(TTF_Init() == -1)
{
	fprintf(stderr, "Error initializing TTF_Init : %s\n", TTF_GetError());
	exit(EXIT_FAILURE);
}
\end{Csource}
إذا كان هناك خطأ في تشغيل
\textenglish{SDL\_ttf}،
سيتم إنشاء ملف
\InlineCode{stderr.txt}
(في
\textenglish{Windows}
على الأقل) يحتوي على رسالة تشرح الخطأ.\\
للذين يطرحون السؤال: الدالة
\InlineCode{TTF\_GetError}
تقوم بإرجاع آخر رسالة خطأ لـ\textenglish{SDL\_ttf}،
و لهذا استعملتها في
\InlineCode{fprintf}.

\subsection{إيقاف \textenglish{SDL\_ttf}}

 لنوقّف المكتبة، نستدعي الدالة
\InlineCode{TTF\_Quit}.
هي أيضًا لا تحتاج أي معامل. يمكنك استدعاؤها قبل أو بعد
\InlineCode{SDL\_Quit}
هذا لا يهم:

\begin{Csource}
TTF_Quit();
\end{Csource}

\subsection{تحميل خط}

حسنًا كان كلّ شيء جيدًا وغير معقّدٍ، لكننا لم نستمتع بعد. لننتقل إلى الأهمّ إذا أردت ذلك: والآن بما أنه تم تحميل 
\textenglish{SDL\_ttf}،
يجب علينا أن نقوم بتحميل خط ما. ما إن يتم هذا الشيء، يمكننا أخيرًا كتابة النص!

هنا أيضًا، توجد دالتان:
\begin{itemize}
	\item \InlineCode{TTF\_OpenFont}:
	تفتح ملف خط
	(\InlineCode{.ttf}).
	\item \InlineCode{TTF\_CloseFont}: 
	تغلق الملف المفتوح.
\end{itemize}

يجدر بالدالة
\InlineCode{TTF\_OpenFont}
أن تخزّن النتيجة في متغير من نوع
\InlineCode{TTF\_Font}.
لهذا يجب عليك إنشاء مؤشّر من نوع
\InlineCode{TTF\_Font}
كالتالي:

\begin{Csource}
TTF_Font *font = NULL;
\end{Csource}

يحتوي المؤشّر
\InlineCode{font}
إذا على معلومات خاصة بالخط المفتوح.

تأخذ الدالة 
\InlineCode{TTF\_OpenFont}
معاملين:

\begin{itemize}
	\item اسم ملف الخط (بصيغة
	\InlineCode{.ttf})
	الذي نريد فتحه. الأمثل هو وضع ملف الخط في مجلّد المشروع. مثال عن ملف:
	\InlineCode{arial.ttf}
	(من أجل الخط
	\textenglish{Arial}).
	\item حجم الخط الذي نريد استعماله. يمكنك مثلا استعمال حجم 22.
	
	إنها نفس الحجوم التي تستعملها في برامج معالجة النصوص مثل
	\textenglish{Word}.
\end{itemize}

لم يتبقّ لنا سوى إيجاد الخطوط ذات الصيغة
\InlineCode{.ttf}.
أنت تملك أصلًا العديد منها على حاسوبك، لكن يمكنك تنزيلها من الأنترنت كما سنرى الآن.

\subsubsection{على حاسوبك}

لديك أصلا خطوط على حاسوبك!\\
إن كنت تعمل بـ\textenglish{Windows}،
 ستجد الكثير من هذه الملفات في المجلّد
\InlineCode{C:\textbackslash Windows\textbackslash Fonts}.\\
ليس عليك سوى نسخ الملف الخاص بالخط الذي يعجبك ولصقه في مجلّد المشروع. 

إذا كان اسم الملف يحتوي على حروف "غريبة" كالفراغات، الحروف ذات العلامات الصوتية 
(\textenglish{accents})
أو حتى الحروف الكبيرة، أنصحك بإعادة تسمية هذا الملف. ولكي نكون متيقنين من عدم وجود أيّ مشكل، لا تستعمل سوى الأحرف الصغيرة وتجنّب الفراغات.

\begin{itemize}
	\item مثال عن اسم خاطئ: 
	\InlineCode{TIMES NEW ROMAN.TTF}.
	\item مثال عن اسم صحيح:
	\InlineCode{times.ttf}.
\end{itemize}

\subsubsection{على الأنترنت}

الخيار الآخر: احصل على خطّ من الأنترنت. ستجد الكثير من المواقع التي تقترح خطوطا مجانية وأصلية للتنزيل.

أنصحك شخصيا بزيارة الموقع
\href{http://www.dafont.com/}{\textenglish{dafont.com}}
لأنّه مصنّف بشكل جيّد ومحتواه منظّم ومنوّع.

لاحظ الصور التالية التي ستعطيك فكرة عن الخطوط التي ستجدها هناك بسهولة:

\begin{figure}[H]
	\centering
	\includegraphics[height=0.07\textheight]{Chapter_III-7_Font1}
\end{figure}
\begin{figure}[H]
	\centering
	\includegraphics[height=0.07\textheight]{Chapter_III-7_Font2}
\end{figure}
\begin{figure}[H]
	\centering
	\includegraphics[height=0.07\textheight]{Chapter_III-7_Font3}
\end{figure}

\subsubsection{تحميل الخط}

 أقترح عليك استعمال الخط 
\textenglish{Angelina} (\url{http://www.dafont.com/angelina.font})
 لبقية الأمثلة.

فلنفتح الخط كالتالي:

\begin{Csource}
font = TTF_OpenFont("angelina.ttf", 65);
\end{Csource}

الخط المستعمل سيكون
\InlineCode{angelina.ttf}.
لقد قمت وضع هذا الخط في مجلّد المشروع كما قمت بإعادة تسميته لكي يكون كلّه بحروف صغيرة.\\
سيكون للخط الحجم 65. ستبدو الكتابة كبيرة لكنه خطّ خاص يستلزم ذلك لكي يظهر بشكل جيد.

الأمر المهم هو أن
\InlineCode{TTF\_OpenFont}
تخزّن النتيجة في المتغير
\InlineCode{font}،
ستعيد استعمال هذا المتغير الآن بكتابة نص. فهي تسمح بالإشارة إلى الخط الذي نريد أن نستعمله لكي نكتب النص.

\begin{information}
لا تحتاج إلى فتح الخط في كلّ مرة تريد فيها الكتابة به: افتحه مرّة واحدة في بداية البرنامج وأغلقه في نهايته.
\end{information}

\subsubsection{غلق الخط}

يجب التفكير في غلق كل خط قمنا بفتحه قبل استدعاء
\InlineCode{TTF\_Quit}.
في حالتي، هذا ما تكون عليه الشفرة:

\begin{Csource}
TTF_CloseFont(font); // Must be before TTF_Quit();
TTF_Quit();
\end{Csource}

هكذا يكون العمل!

\section{الطرق المختلفة للكتابة}

و الآن، بما أنه تم تحميل
\textenglish{SDL\_ttf}
و أن لدينا متغيرا 
\InlineCode{font}
محمّلا هو الآخر، لن يمنعنا أي شيء وأي شخص من كتابة نص في نافذة 
\textenglish{SDL}!

جيد: كتابة النص هو أمر جيد، لكن بواسطة أي دالة؟ من خلال التوثيق يوجد ما لا يقلّ عن 12 دالة لفعل ذلك!

في الواقع، توجد 3 طرق مختلفة لـ\textenglish{SDL\_ttf}
لكي ترسم نصًا.


\begin{itemize}
	\item \textbf{\textenglish{Solid}}
	(الصورة 1): هي التقنية الأكثر سرعة. ستتم كتابة النص بسرعة في
	\InlineCode{SDL\_Surface}.
	ستكون المساحة شفافة لكنها لن تستخدم إلا مستوًى واحدًا من الشفافية (لقد تعلّمنا ذلك في الفصول السابقة). هذا أمر عملي، لكن النص لن يكون جميلًا لأنه حوافه لن تكون منحوتة بشكل جيد وخاصة إن كان مكتوبا بحجم ضخم. استعمل هذه التقنية حينما يكون عليك تغيير النص كثيرًا، مثلا لإظهار الوقت المنقضي أو عدد الـ\textenglish{FPS}
	الخاص بلعبة.
	\item \textbf{\textenglish{Shaded}}
	(الصورة 2): هذه المرة، سيكون النص جميلًا. فالحروف ستكون محسّنة أكثر (هذا يعني أن محيط الحواف سيكون مُـلطّفا بشكل مُريح لعين الإنسان) وسيظهر النص أكثر نعومة. يوجد عيب في هذه التقنية: يجب أن تكون الخلفية ذات لون واحد موّحد. يستحيل جعل خلفية الـ\InlineCode{SDL\_Surface}
	شفافة بطريقة الـ\textenglish{Shaded}.
	\item \textbf{\textenglish{Blended}}
	(الصورة 3): هي التقنية الأكثر قوّة، لكنها بطيئة. في الواقع، هي تأخذ الوقت اللازم الذي تأخذه التقنية 
	\textenglish{Shaded}
	لإنشاء الـ\InlineCode{SDL\_Surface}.
	الاختلاف الوحيد بينها وبين الـ\textenglish{Shaded}،
	هي أنه يمكنك لصقُ النص على صورة وسيتم احترام الشفافية (على عكس
	\textenglish{Shaded}
	التي تفرض وجود خلفية موحّدة اللون). احذر: عملية اللصق بهذه الطريقة أبطأ من تلك الخاصة بالـ\textenglish{Shaded}.
\end{itemize}

\begin{figure}[H]
	\centering
	\includegraphics[width=0.3\textwidth]{Chapter_III-7_Solid}
\end{figure}
\begin{figure}[H]
	\centering
	\includegraphics[width=0.3\textwidth]{Chapter_III-7_Shaded}
\end{figure}
\begin{figure}[H]
	\centering
	\includegraphics[width=0.3\textwidth]{Chapter_III-7_Blended}
\end{figure}

ملخّص:

\begin{itemize}
	\item إذا كان لديك نص يتغير محتواه كثيرًا، كعداد عكسي، استعمل التقنية 
	\textenglish{Solid}.
	\item إذا كان النص لا يتغير كثيرًا وأنك تريد لصق النص على خلفية موحدة اللون، استعمل التقنية 
	\textenglish{Shaded}.
	\item إذا كان النص لا يتغير كثيرًا ولكنك تريد لصقه على خلفية غير موحدّة اللون (كصورة مثلًا) استعمل التقنية 
	\textenglish{Blended}.
\end{itemize}

هكذا إذا، يجدر بك أن تكون قد تعوّدت قليلًا على هذه الأساليب الخاصة بـ\textenglish{SDL\_ttf}
في الكتابة.

لقد قلتُ لك أنه توجد 12 دالة لذلك.\\
في الواقع، من أجل كلّ طريقة في الكتابة، توجد 4 دوال لذلك. كلّ دالة تكتب النص بالاستعانة بمجموعة محارف 
(\textenglish{Charset})
مختلفة. هذه الدوال هي:

\begin{itemize}
	\item \textenglish{Latin1}،
	\item \textenglish{UTF8}،
	\item \textenglish{Unicode}،
	\item \textenglish{Unicode Glyph}.
\end{itemize}

الأمثل أن تختار
\textenglish{Unicode}
لأنها مجموعة محارف تحوي أغلب الحروف والإشارات الموجودة على وجه الأرض. ولكن، استعمال
\textenglish{Unicode}
ليس سهلا دائمًا (محرف واحد يأخذ حجما أكبر من حجم
\InlineCode{char}
في الذاكرة)، فلن نرى كيفية استعمالها هنا.

إذا كان برنامجك مكتوبًا بالفرنسية فمجموعة
\textenglish{Latin1}
تكفي بإسهاب، يمكنك الاكتفاء بهذه الأخيرة.

الدوال الثلاثة التي تستعمل نظام التشفير
\textenglish{Latin1}
هي:

\begin{itemize}
	\item \InlineCode{TTF\_RenderText\_Solid}،
	\item \InlineCode{TF\_RenderText\_Shaded}،
	\item \InlineCode{TTF\_RenderText\_Blended}.
\end{itemize}

\subsection{مثال عن كتابة نص بطريقة الـ\textenglish{Blended}}

لكيّ نختار لونا بـ\textenglish{SDL\_ttf}،
لن نستعمل نفس النوع كما بـ\textenglish{SDL}
(إنشاء متغير من نوع
\InlineCode{Uint32}
بالاستعانة بالدالة
\InlineCode{SDL\_MapRGB}).\\
بالعكس، سنستعمل هيكلا جاهزا من طرف
\textenglish{SDL}
و هو: 
\InlineCode{SDL\_Color}.
هذا الهيكل يحتوي ثلاثة مركّبات: كمية الأحمر، الأخضر والأزرق.

إذا أردت إنشاء متغير
\InlineCode{blackColor}،
يجب عليك أن تكتب إذا:

\begin{Csource}
SDL_Color blackColor = {0, 0, 0};
\end{Csource}

\begin{warning}
احذر لكي لا تخلط بينها وبين الألوان التي تستعملها عادة 
\textenglish{SDL}!\\
\textenglish{SDL}
تستعمل متغيرات
\InlineCode{Uint32}
يتم إنشاؤها بمساعدة 
\InlineCode{SDL\_MapRGB}.\\
بينما
\textenglish{SDL\_ttf}
تستعمل متغيرات
\InlineCode{SDL\_Color}.
\end{warning}

سنقوم بكتابة نص بالأسود في
\InlineCode{SDL\_Surface}،
نسميها
\InlineCode{text}.

\begin{Csource}
text = TTF_RenderText_Blended(font, "Salut les Zér0s !", blackColor);
\end{Csource}

أنت ترى  المعاملات التي بعثتاها بالترتيب: الخط (من نوع
\InlineCode{TTF\_Font})،
النص الذي نريد كتابته وأخيرًا اللون (من نوع
\InlineCode{SDL\_Color}).\\
يتم تخزين النتيجة في مساحة. تحسب
\textenglish{SDL\_ttf}
تلقائيًا الحجم اللازم للمساحة بدلالة حجم النص وعدد الحروف التي تريد كتابتها.

كما هو الحال بالنسبة لأي مساحة، سيحتوي المؤشّر
\InlineCode{text}
المركّبات
\InlineCode{w}
و
\InlineCode{h}
التي تشير بالترتيب إلى عرض وارتفاع المساحة. إذن فهذه طريقة جيدة لمعرفة أبعاد النص ما إن تتم كتابة هذا الأخير على المساحة. لن يكون عليك سوى كتابة:

\begin{Csource}
text->w // Gives the width
text->h // Gives the height
\end{Csource}

\subsection{الشفرة المصدرية الكاملة لكتابة نص}

أنت تعرف الآن كلّ ما يجب أن تتم معرفته بخصوص
\textenglish{SDL\_ttf}،
فلنرى الشفرة المصدرية التي تلخّص كتابة نص بطريقة الـ\textenglish{Blended}:

\begin{Csource}
#include <stdlib.h>
#include <stdio.h>
#include <SDL/SDL.h>
#include <SDL/SDL_image.h>
#include <SDL/SDL_ttf.h>
int main(int argc, char *argv[])
{
	SDL_Surface *screen = NULL, *text = NULL, *wallpaper = NULL;
	SDL_Rect position;
	SDL_Event event;
	TTF_Font *font = NULL;
	SDL_Color blackColor = {0, 0, 0};
	int cont = 1;
	SDL_Init(SDL_INIT_VIDEO);
	TTF_Init();	
	screen = SDL_SetVideoMode(640, 480, 32, SDL_HWSURFACE | SDL_DOUBLEBUF);
	SDL_WM_SetCaption("Gestion du texte avec SDL_ttf", NULL);
	wallpaper = IMG_Load("moraira.jpg");
	// Loading the font 
	font = TTF_OpenFont("angelina.ttf", 65);
	// Writing the text on the surface with blended mode (the optimal one)
	text = TTF_RenderText_Blended(font, "Salut les Zér0s !", blackColor);
	while (cont)	
	{
		SDL_WaitEvent(&event);
		switch(event.type)
		{
			case SDL_QUIT:
			cont = 0;
			break;
		}
		SDL_FillRect(screen, NULL, SDL_MapRGB(screen->format, 255, 255, 255));
		position.x = 0;
		position.y = 0;
		SDL_BlitSurface(wallpaper, NULL, screen, &position); // Blitting the wallpaper
		position.x = 60;
		position.y = 370;
		SDL_BlitSurface(text, NULL, screen, &position); // Blitting the text
		SDL_Flip(screen);
	}
	TTF_CloseFont(font);
	TTF_Quit();
	SDL_FreeSurface(text);
	SDL_Quit();
	return EXIT_SUCCESS;
}
\end{Csource}

النتيجة تمثّلُها الصورة التالية:

\begin{figure}[H]
	\centering
	\includegraphics[width=0.6\textwidth]{Chapter_III-7_Blended-text}
\end{figure}

إذا أردت تغيير طريقة الكتابة للتجريب، لا يوجد سوى سطر للتعديل: السطر الخاص بإنشاء المساحة (استدعاء الدالة 
\InlineCode{TTF\_RenderText\_Blended}).

\begin{warning}
تأخذ الدالة
\InlineCode{TTF\_RenderText\_Shaded}
معاملا رابعا على عكس الآخرتين. هذا المعامل الأخير هو لون الخلفية الذي نريد استعماله. يجب عليك إذا إنشاء متغير من نوع
\InlineCode{SDL\_Color}
للإشارة إلى لون الخلفية (مثلا أبيض).
\end{warning}

\subsection{خصائص كتابة نص}

يمكن أيضًا تحديد خصائص الخط، كـغليظ مثلًا، مائل ومسطّر. 

يجب أولاّ أن يتم تحميل الخط ولهذا يجب أن يتوفر لديك متغير
\InlineCode{font}
صحيح. ويمكنك إذا استدعاء الدالة 
\InlineCode{TTF\_SetFontStyle}
التي ستقوم بالتعديل على الخط لكي يكون غليظا، مائلا أو مسطّرا حسب الرغبة. الدالة تأخذ معاملين:

\begin{itemize}
	\item الخط الذي نريد تعديله.
	\item دمج أعلام للإشارة إلى نمط الكتابة الذي نريد إعطاءه: غليظ، مائل أو مسطّر.
\end{itemize}

بالنسبة للأعلام، يجب عليك استعمال الثوابت التالية:

\begin{itemize}
	\item \InlineCode{TTF\_STYLE\_NORMAL}:
	عادي.
	\item \InlineCode{TTF\_STYLE\_BOLD}:
	غليظ.
	\item \InlineCode{TTF\_STYLE\_ITALIC}:
	مائل.
	\item \InlineCode{TTF\_STYLE\_UNDERLINE}:
	مسطّر.
\end{itemize}

بما أنها قائمة من الأعلام، يمكنك الدمج بينها باستعمال الإشارة
\InlineCode{|}
كما تعلّمنا القيام بذلك سابقًا.

فلنجرّب:

\begin{Csource}
// Loading the font
font = TTF_OpenFont("angelina.ttf", 65);
// The text will be italic and underlined
TTF_SetFontStyle(font, TTF_STYLE_ITALIC | TTF_STYLE_UNDERLINE);
// Writing the text in italic and underlined modes
text = TTF_RenderText_Blended(font, "Salut les Zér0s !", blackColor);
\end{Csource}

النتيجة: النص مكتوب بخاصية مائل ومسطّر:

\begin{figure}[H]
	\centering
	\includegraphics[width=0.6\textwidth]{Chapter_III-7_Italic-underlined-text}
\end{figure}

لإرجاع خط ما إلى حالته العاديّة، يكفي أن نعيد استدعاء الدالة
\InlineCode{TTF\_SetFontStyle}
باستعمال العلم
\InlineCode{TTF\_STYLE\_NORMAL}
هذه المرّة.

\subsection{تمرين: العداد}

سيجمع هذا التمرين بين المفاهيم التي تعلّمتها في هذا الفصل وفصل التحكّم في الوقت. مهمّتك، إن قبلتها، هي إنشاء عداد تتصاعد قيمته كلّ أعشار الثانية، أي أنه سيُظهر بشكل تقدّمي القيم التالية: 0، 100، 200، 300، 400\dots
بعد ثانية، يجدر بالرقم 1000 أن يظهر.

\subsubsection{طريقة للكتابة في سلسلة محارف}

لكي تحلّ هذا التمرين، ستحتاج إلى معرفة كيفية الكتابة داخل سلسلة محارف في الذاكرة.\\
في الواقع يجب عليك أن تعطي للدالة
\InlineCode{TTF\_RenderText}
متغيرا من نوع
\InlineCode{char*}
لكن ماهو متوفّر لديك هو عدد (من نوع
\InlineCode{int}
مثلا). كيف يمكننا تحويل عدد إلى سلسلة محارف؟

يمكننا أن نستعمل من أجل هذا الدالة
\InlineCode{sprintf}.\\
إنها تعمل بنفس الطريقة التي تعمل بها
\InlineCode{fprintf}،
الاختلاف الوحيد هو أنه في عوض الكتابة في ملف، ستتم الكتابة في سلسلة محارف (الحرف
\textenglish{s}
يختصر الكلمة
\textenglish{string}
و التي تعني "سلسلة محارف" بالإنجليزيّة).\\
أوّل معامل تقدّمه سيكون إذا مؤشّرا نحو جدول من
\InlineCode{char}.

\begin{critical}
قم بحجز مكان كافٍ من أجل جدول
\InlineCode{char}
إذا أردت ألا تتجاوز في الذاكرة!
\end{critical}

مثال:

\begin{Csource}
sprintf(time, "Temps : %d", counter);
\end{Csource}

هنا، المتغير
\InlineCode{time}
هو جدول محارف (20 محرفا)، و
\InlineCode{counter}
هو متغير من نوع
\InlineCode{int}
يحوي الزمن.\\
بعد هذه التعليمة، سلسلة المحارف 
\InlineCode{time}
ستحتوي مثلا على
"\textenglish{Temps : 500}".

هيّا، حان وقتُ العمل!

\subsection{التصحيح}

هذا تصحيح ممكن للتمرين:

\begin{Csource}
int main(int argc, char *argv[])
{
	SDL_Surface *screen = NULL, *text = NULL;
	SDL_Rect position;
	SDL_Event event;
	TTF_Font *font = NULL;
	SDL_Color blackColor = {0, 0, 0}, whiteColor = {255, 255, 255};
	int cont = 1;
	int currentTime = 0, previousTime = 0, counter = 0;
	char time[20] = ""; // A table of char big enough
	SDL_Init(SDL_INIT_VIDEO);
	TTF_Init();
	screen = SDL_SetVideoMode(640, 480, 32, SDL_HWSURFACE | SDL_DOUBLEBUF);
	SDL_WM_SetCaption("Gestion du texte avec SDL_ttf", NULL);
	// Loading the police
	font = TTF_OpenFont("angelina.ttf", 65);
	// Time and text initialization
	currentTime = SDL_GetTicks();
	sprintf(time, "Temps : %d", counter);
	text = TTF_RenderText_Shaded(font, time, blackColor, whiteColor);
	while (cont)
	{
		SDL_PollEvent(&event);
		switch(event.type)
		{
			case SDL_QUIT:
			cont = 0;
			break;
		}
		SDL_FillRect(screen, NULL, SDL_MapRGB(screen->format, 255, 255, 255));
		currentTime = SDL_GetTicks();
		if (currentTime - previousTime >= 100) // If 100ms at least have passed
		{
			counter += 100; // We add 100ms to the counter
			sprintf(time, "Temps : %d", counter); // We write in the string "time" the new time
			SDL_FreeSurface(text);// We delete the previous surface
			text = TTF_RenderText_Shaded(font, time, blackColor, whiteColor); // We write the sring "time" in SDL_Surface
			previousTime = currentTime; // We update the previous time
		}
		position.x = 180;
		position.y = 210;
		SDL_BlitSurface(text, NULL, screen, &position); // Blitting the text
		SDL_Flip(screen);
	}
	TTF_CloseFont(font);
	TTF_Quit();
	SDL_FreeSurface(text);
	SDL_Quit();
	return EXIT_SUCCESS;
}
\end{Csource}
 
الصورة التالية تمثّل النتيجة في غضون 13,9 ثانية بالتحديد:

\begin{figure}[H]
	\centering
	\includegraphics[width=0.6\textwidth]{Chapter_III-7_Time-text}
\end{figure}

لا تتردد في تنزيل المشروع إذا أردت دراسته بالتفصيل وتحسينه. هو ليس مثاليًا بعد: يمكننا مثلًا استعمال
\InlineCode{SDL\_Delay}
لتجنّب استعمال المعالج بنسبة 100\%.

\textenglish{\url{https://openclassrooms.com/uploads/fr/ftp/mateo21/ttf_exercice_temps.zip} (437 Ko)}

\subsubsection{للذهاب بعيدا}

إذا أردت التقدّم وتحسين هذا البرنامج، يمكنك أن تحاول صنع لعبة أين يجب النقر بالفأرة العدد الأقصى من المرات الممكنة في النافذة في وقت محدود حيث تتزايد قيمة العداد بعد كلّ نقرة.

يجب أن يتم إظهار عداد عكسي. حينما يصل إلى الصفر، نظهر عدد النقرات التي تم القيام بها ونطلب من المستعمل ما إن كان يريد إعادة المحاولة.

يمكنك أيضًا معالجة أفضل النتائج وتسجيلها في ملف. هذا سيساعدك في التدرب من جديد على استخدام الملفات في
\textenglish{C}.

حظًا موفقا!

\section*{ملخّص}

\begin{itemize}
	\item لا يمكننا أن نكتب نصًا في
	\textenglish{SDL}،
	إلا إن استعملنا تمديدًا كالمكتبة 
	\textenglish{SDL\_ttf}.
	\item تسمح هذه المكتبة بتحميل ملفات خطوط ذات صيغة 
	\InlineCode{.ttf}
	بالاستعانة بالدالة
	\InlineCode{TTF\_OpenFont}.
	\item توجد ثلاث طرق لكتابة نص، ترتيبها من الأبسط إلى الأكثر تعقيدا:
	\textenglish{Solid}،
	ثم
	\textenglish{Shaded}
	ثم
	\textenglish{Blended}.
	\item يمكننا الكتابة في
	\InlineCode{SDL\_Surface}
	عن طريق دوال مثل
	\InlineCode{TTF\_RenderText\_Blended}.
\end{itemize}

  \chapter{تشغيل الصوت بـ\textenglish{FMOD}}

منذ أن اكتشفنا الـ\textenglish{SDL}،
تعلّمنا موضعة صور على النافذة، التفاعل مع المُستعمل بالفأرة و لوحة المفاتيح، كتابة نصوص، لكن ينقص أمر بالتأكيد : الصوت !

سيسدّ هذا الفصل ذلك النقص. بما أن الإمكانيّات التي توفّرها لنا الـ\textenglish{SDL}
من ناحية الصوت محدودة جداً، سنكتشف هنا مكتبة متخصصة في الصوت :
\textenglish{FMOD}.

\section{تثبيت \textenglish{FMOD}}

\subsection{لماذا \textenglish{FMOD} ؟}

أنت تعرف ذلك الآن: الـ\textenglish{SDL}
ليست فقط مكتبة رسومية. هي تسمح أيضاً بمعالجة الصوت عن طريق وحدة تسمّى
\textenglish{SDL\_audio}.
فلماذا إذاً سنحضّر مكتبة خارجية لا علاقة لها بالـ\textenglish{SDL}
كـ\textenglish{FMOD} ؟

في الواقع هو اختيار قمتُ به بعد عدّة اختبارات. كان بإمكاني أن أشرح لك طريقة معالجة الصوت بالـ\textenglish{SDL}
لكنّي فضّلت عدم فعل ذلك.\\
سأشرح موقفي أكثر.

\subsection{لماذا قمتُ بتجنّب \textenglish{SDL\_audio} ؟}

يعتبر التحكم في الصوت بالـ\textenglish{SDL}
"منخفض المستوى". هذا يعني أنه يجب القيام بالعديد من التعامُلات الدقيقة كي نستطيع تشغيل الصوت. بمعنى آخر، سيكون الأمر صعباً و لا أجد ذلك ممتعاً. توجد مكتبات أخرى تسمح بتشغيل الصوت بشكل بسيط.

\begin{information}
تذكير بسيط : مكتبة "منخفضة المستوى" هي مكتبة قريبة من الحاسوب. يجب أن نتعرّف إذا على قليل من العمل الداخلي للحاسوب كي نستفيد منها و يتطلب الأمر في الواقع وقتاً أكثر من الوقت اللازم للقيام بنفس الشيء مع مكتبة "عالية المستوى".\\
لا تنس أنّ كلّ شيء نسبيّ : لا توجد مكتبات منخفضة المستوى من جهة و أخرى عالية المستوى من جهة أخرى. هي فقط أكثر أو أقل من بعضها البعض في المستوى. مثلا، المكتبة
\textenglish{FMOD}
عالية المستوى مقارنة بالوحدة
\textenglish{SDL\_audio}
من الـ\textenglish{SDL}.
\end{information}

تفصيل آخر مهم، تسمح الـ\textenglish{SDL}
بتشغيل صوت بصيغة
\textenglish{WAV}
فقط. صيغة الصوت هذه ليست مضغوطة. أي أن موسيقى من 3 دقائق تأخذ عشرات الميغا أوكتي،
على عكس الصيغ المضغوطة مثل
\textenglish{MP3}
أو
\textenglish{Ogg}
التي تحجز حجم ذاكرة أقلّ بكثير (من 2 إلى 3 ميغا أوكتي).

في الواقع، لو نفكّر في الأمر جيًداً، كان الأمر مشابهاً بالنسبة للصور، فالـ\textenglish{SDL}
لا تتعامل إلا مع الصيغة
\textenglish{BMP}
(صُور غير مضغوطة) بشكل مبدئي. مما استوجب علينا تسطيب مكتبة إضافية و هي
\textenglish{SDL\_image}
لنتمكّن من قراءة صيغ الصور الأخرى كـ\textenglish{JPEG}، \textenglish{PNG}، \textenglish{GIF}،
إلخ.

اعلم أنه هناك مكتبة مكافئة بالنسبة للصوت و هي :
\textenglish{SDL\_mixer}.
هي قادرة على قراءة عدد كبير من صيغ الصوت، من بينها
\textenglish{MP3}، \textenglish{Ogg}، \textenglish{Midi} \dots
و رغم ذلك، لم أكلّمك عن هذه المكتبة. لماذا ؟

\subsection{لماذا قمتُ بتجنّب \textenglish{SDL\_mixer} ؟}

\textenglish{SDL\_mixer}
هي مكتبة نضيفها للـ\textenglish{SDL}،
بطريقة 
\textenglish{SDL\_image}.
هي سهلة للاستعمال و تقرأ العديد من صيغ الصوت المختلفة. لكن، و بعد الاختبارات التي قمتُ بها، تبيّن لي أن هذه المكتبة تحتوي عِلَلا مزعجة بالإضافة إلى كونها محدودة من ناحية المزايا التي تمنحها.

من أجل هذه الأسباب توجّهت مباشرة إلى
\textenglish{FMOD}،
مكتبة لا علاقة لها بالـ\textenglish{SDL}
بالتأكيد لكن لها الأفضلية كونها قوية و متداولا عليها.

\subsection{تنزيل \textenglish{FMOD}}

إن كنت قد حكيت لك كل هذا، فهذا فقط لأخبرك بأن اختيار
\textenglish{FMOD}
لم يكن عشوائياً. ببساطة هي أفضل مكتبة مجانية استطعت إيجادها.\\
كما أنها سهلة الإستخدام كـ\textenglish{SDL\_mixer}
بأفضلية لا يمكن تجاهلها : لا توجد بها مشاكل برمجية.

تسمح
\textenglish{FMOD}
بالقيام بالعديد من الوظائف التي لا تسمح بها
\textenglish{SDL\_mixer}،
كالتأثيرات الصوتيّة ثلاثية الأبعاد.

\begin{warning}
\textenglish{FMOD}
هي مكتبة مجانية لكن ليست تحت رخصة
\textenglish{LGPL}
على عكس الـ\textenglish{SDL}.
هذا يعني أنه بإمكانك أن تستخدمها مادامت لم تحقق بها برامج مدفوعة. إذا أردت أن يكون البرنامج غير مجاني، يجب أن تدفع رسوماً لمؤلّف المكتبة (سأتركك تطّلع على الأسعار من خلال الموقع الرسمي لـ\textenglish{FMOD}).\\
كثير من الألعاب التجارية تستعمل
\textenglish{FMOD}
و من أشهر هذه الألعاب :
\textenglish{Starcraft II}، \textenglish{World of Warcraft : Cataclysm}، \textenglish{Crysis 2}،
إلخ.
\end{warning}

تتوفر العديد من نسخ
\textenglish{FMOD}،
و النسخة الموجهّة إلى الاستعمال في أنظمة التشغيل المألوفة
(\mbox{\textenglish{GNU/Linux}}، \textenglish{Windows}، \mbox{\textenglish{Mac OS X}}، \dots)
تُدعى
\textenglish{FMOD Ex Programmers API}.

نزّل إذا نسخة
\textenglish{FMOD Ex}
المناسبة لنظام التشغيل الخاص بك. خذ النسخة المسمّاة "مستقرة"
(\textenglish{stable}).

و تأكد بشكل خاص ما إن كان لديك نظام تشغيل
\textenglish{32 bits}
أو
\textenglish{64 bits}
(في
\textenglish{Windows}،
 قم بنقر يميني على جهاز الكمبيوتر
(\textenglish{Computer})
ثم في قسم الخصائص
(\textenglish{Properties})
تجد المعلومة اللازمة).

\url{http://www.fmod.org/fmod-downloads.html#FMODExProgrammersAPI}

\subsection{تثبيت \textenglish{FMOD}}

يعمل التثبيت بنفس مبدأ عمل المكتبات السابقة، أي مثل الـ\textenglish{SDL}.

يجدر بالملف الذي حمّلته أن يكون ملفاً تنفيذياً (في
\textenglish{Windows})،
 أو أن يكون أرشيفا 
(\InlineCode{.dmg}
في
\mbox{\textenglish{Mac OS X}}
و
\InlineCode{.tar.gz}
في
\mbox{\textenglish{GNU/Linux}}).

\begin{enumerate}
	\item ثبّت
	\mbox{\textenglish{FMOD Ex}}
على قرصك الصلب. الملفات التي نحتاجها يجب أن تتواجد في مجلّد يشبه هذا :\\
	\InlineCode{C:\textbackslash Program Files\textbackslash FMOD SoundSystem\textbackslash FMOD Programmers API Win32\textbackslash api}.
	\item في هذا المجلّد تجد الـ\textenglish{DLL}
	الخاصّ بـ\mbox{\textenglish{FMOD Ex}}
	(\InlineCode{fmodex.dll})
	و يجب أن يوضع في مجلّد المشروع. الـ\textenglish{DLL}
	الأخرى، أي
	\InlineCode{fmodexL.dll}
	تعمل على تنقيح العلل البرمجية. لن نقوم بذلك هنا. تذكّر فقط بأن الملف
	\InlineCode{fmodex.dll}
	هو الذي يجب أن تُعطيه مع الملف التنفيذي للبرنامج.
	\item في المجلّد
	\InlineCode{api/inc}،
	تجد الملفات
	\InlineCode{.h}.
	ضعها كلّها إلى جانب الملفات الرأسية التي هي في مجلّد البيئة التطويرية. مثلا :
	\InlineCode{Code Blocks/mingw32/include/fmodex}
	(لقد أنشأت مجلّدا خصيصاً لأجل
	\textenglish{FMOD}
	كما مثل الـ\textenglish{SDL}).
	\item في المجلّد
	\InlineCode{api/lib}،
	استرجع الملف الموافق للمترجم. يجدر بملف نصّي أن يشير إلى أي ملف يجب أن نأخذ.
	\begin{itemize}
		\item إذا كنت تستعمل
		\textenglish{Code::Blocks}،
		فالمترجم هو
		\textenglish{mingw}.
		أنسخ الملف
		\InlineCode{libfmodex.a}
		في المجلّد
		\InlineCode{lib}
		للبيئة التطويرية.
		
		في
		\textenglish{Code::Blocks}،
		إنه المجلّد
		\InlineCode{CodeBlocks/mingw32/lib}.
		\item إذا كنت تستعمل
		\textenglish{Visual C++}،
		استرجع الملف
		\InlineCode{fmodex\_vc.lib}.
	\end{itemize}
	\item أخيراً، الشيء الأكثر أهمية ربّما، يوجد مجلّد 
	\InlineCode{documentation}
	في المجلّد
	\textenglish{FMOD Ex}.
	من المفروض أن تتم إضافة اختصارات إلى قائمة "إبدأ" نحو هذه الملفات التوجيهية. أبق نظرك عليها لأننّا لا يمكن ألا نكتشف كلّ ميزات
	\textenglish{FMOD Ex}
	في هذا الفصل. ستحتاج إلى هذه الملفات في أقرب الآجال بالتأكيد.
	
	يبقى أن نخصص المشروع. هنا أيضاً و مثل كلّ مرة : افتح المشروع بواسطة البيئة التطويرية المفضّلة و أضف الملف
	\InlineCode{.a}
	(أو
	\InlineCode{.lib})
	إلى قائمة الملفات التي يجب أن يسترجعها محرر الروابط.\\
	في
	\textenglish{Code::Blocks}
	(يخالجني شعور بأنني أقوم بالتكرار)، إذهب إلى قائمة
	\InlineCode{Project} / \InlineCode{Build Options}
	ثم قسم
	\InlineCode{Linker}،
	أنقر على
	\InlineCode{Add}
	و أشر إلى المسار الذي يوجد به الملف
	\InlineCode{.a}
	إذا ظهرت لك الرسالة~:
	"\textenglish{Keep as a relative path ?}"،
	أنصحك بأن تجيب بالسلب لكن يجدر بالأمور أن تشتغل في كلتا الحالتين.
	
	تم تثبيت
	\textenglish{FMOD Ex}،
	فلنَرَ بسرعة مما هي مُشَكَّلَة.
\end{enumerate}

\section{تهيئة و تحرير غرض نظامي}

المكتبة
\textenglish{FMOD Ex}
متوفّرة من أجل اللغتين
\textenglish{C}
و
\textenglish{C++}.\\
الشيء الخاص فيها هو أن مطوّري هذه المكتبة احتفظوا ببعض التناسق في "تركيب الكلمات"
(\textenglish{syntax})
بين اللغتين. الميزة الأولى هي أنه إذا تعلّمت التعامل مع
\textenglish{FMOD Ex}
في لغة الـ\textenglish{C}
ستتمكن من فعل ذلك في الـ\textenglish{C++}
بنسبة 95\%.

\subsection{تضمين الملف الرأسي}

قبل كلّ شيء، يلزمك أن تقوم بتضمين الملف الرأسي الخاص بـ\textenglish{FMOD}.
 لابأس في التذكير بكتابته :

\begin{Csource}
#include <fmodex/fmod.h>
\end{Csource}

لقد وضعت هذا الملف في المجلّد الداخلي
\InlineCode{fmodex}.
عدّل على هذا السطر من الشفرة على حسب المسار الذي يتواجد به الملف عندك.\\
إذا كنت تعمل على
\mbox{\textenglish{GNU/Linux}}،
 يجدر بالتسطيب أن يتم تلقائيّا في المجلّد
\InlineCode{fmodex}.

\subsection{إنشاء و تهيئة غرض نظامي}

الغرض النظامي هو عبارة عن متغير نستفيد منه على طول البرنامج لكي نعرّف معاملات المكتبة.\\
تذكّر أنه بالـ\textenglish{SDL}
مثلاً، كان يجب أن نهيّئ المكتبة بشكل مباشر بواسطة دالة. هنا، دليل الاستعمال مختلف قليلاً : في عوض تهيئة كلّ المكتبة، لن نعمل إلا بغرض
(\textenglish{Object})
دوره تعريف سلوك هذه الأخيرة.

لكي ننشئ غرضا نظاميا، يكفي أن نعرّف مؤشّرا من نوع
\InlineCode{FMOD\_SYSTEM}.
مثلا :

\begin{Csource}
FMOD_SYSTEM *system;
\end{Csource}

لكي نحجز مكاناً في الذاكرة من أجل هذا الغرض النظامي، نستعمل الدالة
\InlineCode{FMOD\_System\_Create}
و التي نموذجها هو الآتي :

\begin{Csource}
FMOD_RESULT FMOD_System_Create(FMOD_SYSTEM ** system);
\end{Csource}

لاحظ أن هذه الدالة تأخذ مؤشّرا نحو مؤشّر يؤشّر نحو
\InlineCode{FMOD\_SYSTEM}.
القرّاء الأكثر حرصاً كانوا قد لاحظوا أنه لدى تعريف المؤشّر
\InlineCode{FMOD\_SYSTEM}،
لم يتم حجزه بواسطة
\InlineCode{malloc}
أو أي دالة أخرى. لهذا السبب تماماً تأخذ الدالة
\InlineCode{FMOD\_SYSTEM}
معاملا من ذلك النوع لكي تحجز مكاناً للمؤشّر النظامي.

بعد تعريف الغرض النظامي ، تكفي كتابة :

\begin{Csource}
FMOD_SYSTEM *system;
FMOD_System_Create(&system);
\end{Csource}

هكذا إذا، بما أننا نتوفّر الآن على الغرض النظامي، لم يتبّق علينا سوى تهيئته. لفعل هذا، نستعمل الدالة
\InlineCode{FMOD\_System\_Init}
ذات النموذج :
\begin{Csource}
FMOD_RESULT FMOD_System_Init(
	FMOD_SYSTEM *  system,
	int  maxchannels,
	FMOD_INITFLAGS  flags,
	void *  extradriverdata
);
\end{Csource}

\begin{itemize}
	\item المعامل
	\InlineCode{system}
	هو المعامل الذي يهمّنا أكثر، لأنه المؤشّر الذي سنقوم بتهيئته.
	\item المعامل
	\InlineCode{maxchannels}
	يمثّل العدد الأقصى للقنوات التي يجب أن تديرها
	\InlineCode{FMOD}.
	بمعنى آخر، هو العدد الأقصى للأصوات التي يمكن أن يتم تشغيلها في نفس الوقت. هذا يعتمد على قوة بطاقة الصوت لديك . ننصح عادة بقيمة 32 (قيمة كافية من أجل معظم الألعاب البسيطة). لمعلوماتك، يمكن نظرياً لـ\InlineCode{FMOD}
	إدارة 1024 قناة مختلفة، لكن بهذا المستوى ستخاطر بجعل حاسوبك يشتغل كثيرا !
	\item المعامل
	\InlineCode{flag}
	لا يهمّنا كثيراً في هذا الدرس، سنكتفي بإعطائها القيمة
	\InlineCode{FMOD\_INIT\_NORMAL}.
	\item المعامل
	\InlineCode{extradriverdata}
	لا يهمّنا أيضاً، سنعطيه القيمة
	\InlineCode{NULL}.
\end{itemize}

مثلا، لكي نعرّف، نحجز، و نهيّئ غرضا نظاميا،  نقوم بكتابة التالي :

\begin{Csource}
FMOD_SYSTEM *system;
FMOD_System_Create(&system);
FMOD_System_Init(system, 2, FMOD_INIT_NORMAL, NULL);
\end{Csource}

نتوفّر الآن على غرض نظامي جاهز للإستعمال.

\subsection{غلق و تحرير غرض نظامي}

نغلق ثمّ نحرر الغرض النظامي بواسطة دالتين :

\begin{Csource}
FMOD_System_Close(system);
FMOD_System_Release(system);
\end{Csource}

هل يجدر بي أن أعلّق على هذه الشفرة ؟

  \chapter{عمل تطبيقي : الإظهار الطيفي للصوت}

هذا العمل التطبيقي سيقترح عليك التعامل مع الـ\textenglish{SDL}
و الـ\textenglish{FMOD}
في نفس الوقت. هذه المرّة، لن نعمل على لعبة. كما نعرف فالـ\textenglish{SDL}
مخصصة لهذا، لكن يمكن استعمالها في ميادين أخرى. سيقوم هذا الفصل بإثبات أنها صالحة لأجل أشياء أخرى.

سنحقق هنا إظهاراً للطيف الصوتي بالـ\textenglish{SDL}.
يتوقّف هذا على إظهار تركيبة الصوت الذي نشغّله، مثلاً موسيقى. نجد هذه الخاصية في كثير من برامج قراءة الأصوات. إنه أمرٌ ممتع و ليس بقدر الصعوبة التي يبدو عليها !

سيسمح لك هذا الفصل بالعمل على مفاهيم قُمنا باستكشافها مؤخّراً :

\begin{itemize}
	\item التحكّم في الوقت.
	\item المكتبة 
	\textenglish{FMOD}.
\end{itemize}

سنتعرّف علاوة على ذلك، على كيفية التعديل على مساحة بيكسلا ببيكسل.

الصورة التالية تعطيك مظهراً للبرنامج الذي سنكتبه في هذا الفصل.

\Picture{Chapter_III-9_Window-spectral}

هو نوع الإظهار الذي نجده في قارئي الأصوات كـ\textenglish{Winamp}،
\textenglish{Windows Media Player} أو \textenglish{AmaroK}.\\
كما قلتُ لك إن الأمر ليس صعبَ التحقيق. على عكس العمل التطبيقي الخاص بـ\textenglish{Mario Sokoban}،
هذه المرّة ستقوم بنفسك بالعمل. سيمثّل هذا بالنسبة إليك تمريناً جيداً.

\section{التعليمات}

التعليمات بسيطة. إتّبعها خطوة بخطوة بالترتيب، و لن تواجه أي مشاكل.

\subsection{قراءة ملف \textenglish{MP3}}

لكي تبدأ، يجب عليك إنشاء برنامج يقوم بقراءة ملف
\textenglish{MP3}. ليس عليك سوى إعادة
الأغنية 
"\textenglish{Home}"
للمجموعة
"\textenglish{Hype}"
و التي استعملناها في الفصل الخاص بـ\textenglish{FMOD}
لتلخيص كيفية عمل تشغيل الموسيقى.

إذا اتّبعت جيّدا الفصل حول
\textenglish{FMOD}،
لا تحتاج أكثر من بضعة دقائق لكي تقوم بالعملية. أنصحك بالمناسبة أن تقوم بنقل الملف
\textenglish{MP3}
إلى مجلّد المشروع.

\subsection{استرجاع المعلومات الطيفية للصوت}

لكي نعرف كيف يعمل الإظهار الطيفي للصوت، من الواجب أن أشرح لك كيفية يعمل الأمر من الداخل (بشكل تقريبي فقط، و إلا سندخل في درس رياضيات).

يمكن أن يتم تقسيم الصوت إلى ترددات 
(\textenglish{Frequencies}).
بعض الترددات منخفضة، بعضها متوسطة و بعضها مرتفعة. ما سنقوم به في عملية الإظهار هو إظهار كمية كلّ واحدة من الترددات على شكل شرائط و كلّما يكون الشريط كبيراً، كلما يكون التردد مستعملاً أكثر :

\Picture{Chapter_III-9_Frequencies}

على يسار النافذة، نقوم بإظهار الترددات المنخفضة، و على اليمين الترددات المرتفعة.

\begin{question}
لكن كيف نسترجع كميّة كلّ تردد ؟
\end{question}

ستهتم
\textenglish{FMOD}
بهذا العمل. يمكننا استدعاء الدالة
\InlineCode{FMOD\_Channel\_GetSpectrum}
ذات النموذج :

\begin{Csource}
FMOD_RESULT FMOD_Channel_GetSpectrum(
	FMOD_CHANNEL *  channel,
	float *  spectrumarray,
	int  numvalues,
	int  channeloffset,
	FMOD_DSP_FFT_WINDOW  windowtype
);
\end{Csource}

و هاهي المعاملات التي تحتاجها الدالة :

\begin{itemize}
	\item القناة التي تشتغل فيها الموسيقى. يجب إذا استرجاع مؤشّر نحو هذه القناة.
	\item جدول
	\InlineCode{float}.
	يجب أن يتم حجز الذاكرة من أجل هذا الجدول مسبّقاً، بشكل ثابت أو حيّ، لكي نسمح لـ\textenglish{FMOD}
	بملئه بشكل صحيح.
	\item حجم الجدول. يجب أن يكون حجم الجدول إجبارياً عبارة عن قوّة للعدد 2، مثلا 512.
	\item يسمح هذا المعامل بتعريف بأي مخرج نحن مهتمون. مثلاً لو أننا في
	\textenglish{stereo}،
	فـ$ 0 $ تعني اليسار و $ 1 $ تعني اليمين.
	\item هذا المعامل معقّد قليلاً، و لا يهمّنا حقيقة في هذا الفصل. سنكتفي بإعطائه القيمة\\ 
	\InlineCode{FMOD\_DSP\_FFT\_WINDOW\_RECT}.
\end{itemize}

\begin{information}
تذكير : النوع
\InlineCode{float}
هو نوع عشري، مثل
\InlineCode{double}.
الاختلاف بين الإثنين يكمن في كون الـ\InlineCode{double}
أكثر دقّة من الآخر، لكن في حالتنا يكفينا الـ\InlineCode{float}.
هذا الأخير مستعمل من طرف
\textenglish{FMOD}
هنا. و لذلك، هو ما سنستعمله نحن أيضاً.
\end{information}

بشكل واضح، نعرّف جدول الـ\InlineCode{float} :

\begin{Csource}
float spectrum[512];
\end{Csource}

ثم، حين يتم تشغيل الموسيقى، نطلب من 
\textenglish{FMOD}
ملئ جدول الأطياف بالقيام مثلاً بـ :

\begin{Csource}
FMOD_Channel_GetSpectrum(channel, spectrum, 512, 0, FMOD_DSP_FFT_WINDOW_RECT);
\end{Csource}

يمكننا بعد ذلك تصفّح الجدول لكي نتحصّل على قيم الأطياف :

\begin{Csource}
spectrum[0] // The lowest frequency (Left)
spectrum[1]
spectrum[2]
...
spectrum[509]
spectrum[510]
spectrum[511] // The highest frequency (Right)
\end{Csource}

كلّ تردد هو عبارة عن عدد عشري محصور بين $ 0 $ (لا شيء) و $ 1 $ (قيمة قصوى). ينصّ عملك على إظهار كلّ شريط سواء كان قصيراً أو كبيراً بدلالة القيمة التي تحتويها كلّ من خانات الجدول.

مثلاً، إذا كانت القيمة هي $ 0.5 $ يجدر بك رسم شريط يكون علّوه مساوياً لنصف علوّ النافذة. إذا كانت القيمة هي $ 1 $، فسيأخذ الشريط كلّ علو النافذة.

بشكل عام، تكون القيم ضعيفة (أكثر قرباً من $ 0 $ على $ 1 $). أنصحك بضرب كلّ القيم بـ20 لكي ترى الطيف بشكل أفضل.\\
احذر : إذا قمت بهذا، تأكد بأنك لن تتجاوز $ 1 $ (قم بتدوير القيمة إلى $ 1 $ إذا احجت إلى ذلك). إذا وجدت أنك تتعامل مع أعداد تفوق $ 1 $، فقد تواجه مشاكل لاحقاً في رسم الشرائط العموديّة لاحقا !

\begin{question}
لكن يجدر بالشرائط أن تتحرّك في نفس الوقت الذي يتم فيه تشغيل الصوت، أليس كذلك ؟ بما أن الصوت يتحرّك كلّ الوقت، يجب تحديث الصورة الرسومية، ما العمل ؟
\end{question}

سؤال جيد. في الواقع، الجدول الخاص المتكون من 
512 \InlineCode{float}
الذي ترجعه لنا
\textenglish{FMOD}
يتغيّر كل 25 مث (لكي نكون في نفس الفاصل الزمني بالنسبة للصوت الحالي). يجب إذا في الشفرة المصدرية أن تعيد قراءة جدول الـ512
\InlineCode{float}
 (بإعادة استدعاء
\InlineCode{FMOD\_Channel\_GetSpectrum}
 كلّ 25 مث)، ثم تقوم بتحديث رسمك ذي الشرائط.
 
أعد قراءة الفصل حول التحكّم في الوقت بالـ\textenglish{SDL}
لكي تتذكّر كيفية عمل ذلك. لديك الخيار بين
\InlineCode{GetTicks}
و الـ\textenglish{callbacks}.
استعمل ما تراه أكثر سهولة لك.

\subsection{إنشاء التدرّج اللوني}

في البداية، يمكنك تحقيق الشرائط بلون موحّد. يمكنك إذا إنشاء مساحات. يجب إذا أن تكون هناك 512 مساحة : واحدة من أجل كلّ شريط. كلّ مساحة تأخذ إذا بيكسلا واحدا كعُرض. و يختلف علوّ الشرائط بدلالة شدّة كلّ تردد.

أنصحك بعدها أن تقوم بتحسين : يجب على الشريط أن يميل للأحمر كلّما زادت كثافة الصوت. أي أنه على الشريط أن يكون أخضراً من الأسفل و أحمراً من الأعلى.

\begin{question}
لكن \dots المساحة الواحدة لا يمكنها أن تأخذ سوى لونٍ واحدٍ لو عندما نستعمل الدالة
\InlineCode{SDL\_FillRect}.
لا يمكننا إنشاء تدرّح لوني !
\end{question}

في الواقع، يمكننا بالتأكيد إنشاء مساحات بعَرْض 1 بيكسل و عُلُو 1 بيكسل من أجل كلّ لون في التدرّج. لكن هذا سيأخذ بنا إلى إنشاء مساحات عديدة و لن يكون التحكّم فيها مثالياً !

كيف يمكن لنا أن نرسم بيكسلا ببيكسل ؟\\
لم أعلّمك هذا من قبل، لأنّ هذه التقنية لا تستحقّ فصلاً كاملاً. ستجد أنها في الواقع ليست صعبة. 

في الواقع، لا تقترح الـ\textenglish{SDL}
أية دالة للرسم بيكسلا ببيكسل. لكن لنا الحق في أن نكتبها بأنفسنا. لكي نقوم بهذا، يجب إتّباع هذه الخطوات النموذجية بالترتيب :

\begin{enumerate}
	\item استدع الدالة
	\InlineCode{SDL\_LockSurface}
	لنعلن للـ\textenglish{SDL}
	أننا سنقوم بالتعديل على المساحة يدوياً. هذا "يعطّل" المساحة للـ\textenglish{SDL}
	و ستكون وحدك قادراً على التحكّم فيها مادامت المساحة معطّلة.
	
	هنا، أنصحك بأن تعمل بمساحة واحدة فقط : الشاشة. إذا أردت رسم بيكسل في منطقة محددة من الشاشة، يجب عليك تعطيل المساحة 
	\InlineCode{screen} :
	
\begin{Csource}
SDL_LockSurface(screen);
\end{Csource}

	\item يمكنك بعد ذلك تغيير محتوى كلّ بيكسل من المساحة. بما أن الـ\textenglish{SDL}
	لا تقترح أية دالة للقيام بهذا، يجب أن نكتبها بأنفسنا في البرنامج.
	
	سأعطيك هذه الدالة، و التي استخرجتها من الملفات التوجيهية للـ\textenglish{SDL}.
	هي معقدّة أكثر لأنها تعمل على المساحة مباشرة و تتحكم في كلّ أعماق اللون الممكنة (بيتات على البيكسل). لا تحتاج لحفظها أو فهمها، قم بنسخها ببساطة في البرنامج لكي تتمكّن من استعمالها :
	
\begin{Csource}
void setPixel(SDL_Surface *surface, int x, int y, Uint32 pixel)
{
	int bpp = surface->format->BytesPerPixel;
	
	Uint8 *p = (Uint8 *)surface->pixels + y * surface->pitch + x * bpp;
	
	switch(bpp) {
		case 1:
		*p = pixel;
		break;
		
		case 2:
		*(Uint16 *)p = pixel;
		break;
		
		case 3:
		if(SDL_BYTEORDER == SDL_BIG_ENDIAN) {
			p[0] = (pixel >> 16) & 0xff;
			p[1] = (pixel >> 8) & 0xff;
			p[2] = pixel & 0xff;
		} else {
			p[0] = pixel & 0xff;
			p[1] = (pixel >> 8) & 0xff;
			p[2] = (pixel >> 16) & 0xff;
		}
		break;
		
		case 4:
		*(Uint32 *)p = pixel;
		break;
	}
}
\end{Csource}
\end{enumerate}

هي سهلة الإستعمال. ابعث لها المعاملات التالية :

\begin{itemize}
	\item المؤشّر نحو المساحة التي تريد التعديل عليها (يجب أن تكون معطّلة بواسطة
	\InlineCode{SDL\_LockSurface}).
	\item وضعية الفاصلة الخاصة بالبيكسل الذي نريد التعديل عليه في المساحة
	(\InlineCode{x}).
	\item وضعية الترتيبة الخاصة بالبيكسل الذي نريد التعديل عليه في المساحة
	(\InlineCode{y}).
	\item اللون الجديد الذي نعطيه للبيكسل. يجب أن يكون هذا اللون بصيغة
	\InlineCode{Uint32}.
	يمكنك إذا توليده بالإستعانة بالدالة
	\InlineCode{SDL\_MapRGB}
	التي تتقننها جيداً الآن.
	\item أخيراً، حينما تنتهي من العمل على المساحة، يجب ألا تنسى أن تزيل تعطيلها باستدعاء\\ 
	\InlineCode{SDL\_UnlockSurface}.
\end{itemize}

\begin{Csource}
SDL_UnlockSurface(screen);
\end{Csource}

\subsection{شفرة  ملخّصة للمثال}

لو نلخّص، ستجد بأن كلّ شيء سهل.\\
هذه الشفرة ترسم بيكسلا أحمرا في منتصف المساحة
\InlineCode{screen}
(أي في منتصف النافذة).

\begin{Csource}
SDL_LockSurface(screen); // We lock the surface
setPixel(screen, screen->w / 2, screen->h / 2, SDL_MapRGB(screen->format, 255, 0, 0)); // We draw a red pixel in the middle of the screen
SDL_UnlockSurface(screen); // We unlock the surface
\end{Csource}

من هذه القاعدة، يجدر بك أن تتمكن من تحقيق التدرّج اللوني من الأخضر للأحمر (يجب أن تستعمل الحلقات التكرارية).

\section{التصحيح}

إذا، كيف وجدت الموضوع ؟ ليس صعب الفهم، يجب فقط القيام ببعض الحسابات، خاصة من أجل تحقيق التدرّج اللوني. مستوى التمرين هو مستوً عام، يجب فقط أن تفكّر أكثر. 

بعض الأشخاص يأخذون وقتاً أطول من آخرين لإيجاد التصحيح. إذا لم تتمكّن من حلّ التمرين، هذا ليس سيئاً. ما يهمّ هو أن ننتهي بالوصول إلى هدفنا. مهما كان المشروع الذي تعمل عليه، فسيكون هناك بالتأكيد أوقات نجد فيها أنه لا ينقصنا أن نجيد البرمجة لكي نتمكّن من حلّ المشكل، يجب أيضاً أن نكون منطقيين و نجيد التفكير.

سأعطيك الشفرة المصدرية الكاملة. لقد علّقت عليها بشكل كافٍ :

\begin{Csource}
#include <stdlib.h>
#include <stdio.h>
#include <SDL/SDL.h>
#include <fmodex/fmod.h>
#define WINDOW_WIDTH 512 // MUST stay equal to 512 because there are 512 bars corresponding to 512 floats
#define WINDOW_HIGHT 400 // You can change this.
#define RATIO (WINDOW_HIGHT / 255.0)
#define LIMIT_TIME_TO_REFRESH 25 // Time in ms between two updates of the graph (25 is the minimum)
#define SPECTERUM_SIZE 512
void setPixel(SDL_Surface *surface, int x, int y, Uint32 pixel);
int main(int argc, char *argv[])
{
	SDL_Surface *screen = NULL;
	SDL_Event event;
	int cont = 1, barHeight = 0, currentTime = 0, previousTime = 0, i = 0, j = 0;
	float spectrum[SPECTERUM_SIZE];
	/* Initializing FMOD:
	Load FMOD, the music and start playing the music
	*/
	
	FMOD_SYSTEM *system;
	FMOD_SOUND *music;
	FMOD_CHANNEL *channel;
	FMOD_RESULT result;
	FMOD_System_Create(&system);
	FMOD_System_Init(system, 1, FMOD_INIT_NORMAL, NULL);
	// We open the music
	result = FMOD_System_CreateSound(system, "hype_home.mp3", FMOD_SOFTWARE | FMOD_2D | FMOD_CREATESTREAM, 0, &music);
	// We check if it has been opened correctly (IMPORTANT)
	if (result != FMOD_OK)
	{
		fprintf(stderr, "Can't read the mp3 file\n");
		exit(EXIT_FAILURE);
	}
	// We play the music
	FMOD_System_PlaySound(system, FMOD_CHANNEL_FREE, music, 0, NULL);
	
	// We get the channel pointer
	FMOD_System_GetChannel(system, 0, &channel);
	/*
	Initializing the SDL:
	---------------------
	We load the SDL, open a window and write in its title bar.
	We get also a pointer to the surface screen which will be the only surface to use in this program 
	*/
	SDL_Init(SDL_INIT_VIDEO);
	screen = SDL_SetVideoMode(WINDOW_WIDTH, WINDOW_HIGHT, 32, SDL_SWSURFACE | SDL_DOUBLEBUF);
	SDL_WM_SetCaption("Sound spectrum visualisation", NULL);
	// Main loop	
	while (cont)
	{
		SDL_PollEvent(&event); // We have to use PollEvent because we don't have to wait for the user's event to refresh the window
		switch(event.type)
		{
			case SDL_QUIT:
			cont = 0;
			break;
		}
		// We clear the screen every time before drawing the graph (black wallpaper)
		SDL_FillRect(screen, NULL, SDL_MapRGB(screen->format, 0, 0, 0));
		/* Managing the time
		-----------------
		We compare between the current time and the previous one (the last iteration of the loop).
		If the difference is less than 25 ms (updating time limit)
		Then we wait until 25ms pass.
		After that, we update previousTime with the new time. */
		currentTime = SDL_GetTicks();
		if (currentTime - previousTime < LIMIT_TIME_TO_REFRESH)
		{
			SDL_Delay(LIMIT_TIME_TO_REFRESH-(currentTime-previousTime));
		}
		previousTime = SDL_GetTicks();
		/* Drawing the sound spectrum
		------------------------
		It's the most important part. We have to think a little bit before drawing the spectrum. Maybe it's hard but it's possible, here is the proof.
		We fill the 512 floats table via FMOD_Channel_GetSpectrum()
		Then we work pixel by pixel on the surface screen to draw the bars.
		We make a first loop to browse the window in width.
		The second loop browses the window in height to draw the bars.
		*/
		
		/* We fill the 512 floats table. I've chosen to be interested in the left output */
		FMOD_Channel_GetSpectrum(channel, spectrum, SPECTERUM_SIZE, 0, FMOD_DSP_FFT_WINDOW_RECT);
		SDL_LockSurface(screen);
		/* We block the surface screen because we're going to directly modify its pixels */
		
		/* LOOP 1 : We browse the window in width (for every vertical bar) */
		for (i = 0 ; i < WINDOW_WIDTH ; i++)
		{
			/* We calculate the vertical bar's height that we're going to draw.
			spectrum[i] will return a number between 0 and 1 that we're going to multiply by 20 to zoom in order to have a better view (As I said).
			
			The, we multiply by WINDOW_HEIGHT so the bar will be expanded comparing to the window's size. */
			
			barHeight = spectrum[i] * 20 * WINDOW_HIGHT;
			/* We verify that the bar doesn't exceed the height of the window
			If it's the case, we crop the bar so it become equal to the window's height. */
			if (barHeight > WINDOW_HIGHT)	
				barHeight = WINDOW_HIGHT;
			/* LOOP 2 : we browse in height the vertical bar to draw it */
			
			for (j = WINDOW_HIGHT - barHeight ; j < WINDOW_HIGHT ; j++)	
			{
				/* We draw each pixel of the bar with the right colour.
				We simply vary the red and green colours, each one in a different way.
				
				j doesn't vary between 0 and 255 but between 0 and WINDOW_HEIGHT.
				
				If we want to adapt it proportionally to the window's height, we can simply calculate j / RATIO, where RATIO is equal to (WINDOW_HEIGHT / 255.0).
				
				It tooks for me 2-3 minutes so I can find the write calculation to do, every one can do it. You just have to think a little bit */
				
				setPixel(screen, i, j, SDL_MapRGB(screen->format, 255 - (j / RATIO), j / RATIO, 0));
			}
		}
		SDL_UnlockSurface(screen); /* We have finished working on the screen, we block the surface */
		SDL_Flip(screen);
	}
	/* The program is finished.
	We free the music from the memory
	And we close FMOD and SDL */
	
	FMOD_Sound_Release(music);
	FMOD_System_Close(system);
	FMOD_System_Release(system);
	SDL_Quit();
	return EXIT_SUCCESS;
}
/* The function setPixel lets us draw a surface pixel by pixel */

void setPixel(SDL_Surface *surface, int x, int y, Uint32 pixel)
{
	int bpp = surface->format->BytesPerPixel;
	Uint8 *p = (Uint8 *)surface->pixels + y * surface->pitch + x * bpp;
	switch(bpp) 
	{
		case 1:
		*p = pixel;
		break;
		case 2:
		*(Uint16 *)p = pixel;
		break;
		case 3:
		if(SDL_BYTEORDER == SDL_BIG_ENDIAN)
		{
			p[0] = (pixel >> 16) & 0xff;
			p[1] = (pixel >> 8) & 0xff;
			p[2] = pixel & 0xff;
		} 
		else
		{
			p[0] = pixel & 0xff;
			p[1] = (pixel >> 8) & 0xff;
			p[2] = (pixel >> 16) & 0xff;
		}
		break;
		case 4:
		*(Uint32 *)p = pixel;
		break;
	}
}
\end{Csource}

يجدر بك أن تتحصّل على نتيجة تشبه الصورة التالية :

\Picture{Chapter_III-9_Window-spectral}

من المعلوم أن النتيجة المرئية أفضل لتقدير النتيجة. أنصحك بالاطلاع عليها من هنا :

\textenglish{\url{https://openclassrooms.com/uploads/fr/ftp/mateo21/spectre.html} (4.3 Mo)}

لاحظ أن ضغط الملف أنقص من جودة الصوت و عدد الصور في الثانية.\\
الأفضل هو أن تقوم بتحميل البرنامج كاملاً (مرفقاً بالشفرة المصدرية) لكي تجربه عندك. يمكنك حينها تقدير البرنامج في ظروف أفضل.

تنزيل الملف التنفيذي و الشفرة المصدرية الكاملة :

\url{https://openclassrooms.com/uploads/fr/ftp/mateo21/spectre.zip}

\begin{critical}
يجب قطعاً أن يكون الملف
\InlineCode{Hype\_Home.mp3}
متواجداً في مجلّد المشروع لكي يشتغل البرنامج (و إلا فسيتوقف حالاً).
\end{critical}

\section*{أفكار للتحسين}

يمكن دائما تحسين البرنامج. هنا، لدي مثلاً أفكار تمديد كثيرة يمكنها أن تصل بك إلى إنشاء برنامج صغير لقراءة الملفات
\textenglish{MP3}.

\begin{itemize}
	\item سيكون من الجيد أن نختار بأنفسنا الملف
	\textenglish{MP3}
	الذي نريد قراءته. يمكن مثلاً أن نقدّم لائحة تضم كلّ الملفات بذات الصيغة و المتواجدة في مجلّد المشروع. لم نرَ كيف نقوم بذلك، لكن يمكنك وحدك أن تكتشف ذلك. كمساعدة : استعمل المكتبة
	\textenglish{dirent}
	( قم بتضمين الملف 
	\InlineCode{dirent.h}).
	عليك بالبحث في الأنترنت لتعرف كيفيّة العمل بها.
	\item إذا كان البرنامج قادرا على التحكّم في لائحات التشغيل
	(\textenglish{Playlists})
	المشغّلة، سيكون أمراً أفضل. توجد كثير من صيغ اللائحات و أشهرها الصيغة
	\textenglish{M3U}.
	\item يمكنك إظهار اسم الـ\textenglish{MP3}
	التي أنت بصدد تشغيله في النافذة مثلاً (يجب استعمال
	\textenglish{SDL\_ttf}).
	\item يمكنك إظهار مؤشّر يشير إلى المكان في القراءة الّذي وصل إليه التشغيل، هذا ما يفعله أغلب قارئي الـ\textenglish{MP3}.
	\item يمكنك أيضاً أن تقترح التعديل على قوة الصوت.
	\item إلى آخره.
\end{itemize}

باختصار، هناك الكثير لفعله. لديك إمكانيّة إنشاء قارئي أصوات ممتازة، ليس عليك سوى كتابة الشفرة الخاصة بها !

  \part{هياكل البيانات}
  \chapter{السلاسل المتّصلة}

لكي نخزّن المعلومات في الذاكرة، استعملنا متغيّرات بسيطة (من نوع 
\InlineCode{int}، \InlineCode{double} \dots)،
كما استعملنا جداول و هياكل مخصّصة. إذا أردت تخزين سلسلة من البيانات، فالأبسط غالباً هو استعمال جداول. 

لكن تصبح الجداول أحياناً محدودة جداً. مثلاً، إذا أنشأت جدولاً ذو 10 خانات ثم تبيّن لك لاحقاً في البرنامج أنك تحتاج إلى حجم أكبر، سيكون من المستحيل تكبير حجم الجدول. و أيضاً لا يمكنك إدخال خانة إلى وسط الجدول.

تمثّل السلاسل المتصّلة طريقة لتنظيم البيانات في الذاكرة بطريقة أكثر مرونة. و بما أن لغة 
\textenglish{C}
لا تقترح قاعدياً هذا النظام من التخزين، سيكون علينا أن ننشئه بأنفسنا. سيكون تمريناً ممتازاً يساعدك على أن ترتاح أكثر مع هذه اللغة.

  \chapter{المكدّسات و الطوابير (\textenglish{Stacks and Queues})}

لقد اكتشفنا مع القوائم المتسلسلة طريقة جديدة أكثر مرونة من الجداول لتخزين البيانات. هذه القوائم مرنة بشكل خاص لأنه يمكننا أن نُدرج فيها و نحذف منها بيانات من أي مكان أردنا و في أية لحظة.

المكدّسات و الطوابير التي سنكتشفها هنا هما شكلان خاصّان نوعاً ما من القوائم المتسلسلة. فهما تسمحان بالتحكّم بالطريقة التي تُضاف بها العناصر الجديدة إليها. هذه المرة لن نقوم بإضافة عناصر جديدة في وسط القائمة، بل فقط في البداية أو النهاية. 

المكدّسات و الطوابير تعتبران مفيدتان للغاية من أجل البرامج التي تحلل المعطيات الّتي تصل بالتدريج. سنرى بالتفصيل كيف تعملان في هذا الفصل.

تتشابه المكدّسات و الطوابير كثيراً، لكنهما تختلفان اختلافاً بسيطاً ستتعرف عليه بسرعة. سنكتشف أولاً المكدّسات و التي ستذكّرك بالقوائم المتّصلة بشكل كبير. 

بشكل عام، سيكون هذا الفصل بسيطاً إذا كنت قد فهمت جيّداً كيفية عمل القوائم المتسلسلة. إن لم تكن هذه حالتك، فأعد قراءة الفصل السابق لأننا سنحتاج إليه.

\section{المكدّسات (\textenglish{Stacks})}

تخيّل مكدّساً للقطع النقدية (الصورة التالية). يمكنك إضافة قطع أخرى واحدة تلو الأخرى في أعلى المكدّس، و يمكنك أيضاً نزع القطع من أعلى المكدّس. بالمقابل، لا يمكن نزع قطعة من أسفل المكدّس. إن أردت التجريب، أتمنى لك حظًاً موفّقاً !

\begin{figure}[H]
	\centering
	\includegraphics[height=0.2\textheight]{Chapter_IV-2_Money-stack}
\end{figure}

\subsection{كيفية عمل المكدّسات}

مبدأ عمل المكدّسات في البرمجة ينصّ على تخزين البيانات مع وصولها على التوالي واحدة فوق الأخرى لكي نستطيع استرجاعها فيما بعد. مثلا، تخيّل مكدّساً للأعداد الصحيحة من نوع 
\InlineCode{int}
(الصورة الموالية). لو أضيف عنصراً (نتكلّم عن
\textbf{التكديس})،
فستتم إضافته في أعلى المكدّس، تماماً كما في لعبة 
\textenglish{Tetris} :

\begin{figure}[H]
	\centering
	\includegraphics[height=0.2\textheight]{Chapter_IV-2_Stack}
\end{figure}
\begin{figure}[H]
	\centering
	\includegraphics[height=0.3\textheight]{Chapter_IV-2_Stack-add}
\end{figure}

الأكثر أهمية هو وجود عملية تقوم باستخراج الأعداد من المكدّس. نحن نتكلّم عن
\textbf{إلغاء التكديس}.
نسترجع القيم واحدة تلو الأخرى، بدءً من الأخيرة الموضوعة أعلى المكدّس (الصورة الموالية). ننزع البيانات على التوالي حتى نصل إلى قاع المكدّس.

\begin{figure}[H]
	\centering
	\includegraphics[height=0.3\textheight]{Chapter_IV-2_Stack-remove}
\end{figure}

نسمي هذا بخوارزمية
\textbf{\textenglish{LIFO}}،
و التي تعني
"\textenglish{Last In First Out}".
الترجمة : "آخر عُنصر تمت إضافته، هو أول عنصر يخرج".

عناصر المكدّس مرتبطة ببعضها بطريقة القوائم المتسلسلة. فهي تحمل مؤشّراً نحو العنصر الموالي و لا تتموضع بالضرورة بجنب بعضها في الذاكرة. يجب على آخر عُنصر (في أقصى أسفل المكدّس) أن يؤشّر نحو
\InlineCode{NULL}
لكي يقول أنّنا \dots لمسنا القاع :

\begin{figure}[H]
	\centering
	\includegraphics[height=0.2\textheight]{Chapter_IV-2_Stack-numbers}
\end{figure}

\begin{question}
 فيما ينفع كلّ هذا، واقعيّا ؟
\end{question}

توجد برامج تحتاج فيها إلى تخزين اليبانات مؤقّتاً لإخراجها اعتماداً على ترتيب محدد : يجب على آخر عُنصر أدخلته أن يخرج هو الأول.

لأعطي مثالاً واقعياً، يستعمل نظام التشغيل في حاسوبك هذا النوع من الخوارزميات لكي يتذكر الترتيب الذي تم استدعاء الدوال فيه. تخيّل مثالا :

\begin{enumerate}
	\item يبدأ البرنامج بالدالة
	\InlineCode{main}
	(مثل كل مرّة).
	\item تستدعي فيها الدالة 
	\InlineCode{play}.
	\item تقوم هذه الدالة 
	\InlineCode{play}
	بدورها باستدعاء الدالة
	\InlineCode{load}.
	\item ما إن تنتهي الدالة
	\InlineCode{load}،
	نعود إلى الدالة
	\InlineCode{play}.
	\item ما إن تنتهي الدالة 
	\InlineCode{play}،
	نعود إلى الدالة 
	\InlineCode{main}.
	\item أخيراً، ما إن تنتهي الدالة
	\InlineCode{main}.
	لا تبقى أية دالة تحتاج إلى الاستدعاء، ينتهي البرنامج.
\end{enumerate}

لكي "يتذكّر" الترتيب الذي تم فيه استدعاء الدوال، يُنشئ الحاسوب مكدّساً لهذه الدوال على التوالي :

\begin{figure}[H]
	\centering
	\includegraphics[width=0.6\textwidth]{Chapter_IV-2_Stack-history}
\end{figure}

هذا مثال واقعيّ عن استعمال المكدّسات. و بفضل هذه التقنية يعرف الجهاز الآن إلى أي دالة يجب عليه أن يعود. يمكنه أن يكدّس 100 استدعاء للدوال إن وجب الأمر، لكنّه سيرجع ليجد الدالة الرئيسية في أسفل المكدّس !

\subsection{إنشاء نظام مكدّس}

و الآن بما أننا فهمنا مبدأ عمل المكدّسات، فلنحاول بناء واحد. تماما مثل القوائم المتسلسلة، لا يوجد نظام مكدّس متضمّن في لغة الـ\textenglish{C}.
يجب أن ننشئه بأنفسنا.

سيكون لكل عنصر من المكدّس هيكل مماثل للهيكل الخاص بالقائمة المتسلسلة :

\begin{Csource}
typedef struct Element Element;
struct Element
{
	int number;
	Element *next;
};
\end{Csource}

يحتوي هيكل التحكّم على عنوان أول عنصر من المكدس، أي العنصر المتواجد في الأعلى :

\begin{Csource}
typedef struct Stack Stack;
struct Stack
{
	Element *first;
};
\end{Csource}

سنحتاج ككل إلى الدوال التالية :

\begin{itemize}
	\item تكديس  عنصر.
	\item إلغاء تكديس عنصر.
\end{itemize}

ستلاحظ أنه، على خلاف القوائم المتسلسلة، لا نتكلّم لا عن الإضافة و لا عن الحذف. نتكلّم عن التكديس و إلغاء التكديس. لأن هاتين العمليتين محدودتان على عنصر واحد محدد، كما رأينا. بهذا، يمكننا إضافة و نزع عنصر من المكدّس من الأعلى فقط.

يمكننا أيضاً كتابة دالة لإظهار محتوى المكدّس، أمر عمليّ للتأكد من أن البرنامج يعمل بشكل صحيح.

هيا بنا  !

\subsubsection{التكديس (\textenglish{stacking})}

يجدر بدالتنا
\InlineCode{stack}
أن تأخذ كمعاملين هيكل التحكّم في المكدّس (من نوع
\InlineCode{Stack})
و العدد الجديد لتخزينه.\\
أذكّرك بأننا نخزّن هنا أعداداً صحيحة
\InlineCode{int}،
لكن لا شيء يمنعنا من تعديل هذه الأمثلة لأنواع أخرى من البيانات. يمكننا تخزين أي شيء : أعداد
\InlineCode{double}،
محارف
\InlineCode{char}،
سلاسل محارف، جداول و حتى هياكل أخرى !

\begin{Csource}
void stack(Stack *stk, int newNumber)
{
	Element *new = malloc(sizeof(*new));
	if (stk == NULL || new == NULL)
	{
		exit(EXIT_FAILURE);
	}
	new->number = newNumber;
	new->next = stk->first;
	stk->first = new;
}
\end{Csource}

تتم الإضافة في بداية المكدّس لأنه، كما رأينا، يستحيل القيام بذلك في المنتصف. هذا مبدأ عمل المكدّسات، نضيف دائما من الأعلى. \\
بهذا، على عكس القوائم المتسلسلة، لا يجب أن ننشئ دالة لإدراج عنصر في منتصف المكدّس. يجب أن تكون الدالة
\InlineCode{stack}
هي الوحيدة الّتي يمكنها إضافة عناصر جديدة للمكدس.

\subsubsection{إلغاء التكديس (\textenglish{unstacking})}

دور دالة إلغاء التكديس هو حذف العنصر المتواجد في أعلى المكدّس، قد تشك في ذلك. لكن يجب على هذه الدالة أيضا أن تُرجع إلينا العنصر الذي حذفته. في حالتنا، هو العدد الّذي كان موجودا في أعلى المكدّس.

هذه هي الطريقة التي نصل بها إلى عناصر المكدّس : ننزعها واحداً تلو الآخر. نحن لا نتقدّم فيها باحثين عن الوصول إلى ثاني و ثالث عنصر. بل نطلب دائماً استرجاع على أول عنصر.

دالتنا
\InlineCode{unstack}
ستُرجع إذا
\InlineCode{int}
يوافق العدد المتواجد في رأس المكدّس  :

\begin{Csource}
int unstack(Stack *stack)
{
	if (stack == NULL)
	{
		exit(EXIT_FAILURE);
	}
	int unstackedNumber = 0;
	Element *unstackedElement = stack->first;
	if (stack != NULL§\footnotemark§ && stack->first != NULL)
	{
		unstackedNumber = unstackedElement->number;
		stack->first = unstackedElement->next;
		free(unstackedElement);
	}
	return unstackedNumber;
}
\end{Csource}

\begin{tcolorbox}[title={\footnotemark[1]ملاحظة مُرَاجِع الكتاب}, colback=orange!20, colframe=orange!70, fontupper=\small, coltitle=white, fonttitle=\normalsize, attach title]
يبدو لي أن مؤلّف الكتاب قد أضاف جزءً عديم الفائدة في الشرط الثاني :
\InlineCode{stack != NULL}،
فَبِوُصول البرنامج إلى ذلك السطر سيكون هذا الجزء من الشرط خاطئا بكلّ تأكيد لأنّنا قد تحقّقنا من عكسه في الشرط الأوّل 
(\InlineCode{if(stack == NULL)}).
و حتى عند عدم وجود الشرط الأوّل، فالتعليمة
\InlineCode{stack->first}
كانت ستعطّل البرنامج لأنّ مؤشّراً
\InlineCode{NULL}
لا يملك أيّة مركّبات !
\end{tcolorbox}

نسترجع العدد الذي في رأس المكدّس لنبعثه في نهاية الدالة. نعدّل عنوان أول عنصر من المكدّس بما أن هذا الأخير يتغير.
أخيراً، نحذف بالتأكيد رأس المكدّس القديم باستعمال
\InlineCode{free}.

\subsubsection{إظهار محتوى المكدّس}

بالرغم من أن هذه الدالة غير ضروريّة (الدالتان
\InlineCode{stack}
و 
\InlineCode{unstack}
كافيتان للتحكّم في المكدّس !)، ستكون مهمّة لاختبار عمل المكدّس و خاصّة "لمعاينة" النتيجة :

\begin{Csource}
void displayStack(Stack *stack)
{
	if (stack == NULL)
	{
		exit(EXIT_FAILURE);
	}
	Element *current = stack->first;
	while (current != NULL)
	{
		printf("%d\n", current->number);
		current = current ->next;
	}
	printf("\n");
}
\end{Csource}

بما أن هذه الدالة بسيطة بسخافة، فهي لا تحتاج شرحاً.

بالمقابل، حان الآن إذاً لكتابة الدالة الرئيسية لاختبار سلوك مكدّسنا :

\begin{Csource}
int main()
{
	Stack *myStack = initialization(); // See the previous chapter
	stack(myStack, 4);
	stack(myStack, 8);
	stack(myStack, 15);
	stack(myStack, 16);
	stack(myStack, 23);
	stack(myStack, 42);
	printf("\nStack's state :\n");
	displayStack(myStack);
	printf("I unstack %d\n", unstack(myStack));
	printf("I unstack %d\n", stack(myStack));
	printf("\nStack's state :\n");
	displayStack(myStack);
	return 0;
}
\end{Csource}

نُظهر حالة المكدّس بعد الكثير من التكديس و مرة أخرى بعد كثير من إلغاء التكديس. نُظهر أيضاً العدد الذي قُمنا بحذفه في كلّ مرة نقوم بإلغاء التكديس. النتيجة في الكونسول هي التالية :

\begin{Console}
Stack's state :
42
23
16
15
8
4

I unstack 42
I unstack 23

Stack's state :
16
15
8
4
\end{Console}

تأكّد أنك تفهم جيّداً ما يحصل في البرنامج. إذا فهمت هذا، فقد فهمت كيفية عمل المكدّسات !

يمكنك تنزيل مشروع المكدّسات كاملاً لو أردت :

\url{http://www.siteduzero.com/uploads/fr/ftp/mateo21/c/piles.zip}

\section{الطوابير (\textenglish{Queues})}

تشبه الطوابير المكدّسات كثيراً، إلا أنها تعمل بالإتجاه المعاكس !

\subsection{كيفية عمل الطوابير}

في هذا النظام، يتم وضع العناصر الواحد بعدَ الآخر. أوّل عنصر يخرج من الطابور هو أول عنصر يدخل إليه. نتكلّم هنا عن خوارزمية 
\textenglish{FIFO} (\textenglish{First In First Out})،
و هذا يعني : "أول من يصل هو أول من يخرج".

تسهل المشابهة بالحياة اليوميّة. حينما تشتري تذكرة لمشاهدة السينما، تقف في طابور شبّاك التذاكر (الصورة الموالية). باستثناء إن كنت أحد إخوة بائع التذاكر، يجدر بك الوقوف في الطابور و انتظار دورك مثل كلّ الآخرين. أول الواصلين هو أول من تتم خدمته.

\begin{figure}[H]
	\centering
	\includegraphics[width=0.8\textwidth]{Chapter_IV-2_Humans-queue}
\end{figure}

 في البرمجة، تفيد الطوابير في إيقاف مؤقّت للمعلومات حسب الترتيب الذي وصلت به. مثلاً، في برنامج مُحادثة، لو تتلقى ثلاثة رسائل يفصلها فارق زمني قصير جداً، يتم صفُّها في طابور واحدة تلو الأخرى في الذاكرة. ثم يتم إظهار أوّل رسالة وصلت ثم الثانية و هكذا.
 
يتم تخزين الأحداث التي تبعثها المكتبة
\textenglish{SDL}
التي قُمنا بدراستها أيضا في طابور. إذا حرّكت الفأرة، يتم توليد حدث من أجل كلّ بيكسل تحرّك فوقه مؤشّر الفأرة. تخزن الـ\textenglish{SDL}
الأحداث في طابور ثم تبعثها لنا واحدا واحدا في كلّ مرة نستدعي فيها الدالة
\InlineCode{SDL\_PollEvent}
(أو
\InlineCode{SDL\_WaitEvent}).

في لغة الـ\textenglish{C}.
الطابور هو قائمة متسلسلة أين يقوم كلّ عنصر فيها بالتأشير على العنصر الموالي، تماماً مثل المكدّسات. آخر عنصر من الطابور يؤشّر نحو
\InlineCode{NULL} :

\begin{figure}[H]
	\centering
	\includegraphics[width=0.5\textwidth]{Chapter_IV-2_Queue}
\end{figure}

\subsection{إنشاء نظام طابور}

نظام الطابور يشبه ذلك الخاص بالمكدّسات. يوجد اختلاف بسيط في كون أن العناصر تخرج في الإتجاه المعاكس، لا يوجد شيء صعب إن كنت قد فهمت المكدّسات.

سننشئ هيكل
\InlineCode{Element}
و هيكل تحكّم
\InlineCode{Queue} :

\begin{Csource}
typedef struct Element Element;
struct Element
{
	int number;
	Element *next;
};
typedef struct Queue Queue;
struct Queue
{
	Element *first;
};
\end{Csource}

تماما كالمكدّسات، كل عنصر من الطابور سيكون من نوع
\InlineCode{Element}.
بالإستعانة بالمؤشّر 
\InlineCode{first}
سنتوفّر دائماً على العنصر الأول و يمكننا من خلاله الصعود إلى آخر عنصر.

\subsubsection{إضافة عنصر إلى الطابور (\textenglish{enqueuing})}

الدالة التي تضيف عُنصُراً إلى الطابور تسمّى دالة "الإلحاق"
(\textenglish{enqueuing}).
توجد حالتان :

\begin{itemize}
	\item إما أن الطابور فارغ، في هذه الحالة يجب أن ننشئ الطابور بجعل المؤشّر
	\InlineCode{first}
	يؤشّر نحو العنصر الجديد الذي نحن بصدد انشائه.
	\item إما أن الطابور غير فارغ، في هذه الحالة يجب أن نتقدّم في الطابور إنطلاقاً من العنصر الأول حتى نصل إلى آخر عنصر. نضيف العنصر الجديد بعد آخر عنصر.
\end{itemize}
إليك ما يمكننا فعله عملياً :

\begin{Csource}
void enqueue(Queue *q, int newNumber)
{
	Element *new = malloc(sizeof(*new));
	if (q == NULL || new == NULL)
	{
		exit(EXIT_FAILURE);
	}
	new->number = newNumber;
	new->next = NULL;
	if (q->first != NULL) // The queue is not empty
	{
		// We move to the end of the queue
		Element *currentElement = q->first;
		while (currentElement->next != NULL)
		{
			currentElement = currentElement->next;
		}
		currentElement->next = new;
	}
	else /* The queue is empty, it's our first element */
	{
		q->first = new;
	}
}
\end{Csource}

ترى في هذه الشفرة المصدرية تحليل كلتا الحالتين، كلّ منهما يجب أن تتم دراستها على حدى. الاختلاف مقارنةً بالمكدّسات، و الذي يضيف لمسة صعوبة صغيرة، هو أنه يجب التموضع في نهاية الطابور لوضع العنصر الجديد. لكن لابأس فحلقة
\InlineCode{while}
كافية للقيام باللازم، هذا ما يمكنك ملاحظته.

\subsubsection{إزالة عنصر من الطابور (\textenglish{dequeuing})}

عملية إلغاء الإلحاق 
(\textenglish{dequeuing})
تشابه كثيراً عملية إلغاء التكديس. بامتلاكنا مؤشّرا نحو أول عنصر من الطابور، يكفي أن ننزعه و أن نُرجع قيمته.

\begin{Csource}
int dequeue(Queue *queue)
{
	if (queue == NULL)
	{
		exit(EXIT_FAILURE);
	}
	int dequeuedNumber = 0;
	// We verify if there's something to dequeue
	if (queue->first != NULL)
	{
		Element *dequeuedElement = queue->first;
		dequeuedNumber = dequeuedElement->number;
		queue->first = dequeuedElement->next;
		free(dequeuedElement);
	}
	return dequeuedNumber;
}
\end{Csource}

\subsubsection{حان دورك !}

تبقى دالة إظهار محتوى الطابور
\InlineCode{displayQueue}
و عملها مشابه لما قمنا به مع المكدّسات. سيسمح لك هذا بالتأكد من سلامة عمل الطابور.

قم بعد ذلك بكتابة
\InlineCode{main}
من أجل تجريب البرنامج. يجدر بنتيجة البرنامج أن تشبه هذه :

\begin{Console}
Queue's state :
4 8 15 16 23 42

I dequeue 4
I dequeue 8

Queue's state :
15 16 23 42
\end{Console}

يجدر أن تكون قادراً على إنشاء مكتبة الطوابير الخاصة بك، بملفات
\InlineCode{queue.h}
و
\InlineCode{queue.c}
مثلاً.

أقترح عليك تنزيل مشروع التحكّم في الطوابير كاملاً إن أردت. إنه يتضمّن الدالة
\InlineCode{displayQueue} :

\url{http://www.siteduzero.com/uploads/fr/ftp/mateo21/c/files.zip}

\section*{ملخّص}

\begin{itemize}
	\item تسمح المكدّسات و الطوابير بتنظيم معطيات في الذاكرة عند وصولها بالتوالي.
	\item تستعمل المكدّسات و الطوابير نظام قوائم متسلسلة لتجميع العناصر.
	\item في حالة المكدّسات، تتم إضافة المعطيات الواحدة فوق الأخرى. و إن أردنا استخراج بيانة، فسنستخرج آخر واحدة و التي كنا بصدد إضافتها (الأحدث). نتكّلم هنا عن خوارزمية 
	\textenglish{LIFO} (\textenglish{Last In First Out}).
	\item في حالة الطوابير، تتم إضافة المعطيات الواحدة بعد الأخرى. نقوم باستخراج البيانة الأولى و التي قمنا بإضافتها أولا للطابور (الأقدم). نتكلّم عن خوارزمية
	\textenglish{FIFO} (\textenglish{First In First Out}).
\end{itemize}

  \chapter{جداول التجزئة}

للقوائم المتسلسلة نقطة ضعف كبيرة في حال أردنا قراءة محتواها : يستحيل الوصول إلى عنصر معيّن مباشرة. يجب التقدّم في القائمة عنصراً بعنصر حتى نجد العنصر الذي نريد. هذا يطرح مشاكل من ناحية الأداء ما إن يكون حجم القائمة المتسلسلة ضخماً. تخيّل قائمة متسلسلة تتكوّن من 1000 عنصر بينما العنصر الذي نبحث عنه موجود في آخرها !

تمثّل جداول التجزئة طريقة أخرى لتخزين البيانات. حيث أنها تستند على مبدأ الجداول في لغة الـ\textenglish{C}
و التي نعرف التعامل معها جيّداً. ماهي فائدتها الكُبرى ؟ هي تسمح بإيجاد سريع لعنصر محدد، سواء كان الجدول يحتوي 100، 1000، 10000 خانة أو حتى أكثر !

  \chapter*{خاتمة}

هل تريد
\textit{المزيد}
؟

لماذا لا تتعلّم لغة الـ\textenglish{C++}
؟

\url{http://www.siteduzero.com/tuto-3-5395-0-apprenez-a-programmer-en-c.html}

 هذا درس آخر كتبتُه حول هذه اللغة قريبة الـ\textenglish{C}.
 إذا كنت تعرف الـ\textenglish{C}،
فلن تكون ضائعاً بل ستفهم بسرعة فائقة الفصول الأولى !\\
فليكن في علمك أنني كتبت درساً قصيرا يسمّى "من الـ\textenglish{C}
إلى الـ\textenglish{C++}"
الذي يبيّن جزءً من الاختلاف بين الـ\textenglish{C}
و الـ\textenglish{C++}.

\url{http://www.siteduzero.com/tutoriel-3-430167-du-c-au-c.html}

بلغة الـ\textenglish{C++}،
يمكنك البدء في البرمجة غرضية التوجّه (أو البرمجة الكائنية) (\textenglish{OOP}).
قد يكون هذا المبدأ معقّدا قليلا في البداية، لكن ستجد بأن هذه الطريقة في البرمجة ناجعة جداً ! ستكتشف أيضاً معها المكتبة
\textenglish{Qt}
التي تسمح بإنجاز واجهات رسومية كاملة جدّا.

أشكر كثيرا
\href{http://www.siteduzero.com/membres-294-45753.html}{Taurre}
و
\href{http://www.siteduzero.com/membres-294-181268.html}{Pouet\_forever}
لمساعدتهم الكبيرة في القيام بالمراجعات الأخيرة لهذه الدروس.

\end{document}
