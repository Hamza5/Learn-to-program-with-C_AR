\documentclass[11pt, a4paper]{book}
\usepackage[svgnames,table]{xcolor} % For color names
\usepackage{fontspec} % font selecting commands
\usepackage{tcolorbox} % For colored boxes
\usepackage{graphicx} % For pictures
\usepackage[export]{adjustbox}
\usepackage{polyglossia} % Babel alternative in XeLaTex
\usepackage{fancyhdr} % For page headers
\usepackage[top=2.5cm, bottom=2.5cm, left=2cm, right=3cm]{geometry} % To control page margins
\usepackage{listings} % For source code
\usepackage[colorlinks=true,
            urlcolor=blue,
            unicode=true,
            pdftitle={تعلّم البرمجة بلغة الـC},
            pdfauthor={عدن بلواضح,حمزة عباد,أحمد زبوشي}
            pdfdisplaydoctitle=true]{hyperref} % For hyperlinks and PDF metadata
\usepackage{float}
\usepackage{tabu,booktabs}
\usepackage{bidi}
% Language settings
\setmainlanguage[locale=algeria]{arabic}
\setotherlanguage{english}
\addto\captionsarabic{ % Without this, changes won't take effect, because of Polyglossia
  \renewcommand{\partname}{الجزء}
  \renewcommand{\chaptername}{الفصل}
}
% Font settings
\defaultfontfeatures{Ligatures=TeX}
\newfontfamily\arabicfont[Script=Arabic, Scale=1.2]{Amiri} % An arabic font
\newfontfamily\englishfont[Script=Latin]{Arial} % Font used for latin text in the document
\newfontfamily\arabicfonttt{Courier New} % Monospace font, for displaying codes
% Boxes definitions
\tcbset{boxrule=0mm, arc=2pt}
\newtcolorbox{question}{colback=blue!70!green!10, colframe=blue!80!green!5, fontupper=\itshape} % Used for question boxes
\newtcolorbox{critical}{colback=red!20, colframe=red!50} % Used for critical warning boxes
\newtcolorbox{warning}{colback=yellow!20, colframe=yellow!50} % Used for warning boxes
\newtcolorbox{information}{colback=blue!5!green!20, colframe=blue!10!green} % Used for information boxes
\newcommand\InlineCode[1]{\fcolorbox{LightGray}{Snow}{\ttfamily \LR{#1}}}
% Titles settings (Make them orange)
\makeatletter
\let\oldchapter\chapter
\newcommand{\@chapterstar}[1]{\cleardoublepage\phantomsection\addcontentsline{toc}{chapter}{#1}{\color{green!30!blue!80}\oldchapter*{#1}}}
\newcommand{\@chapternostar}[1]{{\color{green!30!blue!80}\oldchapter{#1}}}
\renewcommand{\chapter}{\@ifstar{\@chapterstar}{\@chapternostar}}
\let\oldpart\part
\newcommand{\@partstar}[1]{\cleardoublepage\phantomsection\addcontentsline{toc}{part}{#1}{\color{orange}\oldpart*{#1}}}
\newcommand{\@partnostar}[1]{{\color{orange}\oldpart{#1}}}
\renewcommand{\part}{\setcounter{chapter}{0}\@ifstar{\@partstar}{\@partnostar}}
\let\oldsection\section
\newcommand{\@sectionstar}[1]{\phantomsection\addcontentsline{toc}{section}{#1}{\color{orange}\oldsection*{#1}}}
\newcommand{\@sectionnostar}[1]{{\color{orange}\oldsection{#1}}}
\renewcommand\section{\@ifstar{\@sectionstar}{\@sectionnostar}}
\makeatother
% Pictures settings
\graphicspath{{Pictures/}} % Folder of pictures
\newcommand\Picture[2][]{ % This command automatically centers the picture and fits its size to the page. It supports captions too.
  \begin{center}
    \includegraphics[max size={0.8\textwidth}{0.5\textheight}]{#2}\\
    #1
  \end{center}
}
% Paragraphs settings
\setlength{\parskip}{5mm plus2mm minus2mm} % Spacing between paragraphs (+/-)
% Page header and footer settings
\setlength{\headheight}{15pt}
\pagestyle{fancy}
\renewcommand{\chaptermark}[1]{ \markboth{{\thechapter\ #1}}{} }
\renewcommand{\sectionmark}[1]{ \markright{\thesection\ #1} }
\fancyhead{}
\fancyhead[OR,EL]{\rightmark}
\fancyhead[ER,OL]{\leftmark}
\setlength{\footskip}{1.5cm}
% Fixing the issues of the numbering
\renewcommand{\thepart}{\Alph{part}}
\renewcommand{\thechapter}{\Alph{part}.\arabic{chapter}}
\renewcommand{\thesection}{\Alph{part}.\arabic{section}.\arabic{chapter}}
\renewcommand{\thesubsection}{\Alph{part}.\arabic{subsection}.\arabic{section}.\arabic{chapter}}
\renewcommand{\thesubsubsection}{\Alph{part}.\arabic{subsubsection}.\arabic{subsection}.\arabic{section}.\arabic{chapter}}
% Global settings for code and console
\lstset{frame=single, basicstyle=\ttfamily, breaklines=true, showlines, aboveskip=\parskip}
% C source code
\lstdefinestyle{C}{language=C, showstringspaces=false, numbers=left,
        keywordstyle=\bfseries\color{Cyan}, commentstyle=\itshape\color{Gray},
        numberstyle=\color{Gray}, stringstyle=\color{Crimson},
        directivestyle=\color{DarkOrange},
        deletekeywords={return,if,else,switch,for,while,do,const,static},
        morekeywords=[2]{return,if,else,switch,for,while,do,const,static}, keywordstyle=[2]\bfseries\color{Magenta},
        morekeywords=[3]{printf,scanf}, keywordstyle=[3]\color{Cyan},
}
\lstnewenvironment{Csource}{\lstset{style=C}\setLTR}{\unsetLTR}
% Console
\lstnewenvironment{Console}{\setLTR}{\unsetLTR}
% Table settings
\setlength{\tabulinesep}{2pt}
\setlength{\arrayrulewidth}{2pt}
\taburulecolor{White}
\newenvironment{Table}[1]{ % Accepts 1 parameter which is the number of columns
\taburowcolors[2] 2{LightGray!40 .. LightGray!80}
\begin{center}
  \begin{tabu}{*{#1}{|r}|}
    \toprule
    \rowfont{\bfseries\color{White}}
    \rowcolor{OrangeRed}
    \everyrow{\hline}
}{
  \end{tabu}
\end{center}
}
\newenvironment{Table*}[1]{
\taburowcolors[1] 2{LightGray!40 .. LightGray!80}
\begin{center}
  \begin{tabu}{*{#1}{|r}|}
    \toprule
    \everyrow{\hline}
}{
  \end{tabu}
\end{center}
}

\title{تعلّم البرمجة بلغة الـC}
\begin{document}
  \chapter*{تقديم}
إن التحرّر الفكري في بداية القرن العشرين أدّى إلى توسّع في البحوث العلمية التي شملت كل الميادين لاسيّما التكنولوجية منها كعلوم الحاسوب. هذه الأخيرة أعقبتها ثورة في لغات البرمجة التي تعتبر ركيزة أساسية تقوم عليها البرامج. من بين هذه اللغات نجد لغة الـ\textenglish{C}،
إذ تعتبر من أقوى لغات البرمجة و أكثرها شيوعاً، فهي مستلهمة من طرف لغتي
 \textenglish{B}
 و
 \textenglish{BCPL}
حيث تمّ تطويرها في عام 1972 من طرف
\textenglish{Ken Thompson}
و
 \textenglish{Dennis Ritchie}،
و في ظرف سنة واحدة توسّعت لتكون عِـماد نظام التشغيل
\textenglish{UNIX}
بنسبة
90\%
ثم تم توزيعها في العام المـُوالي رسمياً عبر الجامعات لتصبح بذلك لغة برمجة عالمية. و اشتهرت لغة الـ\textenglish{C}
 كونـُها لغة عالية المستوى، لها مُترجم سريع و فعّال. كما أنها لغة برمجية نقّالة، هذا يعني أن أي برنامج يحترم المعيار
\textenglish{AINSI}
يمكن أن يتمّ تشغيله على أيّة منصّة تحتوي على مترجم
\textenglish{C}
 دون أيّة تخصيصات.

يعتبر هذا الكتاب بوابة سهلة لكلّ مبتدئ لتعلّم لغة الـ\textenglish{C}
خطوة بخطوة بدءً من الأساسيات وصولاً إلى تطوير ألعاب ثنائية الأبعاد و التحكّم في هياكل البيانات الأكثر تعقيداً. الكتاب مرفق بجملة من التمارين و الأعمال التطبيقية المحلولة التي تساعد على هضم المفاهيم المكتسبة و تطبيقها على أيّ مشكل برمجي مهما كان نوعه. و لأن الكثير من لغات البرمجة تعتمد أساساً على الـ\textenglish{C}
كالـ\textenglish{Java}
و الـ\textenglish{C++}
و الـ\textenglish{C\#}
(لغات برمجية غرضية التوجّه) و حتى
\textenglish{PHP}
(لغة لبرمجة المواقع) فإن تعلّم لغة الـ\textenglish{C}
 سيساعد على تعلّم أيّة لغة برمجية كانت. تبقى الإرادة و حبّ العمل و الشغف المفاتيح الرئيسية للنجاح و الوصول إلى الاحترافية.

\vfill

\hfill\parbox{0.3\textwidth}{\centering
عدن بلواضح

\vspace{1em}
الجزائر\\[0.5em]
في
24 ذو القعدة 1438\\[0.3em]
الموافق لـ17 أوت 2017
%\Hijritoday\\[0.3em]
%الموافق لـ\today

}


  \chapter*{مقدّمة}

\vspace{-0.6em}
تحبّ تعلّم البرمجة لكن لا تعرف من أين تبدأ ؟ هذه الدروس لتعليم لغة الـ\textenglish{C}
للمبتدئين قد جُعلت خصّيصاً من أجلك !

\vspace{-0.1em}
لغة الـ\textenglish{C}
هي لغة لا مفرّ منها، أُستلهمَت منها العديد من اللغات الأخرى. تمّ اختراعها في السبعينات و لا تزال مستعملة لحدّ الآن في البرمجة النظامية و عالم الروبوتات. تعتبر لغة الـ\textenglish{C}
لغة معقّدة، لكن إن استطعت تعلّمها ستكوّن لك قاعدة برمجية صلبة !

\vspace{-0.1em}
في هذه الدروس، ستبدأ باكتشاف مبدأ عمل الذاكرة، المتغيرات، الشروط و الحلقات. ثم ستقوم باستعمال كلّ ما تعلّمته في إنشاء واجهات رسومية بالاستعانة بالمكتبة
\textenglish{SDL}
 (ألعاب فيديو، تسجيلات صوتية \dots). أخيراً، ستتعلّم كيف تتعامل مع هياكل البيانات الأكثر شيوعاً من أجل تنظيم المعلومات في الذاكرة : قوائم متسلسلة، مكدّسات، طوابير، جداول تجزئة \dots

\vspace{-0.1em}
التحق بي في هذه الدروس من أجل اكتشاف البرمجة بلغة الـ\textenglish{C} !

\begin{figure}[H]
	\centering
	\includegraphics[height=0.3\textheight]{Introduction_original}\\
\small بعض الإنجازات الّتي سنقوم بها في هذا الكتاب
\end{figure}

\vfill
\hfill\parbox{0.3\textwidth}{\centering \textenglish{Mathieu Nebra}\\[0.2em]
مؤسس مشارك لموقع
\href{http://openclassrooms.com/}{\textenglish{OpenClassrooms}}
}

  \part{أساسيّات البرمجة بلغة الـC}
  \chapter{قلت برمجة ؟}
\section{ما هي البرمجة؟}
\begin{question}
  ما الذي تعنيه كلمة "بَرْمَجَ"؟
\end{question}

لن أتعبك وأعطيك أصل كلمة "بَرْمَجَ"، لكنني سأختصر كل شيء في جملة: البرمجة تعني إنشاء برامج حاسوب. وهذه البرامج التي تنشئها تأمر الجهاز بالقيام بتعليمات وأفعال معيّنة.
حاسوبك الخاص يحتوي على كثير من هذه البرامج وبمختلف أنواعها:

\begin{itemize}
  \item الآلة الحاسبة تعتبر برنامجاً.
  \item معالج النصوص يعتبر برنامجاً أيضاً.
  \item وكذلك برنامج المحادثة.
  \item ألعاب الفيديو هي برامج كذلك.
\end{itemize}

\Picture[\caption{نسخة عن لعبة \textenglish{MetalSlug} الشهيرة تم إنشاؤها من طرف العضو \href{http://www.siteduzero.com/membres-294-176405.html}{\textenglish{joe87}}}]{Chapter_I-1_MetalSlug}
باختصار البرامج موجودة في كل جهاز، وهي التي تعطي الحاسوب قدرته على إنجاز مختلف المهام التي تُخوَّل إليه. يمكنك أن تنشئ برنامج تشفير أو لعبة ثنائية / ثلاثية الأبعاد باستخدام لغة برمجة مثل \textenglish{C}.

ملاحظة: لم أقل أن إنشاء لعبة يتم برمشة عين، لقد قلت فقط بأنه شيء ممكن، لكن كن متأكداً، سوف يتطلب ذلك جهدا كبيراً!

وبما أننا في بداية الطريق، فإّننا لن نقوم بإنشاء لعبة ثلاثية الأبعاد! لكنّنا سنبدأ بكيفية عرض نص على الشاشة، طبعا ستقول ما علاقة هذا بإنشاء الألعاب؟ لكن ثِق بي، هذا الأمر ليس بسيطا كما يبدو!

بالطبع هذا ليس شيئا مُبهراَ، ولكن يجب علينا أن نبدأ من هنا؛ وشيئا فشيئا يمكنك أن تنشئ برامج معقّدة أكثر. فالهدف من هذا الدرس هو أن أعرفك على كل ما يتعلق بهذه اللغة.

\section{البرمجة، بأي لغة يا ترى؟}
حاسوبك هو آلة غريبة جداً، هذا أقل ما يمكن أن نقوله عنه. يمكننا أن نخاطبه فقط بالصفر والواحد، فمثلا إذا طلبنا منه حساب 3+5 فيمكن لهذا أن يعطينا نتيجة كالتالي (هذه ليست ترجمة دقيقة ولكنها تشبه ما يحدث بالفعل):
\InlineCode{0010110110010011010011110}

ما تَرَاه هنا يسمى اللغة الثنائية
(\textenglish{Binary language})
أو لغة الآلة
(\textenglish{Machine language})،
وحاسوبك لا يفهم سوى هذه اللغة، وكما تلاحظ، هذه اللغة غير مفهومة على الإطلاق!

مشكلتنا الآن:
\begin{question}
  كيف يمكننا التعامل مع حاسوب لا يفهم سوى اللغة الثنائية؟
\end{question}

حاسوبك لا يتحدث الإنجليزية، ولا العربية، ولا أي لغة غير هذه اللغة، ولكنها صعبة جدا لدرجة أن حتى أكبر خبراء الحاسوب لا يستخدمونها.
لهذا قام بعض مهندسي الحواسيب باختراع لغات يمكن أن تُتَرجَمَ إلى اللغة الثنائية، لكن الشيء الأصعب هو إنشاء البرامج الّتي تقوم بهذه الترجمة. ولحسن الحظ فقد قاموا بهذا العمل نيابة عنا. هذه البرامج تقوم بترجمة الأوامر الّتي تكتبها (مثلا: "أُحسب 3+5") إلى شيء يشبه هذا:
\InlineCode{0010110110010011010011110}.

هذا المخطط يلخص ما كنت أشرح:

\Picture{Chapter_I-1_Translation}

\section{قليل من المفردات}
حتّى الآن كنت أتحدّث إليك بكلمات بسيطة، لكن يجب أن تعلم أنه في المعلوماتية توجد مصطلحات علمية لكل ما ذكرت. طوال هذا الدرس، سوف تتعلم استخدام المفردات المناسبة. هذا سيفيدك كثيرا خصوصا عندما تتحدث مع مبرمجين آخرين، حيث أنك سوف تتفاهم معهم بكل سهولة.

نعود إلى الحديث عن المخطط السابق في المستطيل الأول قلت أن "برنامجك مكتوب بلغة مُبَسَّطة"، في الواقع هذا النوع من اللغات يُعرف باسم لغات البرمجة عالية المستوى (\textenglish{High-level programming languages}). هناك مستويات عديدة من لغات البرمجة، وكلما كان مستوى اللغة أعلى كانت أقرب إلى اللغة الحقيقية وكان استخدامها أسهل. إذن، اللغات عالية المستوى سهلة الاستخدام لكنها تتضمن بعض السلبيّات سوف نتعرّف عليها لاحقا.

توجد العديد من لغات البرمجة، وهي متفاوتة المستوى، منها:
\begin{itemize}
  \item \textenglish{C}
  \item \textenglish{C++}
  \item \textenglish{Java}
  \item \textenglish{Visual Basic}
  \item \textenglish{Delphi}
  \item و العديد غيرها
\end{itemize}

كما تلاحظ، لم أرتبها حسب مستوياتها، لذلك لا تعتقد أن اللغة الأولى في القائمة هي الأسهل أو العكس. عموما، لائحة اللغات الموجودة طويلة جدا لدرجة أنه لا يمكنني كتابتها كلها هنا.

مصطلح آخر يجب تذكّره هو
\underline{الشفرة المصدرية}
(\textenglish{Source code})،
 وهي ببساطة الشفرة الخاصة ببرنامجك الذي تكتبه بلغة عالية المستوى والذي يتم ترجمته فيما بعد إلى اللغة الثنائية.

 ثم يأتي دور البرنامج الذي يحوّل هذه اللغة عالية المستوى إلى اللغة الثنائية، هذا النوع من البرامج يعرف باسم
 \underline{المترجم}
  أو
  \underline{المصنّف}،
 والعملية الّتي يقوم بها تسمى
 \underline{الترجمة}
 أو
 \underline{التصنيف}.

\begin{information}
  يوجد لكل لغة عالية المستوى مترجم خاص، وهذا شيء منطقي، فاللغات مختلفة فيما بينها، فلا يمكننا ترجمة لغة
\textenglish{C}
بنفس الطريقة الّتي نترجم بها
\textenglish{Delphi}
مثلا.
  بعض اللغات مثل
\textenglish{C}
تملك العديد من المترجمات، فمنها من هو مكتوب من طرف
\textenglish{Microsoft}
، و منها من
\textenglish{GNU}
، إلخ… سوف نتعرّف على كل هذا في الدرس القادم.
  لحسن الحظ، هذه المترجمات متطابقة تقريبا (رغم وجود اختلافات طفيفة بينها سوف نتعرف عليها لاحقا).
\end{information}

أخيرا، البرنامج الثنائي المنشئ بواسطة المترجم يسمى الملف
\underline{القابل للتنفيذ}
أو
\underline{التنفيذي}
(\textenglish{Executable}).
 لهذا السبب تملك البرامج
 (على الأقل برامج
 \textenglish{Windows})
 الامتداد
\textenglish{.exe}
 والذي هو اختصار كلمة
 \textenglish{EXEcutable}.

 نعود إلى مخططنا السابق، وهذه المرة سنستخدم المصطلحات الصحيحة:

 \Picture{Chapter_I-1_Compilation}

 \section{لماذا نختار تعلّم \textenglish{C}؟}
 كما قلت سابقا، يوجد كثير من اللغات عالية المستوى، فلماذا ينبغي علينا أن نبدأ بإحداها على وجه الخصوص؟ سؤال عظيم!

على أية حال يجب علينا أن نختار بأي لغة سنبدأ البرمجة عاجلا أم آجلا، وبالتالي لديك الخيار في البدء بـ:
\begin{itemize}
  \item \textbf{لغة ذات مستوى عالي جدّا}:
 وتكون سهلة جدّا أوعامة، نذكر من بينها
 \textenglish{Python}، \textenglish{Ruby}، \textenglish{Visual Basic}،
 وغيرها. هذه اللغات تسمح بكتابة برامج بشكل أسرع. عامّة تحتاج لأن تُرفق معها ملفات مُسَاعِدة لكي تعمل (كَمُفَسِّرٍ مثلا).
  \item \textbf{لغة ذات مستوى منخفض قليلا}:
هي أكثر صعوبة نوعا ما، ولكن مع لغة مثل
\textenglish{C}
 سوف تتعلم كثيرا عن البرمجة وحول طريقة عمل حاسوبك. ستكون بعد ذلك قادراً على تعلّم لغة برمجة أخرى إن أردت وبكل يُسْرٍ.
\end{itemize}

من ناحية أخرى،
\textenglish{C}
لغة برمجة واسعة الإنتشار، أُستخدمت في برمجة العديد من البرامج التي تعرفها. حتى أنها كثيرا ما تدرّس في الدراسات العليا في مجال المعلوماتية.
هذه هي الأسباب الّتي جعلتني أتحمّس لتعليمك لغة
\textenglish{C}
بالتحديد. لم أقل أنّه يجب عليك أن تبدأ بها، لكنّي قلت إنه خيار جيّد لكي أقدّم لك معرفة صلبة في هذا الدرس.

\begin{information}
  بعض لغات البرمجة موجّهة أكثر للشبكة العنكبوتية
 (\textenglish{Web})
 مثل
 \textenglish{PHP}
 أكثر منها من إنشاء البرامج المعلوماتية.
\end{information}

سوف أفترض في هذا الكتاب أنّ هذه هي لغة برمجتك الأولى وأنّه لم يسبق لكم أن برمجت من قبل. فإن كنت قد برمجت قليلا من قبل فلا مضرّة في أن تعيد من الصفر.

\begin{question}
  ما هو الفرق بين
  \textenglish{C}
  و
  \textenglish{C++}
  ؟
\end{question}

هاتان اللغتان قريبتان جدّا من بعضهما، وكلاهما مستخدمتان بكثرة. ولكي تعرف كيف نشأتا يجب عليك أن تدرس التاريخ قليلا:
\begin{itemize}
  \item في البداية، عندما كانت الحواسيب تَزِنُ أطنانا وتشغل مكانا قَدْرُهُ حجم منزلك، تمّ إختراع لغة برمجة تسمّى
\textenglish{Algol}.
  \item بعدها تطوّرت الأمور أكثر واختُرعَت لغة برمجة جديدة عُرِفَتْ باسْمِ
\textenglish{CPL}
 والّتي تطوّرت فيما بعد إلى لغة
\textenglish{BCPL}
 ثم أخذت إسم اللغة
\textenglish{B}.
  \item مع مضيّ الزمن توصّل الخبراء إلى ابتكار اللغة
\textenglish{C}
 وقد تمّ إدخال بعض التعديلات عليها إلّا أنها لا تزال من أحد اللغات الأكثر استخداما اليوم.
  \item وبعد زمن، أراد الخبراء أن يضيفوا بعض الأشياء إلى
\textenglish{C}
، يمكن اعتبارها نوعا من التحسينات. والنتيجة كانت بما يعرف بلغة
\textenglish{C++}
، وهي لغة
\textenglish{C}
 مع إضافات تمكّننا من البرمجة بطريقة مختلفة.
\end{itemize}

\begin{information}
  الـ
\textenglish{C++}
ليست أحسن من الـ
\textenglish{C}
، هي فقط تمكننا من البرمجة بطريقة مختلفة وتساعد المبرمج على تنظيم شفرة برنامجه. رغم ذلك هي تشبه الـ
\textenglish{C}
كثيرا. وإن كنت تنوي تعلّم الـ
\textenglish{C++}
فيما بعد فَسَوْفَ تجد ذلك سهلا.
\end{information}

ولو اعتُبرت
\textenglish{C++}
 تطويرا لـ
\textenglish{C}
 فإن هذا لا يعني أنه يجب استخدام
\textenglish{C++}
 فقط لإنشاء البرامج. لغة
\textenglish{C}
 ليست لغة عجوزا منسيّة، بالعكس هي مستخدمة بكثرة اليوم. بل إنها أساس أنظمة التشغيل الكبيرة مثل
\textenglish{Unix }
(ومنه
\textenglish{GNU/Linux}
 و
\textenglish{Mac OS}) و
\textenglish{Windows}.

\section{هل البرمجة صعبة؟}
هذا سؤال يعذّب روح كل من يريد تعلّم البرمجة! هل يجب أن تكون أستاذ رياضيات كبير درس 10 سنوات من التعليم العالي حتّى تبدأ البرمجة؟

الجواب هو لا بالطبع. كل ما تحتاج إليه هو معرفة العمليات الأربع الأساسية:
\begin{itemize}
  \item الجمع
  \item الطرح
  \item الضرب
  \item القسمة
\end{itemize}
هذا ليس مخيفا! سوف أشرح لك في درس لاحق كيف يقوم الحاسوب بهذه العمليات الأساسية في برامجك.

باختصار، لا توجد صعوبات غير قابلة للحلّ. في الواقع، هذا يعتمد على طبيعة برنامجك، فإذا كنت تريد إنشاء برنامج تشفير فيجب عليك معرفة بعض الأشياء في الرياضيات، وإن كان برنامجك يقوم بالرسم ثلاثي الأبعاد فيجب أن تكون لديك بعض المعرفة بالهندسة الفضائية.

كل حالة تعامل بطريقة خاصّة. ولكن لتعلّم لغة
\textenglish{C}
 نفسها لا تحتاج إلى أيّة معارف قبليّة.

\begin{question}
  إذن أين هو الفخ؟ وأين تكمن الصعوبة؟
\end{question}

يجب أن تعرف كيف يعمل الحاسوب، لتفهم ما الّذي نقوم به في C. من هذا المنطلق، كن متيقّنا أنّي سأعلّمك كلّ هذا شيئا فشيئا.

اعلم أن للمبرمج صفات أيضا مثل:
\begin{itemize}
  \item الصبر: البرنامج لا يعمل عادة من أوّل محاولة، يجب أن تكون مثابراً.
  \item حسّ المنطق: صحيح أنّك لست بحاجة إلى أن يكون لديك مستوى جيّد في الرياضيّات، لكنّ هذا لا يمنع من التفكير وتحليل المشكلات بالمنطق.
  \item الهدوء: فيجب عليك ألّا تضرب حاسوبك بالمطرقة، فهذا لن يجعل برنامجك يعمل!
\end{itemize}

  \chapter{برنامجك الأوّل}

لقد قمنا بتحضير كلّ شيء إلى حد الآن ويمكننا أن نبدأ قليلا من البرمجة. مع نهاية هذا الفصل ستكون قد نجحت في إنشاء أوّل برنامج لك.

لكي أصدقك القول، سيظهر البرنامج بالأبيض والأسود ولن يقوم بشيء سوى إلقاء التحيّة. يبدو عديم الفائدة، لكنّه برنامجك الأوّل وأؤكّد لك أنّك ستكون فخورا به.

\section{كونسول أو نافذة ؟}

لقد تحدثنا سابقا عن فكرة برامج الكونسول وبرامج النوافذ في الفصل السابق. البيئة التطويرية تطلب منا تحديد أي نوع من البرامج نريد أن ننشئها. ولقد قلنا إننا سننشئ برامج من نوع كونسول.

يوجد نوعان من البرامج، لا أكثر :

\begin{itemize}
  \item ،برامج بنوافذ
  \item برامج تعمل في الكونسول.
\end{itemize}

\subsection{البرامج الّتي تملك نوافذ}

هي البرامج التي نعرفها جميعا. هذا مثال على برنامج من نوع نافذة، مثل الرسام.

\begin{figure}[H]
	\centering
	\includegraphics[width=0.6\textwidth]{Chapter_I-3_Paint}
\end{figure}

أعتقد أنّك تحب إنشاء برامج كهذه، لكنّ هذا ليس في مقدورك حاليا. في الواقع، إنشاء برامج بنوافذ هو أمر ممكن بلغة \textenglish{C}، لكنّ بالنسبة لمبتدئ، هذا أمر معقّد جدّا. كبداية، يستحسن إنشاء برامج الكونسول.

\begin{question}
  لكن ماذا يعنى برنامج
\textenglish{Console}
؟
\end{question}

\subsection{البرامج الّتي تعمل في الكونسول}

برامج الكونسول هي أول ما ظهر من برامج. في ذلك الوقت، شاشات الحواسيب لم تكن سوى بالأبيض والأسود، ولم تكن فعّالة لكي تتمكّن من رسم النوافذ كما هو الحال مع حواسيبنا حاليّا.

مرّ الزمن بسرعة وزادت شعبية الويندوز نظراً لبساطته إلى أن نسي كثير من الناس ما هي الكونسول.

لديّ خبر جيّد لك !
\textbf{الكونسول لم تمت بعد} !
 في الواقع،
\textenglish{GNU/Linux}
 قد أعاد الكونسول إلى الحياة. هذه صورة لكونسول على
\textenglish{GNU/Linux}.

\begin{figure}[H]
	\centering
	\includegraphics[width=0.6\textwidth]{Chapter_I-3_Console}
\end{figure}

مرعب ! صحيح ؟ لكن على الأقل عرفت ما هي الكونسول، وهذه بعض الملاحظات :

\begin{itemize}
  \item اليوم، يمكننا عرض الألوان في الكونسول. ليس كلّ شيء بالأبيض والأسود كما تتخيّل.
  \item الكونسول هو الأسهل من ناحية البرمجة بالنسبة للمبتدئين.
  \item أداة عالية الإمكانيّات إذا عرفنا كيف نستخدمه.
\end{itemize}

كما قلت لك، إنشاء برامج كونسول أمر سهل جدّا وملائم للمبتدئين (وهذا عكس برامج النوافذ). ليكن في علمك أيضا أنّ الكونسول قد تطوّرت وبإمكانها عرض الألوان، ولا شيء يمنعك من إضافة صورة خلفيّة لها.

\begin{question}
  وفي الويندوز ألا توجد
\textenglish{Console}
 ؟
\end{question}

بلى، لكنّها مخفيّة لو صح القول. يمكنك فتحها بالذهاب إلى "إبدأ"
(\InlineCode{Start})
 ثمّ "ملحقات"
(\InlineCode{Accessories})
 ثمّ "موجه الأوامر"
(\InlineCode{Command prompt})
 أو بالذهاب إلى "إبدأ" ثمّ "تشغيل"
(\InlineCode{Run})
 واكتب فيها
\InlineCode{cmd}
 واضغط على "موافق".

\begin{figure}[H]
	\centering
	\includegraphics[width=0.6\textwidth]{Chapter_I-3_Console-Windows}
\end{figure}

إذا كنت تستخدم نظام ويندوز، فاعلم بأن أولى برامجك ستكون في نوافذ شبيهة بهذه. أنا لم أختر البداية هكذا لجعلك تشعر بالملل، بل لتعليمك الأساسيّات اللازمة لكي تتمكّن لاحقا من إنشاء النوافذ.

إذن فلتكن متيقّناً، بمجرّد أن تصل إلى المستوى اللازم لإنشاء النوافذ، سوف أعلّمك كيف تفعل ذلك.

\section{الحدّ الأدنى من الشفرة المصدرية}

من أجل أي برنامج، يجب كتابة قدر معيّن من الشفرة المصدرية. هذه الشفرة لا تقوم بشيء خاصّ لكنّها ضروريّة. هذه الشفرة التي سنكتشفها الآن ستكون أساس أغلب برامجك الّتي ستكتبها بلغة \textenglish{C}.

\subsection{أطلب من البيئة التطويرية الخاصة بك تزويدك بالحد الأدنى من الشفرة المصدرية}

لقد لاحظت أن طريقة إنشاء مشروع جديد تختلف من بيئة تطويرية إلى أخرى. إليك تذكيراً بسيطا : في برنامج
\textenglish{Code::Blocks}
 (الذي سنستخدمه في هذا الكتاب)، عليك التوجه نحو
\InlineCode{File}
 ثمّ
\InlineCode{New}
 ثمّ
\InlineCode{Project}
 ثم تختار
\InlineCode{Console Application}
 وبعدها اللغة
\textenglish{C}.
سيولّد لك الحد الأدنى من الشفرة المصدرية
 \textenglish{C}
 التي تحتاجها. ها هي :

\begin{Csource}
#include <stdio.h>
#include <stdlib.h>

int main()
{
    printf("Hello world!\n");
    return 0;
}

\end{Csource}

\begin{information}
لاحظ أنّه يوجد سطر فارغ في نهاية الشفرة. يفترض أن ينتهي كل ملف مكتوب بلغة
\textenglish{C}
هكذا. إن لم تفعل ذلك، فهذه ليست بمشكلة، لكن توقّع أن يعرض لك المترجم تحذيراً
(\textenglish{Warning}).
\end{information}

علماً أنّ السطر :
\begin{Csource}
int main()
\end{Csource}
\dots
بإمكانه أن يُكتب كالتالي :

\begin{Csource}
int main(int argc, char *argv[])
\end{Csource}

كلتا العبارتين تحملان نفس المعنى لكن الثانية، الأكثر تعقيدا، هي الأكثر شيوعا، لذلك فإنّنا سنستخدمها في الفصول القادمة.\\
إستخدامنا للشكل الأوّل أو الثاني لا يغيّر شيئا بالنسبة لنا. لذلك لا داعي لإضاعة الوقت هنا، خصوصاً أنّك لا تملك المستوى اللازم لفهم ما تعنيه.

إذا كنت تستخدم بيئة تطويرية أخرى فقم بنسخ هذه الشفرة المصدرية وألصقها في الملف \InlineCode{main.c} ليكون لديكم نفس الشفرة.

أخيرا، قم بحفظ عملك في المشروع. أعلم أننا لم نقم بشيء حتّى الآن لكن من الجيّد التعوّد على الحفظ في كلّ مرّة.

\subsection{تحليل أسطر الشفرة المصدرية السابقة}
قد تبدو لك الشفرة المصدرية السابقة أنّها كاللغة الصينيّة، أنا أتخيّل ذلك ! في الواقع هي تسمح بإنشاء برنامج كونسول يعرض نصّا على الشاشة. يجب تعلّم كيفيّة قراءة كلّ هذا.

فلنبدأ بأوّل سطرين :
\begin{Csource}
#include <stdio.h>
#include <stdlib.h>
\end{Csource}

هذان السطران يبدآن بعلامة
\InlineCode{\#}.
وهي أسطر خاصّة تُعرف باسم
\textbf{توجيهات المعالج القبلي}
(\textenglish{Preprocessor directives}). اسم معقّد، أليس كذلك ؟ هذه الأسطر تتمّ قراءتها من طرف البرنامج المسمّى بالمعالج القبلي، وهو برنامج يتمّ تشغيله في بداية الترجمة.

ما رأيناه سابقا كان مخطّطا بسيطا لعمليّة الترجمة. لكنّ في الواقع، هناك الكثير من المراحل التي تحدث في هذه العمليّة. سنقوم بتفصيل هذا لاحقا. حاليّا عليك فقط تذكّر وضع هذين السطرين أعلى كلّ ملفّاتك.

\begin{question}
  حسنا لكن ماذا يعنيه هذان السطران ؟ أريد أن أعرف !
\end{question}

كلمة
 \InlineCode{include}
 بالإنجليزيّة تعني "تضمين". هذان السطران يقومان بتضمين ملفّات في المشروع، أي إضافة هذه الملفّات من أجل عمليّة الترجمة. هناك سطران وبالتالي هناك ملفان يتمّ تضمينهما في المشروع وهما بالترتيب :
\InlineCode{stdio.h}
 و
\InlineCode{stdlib.h}.
هذان الملفّان موجودان بالفعل على حاسوبك وهما ملفّان مصدريّان جاهزان، سوف تعرف مستقبلا أنّنا نسميها
\textbf{مكتبات}
(\textenglish{Libraries}).
 هذه الملفّات تحتوي الشفرة المصدرية اللازمة لعرض نصّ على الشاشة.

 بدون هذين الملفّين، كتابة نصّ على الشاشة سيكون أمرا مستحيلاً. فالحاسوب لا يعرف فعل أي شيء مبدئيا.

 باختصار، السطران الأول والثاني يقومان بتضمين المكتبات التي ستساعدنا في إظهار نصّ على الشاشة بكلّ سهولة.

 نمر للتالي، باقي الأسطر :
 
\begin{Csource}
int main()
{
    printf("Hello world!\n");
    return 0;
}
\end{Csource}

ما تراه هنا هو ما نسميه بـ\textbf{التابع}
أو
\textbf{الدالّة}
(\textenglish{Function}).
 البرنامج في لغة
\textenglish{C}
 يتكوّن من مجموعة دوال. حاليّا برنامجنا لا يحوي سوى دالّة واحدة.

الدالّة تمكّننا من تجميع مجموعة من الأوامر. الغرض من تجميع الأوامر هو جعلها تقوم بوظيفة ما. مثلا يمكننا إنشاء دالّة باسم
 \InlineCode{open\_file}
 وجعلها تحتوي التعليمات التي تشرح للحاسوب كيفيّة فتح ملف.

 دون الدخول في تفاصيل إنشاء الدالّة (الوقت مبكّر، سوف نتحدّث عن الدوال في وقت لاحق) لنحلّل رغم ذلك أجزائه الكبيرة. السطر الأوّل يحتوي اسم الدالّة، إنّه الكلمة الثانية.\\
 أجل، اسم دالّتنا هو
\InlineCode{main}
والذي يعني
"الرئيسية"
. وتشغيل البرنامج دائما يبدأ من الدالة
\InlineCode{main}.

للدالّة بداية ونهاية، وهي محدودة بالحاضنتين
\InlineCode{\{}
و
\InlineCode{\}}.
محتوى الدالّة موجود بين هاتين الحاضنتين. إن كنت قد تابعت جيداً فقد عرفت أنّ الدالّة مشكّلة من سطرين :

\begin{Csource}
printf("Hello world!\n");
return 0;
\end{Csource}

هاته الأسطر في الداخل نسميها
\textbf{التعليمات}
(\textenglish{Instructions})
 (هذه إحدى المصطلحات الّتي يجب عليك حفظها). كلّ تعليمة تمثّل أمراً بالنسبة للحاسوب. فكلّ واحدة منها تطلب منه فعل شيء محدّد.

 كما قلت لك، بتجميع ذكيّ للتعليمات في الدالّة يمكننا إنشاء أجزاء برنامج جاهزة للاستخدام. باستخدام التعليمات المناسبة يمكننا إنشاء دالّة
 \InlineCode{open\_file}
 كما شرحت لك قبل قليل، و أيضا دالّة
\InlineCode{move\_character}
 في لعبة فيديو، على سبيل المثال.

 البرنامج في الواقع ما هو إلّا تتابع لتعليمات : إفعل هذا و إفعل ذاك. أنت تعطي أوامر للحاسوب و هو يقوم بتنفيذها.

 \begin{critical}
هامّ جدّا : لا بدّ أن تنتهي كلّ تعليمة بفاصلة منقوطة
"\InlineCode{;}"
. بهذا يمكن التفريق بين ما إذا كانت هذه تعليمة أم لا. إذا نسيت وضع فاصلة منقوطة نهاية تعليمة ما، فلن تتمّ ترجمة برنامجك.
 \end{critical}

 السطر الأول :
 \InlineCode{printf("Hello world!\\n");}
 يطلب إظهار الرسالة
 "\textenglish{Hello world!}"
  على الشاشة. عندما يصل برنامجك إلى هذا السطر، فسوف يقوم بعرض هذه الرسالة ثمّ المرور إلى التعليمة التالية.

  التعليمة التالية هي
\InlineCode{return 0;}
 و هي تخبرنا أنّ الدالّة
\InlineCode{main}
 قد انتهت و تطلب منه إعادة 0.

 \begin{question}
   لماذا يقوم برنامجي بإعادة العدد 0 ؟
 \end{question}

 في الواقع، كلّ برنامج عندما ينتهي يُرجع قيمة معينة. على سبيل المثال، ليقول أنّ كلّ شيء سار على ما يرام. عمليّا، 0 يعني  أنّ كلّ شيء سار على ما يرام، و كلّ قيمة أخرى تدلّ على حدوث خطأ. في أغلب الأحيان هذه القيمة لا تُستخدم ، لكن يجب رغم ذلك استعمالها.\\
 كان يمكن أن يعمل برنامجك بدون
 \InlineCode{return 0}
، لكن يمكننا القول أن وضعها يعتبر أمراً أكثر نظافة و أكثر جدّية.

إلى هنا نكون قد فصّلنا قليلا في عمل هذه الشفرة المصدرية.

طبعا، نحن لم ندرس كلّ شيء بعمق، و قد تكون لديك بعض الأسئلة عالقة في ذهنك. كن على يقين بأنك ستجد لها أجوبة شيئا فشيئا مع تقدّمنا في الكتاب. لا يمكنني أن أطلعك على كلّ شيء من البداية، لأنّ هناك كثيراً من الأشياء لاستيعابها.

إليك ما يلي : بما أنني في حال جيّدة، سأقوم بوضع مخطّط يضمّ المصطلحات الّتي تعلّمناها في هذا الفصل.

\begin{figure}[H]
	\centering
	\includegraphics[width=0.8\textwidth]{Chapter_I-3_HelloWorld}
\end{figure}

\subsection{لنجرّب برنامجنا}

كلّ ما سنقوم به الآن هو ترجمة المشروع ثمّ تشغيله (اضغط على
\InlineCode{Build \& Run}
 إذا كنت على
\textenglish{Code::Blocks}).
سيطلب منك حفظ مشروعك إذا لم تقم بذلك من قبل.

\begin{critical}
  إن لم تنجح الترجمة و ظهر لك خطأ مثل :\\
\InlineCode{"My-program - Release" uses an invalid compiler. Skipping...}\\\InlineCode{Nothing to be done...}
فهذا يعني أنّك نزلت نسخة
\textenglish{Code::Blocks}
 دون
\InlineCode{mingw}
 (المترجم)، عد و نزّل النسخة التي تحتوي على
\InlineCode{mingw}.
\end{critical}

بعد بُرهة، يظهر برنامجك كما في الصورة :

\begin{figure}[H]
	\centering
	\includegraphics[width=0.8\textwidth]{Chapter_I-3_HelloWorld-run}
\end{figure}

البرنامج يُظهر
"\textenglish{Hello world!}"
 (في السطر الأوّل).\\
الأسطر الّتي أسفله تمّ توليدها من طرف
\textenglish{Code::Blocks}
 وتدلّ على أنّ البرنامج قد تمّ تشغيله بنجاح كما أنها تعطي الوقت الذي استغرقه البرنامج في التشغيل.

 سيطلب منك الضغط على إحدى المفاتيح لإغلاق النافذة. أعلم أن الأمر لم يكن ممتعا جدّا. لكنه برنامجك الأوّل، وهذه لحظة ستتذكرها طيلة حياتك ! ألا تعتقد ذلك ؟

\section{كتابة رسالة على الشاشة}

من الآن سنقوم بإدخال التعديلات على الشفرة المصدرية السابقة. مهمّتك، إن قبلتها : عرض رسالة
"\textenglish{Bonjour}"
 على الشاشة.

\begin{question}
  كيف يمكنني اختيار النص الّذي سيظهر على الشاشة ؟
\end{question}

الأمر بسيط جدا، إذا بدأت من الشفرة التي رأيناها سابقاً، فسيكون عليك استبدال
"\textenglish{Hello world!}"
 بـ"\textenglish{Bonjour}"
 في السطر الذي يستدعي
\InlineCode{printf}.

كما قلت من قبل،
\InlineCode{printf}
 هي
\textbf{تعليمة}
 وهي تعطي أمراً للحاسوب : "قم بعرض هذه الرسالة على الشاشة".\\
يجب أن تعرف أيضا أن
\InlineCode{printf}
 هي دالّة كُتِبَت من قبل من طرف مبرمجين قبلك.

\begin{question}
   أين توجد هذه الدالّة ؟ أنا لا أرى سوى الدالّة \InlineCode{main} !
\end{question}

هل تذكر هذين السطرين ؟

\begin{Csource}
#include <stdio.h>
#include <stdlib.h>
\end{Csource}

قلت لك من قبل أنهما يمكنان البرنامج من إضافة مكتبات. المكتبات في الحقيقة هي ملفّات تحوي أطنانا من الدوال جاهزة للإستخدام. هذه الملفات
(\InlineCode{stdio.h} و \InlineCode{stdlib.h})
 تحوي أغلب الدوال الأساسية التي قد نحتاجها في برنامج ما.
\InlineCode{stdio.h}
 بحد ذاته يحوي دوال تمكّن من عرض أشياء على الشاشة (مثل
 \InlineCode{printf})
 و أيضا الطلب من المستخدم إدخال شيء ما (هذه دوال سنتعرّف عليها لاحقا).

\subsection{لنقل مرحبا للسيّد}

في دالّتنا
\InlineCode{main}
نستدعي الدالّة
 \InlineCode{printf}.
 أي أن لدينا دالّة تستدعي أخرى (هنا
\InlineCode{main}
تستدعي
\InlineCode{printf}).
سترى أن هذا ما يحدث دائما في لغة
\textenglish{C}
: دالّة تحتوي تعليمات تستدعي دوال أخرى، وهكذا.

إذن، لاستدعاء دالّة يكفي كتابة اسمها متبوعا بقوسين، ثم فاصلة منقوطة.

\begin{Csource}
printf();
\end{Csource}

هذا جيد، لكنه غير كاف. يجب أن نُعلم البرنامج بما يجب أن يكتبه في الشاشة. لفعل هذا يجب أن نعطي
\InlineCode{printf}
النص المطلوب عرضه. لفعل هذا نقوم بوضع النص داخل علامات الإقتباس المزدوجة بين القوسين.\\
في حالتنا هذه سنكتب تماما :

\begin{Csource}
printf("Bonjour");
\end{Csource}

آمل ألا تكون قد نسيت رمز الفاصلة المنقوطة في النهاية، وأذكّرك أنّها مهمّة جدا لأنّها تدلّ على نهاية التعليمة.\\
هذه هي الشفرة المصدرية التي يجب أن تحصل عليها :

\begin{Csource}
#include <stdio.h>
#include <stdlib.h>

int main()
{
    printf("Bonjour");
    return 0;
}
\end{Csource}

لدينا إذن تعليمتان تطلبان من الحاسوب القيام بهذين الأمرين بهذا الترتيب :
\begin{enumerate}
  \item عرض
"\textenglish{Bonjour}"
على الشاشة.
  \item نهاية الدالّة
\InlineCode{main}
، إعادة 0. البرنامج يتوقّف.
\end{enumerate}

هذا ما يظهر على شاشتك :

\begin{figure}[H]
	\centering
	\includegraphics[width=0.8\textwidth]{Chapter_I-3_Good-Morning}
\end{figure}

كما ترى، السطر الذي يحتوي الرسالة يكون ملتصقاً قليلا بباقي النص، على خلاف ما رأيناه سابقا.\\
أحد الحلول الممكنة هو إضافة رمز للعودة إلى السطر بعد
 "\textenglish{Bonjour}"
 (كما لو أنّنا ضغطنا على المفتاح
\InlineCode{Enter}).

ولكن ضغط المفتاح
\InlineCode{Enter}
 في الشفرة المصدرية لن يعمل كما تتوقع، لهذا يجب استخدام المحارف الخاصّة
(\textenglish{Special characters}).

\subsection{المحارف الخاصّة}

المحارف أو الرموز الخاصّة هي محارف تمكّن من تعريف عودة إلى السطر، جدولة، إلخ.\\
من السهل التعرّف عليها، فهي مكوّنة من محرفين. الأوّل هو الشَرْطَةُ المائلة الخلفية 
(\textbackslash) (\textenglish{Backslash})
والثاني يكون رقما أو حرفا. إليك محرفين خاصّين قد تحتاجهما كثيرا :

\begin{itemize}
  \item \InlineCode{\textbackslash n} :
 العودة إلى السطر.
 \item \InlineCode{\textbackslash t} :
 الجدولة (فراغ كبير في نفس السطر).
\end{itemize}

في حالتنا هذه، يكفي أن نكتب
\InlineCode{\textbackslash n}
 لإنشاء العودة إلى السطر. إذن، إذا أردنا أن نضع عودة إلى السطر بعد
\textenglish{Bonjour}
، فيكفي أن نكتب :

\begin{Csource}
printf("Bonjour\n");
\end{Csource}

وسيفهم حاسوبك أنّ عليه كتابة
"\textenglish{Bonjour}"
 ويعود إلى السطر.

\begin{figure}[H]
	\centering
	\includegraphics[width=0.8\textwidth]{Chapter_I-3_Good-Morning-backslash-n}
\end{figure}

\begin{information}
  يمكنك الكتابة بعد
\InlineCode{\textbackslash n}
بدون أيّة مشكلة. كلّ ما تكتبه بعد
\InlineCode{\textbackslash n}
 سيوضع في السطر الجديد. يمكنك إذن التدرّب على كتابة :
\InlineCode{printf("Good morning\textbackslash nGood bye\textbackslash n");}\\
و سيتمّ عرض
"\textenglish{Good morning}"
على السطر الأوّل و
"\textenglish{Good bye}"
على السطر الثاني.
\end{information}

\subsection{متلازمة \textenglish{Gérard}}

\begin{question}
  مرحبا، اسمي
\textenglish{Gérard}
و قد حاولت تعديل برنامجك ليقول
"\textenglish{Bonjour Gérard}"،
و لكنّي ألاحظ أنّ حرف
\textenglish{é}
 لا يظهر بشكل جيّد
\dots
 مالّذي عليّ فعله ؟
\end{question}

أوّلا، مرحبا بك
\textenglish{Gérard}
. هذا سؤال جيّد. لكن لديّ خبر سيّء لك. الكونسول الخاصة بـ\textenglish{Windows}
لا تمكّن من عرض الحروف الّتي تحوي علامات النطق الصوتي مثل
\textenglish{é}
، خلافا لكونسول
\textenglish{GNU/Linux}
التي تفعل. لديّ حلّان لهذه المشكلة :

\begin{itemize}
  \item \textbf{استخدم
\textenglish{GNU/Linux}}
. هذا حلّ جذريّ بعض الشيء. أحتاج إلى درس كامل لأعلّمك كيف تعمل على
\textenglish{GNU/Linux}
. إذا لم يكن لديك المستوى، إنس هذا الخيار حاليّا.
  \item \textbf{لا تستخدم الحروف الّتي تحوي علامات النطق الصوتي}.
للأسف إنّه الحل الّذي قد يكون عليك اختياره. الكونسول الخاصة بـ\textenglish{Windows}
لها عيوبها. يجب عليك التعوّد على عدم كتابة مثل هذه الحروف. لكن مستقبلا قد تنشئ برامج بنوافذ ولن تعاني من هذا المشكل. لذلك أنصحك بالصبر على هذه المشكلة حاليّا، فبرامجك المستقبلية "الاحترافية" لن يكون فيها هذا المشكل.
\end{itemize}

لكيلا تنزعج، يمكنك الكتابة دون استخدام الحروف التي تملك علامات النطق الصوتي :

\begin{Csource}
printf("Bonjour Gerard\n");
\end{Csource}

نشكر صديقنا
\textenglish{Gérard}
لتنبيهنا على هذه المشكلة !

\section{التعليقات، مهمّة جدا !}

قبل ختم هذا الفصل الأوّل "الحقيقي" في البرمجة، يجب أن أعرّفك على
\textbf{التعليقات}
(\textenglish{Comments})
. أيّا كانت لغة البرمجة الّتي تستخدمها، ستكون لديك القدرة على إضافة التعليقات للشفرة المصدرية الخاصة بك.

ولكن ما الذي يعنيه "التعليق"؟\\
هذا يعني إمكانية وضع نصّ في وسط برنامجك لشرح دوره، مثلاً : ما الذي يفعله هذا السطر، إلخ. هذا بالفعل أمر ضروريّ، لأنّه حتّى لو كنت عبقرياً في البرمجة، ستكون بحاجة إلى وضع ملاحظات هنا وهناك. هذا يمكنك من :

\begin{itemize}
  \item العثور على ما تبحث عنه بسهولة في الشفرة المصدرية عندما تعود إليه بعد مدّة. من الطبيعيّ أن ننسى كيف تعمل البرامج الّتي كتبناها بعد مدّة. إن توقّفت عن البرمجة لأيّام ثمّ عدت فستكون بحاجة إلى التعليقات لإيجاد ما تريد في شفرة كبيرة جدّا.
  \item إذا أعطيت مشروعك لأحد غيرك (وهو لا يعرف شيئا عن الشفرة المصدرية الخاصة بك)، فالتعليقات تمكّنه من التآلف مع مشروعك بسرعة.
  \item وأخيرا، ستسمح لي بإضافة شروحات وملاحظات حول الشفرة المصدرية في هذه الدروس. وهذا سيفيدك في فهم ما الذي يعنيه كلّ سطر.
\end{itemize}

توجد طريقتان لإضافة تعليق. وهذا يعتمد على طول التعليق المراد إدراجه :

\begin{itemize}
  \item إذا كان تعليقك
\textbf{قصيرا}
: فيمكن كتابته على سطر واحد، ولا يحتوي سوى كلمات قليلة. في هذه الحالة، عليك كتابة شرطتين مائلتين
(\InlineCode{//})
متبوعين بتعليقك. على سبيل المثال :

\begin{Csource}
// This is a comment.
\end{Csource}

بإمكانك إضافة تعليق وحده على السطر، أو على يمين تعليمة معينة. وهذا أمر مهمّ جدّا، لأنّ بهذه الطريقة يمكننا تحديد ما الذي يعنيه السطر الّذي كُتب بجانبه. مثال :

\begin{Csource}
printf("Bonjour"); // This instruction displays 'Bonjour' on the screen
\end{Csource}

  \item  إذا كان تعليقك
\textbf{طويلا}:
لديك الكثير لتقوله، تريد كتابة الكثير من الجمل على كثير من الأسطر. في هذه الحالة، يجب عليك كتابة شفرة تشير إلى "بداية التعليق" وأخرى تشير إلى "نهاية التعليق":

  \begin{itemize}
    \item لبدء التعليق : أكتب شرطة مائلة متبوعة بنجمة 
    (\InlineCode{/*}).
    \item لإنهاء التعليق : أكتب نجمة متبوعة بشرطة مائلة 
    (\InlineCode{*/}).
  \end{itemize}

  يمكنك كتابة هذا على سبيل المثال :
  
  \begin{Csource}
/* This is
a comment
written on several lines */
  \end{Csource}
\end{itemize}
فلنعد إلى الشفرة المصدرية التي تُظهر
"\textenglish{Bonjour}"
على الشاشة ونضيف إليها بعض التعليقات للتدرّب :

\begin{Csource}
/*
Below, the directives of preprocessor.
These lines allow you to add files to your program,
files that we call libraries. Thanks to these libraries, we are ready to use functions for display.
for example, a message on screen.
*/

#include <stdio.h>
#include <stdlib.h>

/*
Following, you have the principal function of the program, called main.
All programs start with this function.
Here, all what does my function is displaying "Bonjour" on the screen.
*/

int main()
{
  printf("Bonjour"); // This instruction displays 'Bonjour' on the screen
  return 0;          // The program returns 0 then it stops.
}
\end{Csource}

هذا هو برنامجنا مع إضافة بعض التعليقات، نعم هو يبدو أكبر نوعا ما، لكنّه في الحقيقة مكافئ للبرنامج السابق. عند الترجمة، كلّ التعليقات يتمّ تجاهلها من طرف المترجم. هذه التعليقات لا تظهر في البرنامج النهائي، فهي تصلح فقط للمبرمجين.

عادة لا نقوم بوضع تعليق لكلّ سطر. لقد قلت وأكرر أنّه من المهم وضع التعليقات في الشفرة المصدرية، لكن يجب عليك معرفة القدر اللازم من التعليقات الواجب وضعه، وضع تعليق في كلّ سطر قد لا يفيد في شيء، بل يضيّع الوقت فقط. مثلا، أنت تعرف أن وظيفة
\InlineCode{printf}
هي عرض نصّ على الشاشة، فلا حاجة لوضع تعليق يشرح ذلك في كلّ مرّة.

من الأحسن التعليق عن عدد من الأسطر دفعة واحدة. هذا يفيد في ذكر وظيفة مجموعة من التعليمات المتتابعة. فيما بعد إن أراد المبرمج إضافة مزيد من التفاصيل في تعليماته، فسيكون بمستوى ذكاء يسمح له بفعل ذلك.

\textbf{تذكر إذن}:
يجب أن تكون التعليقات لإرشاد المبرمج في شفرته المصدرية. حاول التعليق عن مجموعة من الأسطر دفعة واحدة بدل التعليق عن كلّ سطر على حدة.

وإليك هذه المقولة من
\textenglish{IBM} :

\begin{center}
  \itshape\Large
  'إذا قرأت التعليقات الموجودة في برنامج و لم تفهم مبدأ عمله، قم برميه !'
\end{center}

\section*{ملخّص}

\begin{itemize}
  \item البرامج يمكنها التفاعل مع المستخدم عن طريق الكونسول أو عن طريق النافذة.
  \item من السهل على المبرمج في برامجه الأولى استخدام
\textbf{الكونسول}،
رغم أنّ هذه قد تكون غير محبوبة لدى المبتدئ، فهذا لا يمنع من استخدام النوافذ في الجزء الثالث من هذا الكتاب.
  \item البرنامج يتكوّن من
\textbf{تعليمات}
 تنتهي دائما بفاصلة منقوطة.
  \item الدالة
\InlineCode{main}
 (التي تعني الرئيسيّة) هي الدالة الّتي يبدأ بها تنفيذ البرنامج. إنّها الدالة الوحيدة الإجبارية في البرنامج، لا يمكن لأي برنامج أن يُترجم بدونها.
 \item \InlineCode{printf}
 هي دالة تمكننا من عرض رسالة على الشاشة.
 \item \InlineCode{printf}
موجودة في
\textbf{مكتبة}
 تحتوي على كثير من الدوال الأخرى الجاهزة للاستخدام.
\end{itemize}

  \chapter{عالم المتغيّرات}

تعلّمت كيفية إظهار نصّ على الشاشة. جيد، لكنّ هذا ليس شيئا مهماً. هذا لأنك لا تعرف بعد ما يدعى بـ
\underline{المتغيّرات}
(\textenglish{Variables})
في البرمجة.

فائدة هذه المتغيرات هي تمكين الحاسوب من حفظ أعداد في الذاكرة. سنبدأ ببعض الشرح حول ذاكرة الحاسوب وكيفيّة عملها. قد يبدو هذا بسيطا جدّا للبعض، لكنّي أفترض أنّك لا تعرف شيئا عن ذاكرة الحاسوب.

\section{أمر متعلق بالذاكرة}
ما سأعلمك في هذا الدرس هو أمر له علاقة مباشرة بذاكرة حاسوبك.

كل إنسان حيّ له ذاكرة. الأمر عينه بالنسبة للحاسوب، لكن الحاسوب له أنواع عديدة من الذاكرة.

\begin{question}
  لم يملك الحاسوب أنواع عديدة من الذاكرة، واحدة يمكنها أن تكفي، أليس الأمر كذلك؟
\end{question}
كلّا: المشكلة أننا نحتاج ذاكرة سريعة (لاسترجاع المعلومات بسرعة) وفي نفس الوقت كبيرة (لحفظ بيانات كثيرة) قد تضحك إن أخبرتك أننا حتى اليوم لم نتمكن من صنع ذاكرة بهذه المواصفات. أو بالأحرى الذاكرة السريعة باهظة الثمن لذلك لا يتم إنتاج الكثير منها.

لذلك نجد في الحواسيب الحديثة ذاكرة سريعة جدا لكنها ليس ذات سعة كبيرة، وأخرى ذات سعة كبيرة جدّا لكنها غير سريعة.

\subsection{الأنواع المختلفة من الذاكرة}
كي أوضح لك الصورة أكثر، إليك أنواع الذاكرة الموجودة في الحاسوب، من الأسرع إلى الأبطأ:
\begin{enumerate}
  \item السجلات (
\textenglish{Registers}
): ذاكرة سريعة جدّا، موجودة داخل المعالج.
  \item ذاكرة التخبئة (
\textenglish{Cache memory}
): تمثل همزة وصل بين السجلات والذاكرة الحية.
  \item ذاكرة الوصول العشوائي (
\textenglish{Random access memory}
): وهي الذاكرة التي نستخدمها كثيرا، وتدعى اختصارا
\textenglish{RAM}.
  \item القرص الصلب (
\textenglish{Hard disk}
): والذي تعرفه بالطبع، نستعمله لحفظ الملفات.
\end{enumerate}
كما قلت لك، لقد رتبتها من الأسرع (السجلات) إلى الأبطأ (القرص الصلب)، وإن كنت قد تابعت جيدا فقد فهمت أن الذاكرة الأصغر هي الأسرع والأبطأ هي الأكبر.\\
السجلات لا تسع إلا لحمل بضعة أعداد أما القرص الصلب فيمكنه تخزين ملفات ضخمة.

\begin{information}
   عندما أقول ذاكرة بطيئة فهذا بالنسبة لحاسوبك، ففي نظر الحاسوب استغراق 8 ميلي ثانية للوصول إلى القرص الصلب يعتبر زمنا طويلا جدّا!
\end{information}

ما الذي يجب أن أتذكره من كل هذا؟\\
أردت أن أخبرك أننا في الدروس القادمة سوف نستخدم ذاكرة الوصول العشوائي كثيرا. سنتعلم أيضا كيفية القراءة والكتابة في الملفات على القرص الصلب (ليس الآن، لا يزال الوقت مبكّرا على هذا). أمّا بخصوص السجلّات وذاكرة التخبئة فلن نتعامل معهما مطلقا، فالحاسوب هو من سيهتم بأمرهما.

\begin{information}
  في لغات البرمجة منخفضة المستوى، كلغة التجميع (
\textenglish{Assembly language}
) نتعامل مباشرة مع السجلّات، لقد درستها، ويمكنني أن أقول لك أن القيام بعملية ضرب بسيطة يتطلب مجهودا! لحسن الحظ ففي لغة
\textenglish{C}
 (وفي أغلب اللغات الأخرى) الأمر أسهل من ذلك بكثير.
\end{information}

يجب إضافة شيء مهمّ آخر: القرص الصلب هو الوحيد الذي يمكنه حفظ المعلومات بشكل دائم.
\textbf{كل أنواع الذاكرات الأخرى مؤقتة، فبمجرد إطفاء الحاسوب تفقد كل محتواها}!

لحسن الحظ فعند إعادة تشغيل الحاسوب يقوم القرص الصلب بتذكيرها بمحتواها.

\subsection{صورة لذاكرة الوصول العشوائي}
نظرا لأننا سنستعمل ذاكرة الوصول العشوائي خلال لحظات، فمن الأفضل أن أريها لكم (مؤطر بالأحمر):
\Picture{Chapter_I-4_Computer}
لا أطلب منك معرفة كيفية عملها، لكن أردت فقط أن أريك مكانها داخل جهازك. وهذه صورة مقربة لإحدى أشرطتها:
\Picture{Chapter_I-4_RAM}
وهي تدعى اختصارا
\textbf{\textenglish{RAM}}
، لذلك لا تحتر إن سميتها هكذا لاحقا. بالنسبة للذاكرات الأخرى (السجلات والتخبئة) فهي صغيرة لدرجة أنه لا يمكن رؤيتها بالعين المجرّدة.

\subsection{مخطط ذاكرة الوصول العشوائي}
عرض المزيد من الصور لن يفيدك كثيرا، لكن يجب عليك فهم كيف تعمل من الداخل، لذلك سأقدم لك هذا المخطط البسيط الذي يمثل هندسة ذاكرة الوصول العشوائي:
\Picture{Chapter_I-4_RAM-Schema}

كما ترى، يمكننا أن نميز عمودين:
\begin{itemize}
  \item هناك
\textbf{العناوين}
: هي أعداد تسمح للحاسوب بتحديد موضع القيم في الـ
\textenglish{RAM}
. نبدأ بالعنوان 0 وننتهي بالعنوان 3,448,765,900,126 وبعض الأجزاء. لا أعلم بالضبط كم عدد العناوين الموجودة في الـ
\textenglish{RAM}
، لكني أعرف أنها كثيرة جدا. إضافة إلى ذلك، هذا أمر يتعلق بكمية الذاكرة الموجودة في جهازك، فكلما زادت الذاكرة زادت معها العناوين وصار بإمكاننا تخزين معلومات أكثر.
  \item عند كل عنوان يمكننا تخزين
\textbf{قيمة}
(عدد). حاسوبك يقوم بتخزين هذه الأعداد في ذاكرة الوصول العشوائي لكي يتمكن من تذكرها. ولا يمكننا تخزين سوى عدد واحد عند كل عنوان.
\end{itemize}

لا يمكن للذاكرة الحية تخزين شيء سوى الأعداد.

\begin{question}
  لكن كيف يمكننا تخزين الكلمات؟
\end{question}

سؤال جيد. في الواقع حتى الحروف ليست سوى أعداد في نظر الحاسوب! الجملة هي مجرد تتابع لأعداد.\\
يوجد جدول يوافق بين الأعداد والحروف، جدول يقول مثلا بأن العدد 67 يوافق الحرف
\textenglish{Y}
. لن أدخل في التفاصيل أكثر، ستكون لنا فرصة للرجوع إلى هذا لاحقا.

فلنعد إلى مخططنا، الأمور بسيطة جدا: إذا أراد الحاسوب تذكر العدد 5 (الذي قد يمثل عدد الأرواح المتبقية لشخصية في لعبة) فسوف يضعه في مكان ما في الذاكرة أين يتوفر مكان شاغر ويحفظ العنوان الموافق (مثلا 3,062,199,902). لاحقا، عندما يريد معرفة هذا العدد فسيذهب إلى خانة الذاكرة التي تحمل العنوان رقم 3,062,199,902 وسيجد القيمة 5.

هذه آلية عمل الذاكرة بشكل عام. قد يكون الأمر لا زال غامضا في ذهنك حاليا (ما فائدة تخزين عدد إن كان علينا تذكر عنوانه بدلا من ذلك؟) لكن كل شيء سيتضح مع بقية الدروس، أنا أعدك!

\section{التصريح عن متغير}
صدّقني هذه المقدّمة القصيرة عن الذاكرة ستكون مهمّة أكثر مما تعتقد. الآن يمكننا العودة إلى البرمجة.

إذن، ما هو
\underline{المتغير}
(\textenglish{Variable}) ؟\\
إنه معلومة صغيرة نخزنها مؤقتا في الذاكرة الحية. ببساطة يمكننا القول إن المتغير هو قيمة يمكن أن تتغير أثناء اشتغال البرنامج. مثلا عددنا 5 الذي ذكرناه سابقا يمكن أن يتناقص بمرور الزمن. إذا وصل إلى العدد 0 فسنعرف أن اللاعب قد خسر.

في برامجنا سيكون هناك الكثير من المتغيرات. ستراها في كلّ مكان.

في لغة السي، المتغير يتميز بشيئين:
\begin{itemize}
  \item \underline{قيمة}
: هو العدد الذي يحويه، 5 مثلا.
  \item \underline{اسم}
: وهو الذي يمكننا من معرفة المتغيّر. في البرمجة لن يكون علينا تذكّر عناوين الذاكرة. بدلا من ذلك علينا فقط استخدام أسماء المتغيرات. المترجم هو من سيقوم بتحويل الأسماء إلى عناوين.
\end{itemize}

\subsection{إعطاء اسم للمتغير}
في لغة البرمجة
\textenglish{C}
كل متغير يجب أن يملك اسما خاصا به. ومن أجل متغيرنا الذي يحوي عدد الأرواح المتبقية للاعب يمكننا أن نسميه
"\textenglish{Number of lives}"
أو شيء من هذا القبيل.

للأسف توجد بعض الشروط، لا يمكنك تسمية المتغير كيفما شئت:
\begin{itemize}
  \item لا يجب أن يحتوي الاسم سوى على الحروف الصغيرة والكبيرة والأرقام
(\InlineCode{abcABC012}).
  \item يجب أن يبدأ الاسم بحرف.
  \item المسافات ممنوعة. بدلا من ذلك يمكننا استخدام الحرف المعروف باسم
\textenglish{underscore}
 (\InlineCode{\_}).
إنه الحرف الخاص الوحيد غير الحروف والأرقام الذي يمكن استعماله في اسم متغير.
  \item لا يمكنك استخدام حروف غير الحروف الإنجليزية.
\end{itemize}

وأخيرا يجب أن تعرف أن لغة
\textenglish{C}
 تفرّق بين الحروف الصغيرة والكبيرة. ولثقافتك، نقول إن
\textenglish{C}
 حساسة لحالة الأحرف
(\textenglish{Case sensitive}).
كمثال، الأسماء
\InlineCode{width}
 أو
\InlineCode{WIDTH}
 أو
\InlineCode{WiDth}
تعتبر أسماء متغيرات مختلفة، حتى لو كانت تعني لنا الأمر نفسه.

هذه أمثلة عن أسماء متغيرات صالحة:
\InlineCode{numberOfLives}،
\InlineCode{name}،
\InlineCode{surname}،
\InlineCode{phone\_number}،
\InlineCode{phoneNumber}.

لكل مبرمج طريقة خاصة في كتابة أسماء المتغيرات. خلال هذا الدرس سأريك طريقتي:
\begin{itemize}
  \item أبدأ دائما بحرف صغير.
  \item إن كان في الاسم أكثر من كلمة أضع حرف كبيرا في بداية كلّ كلمة.
\end{itemize}

أطلب منك كتابة أسماء متغيراتك بنفس الطريقة التي أتبعها، هذا لكي نكون على تفاهم.

\begin{critical}
  أيّا كان اختيارك، فعليك دائما إعطاء أسماء واضحة لمتغيراتك. كان بإمكاننا اختصار
\InlineCode{numberOfLives}
إلى
\InlineCode{nol}
مثلا. هذا أقصر في الكتابة، لكنه أقل وضوحا عندما تعيد قراءة الشفرة المصدرية. فأنصحك بإعطاء أسماء أطول لمتغيراتك إن كان ذلك يحسّن فهمها.
\end{critical}

\subsection{أنواع المتغيرات}
حاسوبنا كما نعلم ليس سوى آلة كبيرة جدا للحساب. لا يجيد التعامل سوى مع الأعداد. لكن يوجد أنواع كثيرة من الأعداد:
\begin{itemize}
  \item الأعداد الصحيحة الموجبة (الطبيعية) مثل :45، 398، 7650.
  \item الأعداد العشرية، أي التي تحوي فاصلة عشرية: 75.909، 1.7741، 9810.7.
  \item الأعداد الصحيحة السالبة: -87، -916.
  \item الأعداد العشرية السالبة: -76.9، -100.11.
\end{itemize}

حاسوبك المسكين بحاجة للمساعدة! عندما تطلب منه تخزين عدد، يجب أن تذكر له نوعه. هذا ليس لأنّه لا يمكنه التعرف عليه تلقائيّا، ولكن للتنظيم ولعدم أخذ كميات كبيرة من الذاكرة بدون فائدة.

عندما تصرّح عن متغيّر فسيكون عليك تحديد نوعه. إليك أنواع المتغيرات الأساسية في لغة \textenglish{C}:

\begin{Table}{3} % The number of columns is required (here is 3)
  النوع & الحد الأدنى & الحد الأقصى\\
  \LR{\ttfamily signed char} & $-128$ & $127$ \\
  \LR{\ttfamily int} & $-32768$ & $32767$ \\
  \LR{\ttfamily long} & $-2147483648$ & $2147483647$ \\
  \LR{\ttfamily float} & $-1 \times 10^{37}$ & $1 \times 10^{37}-1$\\
  \LR{\ttfamily double} & $-1 \times 10^{37}$ & $1 \times 10^{37}-1$\\
\end{Table}

\begin{warning}
  القيم المعروضة هنا تمثل الحد الأدنى المضمون من طرف اللغة. في الحقيقة قد تتمكن من تخزين أعداد أكبر من هذه. في كلّ الأحوال من المستحسن تذكّر هذه القيم عندما تختار نوع متغيراتك.
\end{warning}

\begin{information}
  للعلم أنّي لم أعرض جميع الأنواع هنا، بل الأساسية منها فقط.
\end{information}

الأنواع الثلاثة الأولى
(\InlineCode{char}، \InlineCode{int}، \InlineCode{long})
تسمح يتخزين الأعداد الصحيحة (1،2،3،4،...).\\
النوعان الأخيران
(\InlineCode{float}، \InlineCode{double})
يسمحان بتخزين الأعداد العشرية (13.8, 16.911…).

سترى أنّنا نتعامل من الأعداد الصحيحة معظم الوقت لأنّها سهلة الإستخدام.

\begin{critical}
  احذر في الأعداد العشرية من استخدام الفاصلة، حاسوبك لا يستخدم سوى النقطة. لذلك لا تكتب
$54,9$
 بدل
$54.9$!
\end{critical}

هذا ليس كلّ شيء، توجد أنواع أخرى تعرف بـ
\InlineCode{unsigned}
 (عديمة الإشارة) تصلح لتخزين الأعداد الموجبة فقط. يجب إضافة كلمة
\InlineCode{unsigned}
إلى النوع لاستخدامها.

\begin{Table}{2}
  النوع & المجال\\
  \LR{\ttfamily unsigned char} & من
$0$
 إلى
$255$ \\
  \LR{\ttfamily unsigned int} & من
$0$
إلى
$65535$ \\
  \LR{\ttfamily unsigned int} & من
$0$
إلى
$4294967295$\\
\end{Table}

كما ترى، مشكلة الأنواع عديمة الإشارة هي عدم القدرة على تخزين الأعداد السالبة، لكن الشيء الإيجابي هي أنّها توفّر لنا ضعف حجم التخزين لكلّ نوع موافق (مثلا
\InlineCode{signed char}
يتوقّف عند 127، بينما
\InlineCode{unsigned char}
يمتد إلى 255.

  \part{تقنيات متقدّمة في لغة الـ\textenglish{C}}
  \chapter{البرمجة المجزّأة
(\textenglish{Modular programming})}
في هذه المرحلة الثانية، سنكتشف مبادئ متقدّمة في لغة الـ
\textenglish{C}
 لن أخفي عليك، هذه المرحلة صعبة الفهم و تحتاج منك التركيز. في نهاية المرحلة، ستكون قادراً على تدبّر أمرك في معظم البرامج المكتوبة بلغة السي. في المرحلة التي تليها نتعلّم كيف نفتح نافذة، كيف ننشئ لعبة ثنائية الأبعاد... الخ

لحدّ الآن عملنا في ملف واحد سمّيناه
\InlineCode{main.c}
. كان أمراً مقبولاً لحدّ الآن لأن برامجنا كانت صغيرة، لكنها ستصبح في القريب العاجل مركّبة من عشرات، لن أقول من مئات الدوال، و إن كنت تريد وضعها كلّها في نفس الملف، فإن هذا الأخير سيصبح ضخماً جداً. لهذا السبب تم اختراع ما نسمّيه بالبرمجة المجزّأة. المبدأ سهل: بدل أن نضع كل الشفرة المصدرية في ملف واحد
\InlineCode{main.c}
، سنقوم بتفريقها إلى عدة ملفات.

\section{النماذج (\textenglish{prototypes})}
لحدّ الآن، كنت عندما تنشئ دالة، أطلب منك وضعها قبل الدالة الرئيسية
\InlineCode{main}
. لماذا؟

لأن للترتيب أهمية حقيقية هنا: فإن قمت بوضع الدالة قبل الـ
\InlineCode{main}
في الشفرة المصدرية، سيقرؤها الجهاز و يتعرف عليها. حينما تقوم باستدعاء الدالة داخل الـ
\InlineCode{main}
، سيعرفها الجهاز و يعرف أيضاً أين يبحث عليها.\\
بالعكس، لو تضع الدالة بعد الـ
\InlineCode{main}
، لن يعمل البرنامج لأن الجهاز لم يتعرّف بعد على الدالة. جرّب ذلك و سترى.
\begin{question}
  لكنه تصميم سيّء نوعاً ما، أليس كذلك ؟
\end{question}
أنا متفق معك! لكن انتبه المبرمجون لهذه النقطة قبلك و عملوا على حلّ المشكل.

بفضل ما سأعلمك إياه الآن، ستتمكن من الدوال في أي ترتيب كان في الشفرة المصدرية، هكذا لن تقلق من هذه الناحية.

\subsection{استعمال النموذج للتصريح عن دالة}
سنقوم بتصريح دوالنا للحاسوب، و هذا بكتابة ما نسميه بـ
\textbf{النماذج}
.لا تنبهر بهذا الاسم، إنه يخبّئ معلومة بسيطة جداً.

تأمل في السطر الأول من دالتنا
\InlineCode{rectangleSurface}
\begin{Csource}
double rectangleSurface(double width, double height)
{
	return width * height;
}
\end{Csource}
قم بنسخ السطر الأول
(\InlineCode{double rectangleSurface...})
المتواجد أعلى الشفرة المصدرية (مباشرة بعد تعليمات التضمين
\InlineCode{\#include}
). أضف
\textbf{فاصلة منقوطة}
في نهاية هذا السطر.\\
و هكذا يمكنك أن تضع الدالة الخاصة بك
\InlineCode{rectangleSurface}
بعد الدالة
\InlineCode{main}
ان أردت !

هذا ما يجب أن تكون عليه الشفرة المصدرية :
\begin{Csource}
#include <stdio.h>
#include <stdlib.h>
// The next line represents the prototype of the function rectangleSurface :
double rectangleSurface(double width, double height);
int main(int argc, char *argv[])
{
	printf("width = 5 and height = 10. Surface = %f\n", rectangleSurface(5, 10));
	printf("width = 2.5 and height = 3.5. Surface = %f\n", rectangleSurface(2.5, 3.5));
	printf("width = 4.2 and height = 9.7. Surface = %f\n", rectangleSurface(4.2, 9.7));

	return 0;
}
// Now, we can put our function wherever we want in the source code:
double rectangleSurface(double width , double height )
{
	return width * height ;
}
\end{Csource}
الشيء الذي تغيّر هنا هو إضافة النموذج أعلى الشفرة المصدرية.\\
النموذج هو عبارة عن إشارة للجهاز، يوحي إليه بوجود دالة تسمى
\InlineCode{rectangleSurface}
و التي تأخذ معاملات إدخال معينة و تُرجِع مخرج من نوع أنت من تحدده.  هذا يساعد الجهاز على تنظيم نفسه.

بفضل ذلك السطر، يمكنك الآن وضع دوالك في أي ترتيب كان دون أي تفكير زائد.

أكتب دائما النموذج الخاص بدوالك. البرامج التي ستكتبها من الآن و صاعداً ستصبح أكثر تعقيداً و تستعمل الكثير من الدوال: من الأحسن أن تتعلّم منذ الآن العادة الجيدة  بوضع نموذج لكل دالة في الشفرة المصدرية.

كما ترى، الدالة
\InlineCode{main}
لا تملك أي نموذج، و كمعلومة فهي الوحيدة التي لا تملك نموذجاً ! لأن الجهاز يعرفها (فهي نفسها مكررة في جميع البرامج).

عليك أن تعرف أنه في سطر النموذج، لست مضطراً إلى تحديد المعاملات التي تتلقاها الدالة كمدخل. الجهاز يحتاج أن يتعرّف إلى نوع المداخل فقط.

يمكننا أن نكتب ببساطة :
\begin{Csource}
double rectangleSurface (double, double);
\end{Csource}
و مع ذلك، فالطريقة التي أريتك إياها أعلاه تعمل أيضاً. الشيء الجيد فيها هو أن كلّ ما عليك فعله هو نسخ و لصق السطر الأول الخاص بالدالة مع إضافة فاصلة منقوطة (طريقة سهلة و سريعة).
\begin{critical}
  لا تنس
\underline{أبدا}
وضع فاصلة منقوطة بعد النموذج، هذا يمكّن الحاسوب من التفريق بين النموذج و بداية الدالة.\\
إن لم تفعل، ستعترضك أخطاء غير مفهومة أثناء عملية الترجمة.
\end{critical}

\section{الملفات الرأسية
(\textenglish{headers})}
لحدّ الآن لا نملك غير ملف مصدري واحد في مشروعنا و هو الذي كنّا نسمّيه
\InlineCode{main.c}.

\subsection{عدة ملفات في مشروع واحد}
تطبيقياً، برامجك لن تكون مكتوبة في ملف واحد
\InlineCode{main.c}.
بالطبع يمكن فعل ذلك، لكن لن يكون من الممتع أن تتجوّل في ملف به 10000 سطر (شخصياً أعتقد هذا). و لهذا فإنه في العادة ننشئ العديد من الملفات في المشروع الواحد.
\begin{question}
  عفوا ... ماهو المشروع ؟
\end{question}
لا ! هل نسيت بسرعة ؟ سأعيد الشرح لأنه من اللازم أن نتّفق على هذا المصطلح.

المشروع هو مجموع الملفات المصدرية الخاصة ببرنامجك. لحد الآن برنامجنا لم تتكون إلا من ملف واحد. و يمكنك التحقق من هذا بالنظر في البيئة التطويرية الخاصة بك، غالبا ما يظهر المشروع في القائمة على اليسار (الصورة الموالية):
\Picture{Chapter_II-1_Project}
كما يمكنك رؤيته في يسار الصورة، هذا المشروع ليس مكوّنا إلا من الملف
\InlineCode{main.c}.

إسمح لي الآن أن أُرِيَكَ صورة لمشروع حقيقي ستقوم به في وقت لاحق من الدروس : لعبة سوكوبان (الصورة الموالية) :
\Picture{Chapter_II-1_Project-Sokoban}
كما ترى، هناك ملفات عديدة. هذا ما يكون عليه المشروع الحقيقي، أي تتواجد به ملفات عديدة في القائمة اليسارية  يمكن التعرّف على الملف
\InlineCode{main.c}
من بين القائمة و الذي يحتوي الدالة
\InlineCode{main}.
بصورة عامة في برامجي، لا أضع إلّا الدالة
\InlineCode{main}
في الـملف
\InlineCode{main.c}.
لمعلوماتك، هذا ليس أمراً إجبارياً، كل واحد ينظّم ملفاته بالشكل الذي يريد. لكن لكي تتبعني جيّداً أنصحك بفعل ذلك.
\begin{question}
  لكن لم يجب عليّ إنشاء ملفات عديدة ؟ و كم من ملف يجب علىّ أن أنشئ في مشروعي ؟
\end{question}
هذا يبقى اختيارك أنت، في الغالب نجمع في نفس الملف المصدري الدوال التي تشترك في الموضوع الذي تعالجه. و هكذا ففي الملف
\InlineCode{editeur.c}
جمعت كلّ الدوال الخاصة ببناء المستوى، و في الملف
\InlineCode{jeu.c}
قمت بتجميع الدوال الخاصة باللعبة نفسها و هكذا ...

\subsection{الملفات
\texttt{\textenglish{.c}}
و
\texttt{\textenglish{.h}}}
كما يمكنك أن تلاحظ، يوجد نوعان مختلفان من الملفات في الصورة السابقة.
\begin{itemize}
  \item \textbf{ملفات ذات الإمتداد
\texttt{\textenglish{.c}}}
: الملفات المصدرية، تحتوي الدوال نفسها.
  \item \textbf{ملفات ذات الإمتداد
\texttt{\textenglish{.h}}}
: تسمى الملفات الرأسية و هي تحتوي النماذج الخاصة بالدوال.
\end{itemize}
عموما، انه لمن النادر وضع نماذج في الملفات من صيغة
\InlineCode{.c}
مثلما فعلنا للتوّ في الـملف
\InlineCode{main.c}
(إلا إذا كان برنامجك صغيرا).

من أجل كل ملف
.\InlineCode{c}
هناك ملف مكافئ له، و الذي يحتوي نماذجا للدوال الموجودة في الملف
\InlineCode{.c}
، تمعّن في الصورة السابقة.
\begin{itemize}
  \item هناك
\InlineCode{editeur.c}
(الشفرة الخاصة بالدوال) و
\InlineCode{editeur.h}
(ملف النماذج الخاصة بالدوال).
  \item هناك
\InlineCode{jeu.c}
و
\InlineCode{jeu.h}.
  \item إلخ...
\end{itemize}
\begin{question}
  لكن كيف يعرف الحاسوب بأن نماذج الدوال موجودة في ملف آخر خارج الملف
\InlineCode{.c}
؟
\end{question}
يجب عليك تضمين الملف الرأسي
\InlineCode{.h}.
مستعيناً بتوجهات المعالج القبلي.\\
كن مستعداً لأنّي سأعطيك الكثير من المعلومات في وقت قصير.

كيف نقوم بتضمين ملف رأسي ؟ أنت تجيد فعل ذلك لأنك قمت بذلك من قبل.

أنظر مثالاً من بداية الملف
\InlineCode{jeu.c} :
\begin{Csource}
#include <stdlib.h>
#include <stdio.h>
#include "jeu.h"
void play(SDL_Surface* screen)
{
// ...
\end{Csource}
التضمين يتم عن طريق توجيهات المعالج القبلي
\InlineCode{\#include}
التي يجدر بك أن تكون قد تعلّمتها من قبل.\\
تمعن في التالي :
\begin{Csource}
#include <stdlib.h>
#include <stdio.h>
#include "jeu.h" // We include jeu.h
\end{Csource}
قمنا بتضمين ثلاثة ملفات من صيغة
\InlineCode{.h}
و هي :
\InlineCode{stdio}، \InlineCode{stdlib} و \InlineCode{jeu}.\\
لاحظ الفرق : الملفات التي قمت بإنشاءها ووضعها في الـمجلّد الخاص بمشروعك يجب أن تكون مضمّنة مع اشارات الاقتباس
(\InlineCode{"jeu.h"})
بينما ملفات المكتبات (التي توجد عادة في البيئة التطويرية الخاصة بك) تكون مضمّنة بعلامات الترتيب
(\InlineCode{<stdio.h>}).

تستعمل إذا :
\begin{itemize}
  \item علامتي الترتيب
\InlineCode{< >}
: لتضمين الملفات المتواجدة في المجلّد
\InlineCode{include}
الخاص بالبيئة التطويرية.
  \item علامتي الاقتباس
\InlineCode{" "}
 : لتضمين  الملفات المتواجدة في مجلّد المشروع (و غالبا بجانب الملفات
\InlineCode{.c}).
\end{itemize}
الأمر
\InlineCode{\#include}
يطلب إدخال محتوى ملف معيّن في الملف
\InlineCode{.c}
فهي تعليمة تقول :"أدخل الملف
\InlineCode{jeu.h}
هنا" مثلا .

\begin{question}
  و في الملف
\InlineCode{jeu.h}
ماذا نجد ؟
\end{question}
لا نجد إلا نماذج خاصة بدوال الملف
\InlineCode{jeu.c}
!
\begin{Csource}
void play(SDL_Surface* screen);
void movePlayer(int map[][NB_BLOCS_HEIGHT], SDL_Rect *pos, int direction);
void moveBox(int *firstBox, int *secondeBox);
\end{Csource}
هكذا يعمل المشروع الحقيقي !
\begin{question}
  ما الهدف من وضع نماذج في ملفات من نوع
  \InlineCode{.h}
  ؟
\end{question}
السبب بسيط للغاية، عندما تستدعي دالة في الشفرة المصدرية الخاصة بك، ينبغى لجهازك أن يكون متعرفا عليها من قبل، و يعرف كم من المعاملات تستعمل...الخ. إن هذا هو الهدف وراء وجود النماذج، انه دليل الاستخدام الخاص بالدالة بالنسبة للجهاز.

كلّ هذا هو مسألة تنظيم، عندما تضع نماذجك في ملفات
\InlineCode{.h}
(ملفات رأسية) مضمّنة في أعلى الملفات
\InlineCode{.c}
، سيعرف جهازكم طريقة استخدام الدوال الموجودة في الملف ما إن يبدأ في قراءته.

عند القيام بهذا، لن يكون عليك القلق حيال الترتيب الذي ستكون عليه دوالك في الملفات
\InlineCode{.c}.
اذا كنت قمت الآن بإنشاء برنامج صغير يحتوي على دالتين أو ثلاث يمكنك أن تفكّر أنه من الممكن للبرنامج أن يتشغل دون وجود النماذج، لكن هذا لن يستمر ذلك طويلا ! فما إن يكبر البرنامج و إن لم تنظّم النماذج في ملفات رأسيّة فستفشل الترجمة دون أدنى شك.
\begin{information}
  عندما تستدعي دالة متواجدة في الملف
  \InlineCode{functions.c}
  إنطلاقا من الملف
  \InlineCode{main.c}
  سيكون عليك تضمين النماذج الخاصة بالملف
  \InlineCode{functions.c}
  في الملف
  \InlineCode{main.c}
  يجب إذن وضع
  \InlineCode{\#include "functions.h"}
  في أعلى الـملف
  \InlineCode{main.c}.\\
  تذكر هذه القاعدة : "في كلّ مرة تستدعي الدالة
  \textenglish{X}
  في ملف، يجب عليك إدراج نموذج هذه الدالة في ملفكم" هذا ما يسمح للـمترجم بمعرفة ما إن كنت قد استدعيتها بشكل صحيح.
\end{information}
\begin{question}
  كيف أقوم بإضافة ملفات
\InlineCode{.c}
 و
\InlineCode{.h}
 إلى مشروعي ؟
\end{question}
هذا راجع للـبيئة التطويرية التي تستخدمها. لكن المبدأ هو نفسه في جميع البرامج :
\InlineCode{File} / \InlineCode{New} / \InlineCode{Source File}\\
هذا يسمح بإنشاء ملف جديد فارغ. هذا الملف ليس حاليا من النوع
\InlineCode{.c}
ولا
\InlineCode{.h}
أنت من يحدد ذلك أثناء عملية حفظ الملف. قم إذن بحفظه (حتّى و إن كان لا يزال فارغا !) و هنا يطلب منكم إدخال اسم للملف، يمكنك هنا اختيار صيغة الملف :
\begin{itemize}
  \item إذا سميته
\InlineCode{file.c}
فسيكون بامتداد
\InlineCode{.c}.
  \item إذا سميته
\InlineCode{file.h}
فسيكون بامتداد
\InlineCode{.h}.
\end{itemize}
هذا سهل. قم بحفظ الملف في المجلّد أين تتواجد باقي الملفات الخاصة بمشروعك (نفس المجلّد أين يتواجد الملف
\InlineCode{main.c}). عموما كل ملفات المشروع تقوم بحفظها في نفس المجلّد سواء كانت ذات صيغة
\InlineCode{.c}
أو
\InlineCode{.h}.

مجلّد المشروع في النهاية سيكون مثل هذا :
\Picture{Chapter_II-1_Project-Sokoban-Folder}
الملف الذي أنشأته محفوظ لكن لم تتم إضافته إلى مشروعك بعد !\\
لإضافته قم بالنقر يمينا على القائمة أيسر الشاشة (الخاصة بملفات المشروع) و اختر
\InlineCode{Add files}
كالتالي :
\Picture{Chapter_II-1_Project-Add-File}
ستظهر لك نافذة تطلب منك اختيار الملفات التي تريد أن تدخلها للمشروع، اختر الملف الذي قمت بإنشاءه، للتو، و سيتم إدخاله أخيرا في المشروع.  ستجده حاضراً في القائمة اليسارية !

\subsection{الـ
\texttt{include}
الخاصّة بالمكتبات النموذجية}
يفترض أنّ لديكم سؤالا يدور في رؤوسكم الآن...\\
إذا ضمّنا الملفات
\InlineCode{stdio.h}
و
\InlineCode{stdlib.h}
فهذا يعني أنهما موجودان في مكان ما و يمكننا البحث عنهما، أليس كذلك ؟

نعم بالطبع !\\
يفترض أنهما مسطبان في المكان الذي تتواجد به البيئة التطويرية الخاصة بك، بالنسبة للبيئة
\textenglish{Code::Blocks}
أجدهم هنا :

\InlineCode{C:\textbackslash Program Files\textbackslash CodeBlocks\textbackslash MinGW\textbackslash include}

على العموم يجب البحث عن مجلد يحمل اسم
\InlineCode{include}.\\
بداخله تجد كمّا هائلا من الملفات، و هي ملفات رأسية
(\InlineCode{.h})
خاصة بمكتبات نموذجية أي مكتبات متوفرة في كل مكان (سواء في الويندوز أو الماك أو اللينكس...)، و ستجد داخلها الملفات
\InlineCode{stdio.h}
و
\InlineCode{stdlib.h}
مع ملفّات أخرى.

يمكنك فتحها إذا أردت، لكن ستتفاجئ بالعديد من الأشياء التي لم أدرّسها لك من قبل خاصة بما يتعلق ببعض توجيهات المعالج القبلي. يمكنك أن تلاحظ بأن الملف مليء بنماذج لدوال نموذجية مثل
\InlineCode{printf}.
\begin{question}
  حسناً، الآن عرفت أين أجد نماذج الدوال النموذجية لكن ألا يمكنني رؤية الشفرة المصدرية الخاصة بالدوال ؟ أين هي الملفات
\InlineCode{.c}
؟
\end{question}
إنها غير موجودة أساسا ، لأنها مترجمة (إلى ملفات ثنائية
(\textenglish{binary files})
يعني إلى لغة الحاسوب). و لهذا فإنه من المستحيل أن تقرأها.

يمكنك إيجاد الملفات المترجمة في المجلّد المسمى
\InlineCode{lib}
(و الذي هو اختصار لكلمة
\InlineCode{library}
أي مكتبة)، بالنسبة لي هي موجودة في المسار :

\InlineCode{C:\textbackslash Program Files\textbackslash CodeBlocks\textbackslash MinGW\textbackslash lib}

ملفات المكتبات المترجمة لها الصيغة
\InlineCode{.a}
في البيئة
\textenglish{Code::Blocks}
و التي تستخدم
\InlineCode{mingw}
كمترجم. و لها صيغة
\InlineCode{.lib}.
في برنامج
\textenglish{Visual C++}
الذي يستخدم المترجم
\InlineCode{Visual}.
لا تحاولوا قراءتها لأنها غير قابلة للقراءة من طرف إنسان عادي.

باختصار، يجب عليك أن تضمّن الملفات الرأسية
\InlineCode{.h}
في الملفات
\InlineCode{.c}
تتمكن من استخدام الدوال النموذجية مثل
\InlineCode{printf}
و كما تعرف فالجهاز على اطلاع على النماذج فهو يعرف ما إن كنت قد طلبت الدوال بشكل صحيح (إن لم تنس أحد المعاملات مثلا).

\section{الـترجمة المنفصلة}
الآن و بعدما عرفت أن المشروع مبنى على أساس ملفات مصدرية عديدة، يمكننا الدخول الآن في تفاصيل عملية الترجمة فلحد الآن لم نر سوى مخطط مبسط عنها.

سأعطيك الآن مخططا مفصلا عنها و من المستحسن أن تحفظه عن ظهر قلب :
\Picture{Chapter_II-1_Compilation-Schema}
هذا مخطط حقيقي عمّا يجري بالضبط أثناء التجميع و سأشرحه لكم :
\begin{enumerate}
  \item \textbf{المعالج القبلي} :
المعالج القبلي هو برنامج ينطلق قبل عملية الترجمة و هو مخصص للقيام بتشغيل تعليمات نطلبها منه عن طريق ما سميناه بتوجيهات المعالج القبلي، و هي الأسطر الشهيرة التي تبدأ يإشارة
\InlineCode{\#}.

لحد الآن توجيهة المعالج القبلي الوحيدة الّتي نعرفها هي
\InlineCode{\#include}
و الّتي تسمح بإدراج ملف في ملف آخر. طبعا للمعالج القبلي مهام اخرى سنتعرف إليها لاحقا لكن ما يهمنا الآن هو ما أعطيتك إياه.
لمعالج القبلي يقوم إذن بـ"استبدال" أسطر
\InlineCode{\#include}
بملفات أخرى نحددها، فهو يضمّن داخل كل الملفات
\InlineCode{.c}
الملفات
\InlineCode{.h}
التي نعينها و نطلب منه تضمينها في السابقة.
  \item \textbf{الترجمة} : هذه الخطوة المهمة التي تسمح بتحويل ملفاتك إلى شفرات ثنائية مفهومة للحاسوب. فالمترجم ييقوم بتجميع الملفات
\InlineCode{.c}
واحدا بواحدا حتى ينهيها جميعها، و لضمان ذلك يجب أن تكون كل الملفات موجزدة في المشروع (بحيث تظهر في القائمة اليسارية).

سيقوم المترجم بتوليد ملف
\InlineCode{.o}
أو
\InlineCode{.obj}
و هذا راجع لنوع المترجم و هي ملفات ثنائية مؤقتة، و على أي حال تحذف هذه الملفات في نهاية الـترجمة و لكن بتعديل الخيارات يمكنك الإبقاء عليها لكن لكن يكون هناك من داع.
  \item \textbf{إنشاء الروابط} :
محرر الروابط
(\textenglish{Linker})
هو برنامج يعمل على جمع الملفات الثنائية من نوع
\InlineCode{.o}
في ملف واحد كبير : الملف التنفيذي النهائي ! هذا الملف يحمل الصيغة
\InlineCode{.exe}
في الويندوز. إن كنت تملك نظام تشغيل آخر فسيأخذ الصيغة المناسبة له.
\end{enumerate}
و هكذا تكون قد تعرفت على الطريقة الحقيقية لعمل الترجمة. كما قلتها و أكررها المخطط أعلاه مهم للغاية، فهو يفرق بين مبرمج يقوم بجمع و نسخ الشفرة المصدرية دون فهم و بين مبرمج يعرف تماما ما عليه فعله !

معظم الأخطاء تحدث في الـترجمة و قد تأتي من محرر الروابط و هذا يعني أنه لم يتمكن من تجميع كل الملفات
\InlineCode{.o}
بطريقة صحيحة (ربمّا لفقدان إحداها).

لا يزال المخطط أعلاه غير كامل، إذ أن المكتبات لم تظهر فيه ! إذن كيف تحدث العملية عندما نستخدم مكتبات برمجية ؟

تبقى بداية المخطط هي نفسها، لكن يقوم محرر الروابط بأعمال اخرى، سيقوم بتجميع ملفاتك
\InlineCode{.o}
(المؤقتة) مع مكتبات جاهزة تحتاجها (
\InlineCode{.a}
أو
\InlineCode{.lib}
وفقا للمترجم) :
\Picture{Chapter_II-1_Compilation-Schema-Libraries}
هكذا ننتهي و يكون مخططنا هذه المرة كاملا، ملفاتك من المكتبات
\InlineCode{.a}
(أو
\InlineCode{.lib})
يتم تجميعها في الملف التنفيذي مع الملفات
\InlineCode{.o}.

فبهذه الطريقة نتحصل في النهاية على برنامج كامل
100\%
و الذي يحتوي كل التعليمات اللازمة للجهاز لتشرح له كيف يعرض نصّا !\\
كمثال، الدالة
\InlineCode{printf}
توجد في ملف
\InlineCode{.a}
و طبعا سيتم تجميعها مع الشفرة المصدرية الخاصّة بنا في الملف التنفيذي.

لاحقا سنتعلم كيف نستخدم المكتبات الرسومية التي نجدها أيضا في ملفات
\InlineCode{.a}
و تعطي للجهاز تعليمات خاصة بكيفية إظهار نافذة على الشاشة كمثال. لكن طبعا، لن ندرسها الآن ، كلّ شيء في وقته.

\section{نطاق الدوال و المتغيرات}
لننهي هذا الدرس، يجب أن أطلعكم عما يسمى بـ
\textbf{نطاق}
المتغيرات و الدوال، سنعرف إمكانية الوصول للدوال و المتغيرات، يعني متى يمكننا استدعاؤها.

\subsection{المتغيرات الخاصة بدالة}
عندما تصرّح عن متغير في داخل دالة يتم حذف هذا المتغير من الذاكرة مع نهاية الدالة.
\begin{Csource}
int triple(int number)
{
	int result = 0; // The variable result is created in the memory
	result = 3 * number;
	return result;
} // The function finished, the variable result is destroyed
\end{Csource}
كلّ متغير تم التصريح عنه في دالة، لا يكون موجودا سوى حينما تكون الدالة في طور الإشتغال.\\
لكن ماذا يعني هذا تحديدا ؟ أنه لا يمكن الوصول إليه من خلال  دالة اخرى !
\begin{Csource}
int triple(int number);
int main(int argc, char *argv[])
{
	printf("The triple of 15 = %d\n", triple(15));
	printf("The triple of 15 = %d",  result); // Error
	return 0;
}

int triple(int number)
{
	int result = 0;
	result = 3 * number;
	return result;
}
\end{Csource}

في الدالة الرئيسية أحاول الوصول إلى المتغير
\InlineCode{result}
و بما أن هذا المتغير تم التصريح عنه داخل الدالة
\InlineCode{triple}
فطبعا لا يمكنني الوصول إليه من خلال الدالة
\InlineCode{main} !

\textbf{تذكّر جيّدا}
: كل متغير تم التصريح عنه داخل دالة، لا يسرى مفعوله إلا في داخل هذه الدالة نفسها ! و نقول أن المتغير محلّي
(\textenglish{local}).

\subsection{المتغيرات الشاملة
(\textenglish{global variables})
: فلتتجنّبها}
\subsubsection{متغير شامل قابل للوصول إليه من خلال كلّ الملفات}
إنه من الممكن التصريح عن متغير يمكن الوصول إليه من خلال كل الدوال من ملفات المشروع.
سأريك كيفية فعل ذلك كي تعرف بأنه أمر موجود، لكن عموما تجنب القيام بذلك.
د يظهر أنها ستسهل لك التعامل مع الشفرة المصدرية لكن قد يؤدي بك هذا لوجود العديد من المتغيرات التي يمكننا الوصول إليها من كلّ مكان مما سيصعّب عليك عملية إدارتها.

للتصريح عن متغير
\underline{شامل}
(\textenglish{global})،
يجب أن تقوم بذلك خارج كلّ الدوال، يعني في أعلى الملف، و عموما بعد أسطر الـ
\InlineCode{\#include}.
\begin{Csource}
#include <stdio.h>
#include <stdlib.h>

int result = 0; // Declaration of a global variable
void triple(int number ); // Prototype of the function
int main(int argc, char *argv[])
{
	triple(15); // We call the function triple which is going to modify the variable result
	printf("The triple of 15 = %d\n", result); // We can access to the variable result
	return 0;
}

void triple(int number)
{
	result = 3 * number;
}
\end{Csource}
في هذا المثال، الدالة
\InlineCode{triple}
لا تُرجع أي شيء
(\InlineCode{void}).
إنها تقوم بتعديل قيمة المتغير الشامل
\InlineCode{result}
التي يمكن للدالة
\InlineCode{main}
أن تسترجعه.

المتغير
\InlineCode{result}
يمكن الوصول إليه من خلال كل الملفات في المشروع و منه يمكننا مناداتها من خلال
\underline{كلّ}
دوال البرنامج.
\begin{warning}
  هذا شيء يجب ألا يتواجد في برامج الـ
\textenglish{C}
الخاصة بك. من المستحسن استعمال التعليمة
\InlineCode{return}
لإرجاع النتيجة بدل التعديل عليه كمتغير شامل.
\end{warning}

\subsubsection{متغير شامل قابل للوصول إليه من خلال ملف واحد}
المتغير الشامل الذي أريتك إياه قبل قليل يمكن الوصول إليه من خلال كل الملفات الخاصة بالمشروع.\\
يمكننا جعل متغيّر شامل مرئيا فقط في الملف الذي تتواجد به. و لكنه يبقى متغيرا شاملا على أية حال حتى و إن كنا نقول أنه ليس كذلك إلا على الدوال المتواجدة في ذات الملف و ليس على كل دوال البرنامج.

لإنشاء متغير شامل مرئي في ملف واحد نستعمل الكلمة المفتاحية
\InlineCode{static}
قبله :
\begin{Csource}
static int result = 0;
\end{Csource}

\subsubsection{متغير ساكن
(\textenglish{static})
بالنسبة لدالة}
حذار: الأمر حساس هنا قليلاً. إن استعملت الكلمة المفتاحية
\InlineCode{static}
عند التصريح عن متغير في داخل دالة، فهذا معنى آخر غير الخاص بالمتغيرات الشاملة.\\
في هذه الحالة، لا يتم حذف المتغير الساكن مع نهاية الدالة، بل حينما نستدعي الدالة مرّة أخرى، سيحفظ المتغير قيمته مثلا :
\begin{Csource}
int triple(int number)
{
	static int result = 0; // The first time when the variable is created
	result = 3 * number;
	return result;
} // when we exit the function, the variable is not destroyed
\end{Csource}
ماذا يعني هذا بالضبط ؟\\
يعني أنه يمكننا استدعاء الدالة لاحقا و يبقى المتغير
\InlineCode{result}
محتفظا بنفس القيمة الاخيرة.

و هذا مثال آخر للفهم أكثر :
\begin{Csource}
int increment();

int main(int argc, char *argv[])
{
	printf("%d\n", increment());
	printf("%d\n", increment());
	printf("%d\n", increment());
	printf("%d\n", increment());

	return 0;
}

int increment()
{
	static int number= 0;

	number++;
	return number;
}
\end{Csource}
\begin{Console}
1
2
3
4
\end{Console}
هنا، في المرة الأولى التي نطلب فيها الدالة
\InlineCode{increment}،
يتم انشاء المتغير
\InlineCode{number}.
ثم نقوم بزيادة 1 إلى قيمته. و ما إن تنتهي الدالة لا يمسح المتغير.

عندما نطلب الدالة للمرة الثانية، يتم ببساطة قفز السطر الخاص بالتصريح بالمتغير، و لا نقوم بإعادة إنشاء المتغير بل فقط نعيد استعمال المتغير الذي أنشأناه سابقا. عندما يأخذ المتغير القيمة 1، تصبح قيمته 2 ثم 3 ثم 4 ... الخ.

هذا النوع من المتغيرات ليس مستعملا بكثرة، لكن يمكنه مساعدتك في بعض الأحيان و لهذا ذكرته في الدرس.

  \chapter{المؤشّرات}
لقد حان الوقت لنكتشف المؤشرات. خذ نفسا عميقا قبل أن تقرر قراءة هذا الدرس لأنه لن يكون درساً للهو و المرح. تمثل المؤشرات واحداً من المبادئ الأكثر أهمية و حساسية في لغة الـ
\textenglish{C}
. و إن كنت أصرّ على أهميتها فهذا  لأنه لا يمكنك إالبرمجة بـ
\textenglish{C}
دون معرفتها و فهمها جيّدا. المؤشرات موجودة في كلّ مكان، و لقد استعملتها من قبل دون أن تعلم بذلك.

كثير من المتعلّمين يصلون إلى المؤشرات و يواجهون صعوبات في فهمها. سنعمل على ألا يكون الأمر مماثلا بالنسبة لك. ضاعف التركيز و خذ الوقت اللازم لفهم المخططات التي سأقدمها لك في هذا الفصل.

\section{مشكل مضجر بالفعل}
واحد من أكبر المشاكل مع المؤشرات هي أنّه بالإضافة إلى أنها صعبة الاستيعاب قليلا بالنسبة للمبتدئين، فإن المتعلّم لا يعرف ما هي أهميتها و فيما يمكننا استعمالها.

يمكنني أن أقول لك بأن "المؤشرات لا يمكن الاستغناء عنها في أي برنامج
\textenglish{C}
، صدقني !"، لكنّي أعرف أن هذه الحجّة ليست كافية لك.

سأطرح عليك مشكلاً لا يمكنك حلّه إلا باستخدام المؤشرات. سيكون هذا مقدّمتنا في هذا الفصل. سنعود إليه في نهاية هذا الفصل و سنترون حلّه باستعمال ما تعلّمتموه في هذا الفصل.

إليك المشكل: أريد كتابة دالة تقوم بإرجاع قيمتين مختلفتين. ستجيبني "هذا مستحيل !". بالفعل، الدالة لا يمكنها ارجاع سوى قيمة واحدة.
\begin{Csource}
int function()
{
	return value;
}
\end{Csource}
اذا استخدمنا
\InlineCode{int}
ترجع لنا قيمة من نوع
\InlineCode{int}
(بفضل التعليمة
\InlineCode{return}).

يمكننا ايضا كتابة دالة لا تُرجع اية قيمة باستخدام الكلمة المفتاحية
\InlineCode{void}.
\begin{Csource}
void function()
{

}
\end{Csource}
لكن إرجاع قيمتين مختلفتين في نفس الوقت... هذا أمر مستحيل لأنه لا يمكننا استعمال تعليمتي
\InlineCode{return}.

لنفرض أنني أريد كتابة دالة أعطيها كمدخل عددا من الدقائق. تقوم الدالة بإرجاع عدد الساعات و الدقائق المواقفة لها.
\begin{itemize}
  \item إذا أعطيت القيمة 45  الدالة ترجع 0 ساعة و 45 دقيقة.
  \item إذا أعطيت القيمة 60 الدالة ترجع القيمة  1 ساعة و 0 دقائق.
	\item إذا أعطيت القيمة 90 الدالة ترجع القيمة  1 ساعة و 30 دقائق.
\end{itemize}
نكن مجانين و لنجرّب ذلك :
\begin{Csource}
#include <stdio.h>
#include <stdlib.h>

/* I put the prototype at the top.
Because it's a short code, I don't put it in a .h file.
In a real program I would have put the prototype
in a separate .h file of course */

void minutesDevision(int hours, int minutes);

int main(int argc, char *argv[])
{
	int hours = 0, minutes = 90;

/* We have a variable "minutes" equals to 90.
   after calling the function, I want from the variable
“hours" to take the value 1 and from my variable
"minutes" to take the value 30 */

	minutesDivision(hours, minutes);
	printf("%d hours and %d minutes", hours, minutes);
	return 0;
}

void minutesDevision(int hours, int minutes)
{
	hours = minutes / 60;  // 90 / 60 = 1
	minutes = minutes % 60; // 90 % 60 = 30
}
\end{Csource}
النتيجة :
\begin{Console}
0 hours and 90 minutes
\end{Console}
لم تشتغل ! ما الأمر يا ترى ؟
في الواقع، عندما نبعث متغيرا إلى دالة، يتم إنشاء نسخة من المتغير، و لهذا فالمتغير
\InlineCode{hours}
في الدالة
\InlineCode{minutesDevision}
هو ليس نفسه الذي في الدالة
\InlineCode{main} !
إنه فقط نسخة !

الدالة
\InlineCode{minutesDevision}
تقوم بعملها. ففي داخلها المتغيران
\InlineCode{hours}
و
\InlineCode{minutes}
يحملان القيمتين الصحيحتين : 1 و 30.

لكن بعد ذلك، تتوقف الدالة مباشرة عند الوصول إلى الحاضنة الغالقة، مثلما تعلمنا سابقا: المتغيرات الخاصة بدالة يتم حذفها مباشرة عند انتهاء الدالة. اذن النسخ عن المتغيرات
\InlineCode{minutes}
و
\InlineCode{hours}
تُمسح.
نرجع بعد ذلك للدالة
\InlineCode{main}.
و التي فيها متغيراتنا
\InlineCode{minutes}
و
\InlineCode{hours}
تحملان القيمتين 0 و 90. لقد فشلنا !
\begin{information}
	لاحظ إذن،بما أن الدالة تقوم بنسخ المتغيرات التي نعطيها لها، لست مضطراً لستمية متغيراتك بنفس الأسماء التي تحملها في الـدالة الرئيسية
\InlineCode{main}.
و بالتالي يمكنك ببساطة كتابة :

\InlineCode{void minutesDivision(int h, int m)}\\
\InlineCode{h}
للساعات و
\InlineCode{m}
للدقائق.\\
إن كانت متغيراتك لم تسمّى بنفس الطريقة في الدالة و في
\InlineCode{main}
فهذا لا يطرح أيّ مشكل !
\end{information}


باختصار، يمكنك إعادة المشكل في كلّ الاتجاهات. يمكنك محاولة بعث قيمة باستخدام الدالة (باستخدام
\InlineCode{return}
و باستخدام النوع
\InlineCode{int}
للدالة) ، لكن لا يمكنك إعادة أكثر من قيمة واحدة من بين القيمتين، هذا مشكل مطروح إذن. كما لا يمكنك استعمال متغيرات شاملة لأن هذا أمر غير مستحسن إطلاقا.

حسناً المشكل لازال مطروحاً، كيف يمكننا حلّه باستخدام المؤشرات ؟

\section{الذاكرة، مسألة عنوان}
\subsection{تذكير بالمكتسبات القبليّة}
سأعود بك قليلاً إلى الوراء، هل تتذكر درس المتغيرات ؟

أيّا كانت إجابتك، أنصحك بأن تعود إلى ذلك الفصل و تقرأ منه الجزء الذي يحمل عنوان (مسألة ذاكرة). هناك مخطط مهم جداً سأقترحه عليك من جديد (الصورة الموالية) :
\Picture{Chapter_II-2_RAM-Schema}
هكذا نقوم تقريباً بتمثيل الذاكرة الحية (الرام) الخاصة بالحاسوب.

يجب قراءة المخطط سطراً بسطر، السطر الأول يمثل "الخانة" الخاصة بأول الذاكرة. لكل خانة رقم، هذا الرقم يمثل
\textbf{عنوانها}
(تذكّر هذا المصطلح جيدا). تحتوي الذاكرة على عدد كبير جداً من العناوين تبدأ من الرقم 0 و تنتهي بالرقم
\textit{(ضع رقماً كبيراً جداً هنا)}.
عدد العناوين التي تتوفر عليها تعتمد على حجم الذاكرة التي يحتوي عليها الجهاز الخاص بك.

في كلّ عنوان يمكننا تخزين عدد واحد فقط. لا يمكننا تخزين عددين في نفس العنوان.

الذاكرة ليست مصنوعة سوى لتخزين الأعداد. لا يمكنها تخزين لا حروف و لا جُمل. و للتخلص من هذا المشكل تم اختراع جدول يقوم بالربط بين الحروف و الأعداد. يقول الجدول مثلاً :"العدد 89 يمثّل الحرف
\textenglish{Y}".
سنعود في درس لاحق إلى عملية التحكّم في الحروف. حاليّا، سنتكفي بالتكلّم عن عمل الذاكرة.

\subsection{عنوان و قيمة}
حينما تنشئ متغيراً
\InlineCode{age}
من نوع
\InlineCode{int}
مثلا، بكتابة :
\begin{Csource}
int age = 10;
\end{Csource}
يطلب البرنامج من نظام التشغيل (الويندوز مثلا) الإذن لاستعمال جزء من الذاكرة. نظام التشغيل يجيب بالإشارة إلى أي عنوان سيسمح لك بتخزين العدد.

هنا تكمن أحد أهم وظائف نظام التشغيل : نقول أنه يحجز الذاكرة للبرامج. يمكننا القول أنه هو القائد، يتحكم في كلّ برنامج و يتأكد من أن هذا الأخير له الإذن لاستعمال الذاكرة في المكان الذي يطلبه.
\begin{information}
إن هذا هو السبب الرئيسي في توقف البرامج عن العمل : إذا حاول برنامجك الوصول إلى مكان غير مسموح له بالوصول إليه في الذاكرة، سيرفض نظام التشغيل و يوقف تشغيله بشكل عنيف (لأنه القائد هنا). بينما يتلقّى المستعمل نافذة خطأ تحتوي على رسالة تشير بأن البرنامج يحاول القيام بعملية غير لائقة.
\end{information}
لنعد للمتغير
\InlineCode{age}
. تم تخزين القيمة 10 في مكان ما من الذاكرة، لنقل مثلاً في العنوان رقم 4655.\\
ما يحدث (و هذا دور المترجم) هو أن الكلمة
\InlineCode{age}
يتم تعويضها بالعنوان 4655 لحظة التنفيذ. مما يعني أنه في كلّ مرة قمت فيها بكتابة الكلمة
\InlineCode{age}
في الشفرة المصدرية، يتم تعوضيها بـ4655، و بهذا يرى الجهاز إلى أي عنوان في الذاكرة عليه الذهاب. و منه يجيب بكلّ فخر بأن المتغير
\InlineCode{age}
يحتوي القيمة 10.

نحن نعرف إذا كيف نسترجع قيمة متغير، يكفي بكلّ بساطة أن نكتب الكلمة
\InlineCode{age}
في الشفرة المصدرية. إذا أردنا إظهار السنّ، يمكننا استعمال الدالة
\InlineCode{printf} :
\begin{Csource}
printf("The value of variable age is : %d", age);
\end{Csource}
النتيجة على الشاشة :
\begin{Console}
The value of variable age is : 10
\end{Console}
لا شيء جديد لحدّ الآن.

\subsection{الخبر المثير لليوم}
أنت تعرف كيف تظهر قيمة متغير، لكن هل تعرف أنه بإمكاننا أيضا إظهار عنوانه ؟

لكي نُظهر عنوان متغير، نستعمل الإشارة
\InlineCode{\%p}
(الحرف
\textenglish{p}
مأخوذ من الكلمة
\textenglish{pointer})
في الدالة
\InlineCode{printf}.
أي أننا لن نبعث للدالة
\InlineCode{printf}
المتغير في حدّ ذاته لكن نبعث لها عنوانه. و لفعل هذا، يجب عليك استعمال الإشارة
\InlineCode{\&}
أمام المتغير
\InlineCode{age}
كما طلبت منك أن تفعل مع الدالة
\InlineCode{scanf}
من قبل دون أن أشرح لك لماذا.

أكتب إذا:
\begin{Csource}
printf("The address of the variable age is  : %p", &age);
\end{Csource}
النتيجة :
\begin{Console}
The address of the variable age is : 0023FF74
\end{Console}
ما تراه هنا هو عنوان المتغير
\InlineCode{age}
في اللحظة التي طلبتُ فيها تنفيذ البرنامج من طرف حاسوبي. نعم نعم 0023FF74 هو رقم، هو فقط مكتوب في النظام الست عشري
(\textenglish{hexadecimal})
عوض النظام العشري الذي تعوّدنا عليه. لو تقوم بتعويض
\InlineCode{\%p}
بـ
\InlineCode{\%d}
فإنه سيظهر لك رقما عشريا كما تعوّدنا.
\begin{information}
	إذا شغّلت البرنامج على حاسوبك فمن المؤكّد أن تحصل على عنوان آخر. الأمر يعتمد على المكان في الذاكرة، البرامج المشتغلة، إلخ.
فيستحيل أن تتوقع العنوان الّذي سيتم تخزين المتغيّر فيه.
إذا قمت بتشغيل البرنامج عدة مرات الواحدة تلو الأخرى قد تتحصل على نفس النتيجة كون الذاكرة لم تتغير في ذلك الزمن القصير.
لكن بالمقابل إن أعدت تشغيل الحاسوب فستتحصل بكل تأكيد على نتائج مختلفة.
\end{information}
إلى أين أريد الوصول بكلّ هذا ؟ أريدك أن تتذكّر التالي:
\begin{itemize}
	\item \InlineCode{age} : تعني قيمة المتغير.
	\item \InlineCode{\&age} : تعني عنوان المتغير.
\end{itemize}
عند استخدام
\InlineCode{age}
سيقرأ الحاسوب قيمة المتغيّر في الذاكرة. أمّا عند استخدام
\InlineCode{\&age}
فسيعيد العنوان الذي يوجد فيه المتغيّر.

\section{إستعمال المؤشرات}
لحدّ الآن، قمنا فقط بإنشاء متغيرات تحتوي على أعداد. الآن سنتعلّم كيف ننشئ متغيرات تحتوي على عناوين: هذا ما نسميه بالمؤشرات.
\begin{question}
	لكن … العناوين هي أعداد أيضاً، أليس كذلك ؟ هذا يعني أننا سنخزن أعداداً دائما !
\end{question}
هذا صحيح، لكن لهذه الأعداد معنى آخر : هي تشير إلى عنوان متغير آخر في الذاكرة.

\subsection{إنشاء مؤشّر}
لإنشاء متغير من نوع مؤشّر، يجب علينا أن نضيف الرمز
\InlineCode{*}
أمام إسم المتغير :
\begin{Csource}
int *myPointer;
\end{Csource}
\begin{information}
	لاحظ أنه يمكننا أيضا أن نكتب
\InlineCode{int* myPointer;}

لهذا نفس المعنى. لكن الطريقة الأولى هي المفضّلة. في الواقع، إن كنت تريد التصريح عن العديد من المؤشرات في نفس السطر، سيكون عليك أن تعيد كتابة النجمة أمام كل اسم :\\

\InlineCode{int *pointer1, *pointer2, *pointer3;}
\end{information}
كما قلت لك، من المهمّ أن تقوم بإعطاء قيم إبتدائية للمتغيرات منذ البداية، و ذلك بإعطائها القيمة 0 مثلا ! إنه من المهم أكثر أن تفعل نفس الشيء مع المؤشرات.\\
لتهيئة مؤشّر، نعطيه قيمة افتراضية، لا نستعمل غالبا القيمة 0 و لكن الكلمة المفتاحية
\InlineCode{NULL}
(أكتبها بأحرف كبيرة).
\begin{Csource}
int *myPointer = NULL;
\end{Csource}
هنا لدينا مؤشر يحمل القيمة الإبتدائية
\InlineCode{NULL}.
هكذا ستعرف لاحقاً في البرنامج أن المؤشر لا يحتوي على أي عنوان.

ما الذي يحصل؟ ستقوم هذه الشفرة المصدرية بحجز خانة في الذاكرة كما لو أننا أنشأنا متغيراً عادياً. الشيء الذي يتغير هو أن المؤشر سيحتوي عنوانا. عنوان متغير آخر.

لم لا عنوان المتغير
\InlineCode{age}
؟ أنت تعرف الآن كيف تشير إلى عنوان متغير في مكان قيمته (باستعمال الرمز
\InlineCode{\&})،
هيا بنا إذن ! هذا ما عليك كتابته :
\begin{Csource}
int age = 10;
int *PointerOnAge = &age;
\end{Csource}
السطر الأول يعني : "أنشئ متغيرا من نوع
\InlineCode{int}
يحمل القيمة 10". السطر الثاني يعني "أنشئ متغيراً من نوع مؤشّر قيمته هي عنوان المتغير
\InlineCode{age}".

يقوم إذا السطر الثاني بمهمّتين معاً. لكي لا تختلط عليك الأمور، إعلم أنه يمكننا تقسيم السطر إلى سطرين :
\begin{Csource}
int age = 10;
int *PointerOnAge; // 1) means “I create the pointer”
PointerOnAge = &age; // 2) means the pointer “PointerOnAge contains the address of age”
\end{Csource}
يمكنك الملاحظة أنه لا يوجد في لغة الـ
\textenglish{C}
نوع نسميه
"\textenglish{pointer}"
كالنوع
\InlineCode{int}
و
\InlineCode{double}.
أي أنه لا يمكننا أن نكتب :
\begin{Csource}
pointer PointerOnAge;
\end{Csource}
في مكان هذا، نستعمل الرمز
\InlineCode{*}
، و لكن نستمر في كتابة
\InlineCode{int}.
ماذا يعني هذا ؟ في الواقع يجب أن نشير إلى نوع المتغير الذي سيحوي عنوانه المؤشر. بما أن المؤشر
\InlineCode{PointerOnAge}
سيحتوي عنوان المتغير
\InlineCode{age}
(الذي هو من نوع
\InlineCode{int})،
إذا فالمؤشر يجب أن يكون من نوع
\InlineCode{int*}
! إذا كان المتغير من نوع
\InlineCode{double}
فإنه يجب عليّ أن أكتب
\InlineCode{double *myPointer}.

\textbf{إصطلاح}
: نقول بأن المؤشّر
\InlineCode{PointerOnAge}
يؤشّر على المتغير
\InlineCode{age}.

المخطط التالي يلخّص ما يحصل في الذاكرة :
\Picture{Chapter_II-2_RAM-Schema-Pointer}
في هذا المخطط، تم تعويض المتغير
\InlineCode{age}
بالعنوان 177450 (أنت ترى بأن قيمته هي 10)، و المؤشّر
\InlineCode{PointerOnAge}
تم تعويضه بالعنوان 3 (هذه محض صدفة).

حينما يتم إنشاء المؤشر، يقوم نظام التشغيل بحجز خانة في الذاكرة كما فعل مع المتغير
\InlineCode{age}.
الشيء المختلف هنا هو أن المتغير
\InlineCode{PointerOnAge}
له معنى آخر. أنظر للمخطط جيداً : قيمته هي عنوان المتغير
\InlineCode{age}.

هذا، عزيزي القارئ، هو السرّ المطلق من وراء كتابة البرامج في لغة الـ
\textenglish{C}.
بهذا نحن ندخل في عالم المؤشرات العجيب !
\begin{question}
	و ... ما هي فائدة هذا ؟
\end{question}
هذا لا يقوم بتحويل الحاسوب إلى آلة صنع القهوة، طبعا. لكن الآن لدينا المؤشر
\InlineCode{PointerOnAge}
يحتوي عنوان المتغير
\InlineCode{age}.

فلنحاول رؤية ما يحتويه المؤشر بالإستعانة بالدالة
\InlineCode{printf} :
\begin{Csource}
int age = 10;
int *PointerOnAge = &age;
printf("%d", PointerOnAge);
\end{Csource}
\begin{Console}
177450
\end{Console}
هذا ليس مفاجئاً، نحن نطلب قيمة
\InlineCode{PointerOnAge}
و قيمته هي عنوان المتغير
\InlineCode{age}
(أي 177450).\\
ماذا نفعل لكي نطلب قيمة المتغير المتواجدة في العنوان الذي يشير إليه المؤشّر
\InlineCode{PointerOnAge}
؟ يجب أن نضع الرمز
\InlineCode{*}
أمام إسم المؤشّر :
\begin{Csource}
int age = 10;
int *PointerOnAge = &age;
printf("%d, *PointerOnAge);
\end{Csource}
\begin{Console}
10
\end{Console}
ها قد وصلنا ! بوضع الرمز
\InlineCode{*}
أمام إسم المؤشّر، يمكننا الوصول إلى قيمة المتغير
\InlineCode{age}.

لو استعملنا الرمز
\InlineCode{\&}
أمام اسم المؤشّر، سنتحصل على العنوان الذي يتواجد به المؤشّر (هنا الرقم 3).
\begin{question}
	ماذا نربح هنا ؟ لقد نجحنا في تعقيد الأمور لا أكثر. لم نكن نحتاج إلى مؤشّر لنظهر قيمة المتغير
\InlineCode{age} !
\end{question}
هذا السؤال (الذي لا مفر من طرحه) شرعي، حاليّا الهدف ليس واضحا، لكن قليلاً بقليل، و مع تقدّم الدروس، ستفهم بأن كلّ هذه المبادئ لم يتم اختراعها من أجل تعقيد الأمور بكلّ سذاجة.

المهم هو أن تفهم المبدأ الآن و بعده ستتوضح الأمور لوحدها رويداً رويداً.

  \chapter{الجداول}
هذا الدرس هو ملحق مباشر للدرس المتعلق بالمؤشرات، و سيعلّمك أهميتها أكثر. إن كنت تعتقد بأنك قادر على تفادي المؤشرات فأنت مخطئ ! هي في كلّ مكان في لغة الـ\textenglish{C}. لقد حذّرتك !

سنتعلم في هذا الدرس كيف ننشئ متغيرات من نوع "جداول". الجدوال مهمّة للغاية في لغة الـ\textenglish{C} لأنها تساعد في تنظيم سلسلة من القيم.

نبدأ هذا الدرس ببعض الشروحات و التفسيرات حول كيفية عمل الجداول في الذاكرة (سأقدم لك الكثير من المخططات التفسيرية). هذه المقدمات حول الذاكرة مهمة جداً : ستساعدك في في معرفة عمل الجداول. فمن المستحسن أن يعرف المبرمج ما يقوم به كي يتحكم في برامجه أكثر، أليس كذلك ؟

\end{document}
